\chapter{Distanzverteilungskomplexit"at}


% gcc
% Gnuplot
% PostScript

\section{Distanzverteilungen}
\label{ddistr-method}

\subsection{Ansatz der Distanzverteilungsanalyse und formale Definition}

Die Lebensformen, die derzeit auf der Erde beobachtet werden,
bilden auf mehreren Ebenen deutlich voneinander abgegrenzte Gruppen, die als Taxons bezeichnet werden.
Diese taxonomische Struktur wird durch die Klassifizierung der Lebensformen 
in verschiedenen \introdef{taxonomische Kategorien} (Spezies,
Gattung, Familie, Ordnung, Klasse, Abteilung bzw.\ Stamm, Reich)
wiedergegeben. Die Spezies ist dabei die speziellste taxonomische
Kategorie, die "ubergeordneten Kategorien sind zunehmend allgemeiner,
das Reich schlie"slich ist die umfassendste, allgemeinste Kategorie.

Jede taxonomische Kategorie ist durch charakteristische Gemeinsamkeiten
gekennzeichnet. Je umfassender die Kategorie ist, desto geringer ist das
charakteristische Ausma"s an Gemeinsamkeiten, desto gr"o"ser ist also
auch die Variabilit"at und Diversit"at in den Taxons der entsprechenden
Kategorie. In einem Taxon einer allgemeinen taxonomischen Kategorie k"onnen Organismen
also nur entfernt miteinander verwandt sein, Organismen, die
in einer speziellen taxonomischen Kategorie zu einem gemeinsamen Taxon
geh"oren, sind dagegen eng miteinander verwandt.

Die "Ahnlichkeit bzw.\ Unterschiedlichkeit von verschiedenen Organismen oder
von Teilkomponenten verschiedener Organismen kann quantitativ durch ein
geeignetes Distanzma"s ausgedr"uckt werden. Je enger zwei Organismen miteinander
verwandt sind, desto kleiner ist die Distanz zwischen beiden. Zwei Organismen,
die zu einem Taxon einer speziellen taxonomischen Kategorie geh"oren,
werden somit eine charakteristische, geringe Distanz zueinander haben.
Eine ebenfalls charakteristische, gr"o"sere Distanz wird man dagegen
zwischen Organismen ermitteln, die verschiedenen Taxons der speziellen
Kategorie angeh"oren und nur auf einer allgemeinen Ebene dem gleichen
Taxon zugerechnet werden.
% hier ggfs. Beispiele

Betrachtet man nun eine Population verschiedener Lebensformen, so kann man nach dieser
"Uberlegung einen Eindruck von der taxonomischen Struktur in dieser Population erhalten,
indem man die Verteilung der Werte in einer Matrix der paarweisen Distanzen untersucht.
Eine H"aufung von Distanzwerten in einem bestimmten Bereich deutet darauf hin, da"s in der
Population Paare von Lebensformen existieren, deren speziellste gemeinsame taxonomische
Kategorie einen in diesem Bereich liegenden charakteristischen Distanzwert hat. Existiert
also nur eine taxonomische Kategorie, ist mit einer H"aufung aller Distanzwerte im Bereich
des entsprechenden einen charakteristischen Distanzwerts zu rechnen. Mehrere Peaks weisen
dagegen auf eine komplexe taxonomische Struktur mit mehreren Kategorien hin. Die Verteilung
der Distanzwerte kann somit als Indiz f"ur die taxonomische Struktur einer Population dienen
(vgl.\ \cite{Higgs92}).
Abb.\ \ref{taxcat-fig} illustriert die Beziehung
zwischen taxonomischen Kategorien und charakteristischen Distanzen.

\begin{figure}

\unitlength1cm
\begin{picture}(16,5.5)
\put(0,0.5){\makebox(8,5)[bl]{\epsfxsize=7cm \epsffile{h.eps}}}
\put(9,0.5){\makebox(7,5)[br]{\epsfxsize=6cm \epsffile{h_distr.eps}}}
\put(0,0){\makebox(16,0.5){(a) Randomisierte Taxonomie}}
\end{picture}

\unitlength1cm
\begin{picture}(16,5.5)
\put(0,0.5){\makebox(8,5)[bl]{\epsfxsize=7cm \epsffile{l.eps}}}
\put(9,0.5){\makebox(7,5)[br]{\epsfxsize=6cm \epsffile{l_distr.eps}}}
\put(0,0){\makebox(16,0.5){Konvergierte Taxonomie}}
\end{picture}

\unitlength1cm
\begin{picture}(16,5.5)
\put(0,0.5){\makebox(8,5)[bl]{\epsfxsize=7cm \epsffile{m.eps}}}
\put(9,0.5){\makebox(7,5)[br]{\epsfxsize=6cm \epsffile{m_distr.eps}}}
\put(0,0){\makebox(16,0.5){Komplexe Taxonomie}}
\end{picture}

\caption[Beziehung zwischen taxonomischen Kategorien und Distanzenverteilungen]
{\label{taxcat-fig}
Beziehung zwischen taxonomischen Kategorien und Distanzenverteilungen. Die Abbildungen zeigen
links Verwandtschaftsb"aume, die zu Populationen unterschiedlicher Charakteristik konstruiert wurden.
Die Skala neben jedem Baum zeigt den Ma"sstab f"ur die L"ange der Kanten. L"angen werden nur in
der senkrechten Richtung dargestellt.
Rechts ist jeweils die Verteilung der Werte in der Distanzmatrix des entsprechenden Baums dargestellt.
(a) Eine Population aus
zufallsgesteuert generierten Genomen weist eine randomisierte Taxonomie auf. Alle Genome sind verwandtschaftlich
weit voneinander entfernt. Die Distanzwerte h"aufen sich in einem engen Bereich um den Erwartungswert
f"ur die Distanz zwischen unkorrelierten Genomen. (b) In einer Population, die in einem kleinen
Teilraum des Sequenzraums konvergiert ist, sind alle Genome paarweise eng miteinander verwandt.
Gr"o"sere Cluster von Genomen sind sogar vollst"andig identisch. Die Distanzen h"aufen sich im
Bereich kleiner Werte. (c) Verwandtschaftsbaum einer Population mit komplexer Taxonomie. Zwei verschiedene
taxonomische Kategorien sind in dem Baum deutlich unterscheidbar. Es existieren drei Cluster mit
engerer Verwandtschaft zwischen den Genomen, die die Ebene einer spezielleren taxonomischer Kategorie
repr"asentieren. Die Verwandtschaftsstruktur der Cluster untereinander stellt eine "ubergeordnete
taxonomische Kategorie dar; Genome, die zu unterschiedlichen Clustern geh"oren, haben eine erheblich
geringere Verwandtschaft als solche, die dem gleichen Cluster angeh"oren. Dementsprechend sind in der
Distanzverteilung zwei klar getrennte Peaks zu sehen. Die Distanzwerte sind deutlich gleichm"a"siger verteilt
als bei einer randomisierten oder einer konvergierten Population.
}
\end{figure}

Die Abbildung zeigt, da"s Distanzverteilungen von randomisierten und von konvergierten Populationen
von einer Verschiebung abgesehen "ahnlich sind, in beiden konzentrieren sich die h"aufigen Distanzwerte
in einem kleinen Bereich, wo sie einen hohen Peak bilden. Diese Gemeinsamkeit kann auf eine strukturelle
"Ahnlichkeit zwischen beiden Populationstypen zur"uckgef"uhrt werden. Eine konvergierte Population befindet
sich vollst"andig in einem kleinen Teilraum des gesamten Sequenzraums, ihre Individuen sind in diesem Teilraum jedoch
relativ gleichm"a"sig verteilt. Sowohl randomisierte als auch konvergierte Populationen k"onnen also als
gleichm"a"sig verteilte Populationen bezeichnet werden, wobei der Streubereich der Verteilung bei
randomisierten Populationen maximal, bei konvergierten Populationen hingegen klein ist. Populationen mit
komplexer Taxonomie geh"oren dagegen nicht zu den gleichm"a"sig verteilten Populationen. Vielmehr zerfallen
sie in mindestens zwei Subpopulationen, die jeweils als gleichm"a"sig verteilte Populationen charakterisiert
werden k"onnen. Das Beispiel in Abb.\ \ref{taxcat-fig}c zeigt eine in drei Subpopulationen eingeteilte
Population. Weitere taxonomische Kategorien entstehen, wenn derartige Subpopulationen ihrereseits in eine
charakteristische Clusterstruktur zerfallen.

Die formale Definition einer \introdef{Distanzverteilung} lautet:
Sei $P = \{g_0, g_1, \ldots , g_{n-1}\}$ eine Population aus $n$ Individuen
und $D$ ein diskretes Distanzma"s. Dann ist die Distanzverteilung $F$ definiert
durch:

\begin{equation}
\label{ddistr-eq}
F(d) := | \{(g_i, g_j) : g_i, g_j \in P, D(g_i, g_j) = d, i < j\} |
\end{equation}

Durch die Bedingung $i < j$ wird erreicht, da"s jede Distanz nur
einmal gez"ahlt wird.

Aus jedem realwertigen Distanzma"s $D$ kann durch die Unterteilung des Wertebereichs
in Intervalle einer Gr"o"se $\delta$ ein diskretes Distanzma"s $D_{\delta}$
abgeleitet werden:

\begin{equation}
D_{\delta}(g_i, g_j) := \left\lfloor \frac{D(g_i, g_j)}{\delta} \right\rfloor
\end{equation}

Damit k"onnen auch mit realwertigen Distanzma"sen Distanzverteilungen
berechnet werden, deren Gestalt nat"urlich durch die Gr"o"se des
Diskretisierungsintervalls $\delta$ ma"sgeblich mitbestimmt wird.


\subsection{Distanzverteilung molekularer Sequenzen}

\begin{figure}
\centerline{\epsfxsize=8cm \epsffile{publ-fullmads-raw.eps}}
\caption[Distanzverteilung der Aminos"auresequenzen von MADS-Boxen]
{\label{mads-ddistr}
Distanzverteilung der Aminos"auresequenzen von MADS-Boxen. Von einer Sequenzmenge aus 95  MADS-Box-Sequenzen
wurde eine Matrix der Hamming-Distanzen erzeugt und von dieser die Distanzverteilung ermittelt. Die MADS-Box-Sequenzen
sind 60 Aminos"auren lang. Der Datensatz stammt aus \protect\cite{Theissen95a}.
}
\end{figure}

Da die Lebensformen der Erde eine komplexe taxonomische Struktur mit mehreren Kategorien bilden, sollten
auch die in S"atzen molekularer Sequenzen gemessenen Distanzwerte eine entsprechend gleichm"a"sige Verteilung
aufweisen. Um dies zu belegen, wurde die Verteilung der Hamming-Distanzen von 95 Aminos"auresequenzen
von MADS-Boxen (s.\ \ref{modellansatz}) berechnet. Der Datensatz der MADS-Boxen stammt aus \cite{Theissen95a}.
Abb.\ \ref{mads-ddistr} zeigt die Distanzverteilung der MADS-Boxen. Ihre Charakteristik kann
eindeutig einer komplexen taxonomischen Struktur zugeordnet werden. Die Werte verteilen sich auf einem relativ
ausgedehnten Bereich, mehrere Peaks sind deutlich erkennbar. Da in diesem Datensatz, der aus \cite{Theissen95a}
entnommen ist, das gesamte Spektrum der taxonomischen Kategorien vertreten ist, sind die Peaks nicht so deutlich
separiert wie in Abb.\ \ref{taxcat-fig}c, sondern sie "uberlagern sich gegenseitig. Eine weitere Gemeinsamkeit
zwischen Abb.\ \ref{taxcat-fig}c und Abb.\ \ref{mads-ddistr} ist das Fehlen extrem hoher Distanzwerte.


% Einfache Distanzmasse: Hamming-Distanz, Editierabstand,
% Komplexe Masse: -> PHYLIP, dnadist, protdist (JC, K2P, ...)
% Sinnvolle Wahl: Normierung diskreter Masse -> kontinuierliches Mass
%    -> Diskretisierung s.d. Groessenordnung in der des diskreten
%    Masses wieder erreicht wird.

\subsection{Visualisierung von Distanzverteilungen}

Einzelne Distanzverteilungen k"onnen als Histogramme graphisch dargestellt
werden. Zur Visualisierung der zeitlichen Entwicklung der Distanzverteilungen
beim Ablauf eines Evolutionsprozesses ist die Histogrammdarstellung jedoch
nicht mehr praktikabel. Stattdessen wird eine Grauwertdarstellung verwendet,
die in Abb.\ \ref{ddistrs-fig} erl"autert wird.

\begin{figure}

\unitlength1cm
\begin{picture}(16,6)
\put(0,1){\makebox(8,5)[b]{\epsfxsize=8cm \epsffile{hgram.eps}}}
\put(9,1){\makebox(7,5)[b]{\epsfxsize=5cm \epsffile{hgpro.eps}}}
{\ttfamily
\put(0,0){\makebox(8,1){(a)}}
\put(9,0){\makebox(7,1){(b)}}
}
\end{picture}

\caption[Darstellung von Zeitserien von H"aufigkeitsverteilungen mithilfe von Grauwerten.]{\label{ddistrs-fig}
Darstellung von Zeitserien von H"aufigkeitsverteilungen mithilfe von Grauwerten.
(a) zeigt eine Serie von 10 Verteilungen in S"aulendiagrammform. Die Stirnfl"achen
der S"aulen sind in Abh"angigkeit von der H"ohe der S"aule gef"arbt; eine dunklere
F"arbung zeigt einen h"oheren H"aufigkeitswert an. \newline
(b) zeigt dieselbe Serie, es wurde jedoch ausschlie"slich die Farbcodierung
verwendet. Anschaulich gesprochen, erh"alt man (b) aus (a), indem man das
S"aulendiagramm (a) von oben betrachtet. Einzelne Werte k"onnen der Grauwertcodierung
nur schwer und mit geringer Genauigkeit entnommen werden. Globale Strukturen
treten in der Grauwertcodierung dagegen deutlicher hervor; die diagonale Streifenstruktur
ist in (b) erheblich besser zu erkennen als in (a).
}
\end{figure}

% dies bei erster Diskussion bei lndga

% Die speziellste taxonomische Kategorie in einer Population kann als
% Quasispecies \cite{Eigen71} aufgefa"st werden. Eine Population, in der
% mehrere Quasispecies koexisitieren, hat daher eine Distanzverteilung
% mit mindestens zwei Peaks, wobei ein Peak bei kleineren Werten von den
% intraspezifischen Distanzen, der andere Peak bei gr"o"seren Werten von
% den interspezifischen Distanzen herr"uhrt. Eine Distanzverteilung mit
% nur einem Peak zeigt dagegen eine Population mit nur einer Quasispecies
% an. Der Distanzbereich, in dem der Peak liegt, h"angt in diesem Fall von
% der Variationsbreite innerhalb der Quasispecies ab; ein Peak im Bereich
% gro"ser Distanzen weist auf gro"se intraspezifische Variablilit"at hin.

% Zeitserien von Distanzverteilungen

% Randomisierte Population -> 1 Quasispecies
% Uniforme Population -> 1 Quasispecies.

% Editierabstand als geeignetes Distanzmass


\section{Distanzverteilungskomplexit"at}
\label{ddc-method}

Aus den in (\ref{ddc-method}) dargelegten "Uberlegungen folgt, da"s die Werte
in der Distanzverteilung umso gleichm"a"siger verteilt sind, je mehr taxonomische
Kategorien die Population aufweist. Wenn alle Individuen einer Population ungef"ahr
gleich eng miteinander verwandt sind, wenn sie also alle eine Quasispecies bilden,
sind die Distanzwerte in dem Distanzbereich, der dem charakteristischen Verwandtschaftsgrad
der Quasispecies entspricht, konzentriert. Sind dagegen mehrere Quasispecies in der
Population, verteilen sich die Distanzwerte in einem breiteren Spektrum. Die
weiteste Verteilung tritt dann auf, wenn die Quasispecies in einer mehrstufigen
Hierarchie taxonomischer Kategorien organisiert sind.

Die Gleichm"a"sigkeit der Verteilung von Werten kann mithilfe der Shannon-Entropie \cite{Shannon}
quantitativ charakterisiert werden. Je gleichm"a"siger die Verteilung der Werte in einer Verteilung
ist, desto h"oher ist die Shannon-Entropie der entsprechenden Verteilung. Da eine gleichm"a"sige
Verteilung der Distanzwerte auf eine komplexe taxonomische Struktur schlie"sen l"a"st, wird
die Shannon-Entropie der Distanzverteilung als \introdef{Distanzverteilungskomplexit"at} (DVK) \cite{Kim95,Kim95a}
definiert:

Sei $F$ eine Distanzverteilung. Die Verteilung der relativen H"aufigkeiten der Distanzwerte $f$
kann dann berechnet werden durch:

\begin{equation}
f(d) := \frac{F(d)}{\sum_i F(i)} = \frac{2 \cdot F(d)}{n(n-1)}
\end{equation}

Dabei ist $n$ die Gr"o"se der Population, die Anzahl der $(i, j)$ mit $j < n$ und $i < j$ (also die
Anzahl Elemente in einem Dreieck der Distanzmatrix ausschlie"slich der Hauptdiagonalen, die entsprechend
Gl.\ \ref{ddistr-eq} in die Distanzverteilung eingehen) ist dann
$n(n-1) / 2$. Mit dieser Normierung der Verteilung ist nun die Definition
der Distanzverteilungskomplexit"at $C_d$:

\begin{equation}
C_d := -\sum_d f(d)\, \log(f(d))
\end{equation}

Eine hohe DVK zeigt eine gleichm"a"sige Verteilung der Distanzwerte in der Distanzmatrix
an und weist somit auf eine komplexe taxonomische Struktur der Population hin.

% Konvergenz und Randomisierung


\section{Technische Ausf"uhrung der Distanzverteilungsanalyse}

Um die taxonomische Struktur einer Population durch die Distanzverteilung und die DVK charakterisieren
zu k"onnen, ist, wie in (\ref{ddistr-method}) gefordert, ein Distanzma"s erforderlich, das den
Verwandtschaftsgrad zweier Individuen quantitativ beschreibt. In der Evolution kommt es zur
Entfernung der Verwandtschaft dadurch, da"s die Menge der mutationsbedingten Unterschiede zwischen
zwei Arten gr"o"ser wird. Ein auf der Ebene genomischer Sequenzen operierendes Distanzma"s sollte
also mit der Anzahl der Mutationen, die zum beobachteten Ausma"s der Unterschiedlichkeit zweier
Sequenzen gef"uhrt haben, korreliert sein.

Bei den Mutationsoperatoren, die in dieser Arbeit bei der Entwicklung der Evolutionsmodelle verwendet
wurden, erf"ullt der Editierabstand (Levenshtein-Distanz) dieses Kriterium. Der Editierabstand zwischen
zwei Sequenzen gibt an, wieviele Zeichen ersetzt, eingef"ugt oder gel"oscht werden m"ussen, um die
eine Sequenz in die andere zu transformieren. Austauschmutationen, Insertionen und Deletionen entsprechen
gerade diesen elementaren Editieroperationen. Die Genduplikation kann in diesem Zusammenhang als
spezielle Form der Insertion betrachtet werden.
Aus diesem Grund wurde der Editierabstand in dieser Arbeit f"ur die Erstellung von Distanzverteilungen
evolvierender Populationen verwendet.

Zur Berechnung von DVK-Werten ist die Verteilung der Editierabst"ande jedoch nicht geeignet. Der
Grund hierf"ur ist, da"s der Wert des Editierabstands zweier Sequenzen auch von der L"ange der
Sequenzen abh"angt. Bei konstanter Mutationsrate nimmt somit die absolute Menge der Unterschiede
zwischen zwei Sequenzen umso schneller zu, je l"anger die Sequenzen sind. Zum Ausgleich dieses
Effekts wurde bei der Berechnung von DVK-Werten stets der mit einer Intervallgr"o"se von 0.01 diskretisierte
relative Editierabstand (s.\ \ref{glossary} verwendet.

Die Ausf"uhrung von Distanzverteilungsanalysen ist bei gr"o"seren Populationen sehr rechenzeitaufwendig, weil
sie stets die Erstellung einer Distanzmatrix erfordert. Der Zeitaufwand hierf"ur ist proportional zum Quadrat
der Gr"o"se der untersuchten Population. Weil auch die Berechnung des Editierabstands eine aufwendige Operation
ist, deren Zeitbedarf proportional zum Produkt der L"angen der untersuchten Sequenzen ist, macht sich die
quadratische Abh"angigkeit des Zeitbedarfs von der Populationsgr"o"se bereits bei Populationen von 50 Individuen
sehr deutlich bemerkbar. Bei erheblich gr"o"seren Populationen erwies sich der Zeitbedarf als prohibitiv.

Um Distanzverteilungsanalysen in der Praxis realisieren zu k"onnen, wurden daher bei Populationen mit mehr als 50
Individuen Stichproben gebildet, indem zufallsgesteuert 50 Individuen ausgew"ahlt wurden. Die Distanzverteilungsanalyse
wurde mit diesen Stichproben durchgef"uhrt. Da auch bei Populationsgr"o"sen von 50 der Rechenzeitbedarf f"ur
die Erstellung einer Distanzmatrix erheblich ist, wurden Distanzverteilungsanalysen nicht in jedem Zeitschritt
ausgef"uhrt. Die Anzahl von Zeitschritten zwischen den Analysen wurde so gew"ahlt, da"s w"ahrend eines Laufs
einige hundert Analysen stattfanden. Durch dieses Vorgehen erkl"art sich die in einigen Diagrammen deutlich wahrnehmbare
Rasterung bei den Ergebnissen der Distanzverteilungsanalyse.


\section{Andere Ma"se und Indikatoren der Komplexit"at}

Zur Messung und Charakterisierung der Komplexit"at verschiedener biologischer Systeme existieren in der Literatur
unterschiedliche Ans"atze, von denen sich die meisten mit der Untersuchung der Komplexit"at individueller Lebewesen
besch"aftigen.

Die morphologische Komplexit"at einer Lebensform kann zun"achst durch blo"se Betrachtung 
intuitiv eingesch"atzt werden.
Durch die Klassifizierung und Abz"ahlung der einzelnen morphologischen Strukturen
kann dieser Eindruck "uberpr"uft und quantifiziert werden. Auf zellul"arer Ebene kann die Anzahl unterschiedlicher
Zelltypen als Komplexit"atsma"s dienen \cite{Bonner88}.

Auch die Molekulargenetik bietet eine Vielzahl von M"oglichkeiten zur Quantifizierung der Komplexit"at. Als erster
Anhaltspunkt kann die Gesamtzahl der Basenpaare im Genom eines Organismus dienen. Diese liefert jedoch aufgrund der
bei verschiedenen Lebensformen sehr unterschiedlich gro"sen Anteilen an repetitiver DNA ein stark verzerrtes Bild.
Eine Korrektur kann auf der Grundlage der Reassoziationskinetik der fragmentierten genomischen DNA erfolgen, die eine
Absch"atzung der Menge nicht repetitiver DNA erm"oglicht ($c_0t$-Messung, s.\ \cite{Kornberg80}). Auf der Sequenzebene
kann die strukturelle oder die algorithmsiche Komplexit"at bestimmt werden \cite{Speicher89}. Die Absch"atzung der
Anzahl der Gene eines Organismus ist eine weitere M"oglichkeit zur Bewertung seiner Komplexit"at.

Bei Computersimulationen der Evolution wurden unter anderem die Shannon-Entropie der H"aufigkeitsverteilung der Zeichen
eines Genoms \cite{Thearling94} und Analysen der Komplexit"at genetisch codierter Funktionalit"at zur Komplexit"atsanalyse
\cite{Horn92} eingesetzt.

Die Komplexit"at von "Okosystemen kann mittels der genetischen Diversit"at charakterisiert werden. Diese ist definiert
als die Shannon-Entropie der relativen H"aufigkeiten der im "Okosystem vorkommenden Species. Diese Analyse wurde bereits
bei Computermodellen der Evolution eingesetzt \cite{Ray94}. Die Analyse des Informationsgehalts der Zeichenpositionen
in \textsl{alignments} \cite{Schneider90} kann ebenfalls als Komplexit"atsanalyse auf der Ebene von Kollektionen von Individuen
eingeordnet werden.

Komplexit"at wird oft als ein Ph"anomen eingeordnet, das in einem Bereich zwischen starrer Ordnung und chaotischer Randomisierung
zu beobachten ist. Dieser Bereich wird als \textsl{edge of chaos} bezeichnet. F"ur Zellularautomaten wurde die 
\textsl{edge of chaos} mithilfe des $\lambda$-Parameters charakterisiert \cite{Langton92}.

Die Beziehung zwischen der DVK und den anderen hier genannten Komplexit"atsma"sen und -in\-di\-ka\-to\-ren werden in den folgenden
Kapiteln anhand von mit verschiedenen Modellen der LindEvol-Familie erhaltenen Ergebnisse untersucht. Zu diesem Zweck
werden verschiedene in LindEvol-Modellen beobachtete Pflanzen hinsichtlich ihrer morphologischen Komplexit"at verglichen.
Weiterhin werden die Anzahl aktiver Gene, die bei LindEvol-Modellen der Anzahl unterschiedlicher Zelltypen entspricht,
sowie die genetische Diversit"at w"ahrend der Simulationsl"aufe protokolliert. Bei den einfacheren Modellen wird zus"atzlich
die Existenz einer \textsl{edge of chaos} gezeigt und mit der DVK in Beziehung gesetzt.



\chapter{Beschreibung der Simulationskomponenten}


\section{Ansatz der Modellbildung}
\label{modellansatz}

In einem Computermodell der Evolution k"onnen niemals alle Komponenten und
Aspekte der Evolution realistisch simuliert werden, sowohl der Zeitbedarf
f"ur die Entwicklung eines solchen Modells als auch der Rechenzeitbedarf
f"ur derartige Simulationen verbieten dies. 

Um die evolution"are Entwicklung morphologischer
und taxonomischer Komplexit"at mit Computermodellen zu beschreiben und zu
untersuchen, m"ussen die f"ur diese evolution"aren Prozesse
relevanten Faktoren zun"achst identifiziert und in formal-abstrakter Weise
beschrieben werden. Die Repr"asentation dieser Faktoren in einem Computermodell
wird auf dieser Grundlage so gestaltet, da"s sie auf formal-abstrakter Ebene
dem molekularbiologisch-physikalischen Vorbild entspricht, gleichzeitig aber
im Computerprogramm effizient realisiert werden kann. Auf diese Weise k"onnen
Modelle entwickelt werden, die alle relevanten Faktoren f"ur die zu
untersuchenden Evolutionsprozesse in formaler
Repr"asentation umfassen, und die gleichzeitig die Durchf"uhrung von Experimenten
mit den verf"ugbaren Rechnerkapazit"aten erm"oglichen. In den Unterabschnitten
\ref{genedescr} bis \ref{physicsdescr} werden nun zun"achst die Faktoren, die
bei der Evolution von Morphologie und der dabei entstehenden Komplexit"at
zusammenspielen, aufgef"uhrt und die hier gew"ahlten Ans"atze zu ihrer
Repr"asentation in formalen Modellen dargelegt und diskutiert. In
(\ref{modeldef}) wird anschlie"send die technische Implementation der Modellkomponenten
besprochen.


\subsection{Genetische Komponenten: Gene, Genome, genetische \\ Mechanismen und Mutationen}
\label{genedescr}

Die r"aumliche Struktur einer nat"urlichen Pflanze, ihr Ph"anotyp,
wird durch ihr Entwicklungsprogramm bestimmt, welches in ihrem Genom
in einer komplexen Form codiert ist.
Beim Ablauf eines Entwicklungsprogramms werden verschiedene Gene der Pflanze
in einer r"aumlich und zeitlich exakt definierten Weise aktiviert. Bei diesem
komplexen Proze"s spielen die Bauplangene der Pflanze die zentrale Rolle.
Diese sind oft Transkriptionsfaktoren, welche durch die Bindung an spezifische
DNA-Sequenzen die Expression entsprechender Zielgene aktivieren oder
reprimieren. Beispiele hierf"ur sind die MADS-Box-Proteine \cite{Shore95} und Hom"oobox-Proteine
\cite{Carroll95}. Die Transkription beginnt dabei im Regelfall stromabw"arts von der Bindungsstelle
des Transkriptionsfaktors.

Die MADS-Box-Proteine bilden eine gro"se Familie von Transkriptionsfaktoren.
Die MADS-Dom"ane bildet dabei den Teil des Proteins, der unmittelbar mit der DNA in Kontakt tritt \cite{Pellegrini95}.
In Pflanzen spielen MADS-Box-Proteine eine Schl"usselrolle bei der Morphogenese der Bl"ute.
Die Expression mancher MADS-Box-Proteine wird durch andere Faktoren dieser Familie reguliert, so da"s die MADS-Box-Proteine
von Pflanzen insgesamt ein regulatorisches Netzwerk ausbilden \cite{Theissen95}. Auch autoregulatorische
Prozesse wurden beobachtet \cite{Troebner92}. Das gesamte Netzwerk steht wiederum in Wechselwirkung mit Phytohormonen und externen
Umwelteinfl"ussen.

Mutationen und Neuentwicklungen von Bauplangenen
k"onnen f"ur einschneidende Ver"anderungen der Morphologie einer Pflanze
ausreichend sein. Beispiele hierf"ur sind die hom"ootischen
Mutationen von MADS-Box-Genen, die u.a.\ bei {\slshape Antirrhinum majus},
{\slshape Arabidopsis thaliana}, {\slshape Petunia hybrida} zu massiven
Ver"anderungen der Bl"ute f"uhren.
Derartige Vorg"ange spielen bei der Evolution neuer Formen oft eine entscheidende
Rolle. Es gibt beispielsweise Hinweise darauf, da"s Mutationen an einigen
MADS-Box-Genen die Entwicklung von Mais ({\slshape Zea mays ssp.\ mays}) aus
Teosinte ({\slshape Zea mays ssp.\ parviglumis}) bewirkt haben \cite{Saedler94}. Weiterhin nimmt man an,
da"s die Entstehung eines komplexen regulatorischen Netzwerks von MADS-Box-Genen
eine Schl"usselrolle bei der Evolution der Bl"ute gespielt hat \cite{Theissen95}.

Gerade bei Pflanzen h"angt auch die Evolution der Formenvielfalt und Diversit"at
haupts"achlich von der Neuentwicklung und Modifikation morphologischer Strukturen
ab. Aus diesen Gr"unden beschr"ankt sich die explizite Repr"asentation genetischer
Information in den Modellen auf die Ebene der Bauplangene. Repr"asentationen f"ur die
Grundfunktionen pflanzlicher Zellen wie Energiestoffwechsel, Biosynthese und DNA-Replikation,
wurden dagegen in die Modelle fest eingebaut. Die Evolution dieser Systeme kann
somit durch die hier entwickelten Modelle nicht simuliert werden.

% Nur Modellierung von unmittelbar morphologierelevante Genen, keine
%   housekeeping-Gene etc.

Die Genome werden in den Modellen als Zeichenketten repr"asentiert. Dadurch
wird die biologische Funktion der DNA als Tr"ager genetischer Sequenzinformation
realistisch modelliert. Weitere biochemische und biophysikalische Eigenschaften
der DNA werden dagegen nicht im Modell repr"asentiert. Dies ist nicht m"oglich, weil
zum einen die derzeit ermittelten Informationen "uber die f"ur genetische Prozesse
relevanten molekularen Mechanismen nicht f"ur die Entwicklung realistischer formaler
Modelle ausreichen, und weil zum anderen der Rechenzeitbedarf zur Simulation eines
einzigen molekularen Prozesses so hoch ist, da"s solche Simulationen aus
praktischen Erw"agungen nicht als Grundlage f"ur Evolutionsmodelle in Frage kommen.

Als Ansatzpunkt f"ur die Repr"asentation genetischer Information wurde
daher die abstrakte Beschreibung eines Gens als strukturelle Einheit aus
einem Operatorteil, in dem Proteine, die die Aktivit"at des Gens regulieren,
mit dem Gen interagieren, und einem Strukturteil, in welchem die biologische
Aktivit"at des Gens (bzw.\ seines Genprodukts) codiert ist, gew"ahlt. Dieses
abstrakte Konzept eines Gens findet sich in Ersetzungsregeln, die in formalen
Grammatiken und auch in Lindenmayer-Systemen vorkommen, wieder. Die linke Seite
einer Regel entspricht dabei dem Operatorteil, sie spezifiziert die Konditionen
f"ur die Aktivierung der Regel. Die rechte Seite codiert die Aktion, die bei der
Aktivierung der Regel ausgel"ost wird; sie entspricht somit dem Strukturteil
eines Gens. Die Ausf"uhrung verschiedener Aktionen kann die Aktivierung von Genen
nach sich ziehen, so da"s die Evolution regulatorischer Netzwerke in den Modellen
m"oglich ist.

Da die molekulare Struktur der DNA nicht explizit in den Modellen repr"asentiert
wird, k"onnen auch Mutationen nicht als molekulare Prozesse explizit modelliert
werden. Die molekularen Details von Mutationen sind f"ur die Evolution jedoch
auch nicht entscheidend, die wesentliche Bedeutung von Mutationen f"ur die
Evolution liegt in der Generierung ungerichteter Variationen. In Computermodellen
der Evolution wird dies traditionell durch die stochastisch kontrollierte Anwendung von
Modifikationsoperatoren auf die Genome realisiert. In den hier entwickelten
Modellen werden Operatoren verwendet, die sich informationstechnisch an
Basenaustauschmutationen, Insertionen, Deletionen und Genduplikationen
anlehnen.

Mutationen an Genen, die
w"ahrend des gesamten Wachstums einer Pflanze nicht aktiviert werden, sind neutral.
Wird dagegen ein Gen betroffen oder erzeugt, welches beim Wachstum der Pflanze
aktiv wird, kann dies zur Modifikation kleiner Details des Ph"anotyps f"uhren, aber
auch die Gesamtstruktur der Pflanze "andern, also hom"ootische und pleiotrope
Effekte haben. Auch letale Mutationen sind m"oglich. Somit ergibt sich aus der
Modellgestaltung das gesamte Spektrum m"oglicher Mutationsfolgen emergent aus der
Interpretation der genetischen Information als Regelwerk zur Steuerung der
Morphogenese.


\subsection{Biologische Komponenten: Umwelteinfl"usse, Interaktion zwischen Individuen}
\label{biodescr}

In der Natur hat die Umwelt eines Individuums vielf"altige
Einfl"usse auf seine Morphogenese und auf den Verlauf seines gesamten Lebens.
Diese Umwelteinfl"usse wirken sich auf die Reproduktionschancen des
Individuums aus und sind somit auch f"ur die Evolution von Bedeutung.
Ein signifikanter Teil der Umwelteinfl"usse geht von anderen Lebewesen
aus. Die Umwelteinfl"usse nebst ihren Auswirkungen auf die Fortpflanzungswahrscheinlichkeiten
von Individuen sind somit nicht von der Evolution unabh"angig; vielmehr werden sie 
durch evolution"are Adaptationsprozesse ihrerseits beeinflu"st.
Dieser Aspekt evolution"arer Dynamik, der von Packard {\slshape intrinsic adaptation}
genannt wurde \cite{Packard89}, wurde bisher nur in wenigen Evolutionsmodellen repr"asentiert.
Gerade f"ur die Evolution von Diversit"at und Komplexit"at haben biogene Ver"anderungen
der Umwelt jedoch eine entscheidende Bedeutung. Sie erzeugen und vernichten "okologische
Nischen und bestimmen dadurch "uber die M"oglichkeiten zur Evolution neuer Arten mit.
In \cite{Stanley73} wird aufgezeigt, da"s die Wechselwirkungen zwischen R"auber und
Beute m"oglicherweise bei der evolution"aren Entstehung der Vielfalt multizellul"arer
Lebensformen eine entscheidende Rolle gespielt hat.

Aus diesem Grund wurde bei der Modellentwicklung im Rahmen dieser Arbeit besonderer
Wert auf mannigfaltige und komplexe M"oglichkeiten zur Interaktion zwischen Pflanzen
gelegt. Die Auswirkungen derartiger Interaktionen wurden jedoch, anders als etwa im
NKC-Modell von Kauffman \cite{Kauffman92}, nicht explizit in den Modellen festgelegt,
da bei einem solchen Ansatz stets nur eine kleine Menge willk"urlich herausgegriffener
Interaktionsm"oglichkeiten repr"asentiert werden kann. Stattdessen wurde die Repr"asentation
der physikalischen Faktoren der Evolution so gestaltet, da"s sich vielf"altige Interaktionsformen
aus der Modellphysik als emergente Eigenschaften ergeben.


\subsection{Physikalische Komponenten: Raum, Zeit und Energie}
\label{physicsdescr}

Die Morphogenese eines Organismus ist die durch sein genetisch codiertes
Entwicklungsprogramm gesteuerte Ausbildung seiner r"aumlichen Struktur.
Zur Modellierung morphogenetischer Prozesse ist daher die Repr"asentation
morphologischer Ph"anotypen variabler r"aumlicher Gestaltung erforderlich.
Als Grundlage f"ur die Repr"asentation der r"aumlichen Struktur von Pflanzen
wurden in den hier entwickelten Modellen diskrete Gitter verwendet.
Pflanzenzellen werden durch elementare Objekte, die genau eine Gitterposition
einnehmen, repr"asentiert, eine Pflanze besteht im Modell aus mehreren, mindestens
jedoch einer Zelle.

Modelle mit diskreten Gittern k"onnen im Vergleich zu Modellen mit kontinuierlichen
R"aumen erheblich effizienter auf Computern implementiert werden.
Evolutionsmodelle mit kontinuierlichem Raum erfordern f"ur ihre technische Umsetzung
Rechner von z.T.\ extremer Leistungsf"ahigkeit (z.B.\ \cite{Sims95}); und in
keinem derartigen Modell wurden bisher alle Faktoren der Evolution ber"ucksichtigt,
deren Repr"asentation Zielsetzung bei der Modellentwicklung in dieser
Arbeit war (vgl.\ auch \cite{Terzopoulos95}).
Die Realisierung eines Modells mit kontinuierlichem Raum kam somit nicht in Betracht.

Weiterhin wurden diskrete Gitter, trotz ihrer Unterschiede zur kontinuerlichen
physikalischen Realit"at, in
verschiedensten Modellen, u.a.\ in Ising-Modellen,
Verkehrssimulationen und in Modellen katalytischer
Prozesse \cite{Gerhardt89} eingesetzt. Durch diese Modelle ist belegt, da"s
es bei allen Unterschieden zwischen diskreten und kontinuierlichen R"aumen
m"oglich ist, Ph"anomene im kontinuierlichen Raum auf der Basis diskreter
Gitter zu modellieren. Dies erstreckt sich auch auf den Bereich emergenter
Ph"anomene, wie die Hodge-Podge-Maschine \cite{Gerhardt89} belegt. Diskrete Gitter
sind somit als Repr"asentation des kontinuierlichen Raums in den hier entwickelten
Modellen gut geeignet.

F"ur die ad"aquate Modellierung der Morphogenese ist die Repr"asentation der
zeitlichen Gestaltung des morphogenetischen Prozesses, seiner Dynamik, ebenso wichtig
wie die Modellierung der R"aumlichkeit. Weiterhin ist die Evolution, auch unabh"angig
von der Ontogenese der evolvierenden Individuen, ein dynamischer Proze"s, dessen
Modellierung ohne die Repr"asentation der Zeitkomponente nicht m"oglich ist.

In den hier entwickelten Modellen wird
der Zeitverlauf durch aufeinanderfolgende, diskrete Zeitschritte repr"asentiert.
In einem Zeitschritt wird jede Zelle einmal prozessiert. Dazu wird ermittelt,
welche Gene in der Zelle aktiviert sind, und die entsprechenden Aktionen werden
ausgef"uhrt. Wie auch bei der Repr"asentation des Raums erm"oglicht auch bei der
Zeit eine diskrete Modellierung eine sehr viel effektivere Implementation des Modells
als eine kontinuierliche Modellierung. Entwicklungsmodelle mit kontinuierlicher
Zeit (z.B.\ \cite{Prusinkiewicz94}) sind bereits f"ur ein einzelnes Individuum
ausgesprochen aufwendig. Die Modellierung von Interaktionen zwischen Pflanzen,
auf die bei der Modellentwicklung im Rahmen dieser Arbeit besonderer Wert gelegt
wurde, f"uhrt zu einer zus"atzlichen Explosion des Rechenzeitbedarfs bei
kontinuierlicher Repr"asentation der Zeit. Diese "Uberlegungen f"uhrten zur
Entscheidung f"ur ein diskretes Zeitmodell. Hinzu kommt die Tatsache, da"s
auf Grundlage dieses Ansatzes unterschiedlichste dynamische Prozesse erfolgreich
simuliert werden konnten. Ein Beispiel hierf"ur sind {\slshape molecular dynamics}
Simulationen.

F"ur den Ablauf der Entwicklung einer Pflanze ist ihre Versorgung mit
Energie notwendig. Die prim"are Energiequelle einer Pflanze ist Licht,
welches von der Pflanze zur Photosynthese organischer Verbindungen
genutzt wird. Die so gewonnen organischen Verbindungen
liefern sowohl den gr"o"sten Teil des Materials als auch die Energie zum Aufbau
der im Laufe der Entwicklung der Pflanze ausgebildeten Morphologie. Die Gestalt
der Pflanze, sowie auch die Formen der Pflanzen in ihrer Umwelt, bestimmen wiederum
die Effektivit"at, mit der eine Pflanze Photosynthese betreiben kann.

In den hier entwickelten Modellen wird Energie durch grob diskretisierte
Quanten repr"asentiert. Energiequanten werden als Photonen in das
System eingebracht. Diese k"onnen von Pflanzenzellen absorbiert werden.
Die Absorption eines Photons bewirkt den "Ubergang der betreffenden
Pflanzenzelle in einen energiereichen Zustand. In diesem Zustand kann die
Zelle eine Zellteilung ausf"uhren und damit zur Morphogenese der Pflanze
beitragen. Nach dieser Aktion ist die Zelle bis zur erneuten Absorption
eines Photons energielos.

Ein Photon durchl"auft das die Welt repr"asentierende Gitter von oben nach unten.
Daher beeinflussen pflanzliche Strukturen oberhalb einer Zelle ihre
Chancen zur Absorption eines Photons. Auf diese Weise werden die komplexen
Beziehungen zwischen der Morphologie einer Pflanze und den Pflanzen in
ihrer Umwelt und der Energiezufuhr einer Pflanze im Modell repr"asentiert.
Auf die Repr"asentation der zum Wachstum einer Pflanze ebenfalls erforderlichen
Versorgung mit "uber das Wurzelsystem aufgenommenen N"ahrsalzen wird dagegen
in den hier vorgestellten Modellen verzichtet.





% Modellierung nicht auf explizit molekularer Ebene, DNA / Genome in
%   Funktion von Sequenzen, nicht raeumliche Molekularstruktur.
% Morphologie erfordert raeumliche Struktur, diskret. Ansatz bewaehrt
%   in Physik (lattice gases, CA, Ising...)
% Zeit in diskreten Schritten, Alternativen dazu sind nicht bekannt.
% Mutationen als ungerichtete Variationen mit Motiven aus Molekulargenetik
%   Beeinflussbarkeit der Mutationsraten
% Funktionale Struktur von Genen: Konditionalteil und Aktions / Anweisunsteil.


\section{Umsetzung in Simulationsmodelle}
\label{modeldef}

Die im vorangehenden Abschnitt aufgef"uhrten biologischen Mechanismen wurden
in eine Reihe von Computermodellen umgesetzt, die insgesamt die LindEvol-Modellfamilie
bilden. In diesem Abschnitt wird nun die technische Realisierung der
in (\ref{modellansatz}) angesprochenen Modellkomponenten beschrieben.
Zu den einzelnen Komponenten geh"oren in der Regel Kontrollparameter, also
Gr"o"sen, die die Arbeitsweise der Komponenten steuern. Neben ihrer nat"urlichsprachlichen
Bezeichnung haben alle Kontrollparameter einen Variablennamen, der z.B.\ in
Parameterdateien (s.\ Anhang \ref{technicalstuff} und auch im Quelltext verwendet wird. Weiterhin werden
Kontrollparameter in mathematischen Ausdr"ucken durch Formelbuchstaben
symbolisiert. Sowohl Variablennamen als auch Formelbuchstaben werden in diesem
Abschnitt f"ur alle Kontrollparameter angegeben und am Ende der Beschreibungen
der einzelnen Komponenten, gemeinsam mit den Bezeichnungen weiterer charakteristischer
Gr"o"sen, tabellarisch zusammengefa"st.


\subsection{Zufallsgenerator}
\label{rndgeneratordef}

Zur Realisierung verschiedener Elemente der Modelle, z.B.\ der
Mutation (\ref{mutationdef}) und der Lichtabsorption
(\ref{energydef}), sind Zufallszahlen erforderlich. 
Alle in einem Simulationsverlauf ben"otigten Zufallszahlen werden mithilfe der
Funktion \verb|random()| von Earl T.\ Cohen erzeugt. Diese Funktion ist in den
C-Funktionsbibliotheken enthalten, die mit den Systemen IRIX 4.0.5f und DEC OSF1 3.0
ausgeliefert werden. Mit der Funktion \verb|initstate()| wird f"ur die Funktion
\verb|random()| ein 256 Bytes langes internes Zustandsfeld ({\slshape state array}\/)
installiert, was laut Dokumentation h"ochstwertige Zufallszahlen gew"ahrleistet.

Zu Beginn einer Simulation wird der Zufallsgenerator mittels der Funktion
\verb|srandom()| initialisiert. Dabei wird der Kontrollparameter \verb|random_seed|
als Argument verwendet. Auf diese Weise ist es m"oglich, jeden Simulationslauf
exakt zu wiederholen.

\medskip
\noindent\begin{tabularx}{\linewidth}{|c|l|X|} \hline
\multicolumn{3}{|c|}{\bfseries Kontrollparameter des Zufallsgenerators} \\ \hline
Formel- & Variablenname & Beschreibung \\
buchstabe &              & \\ \hline
- & \verb|random_seed| & ganzzahliger Wert zur Initialisierung des Zu\-falls\-ge\-ne\-ra\-tors \\ \hline
\end{tabularx}
\medskip


\subsection{R"aumliche Struktur}
\label{topodef}

\subsubsection{Zweidimensionale Struktur}
\label{topo2d}

Eine zweidimensionale \introdef{Welt} ist ein zweidimensionales, orthogonales Gitter. Die Breite und
H"ohe des Gitters sind Kontrollparameter, ihre Formelbuchstaben sind
$w$ und $h$.

Um lokale Interaktionen zu modellieren, mu"s eine \introdef{Nachbarschaft} definiert
werden. Bei den zweidimensionalen LindEvol Modellen wird die Moore-Nachbarschaft
(engl.:\ {\slshape Moore neighborhood}) verwendet. Wie in Abb.\ \ref{neighbor2d}
dargestellt, werden den Moore-Nachbarn eines
Gitterelements Indizes von 0 bis 7 zugerodnet, anhand derer sie eindeutig
bezeichnet werden k"onnen.

Absolute Gitterpositionen werden durch Koordinatenpaare angegeben, die
X-Ko\-or\-di\-na\-te l"auft in Richtung der Breite von $0$ bis $w-1$, die
Y-Ko\-or\-di\-na\-te l"auft, beginnnend am Boden, von $0$ bis $h-1$.

Die Topologie der Welt entspricht derjenigen
einer Zylinderoberfl"ache: Der rechte Nachbar der Position $(w-1, y)$ ist
$(0, y)$, hingegen haben die Gitterpositionen in der untersten Zeile,
dem \introdef{Boden} der Welt,
keine unteren Nachbarn und diejeingen in der obersten Zeile, der \introdef{Decke},
keine oberen Nachbarn.

\begin{figure}[tb]

\unitlength1cm
\begin{picture}(15,5)
\put(1,0){\makebox(15,5)[b]{\epsfxsize=5cm \epsffile{neigh2d.eps}}}
\end{picture}

\caption[Struktur der zweidimensionalen Nachbarschaft]
{\label{neighbor2d}
Die Moore-Nachbarschaft eines Gitterelements X in einem zweidimensionalen Gitter. Die
Nachbarpositionen eines Gitterelements werden entsprechend dem hier dargestellten
Schema indiziert. Position 2 bez"uglich des Elements X bezeichnet also beispielsweise
das Gitterelement rechts unterhalb des Elements X.}
\end{figure}


\subsubsection{Dreidimensionale Struktur}
\label{topo3d}

Eine dreidimensionale Welt ist ein dreidimensionales, orthogonales Gitter
mit quadratischer Grundfl"ache. Kontrollparameter sind die Breite der Grundfl"ache
und die H"ohe des Gitters mit den Formelbuchstaben $w$ und $h$.
Die zweidimensionale Moore-Nachbarschaft
wurde zu einer achtzehnzelligen, dreidimensionalen Nachbarschaft erweitert,
deren Struktur und Indizierung in Abb.\ \ref{neighbor3d} dargestellt ist.

Die gesamte Welt hat die Abmessungen $w * w * h$. Gitterpositionen werden durch
Koordinatentripel bezeichnet, wie bei der zweidimensionalen
Struktur beginnen die Koordinaten mit 0 und am Boden. Der Boden und alle zu ihm
parallelen Ebenen haben Torustopologie, also ist $(0, y, z)$ der rechte Nachbar von
$(w-1, y, z)$, und $(x, w-1, z)$ ist der vordere Nachbar von $(x, 0, z)$.

\begin{figure}[tb]

\unitlength1cm
\begin{picture}(15,10)
\put(1,0){\makebox(5,5)[bl]{\epsfxsize=5cm \epsffile{neigh3dc.eps}}}
\put(9,0){\makebox(5,10)[bl]{\epsfxsize=5cm \epsffile{neigh3ds.eps}}}
\end{picture}

\caption[Struktur der dreidimensionalen Nachbarschaft]
{\label{neighbor3d}
Die achtzehnzellige Nachbarschaft eines Gitterelements X im dreidimensionalen Gitter und das Indizierungsschema
f"ur die Nachbarpositionen. Links ist
die Struktur der Nachbarschaft dargestellt, auf der rechten Seite wurden die Ebenen der
Nachbarschaft separiert, damit alle Gitterpositionen mit ihren Indices gezeigt werden
k"onnen.}
\end{figure}

\medskip
\medskip
\noindent\begin{tabularx}{\linewidth}{|c|l|X|} \hline
\multicolumn{3}{|c|}{\bfseries Kontrollparameter der Welt} \\ \hline
Formel-   & Variablenname & Beschreibung \\ 
buchstabe &               & \\ \hline
$w$ & \verb|world_width| & Breite der Welt bzw.\ Kantenl"ange ihrer Grundfl"ache \\
$h$ & \verb|world_height| & H"ohe der Welt \\ \hline
\end{tabularx}
\medskip
\medskip


\subsection{Pflanzenzellen}
\label{celldef}

Eine \introdef{Pflanzenzelle} besetzt in der Welt genau eine Gitterposition. An einer
Gitterposition kann sich maximal eine Zelle befinden. Innerhalb einer Pflanze werden
die Zellen der Reihenfolge ihrer Entstehung entsprechend indiziert, $C_i$ ist also
die $i$-te Zelle einer Pflanze. Die \introdef{Keimzelle} einer Pflanze tr"agt den Index 0.
Die Koordinaten einer Zelle $C$ werden mit $x(C)$, $y(C)$
und $z(C)$ bezeichnet.

Weiterhin ist eine
Zelle durch ihre Parameter und durch die lokale Struktur der Pflanze in
ihrer Umgebung charakterisiert, die gemeinsam den Zustand der Zelle bestimmen.
Ferner k"onnen Pflanzenzellen verschiedene Aktionen ausf"uhren. Welche Aktion eine
Pflanzenzelle ausf"uhrt, wird durch ein im genom der Pflanze codiertes Regelwerk in
Abh"angigkeit vom Zustand der Zelle gesteuert. In diesem Abschnitt werden nun zun"achst
der Zellzustand und die Aktionen eingef"uhrt. Die Definition des Genoms einer Pflanze
und seine Interpretation als Regelwerk folgen in den Abschnitten (\ref{genomedef})
und (\ref{interpreterdef}).


\subsubsection{Zellparameter}

Jede Pflanzenzelle $C$ hat einen bin"aren Energieparameter $E(C)$, sie ist entweder
energiereich oder energielos. Bei einigen Modellen haben die 
Zellen als weiteren Parameter einen internen Zustand, der je nach Modell 16 oder
32 Bits umfa"st. Der interne Zustand einer Zelle $C$ wird mit $Z_i(C)$ bezeichnet.
Der interne Zustand einer Zelle wird stets mit 0 initialisiert. Er kann durch Aktionen
der Zelle (s.\ \ref{cellactiondef}) ver"andert werden.

\subsubsection{Lokale Struktur}

In der Nachbarschaft (s.\ \ref{topodef}) einer Zelle k"onnen sich weitere
Zellen der Pflanze, zu der sie geh"ort, befinden. Die Anordnung dieser
Nachbarzellen wird als die \introdef{lokale Struktur} einer Zelle bezeichnet. Sie wird
mittels des Indizierungsschemas f"ur die Nachbarschaft auf
nat"urliche Zahlen abgebildet. Dazu wird zun"achst die Nachbarzellfunktion
definiert:

Bezeichne $C$ eine Zelle und $I$ die Menge der Indizes ihrer Nachbarpositionen
(s.\ \ref{topodef}). Dann ist die Nachbarzellfunktion $N_C: I \mapsto \{0, 1\}$
definiert durch:

\begin{equation}
N_C(i) = \left\{
    \begin{array}{ll}
        1, & \mbox{falls an Position } i \mbox{ eine Zelle derselben Pflanze ist} \\
        0, & \mbox{anderenfalls}
    \end{array}
    \right.
\end{equation}

Die Abbildung der lokalen Struktur der Zelle $C$ auf eine ganze Zahl erfolgt
nun durch die Funktion

\begin{equation}
S(C) = \sum_{i=0}^{|I|-1} 2^i N_C(i)
\end{equation}


Zweidimensionale lokale Strukturen werden so auf Achtbit-Zahlen abgebildet, 
dreidimensionale lokale Strukturen werden auf Achtzehnbit-Zahlen abgebildet.
Abb.\ \ref{statespecfig} veranschaulicht die Abbildung im zweidimensionalen
Gitter.


\begin{figure}
\unitlength1.2cm
\begin{picture}(14,3)
\put(0,2){\makebox(3,0.5)[l]{lokale Struktur:}}
\put(4,2){\makebox(3,0.5)[l]{Achtbitzahl:}}
{
\ttfamily
\put(0,1){\framebox(0.5,0.5){5}}
\put(0.5,1){\makebox(0.5,0.5){6}}
\put(1,1){\framebox(0.5,0.5){7}}
\put(0,0.5){\makebox(0.5,0.5){3}}
\put(0.5,0.5){\framebox(0.5,0.5){X}}
\put(1,0.5){\makebox(0.5,0.5){4}}
\put(0,0){\framebox(0.5,0.5){0}}
\put(0.5,0){\makebox(0.5,0.5){1}}
\put(1,0){\makebox(0.5,0.5){2}}
\put(4,0.5){\framebox(0.25,0.5){1}}
\put(4.25,0.5){\framebox(0.25,0.5){0}}
\put(4.5,0.5){\framebox(0.25,0.5){1}}
\put(4.75,0.5){\framebox(0.25,0.5){0}}
\put(5,0.5){\framebox(0.25,0.5){0}}
\put(5.25,0.5){\framebox(0.25,0.5){0}}
\put(5.5,0.5){\framebox(0.25,0.5){0}}
\put(5.75,0.5){\framebox(0.25,0.5){1}}

{
\footnotesize
\put(3,0.2){\makebox(1,0.3)[l]{Bit Nr.}}
\put(4,0.2){\makebox(0.25,0.3){7}}
\put(4.25,0.2){\makebox(0.25,0.3){6}}
\put(4.5,0.2){\makebox(0.25,0.3){5}}
\put(4.75,0.2){\makebox(0.25,0.3){4}}
\put(5,0.2){\makebox(0.25,0.3){3}}
\put(5.25,0.2){\makebox(0.25,0.3){2}}
\put(5.5,0.2){\makebox(0.25,0.3){1}}
\put(5.75,0.2){\makebox(0.25,0.3){0}}
}
}
\put(2,0.5){\makebox(1,0.5)[l]{$\rightarrow$}}
\end{picture}
\caption[Abbildung von zweidimensionalen lokalen Strukturen auf Achtbit-Zahlen]
{\label{statespecfig}
Abbildung von zweidimensionalen lokalen Strukturen auf Achtbit-Zahlen.
Die Numerierung der Bits beginnt bei 0 mit dem niedrigstwertigen Bit.
}
\end{figure}


\subsubsection{Zellzustand}

In Modellen, bei denen die Pflanzenzellen keinen internen Zustand besitzen,
definiert die lokale Struktur einer Zelle $C$ allein ihren Zustand $Z(C)$:

\begin{equation}
\label{stateq-nointernal}
Z(C) = S(C)
\end{equation}

Bei Modellen mit internem Zellzustand sollen sowohl der interne Zustand einer Zelle als auch 
ihre lokale Struktur bei der Bestimmung des Verhaltens der Zelle ber"ucksichtigt werden.
Daher ist der Zustand einer Zelle in diesen Modellen definiert als $S(C)$ bitweise oderiert mit ihrem
internen Zustand. Au"serdem ist das h"ochstwertige Bit des Zellzustands
gesetzt, wenn die Zelle energiereich ist:

\begin{equation}
\label{stateq-internal}
Z(C) = S(C) \vee Z_i(C) \vee E(C) * 2^{\mathit{num\_statebits} - 1}
\end{equation}

{\slshape num\_statebits} bezeichnet hier die Anzahl Bits des internen
Zellzustands. Das h"ochstwertige Bit des Zellzustands wird wegen seiner
speziellen Funktion auch als \introdef{Energiebit} bezeichnet.


\subsubsection{Zellaktionen}
\label{cellactiondef}

Die Pflanzenzellen k"onnen verschiedene \introdef{Aktionen} ausf"uhren. Die meisten
Aktionen verbrauchen Energie. Dies bedeutet, da"s sie nur von einer
energiereichen Zelle ausgef"uhrt werden k"onnen, und da"s die Zelle
nach der Ausf"uhrung der Aktion energielos ist. Die Energie wird
unabh"angig vom Erfolg der Aktion verbraucht.

Die einzelnen Aktionen werden mit englischen Codeworten bezeichnet.
Zu einigen Aktionen geh"ort ein Parameter. Die folgende Tabelle gibt
die Spezifikationen der Aktionen im Einzelnen an.

\label{cellactionlist}
\medskip
\noindent\begin{tabularx}{\linewidth}{l|c|X}
Aktion & Energie- & Beschreibung \\
       & verbrauch & \\ \hline \hline
\verb|divide| {\slshape pos} & ja &  Die Zelle produziert eine energielose
    Tochterzelle. {\slshape pos} gibt dabei den Index der Nachbarposition an,
    an der die Tochterzelle plaziert werden soll. Eine Teilung kann scheitern,
    wenn die spezifizierte Nachbarposition bereits besetzt ist; die exakte Behandlung
    von Konflikten ist bei den verschiedenen Modellen unterschiedlich. Wenn
    an der spezifizierten Position kein Gitterelement existiert (also wenn
    die Position unterhalb des Bodens oder oberhalb der Decke liegt), scheitert
    die Teilung unweigerlich. \\ \hline
\verb|flyingseed| & ja & Eine neue, aus einer energielosen Zelle bestehende
    Pflanze wird an einer zuf"allig ausgew"ahlten freien Gitterposition am
    Boden der Welt generiert. Diese Aktion scheitert,
    wenn alle Gitterelemente am Boden der Welt durch Zellen besetzt sind. \\ \hline
\verb|localseed| & ja & Eine neue einzellige Pflanze wird am Boden der
    Welt senkrecht unterhalb der samenproduzierenden Zelle generiert. Falls
    dieses Gitterelement bereits besetzt ist, wird bei Modellen mit zweidimensionaler
    Topologie das n"achste freie Gitterelement neben dieser Zielposition gesucht
    und die Pflanze dort generiert. Die Aktion scheitert, wenn alle Gitterelemente
    am Boden der Welt besetzt sind. Bei dreidimensionaler Topologie wird nicht
    nach benachbarten, freien Elementen gesucht, wenn die Position senkrecht unterhalb
    der samenproduzierenden Zelle besetzt ist, scheitert die Samenerzeugung. \\ \hline
\verb|mut-| & ja &  Der Mutationsexponent (s.\ \ref{mutationdef}) wird
    um 1 erniedrigt. \\ \hline
\verb|mut+| & ja & Der Mutationsexponent (s.\ \ref{mutationdef}) wird
    um 1 erh"oht. \\ \hline
\verb|statebit| {\slshape num} & nein & Das Bit {\slshape num} im internen
    Zustand der Zelle wird gesetzt, d.h.\ der interne Zustand der Zelle wird
    mit $2^{\mathit{num}}$ bitweise oderiert. Durch \verb|statebit|-Aktionen
    in einem Zeitschritt wird jeweils der interne Zellzustand f"ur den n"achsten
    Zeitschritt festgelegt. Der interne Zustand der Zelle im laufenden
    Zeitschritt bleibt unbeeinflu"st. Eine Zelle, die in einem Zeitschritt keine
    \verb|statebit|-Aktionen ausf"uhrt, hat im n"achsten Zeitschritt den internen
    Zustand $Z_i=0$ \\
\end{tabularx}
\medskip

\medskip
\noindent\begin{tabularx}{\linewidth}{|c|l|X|} \hline
\multicolumn{3}{|c|}{\bfseries Parameter einer Zelle} \\ \hline
Formel- & Variablenname   & Beschreibung \\
buchstabe &               & \\ \hline
$x(C)$    & & X-Koordinate der Zelle $C$ \\
$y(C)$    & & Y-Koordinate der Zelle $C$ \\
$z(C)$    & & Z-Koordinate der Zelle $C$ \\
$N_C$     & & Nachbarzellfunktion der Zelle $C$ \\
$E(C)$    & & Energiezustand der Zelle $C$ \\
$S(C)$    & & lokale Struktur um die Zelle $C$ \\
$Z_i(C)$  & & interner Zustand der Zelle $C$ \\
$Z(C)$    & & Zustand einer Zelle $C$ \\ \hline
\end{tabularx}
\medskip


\subsection{Energie}
\label{energydef}

Pflanzenzellen werden durch die Absorption eines \introdef{Photons} energiereich.
Die Lichtbestrahlung der Welt wird simuliert, indem
in jeder Gitterpositionen der obersten Ebene ein Photon eingesetzt wird.
Von dort wandern die Photonen senkrecht abw"arts. Trifft ein Photon auf
eine Pflanzenzelle, so wird es mit einer Wahrscheinlichkeit von 50\%
absorbiert. Eine energielose Zelle wird dadurch energiereich, der Zustand
einer energiereichen Zelle wird durch die Absorption eines Photons nicht
ver"andert. Wird ein Photon von einer Zelle nicht absorbiert, wandert
es weiter abw"arts. Passiert es den Boden der Welt, ohne absorbiert zu
werden, verschwindet es. Der Algorithmus der Simulation des Lichts
in einer zweidimensionalen Welt in Pseudocode lautet:

\begin{verbatimcmd}
FOR x = 0 TO world_width - 1
  photon_exists = TRUE
  y = world_height - 1
  WHILE photon_exists AND (y >= 0)
    IF \pcodemeta{eine Zelle bei} (x, y) \pcodemeta{existiert}
      IF (random() MOD 2) == 1
        photon_exists = FALSE
        \pcodemeta{Setze Energieparameter der getroffenen Zelle auf energiereich}
      ENDIF
    ENDIF
    y = y - 1
  WEND
NEXT x
\end{verbatimcmd}

Durch die Ausf"uhrung einer der auf Seite \pageref{cellactionlist} genannten
energieverbrauchenden Aktionen gehen Pflanzenzellen vom energiereichen
in den energielosen Zustand "uber. Die Lichteinstrahlung ist somit die
Energiequelle, die alle Lebensvorg"ange in den Simulationen antreibt.


\subsection{Pflanzen}
\label{plantdef}

Eine Pflanze $P$ ist ein Objekt, das aus einer oder mehreren Pflanzenzellen besteht.
Neben ihren Zellen besitzt eine Pflanze als weitere charakteristische Parameter:

\begin{itemize}
\item Ihr Genom (s.\ \ref{genomedef}). Die Aktivit"aten aller Zellen einer Pflanzen
    werden durch ein Genom gesteuert (s.\ \ref{interpreterdef}), damit k"onnen
    somatische Mutationen, Mosaikbildung usw.\ nicht simuliert werden.
\item Einen Mutationsexponenten $\mu$. Der Mutationsexponent einer Pflanze wird durch
    die Ausf"uhrung der Aktionen \verb|mut+| und \verb|mut-| inkrementiert
    bzw.\ dekrementiert. Bei der Mutation des Genoms beeinflu"st er
    die Mutationsraten, dies wird in (\ref{mutationdef}) n"aher erl"autert.
\item Die Gesamtzahl ihrer Zellen, sie wird mit dem Formelbuchstaben $n_c$
    bezeichnet.
\item Die Gesamtzahl ihrer energiereichen Zellen. Sie wird auch ihre (Gesamt-) Energie
    genannt und durch den Formelbuchstaben $n_e$ bezeichnet.
\end{itemize}

\medskip
\noindent\begin{tabularx}{\linewidth}{|c|l|X|} \hline
\multicolumn{3}{|c|}{\bfseries Parameter einer Pflanze} \\ \hline
Formel- & Variablenname & Beschreibung \\
buchstabe &               & \\ \hline
$\mu$ & \verb|mut_counter| & Mutationsexponent \\
$n_c$ & \verb|num_cells| & Anzahl Zellen \\
$n_e$ & \verb|total_energy| & Anzahl Energiereicher Zellen \\ \hline
\end{tabularx}
\medskip


\subsection{Zeit}
\label{timedef}

Alle Simulationen verlaufen in diskreten Zeitschritten. Ein solcher Zeitschritt,
auch als \introdef{Tag} bezeichnet, beginnt stets mit der Simulation von Licht. Dazu wird,
ausgehend von jeder Gitterposition in der obersten Ebene, ein Photonenlauf
wie oben beschrieben durchgef"uhrt. Danach werden die Pflanzen abgearbeitet,
wobei die Reihenfolge in jedem Zeitschritt erneut zufallsgesteuert bestimmt
wird. Die Abarbeitung der Zellen einer Pflanze erfolgt in der Reihenfolge
ihrer Entstehung. In Pseudocode ausgedr"uckt lautet der Algorithmus zur
Simulation eines Zeitschritts:

\begin{verbatimcmd}
\pcodemeta{Simuliere Lichteinstrahlung}
\pcodemeta{Randomisiere die Reihenfolge der Pflanzen}
FOR i = 0 TO psize - 1
  \pcodemeta{Simuliere Wachstum der i-ten Pflanze}
NEXT i
\end{verbatimcmd}


\subsection{Genom}
\label{genomedef}

Genome sind als Zeichenketten von Bytes
modelliert. Die L"ange eines Genoms ist variabel und wird mit
dem Formelbuchstaben $l$ und durch den Variablennamen \verb|genome_length|
bezeichnet. Sie kann durch Mutationen
(s.\ \ref{mutationdef}) ver"andert werden. Die einzelnen
Zeichenpositionen eines Genoms werden mit 0 beginnend indiziert,
das letzte Zeichen hat damit den Index $l - 1$.

Bei allen Modellen wird am Anfang eines Simulationslaufs eine Startpopulation
von Genomen zufallsgesteuert erzeugt. Die L"ange dieser Genome wird durch den
Kontrollparameter $l_s$ (\verb|glen_init|) bestimmt.

\medskip
\noindent\begin{tabularx}{\linewidth}{|c|l|X|} \hline
\multicolumn{3}{|c|}{\bfseries Parameter einer Zelle} \\ \hline
Formel- & Variablenname   & Beschreibung \\
buchstabe &               & \\ \hline
$l$       & \verb|genome_length| & L"ange des Genoms \\
$l_s$     & \verb|glen_init| & L"ange der zufallsgesteuert generierten Genome der Startpopulation \\ \hline
\end{tabularx}
\medskip


\subsection{Interpretation der genetischen Information}
\label{interpreterdef}

F"ur jede Zelle einer Pflanze wird durch die \introdef{Interpretation} des
Genoms der Pflanze bestimmt, welche Aktionen die Zelle ausf"uhren
soll. Der Vorgang der Interpretation eines Genoms erfolgt durch den
Genominterpreter. Im Rahmen der vorliegenden Arbeit wurden zwei
Interpretersysteme entwickelt, die im Folgenden beschrieben werden.

Zur Abarbeitung einer Pflanzenzelle wird das Genom der Pflanze in ein \introdef{Regelwerk}
"ubersetzt. Durch dieses Regelwerk wird f"ur jede Zelle in Abh"angigkeit
von ihrem Zellzustand festgelegt, welche Aktionen die Zelle ausf"uhren
soll. Eine Regel bestimmt eine Aktion f"ur einen Zellzustand oder f"ur eine Klasse
von Zellzust"anden. Der Zustandsteil in einer Regel wird als ihre linke Seite, der
Aktionsteil als ihre rechte Seite bezeichnet.
Wenn der Zustand einer Zelle in die durch die linke Seite einer Regel
spezifizierte Zustandsklasse f"allt, wird diese Regel \introdef{aktiviert}.

Klassen von Zellzust"anden werden durch Bitmasken definiert. In einer
solchen Bitmaske kann der Zustand eines Bits entweder spezifiziert oder
unspezifiziert sein. Die Anzahl der in der linken Seite einer Regel
spezifizierten Bits wird als die Spezifit"at dieser Regel bezeichnet.

Eine Regel wird durch einen Abschnitt von Bytes auf dem Genom codiert.
Ein solcher Abschnitt repr"asentiert ein \introdef{Gen}. Bei einem Byte kann maximal
ein Gen anfangen. Damit ist durch das Genom eine Reihenfolge der Gene
definiert. Die Anzahl der Gene in einem Genom wird durch den
Variablennamen \verb|num_genes| bezeichnet. Ein Gen wird \introdef{exprimiert},
wenn die Regel, die es codiert, aktiviert wird.

Zur textuellen Darstellung einer Zellzustandsklasse werden die spezifizierten
Bits durch ihren jeweiligen Zustand \verb|0| oder \verb|1| dargestellt,
unspezifizierte Bits werden durch \verb|*| repr"asentiert.
Ein Zellustand $Z$ geh"ort zu einer Klasse, wenn der Zustand
s"amtlicher in der Klasse spezifizierten Bits dem Zustand der entsprechenden
Bits in $Z$ entspricht. Zum Beispiel geh"oren zur Klasse
\verb|1*******| s"amtliche Zust"ande, bei denen das Bit 7 gesetzt
ist. Zur Klasse \verb|0000000*| geh"oren nur die beiden Zust"ande
\verb|00000000| und \verb|00000001|.

Zur textuellen Beschreibung einer Regel wird ihre linke Seite in der
beschriebenen Weise dargestellt, darauf folgt das Symbol \verb|->|
und schlie"slich das Codewort f"ur die spezifizierte Aktion
(vgl.\ \ref{cellactiondef}). Diesen Angaben k"onnen zu"atzliche
Informationen "uber das Gen folgen. Die Regeln werden entsprechend ihrer
Reihenfolge auf dem Genom numeriert. Diese Darstellung eines Genoms
wird als sein {\slshape listing} bezeichnet. Ein Beispiel f"ur das
{\slshape listing} eines Genoms ist:

\begin{verbatim}
  1: 00000001 -> flyingseed
  2: 00000000 -> divide 7
  3: 000**001 -> divide 6
  4: ***1*00* -> mut+
  5: 0***00*1 -> statebit 7
\end{verbatim}

Der Zustand einer Keimzelle ist stets 0. Einem Gen, welches eine Regel codiert,
die f"ur diesen Zustand eine Aktion angibt, ist somit erforderlich, um den Wachstumsproze"s
einer Pflanze anzusto"sen und andere Gene zu aktivieren. Ein solches Gen wird als
\introdef{Keimzellgen} bezeichnet.

Wenn durch den Zustand einer Zelle mehrere Gene aktiviert werden, deren
rechte Seite energieverbrauchende Aktionen festlegen, kann nur eine dieser
Aktionen ausgef"uhrt werden. In diesem Fall wird nur das erste Gen auf dem Genom,
welches eine energieverbrauchende Aktion spezifiziert, exprimiert; die
Expression aller weiteren Gene, die energieverbrauchende Aktionen ausl"osen,
wird unterdr"uckt. Im obigen Beispiel
w"urde der Zellzustand \verb|00000001| zur Aktivierung der Regel Nr.\ 1
f"uhren. Regel Nr.\ 3 w"urde nicht aktiviert, obwohl seine Zustandsklasse
den Zellzustand \verb|00000001| ebenfalls einschlie"st. Regel Nr.\ 5 w"urde hingegen
aktiviert, weil \verb|statebit| keine Energie ben"otigt.
Der folgende Pseudocode beschreibt die Ermittlung der von einer Zelle $C$
auszuf"uhrenden Aktionen anhand des im Genom codierten Regelwerks.

\begin{verbatimcmd}
FOR i = 0 TO num_genes - 1
  IF \pcodemeta{\begin{math}Z(C)\end{math} geh{\"o}rt zur Zustandsklasse der i-ten Regel}
    IF \pcodemeta{i-te Regel spezifiziert energieverbrauchende Aktion}
      IF \pcodemeta{C ist energiereich}
        \pcodemeta{F{\"u}hre durch i-te Regel spezifizierte Aktion aus}
        \pcodemeta{Setze \begin{math}E(C)\end{math} auf energielos}
      ENDIF
    ELSE
      \pcodemeta{F{\"u}hre durch i-te Regel spezifizierte Aktion aus}
    ENDIF
  ENDIF
NEXT i
\end{verbatimcmd}


\subsubsection{Blockorientierte Genominterpretation}
\label{blockinterdef}

Bei dem blockorientiert arbeitenden Interpretersystem
besteht jedes Gen aus zwei aufeinanderfolgenden Bytes im Genom. An jeder
geradzahligen Position des Genoms (also bei den Zeichen 0, 2, 4 usw.) beginnt
jeweils ein Gen. Die Gene liegen damit dicht im Genom, es gibt weder
"Uberlappungen noch Bytes, die keinem Gen zugeordnet sind. Abb.\ \ref{blockinterfig}
verdeutlicht die Anordnung der Gene bei der blockorientierten Genominterpretation.

Bei der blockorientierten Genominterpretation kommen keine unspezifizierten Bits auf
der linken Seite einer Regel vor. Das erste Byte eines Gens 
spezifiziert s"amtliche Bits eines Zellzustands,
in ihm ist somit die gesamte linke Seite der Regel codiert. Im zweiten
Byte des Gens ist die rechte Seite der Regel in einem Achtbit-Aktionscode codiert,
seine Codierung wird in (\ref{actioncodedef}) beschrieben.

Die blockorientierte Genominterpretation kann nur in Verbindung mit
Achtbit-Zell\-zu\-st"an\-den sinnvoll verwendet werden, da weitere Bits nicht
in der linken Regelseite spezifiziert werden k"onnen. Eine Erweiterung
des Prinzips der blockorientierten Genominterpretation auf gr"o"sere Bitbreiten ist
zwar leicht konstruierbar, sie hat sich jedoch als unpraktikabel erwiesen.
Auf diesen Punkt wird sp"ater (\ref{lnd2-summary}) n"aher eingegangen.

\begin{figure}

\unitlength1cm
\begin{picture}(16,5)
{\footnotesize
\put(0,2.6){\makebox(2,0.3)[l]{Index:}}
\put(2,2.6){\makebox(1,0.3){0}}
\put(3,2.6){\makebox(1,0.3){1}}
\put(4,2.6){\makebox(1,0.3){2}}
\put(5,2.6){\makebox(1,0.3){3}}
\put(6,2.6){\makebox(1,0.3){4}}
\put(7,2.6){\makebox(1,0.3){5}}
}
\put(0,2){\makebox(2,0.5)[l]{Genom:}}
{\ttfamily
% \put(1,2){\line(1,0){5.5}}
% \put(1,2.5){\line(1,0){5.5}}
\put(2,2){\framebox(1,0.5){\$01}}
\put(3,2){\framebox(1,0.5){\$2B}}
\put(4,2){\framebox(1,0.5){\$88}}
\put(5,2){\framebox(1,0.5){\$47}}
\put(6,2){\framebox(1,0.5){\$A0}}
\put(7,2){\framebox(1,0.5){\$03}}
}
\thicklines
% \put(2,1.5){\line(0,1){0.2}}
\put(2,1.35){\framebox(2,0.35){Gen 0}}
% \put(4,1.5){\line(0,1){0.2}}
\put(4,1.35){\framebox(2,0.35){Gen 1}}
% \put(6,1.5){\line(0,1){0.2}}
\put(6,1.35){\framebox(2,0.35){Gen 2}}
% \put(8,1.5){\line(0,1){0.2}}
\thinlines
\end{picture}

\begin{picture}(16,4.5)
\put(0,4.0){\makebox(2,0.5)[l]{Gen 0:}}
{\ttfamily
\put(2,4.0){\framebox(1,0.5){\$01}}
\put(3,4.0){\framebox(1,0.5){\$2B}}
\put(6,4.0){\makebox(8,0.5)[l]{00000001 -> action code \$2B}}

\put(2.5,4.0){\line(0,-1){0.4}}
\put(2.5,3.6){\line(1,0){4.2}}
\put(6.7,3.6){\vector(0,1){0.4}}

\put(3.5,4.0){\line(0,-1){0.6}}
\put(3.5,3.4){\line(1,0){7.2}}
\put(10.7,3.4){\vector(0,1){0.6}}
}

\put(0,2.5){\makebox(2,0.5)[l]{Gen 1:}}
{\ttfamily
\put(2,2.5){\framebox(1,0.5){\$88}}
\put(3,2.5){\framebox(1,0.5){\$47}}
\put(6,2.5){\makebox(8,0.5)[l]{10001000 -> action code \$47}}

\put(2.5,2.5){\line(0,-1){0.4}}
\put(2.5,2.1){\line(1,0){4.2}}
\put(6.7,2.1){\vector(0,1){0.4}}

\put(3.5,2.5){\line(0,-1){0.6}}
\put(3.5,1.9){\line(1,0){7.2}}
\put(10.7,1.9){\vector(0,1){0.6}}
}

\put(0,1.0){\makebox(2,0.5)[l]{Gen 2:}}
{\ttfamily
\put(2,1){\framebox(1,0.5){\$A0}}
\put(3,1){\framebox(1,0.5){\$03}}
\put(6,1.0){\makebox(8,0.5)[l]{10100000 -> action code \$03}}

\put(2.5,1.0){\line(0,-1){0.4}}
\put(2.5,0.6){\line(1,0){4.2}}
\put(6.7,0.6){\vector(0,1){0.4}}

\put(3.5,1.0){\line(0,-1){0.6}}
\put(3.5,0.4){\line(1,0){7.2}}
\put(10.7,0.4){\vector(0,1){0.6}}
}
\end{picture}

\caption{\label{blockinterfig}
Schema der blockorientierten Genominterpretation.
}
\end{figure}


\subsubsection{Graphische Darstellung von Genomen bei blockorientierter Interpretation}

\begin{figure}[tb]

% {\bfseries Aufbau einer Zeile im {\slshape listing}}
% 
% \begin{tabular}{r@{\ttfamily [}r@{\ttfamily ]:}lcc}
% \verb|2| & \verb|3| & \verb|00000000 -> divide 7   | & \verb|00 07| \\
% Index des Gens & Anzahl Aktivierungen & Regel & codierende Bytesequenz \\
% \end{tabular}
% 

{\bfseries {\slshape Listing} des Genoms:}
\begin{verbatim}
    0 [    1]: 00000000 -> divide 7                 00 07
    1 [    0]: 00000101 -> divide 5                 05 05
    2 [    0]: 10100011 -> divide 0                 a3 b8
    3 [    0]: 11000100 -> flying seed              c4 c0
    4 [    0]: 01001000 -> local seed               48 d7
    5 [    0]: 10100101 -> mut-                     a5 eb
    6 [    0]: 00000011 -> mut+                     03 f4
\end{verbatim}

{\bfseries Graphische Darstellung des Genoms:}

\unitlength1cm
\begin{picture}(16,3)
\put(0,0){\makebox(16,3){\epsfxsize=16cm \epsffile{lstexample.eps}}}
\end{picture}

\caption[Graphische Darstellung von Regeln]{\label{lnd1-graphrule}
Darstellung eines Genoms als {\slshape listing} und in graphischer Form. Im {\slshape listing} gibt die
einer Regel vorangestellte Zahl den Index des Gens innerhalb des Genoms an. Nach dem Index kann, in eckigen
Klammern, angegeben werden, wie oft das Gen w"ahrend der Entwicklung der Pflanze aktiviert wurde.
Hinter einer Regel wird die Bytesequenz, in der die Regel codiert ist, in hexadezimaler Schreibweise
angegeben. \newline
Bei der graphischen Darstellung wird die linke Seite, die bei der blockorientierten Genominterpretation
stets genau einen Zellzustand spezifiziert,
durch eine Abbildung dieser lokalen Struktur visualisiert. Die Darstellung der
rechten Seite h"angt von der in dieser angegebenen Aktion ab. Steht dort die Aktion {\ttfamily divide},
wird die aus der Division resultierende Struktur dargestellt, wobei die Tochterzelle von den
restlichen Zellen durch dunklere Schattierung abgesetzt wird. Andere Aktionen werden durch entsprechende
Symbole repr"asentiert (s.\ Text).
Unterhalb der graphischen Darstellung eines Gens werden in einer weiteren Box sein Index innerhalb
des Genoms und die Anzahl seiner Aktivierungen ({\slshape usage counter}) angegeben.
}
\end{figure}

Zur Verbesserung der Anschaulichkeit
werden bei der blockorientierten Genominterpretation Regelwerke graphisch dargestellt.
Zur Darstellung einer Regel wird dabei die in ihrer linken Seite spezifizierte
lokale Struktur wiedergegeben. Gibt der Aktionsteil der Regel eine Zellteilung an, wird
die nach der Teilung resultierende lokale Struktur gezeichnet, wobei die neu gebildete
Tochterzelle besonders gekennzeichnet wird. F"ur die anderen Aktionen werden folgende
Symbole verwendet:

\medskip
\begin{tabular}{|c|l|} \hline
Symbol & Aktion \\ \hline
+ & \verb|mut+| \\
- & \verb|mut-| \\
$\circ$ & \verb|flyingseed| \\
$\bullet$ & \verb|localseed| \\ \hline
\end{tabular}
\medskip

Als Zusatzinformationen zu den Regeln werden ihr Index sowie die Anzahl ihrer
Aktivierungen angegeben.
Abb.\ \ref{lnd1-graphrule} illustriert die graphische Darstellung von Regelwerken anhand
eines Beispiels.


\subsubsection{Regulatorische Netzwerke bei blockorientierter Genominterpretation}

\begin{figure}

\unitlength1cm
\begin{picture}(16,8)
\put(0,7){\makebox(16,0.5){\bfseries Orthogonale Tochterzellposition}}
\put(0,6.2){\makebox(16,0.5){Beeinflu"ste Zelle in Nachbarschaft von M:}}
\put(0,3.5){\makebox(2.5,2.5){\epsfxsize=2cm \epsffile{nhoinf0.eps}}}
\put(2.6,3.5){\makebox(2.5,2.5){\epsfxsize=2cm \epsffile{nhoinf1.eps}}}
\put(5.2,3.5){\makebox(2.5,2.5){\epsfxsize=2cm \epsffile{nhoinf3.eps}}}
\put(7.8,3.5){\makebox(2.5,2.5){\epsfxsize=2cm \epsffile{nhoinf4.eps}}}
\put(10.4,3.5){\makebox(2.5,2.5){\epsfxsize=2cm \epsffile{nhoinf6.eps}}}
\put(13.0,3.5){\makebox(2.5,2.5){\epsfxsize=2cm \epsffile{nhoinf7.eps}}}
\put(0,2.8){\makebox(16,0.5){Beeinflu"ste Zelle au"serhalb der Nachbarschaft von M:}}
\put(3.9,0){\makebox(2.5,2.5){\epsfxsize=2cm \epsffile{nhoinf2.eps}}}
\put(6.5,0){\makebox(2.5,2.5){\epsfxsize=2cm \epsffile{nhoinf5.eps}}}
\put(9.1,0){\makebox(2.5,2.5){\epsfxsize=2cm \epsffile{nhoinf8.eps}}}
\end{picture}

\begin{picture}(16,8)
\put(0,7){\makebox(16,0.5){\bfseries Diagonale Tochterzellposition}}
\put(0,6.2){\makebox(16,0.5){Beeinflu"ste Zelle in Nachbarschaft von M:}}
\put(2.6,3.5){\makebox(2.5,2.5){\epsfxsize=2cm \epsffile{nhdinf3.eps}}}
\put(5.2,3.5){\makebox(2.5,2.5){\epsfxsize=2cm \epsffile{nhdinf4.eps}}}
\put(7.8,3.5){\makebox(2.5,2.5){\epsfxsize=2cm \epsffile{nhdinf6.eps}}}
\put(10.4,3.5){\makebox(2.5,2.5){\epsfxsize=2cm \epsffile{nhdinf7.eps}}}
\put(0,2.8){\makebox(16,0.5){Beeinflu"ste Zelle au"serhalb der Nachbarschaft von M:}}
\put(1.3,0){\makebox(2.5,2.5){\epsfxsize=2cm \epsffile{nhdinf0.eps}}}
\put(3.9,0){\makebox(2.5,2.5){\epsfxsize=2cm \epsffile{nhdinf1.eps}}}
\put(6.5,0){\makebox(2.5,2.5){\epsfxsize=2cm \epsffile{nhdinf2.eps}}}
\put(9.1,0){\makebox(2.5,2.5){\epsfxsize=2cm \epsffile{nhdinf5.eps}}}
\put(11.7,0){\makebox(2.5,2.5){\epsfxsize=2cm \epsffile{nhdinf8.eps}}}
\end{picture}

\caption[Modifikation lokaler Strukturen durch Zellteilungen]{\label{neighbor-inf}
Die durch die Produktion einer Tochterzelle T durch eine Mutterzelle M ver"anderten
lokalen Strukturen. Der Bereich der lokalen Struktur der Mutterzelle ist durch
ein Gitter dargestellt. Der grau unterlegte Bereich kennzeichnet die Nachbarschaften,
die durch die Produktion von T ver"andert werden. Dabei ist die zentrale Position
dunkler hervorgehoben. An der zentralen Position befindet
sich ggfs.\ die Zelle, deren Zustand durch die Teilung beeinflu"st wird.
Bei der Produktion einer Zelle in einer Position orthogonal zur Mutterzelle liegen
mehr solcher Positionen innerhalb der Nachbarschaft der Mutterzelle. Der
"Uberlappungsbereich zwischen der Nachbarschaft der Mutterzelle und der
jeweils durch die Teilung modifizierten Nachbarschaft ist schraffiert.
}
\end{figure}

\begin{figure}

\unitlength1cm
\begin{picture}(16,12.0)
\put(0,9.5){\makebox(7,5){\epsfxsize=6cm \epsffile{p010.eps}}}
\put(0,9){\makebox(7,0.5){\bfseries (a)}}
\put(8,9.5){\makebox(7,5){\epsfxsize=6cm \epsffile{p015.eps}}}
\put(8,9){\makebox(7,0.5){\bfseries (b)}}
\put(0,5){\makebox(7,5){\epsfxsize=6cm \epsffile{p005.eps}}}
\put(0,4.5){\makebox(7,0.5){\bfseries (c)}}
\put(8,5){\makebox(7,5){\epsfxsize=6cm \epsffile{p012.eps}}}
\put(8,4.5){\makebox(7,0.5){\bfseries (d)}}
\put(0,0.5){\makebox(7,5){\epsfxsize=6cm \epsffile{pxlong0105a.eps}}}
\put(0,0){\makebox(7,0.5){\bfseries (e)}}
\put(8,0.5){\makebox(7,5){\epsfxsize=6cm \epsffile{pxlong0105u.eps}}}
\put(8,0){\makebox(7,0.5){\bfseries (f)}}
\end{picture}
\caption[Graphische Darstellung regulatorischer Netze]{\label{lnd1-graphgenom}
Graphische Darstellung eines LindEvol Genoms. Es wird jeweils das vollst"andige
Genom einer Pflanze mit dem entsprechenden regulatorischen Netzwerk gezeigt. Oberhalb
jeden Genoms ist die resultierende Pflanze dargestellt. Die Genome in den Beispielen
(a) bis (d) wurden manuell erstellt und sind nicht Ergebnis von LindEvol-GA--Simulationen.\newline
(a) Gen 0 ist ein starker Aktivator von Gen 1: Nachdem durch Gen 0 die Produktion einer
Zelle orthogonal oberhalb der Keimzelle ausgel"ost wurde, befindet sich die Mutterzelle
in einer lokalen Struktur, durch die Gen 1 aktiviert wird. (b) In diesem Beispiel ist
Gen 0 ein starker Aktivator f"ur Gen 1, weil die Tochterzelle sich nach der durch Gen
0 gesteuerten Teilung in einer Struktur befindet, die zur Aktivierung von Gen 1
f"uhren kann. Gen 1 ist ein schwacher Aktivator von sich selbst, weil sich an der Position links
oberhalb der von Gen 1 produzierten Tochterzelle eine weitere Zelle befinden k"onnte. In dieser
Zelle w"urde durch die von Gen 1 gesteuerte Zellteilung ein Beitrag zur Aktivierung von 
Gen 1 geleistet. Diese Konstellation tritt jedoch beim Wachstum der Pflanze nicht auf;
in diesem Fall ist dies auch deshalb unm"oglich, weil die angenommene weitere Zelle zu
Beginn eine leere lokale Struktur haben m"u"ste. Dies ist jedoch ausschlie"slich bei der
Keimzelle einer Pflanze der Fall.
(c) Hier ist wiederum Gen 0 ein starker Aktivator von Gen 1. Gen 1
wird zus"atzlich durch sich selbst stark aktiviert, weil die durch Gen 1 gesteuerte
Teilung zu einer Tochterzelle f"uhrt, in der wiederum Gen 1 aktiviert werden kann.
Auf diese Weise kann durch zwei Gene ein Wachstumsprogramm f"ur einen diagonalen
Spro"s unbegrenzter L"ange codiert werden. (d) Gen 0 ist ein schwacher Aktivator
f"ur alle Gene, bei denen in der linken Seite eine Zelle links unterhalb der aktuellen
Mutterzelle spezifiziert ist. Keines dieser Gene wird jedoch tats"achlich aktiviert , weil 
Gene zur Produktion weiterer zur Aktivierung erforderlicher Zellen nicht existieren.
Auch das Gen 4, f"ur das Gen 0 sogar ein starker Aktivator ist, wird nicht aktiviert,
weil kein Gen existiert, welches die zur Aktivierung von Gen 4 erforderliche Zelle
oberhalb der aktuellen Mutterzelle erzeugen k"onnte.
(e) In einem Genom existiert in der Regel eine Vielzahl
starker und schwacher Aktivierungsbeziehungen zwischen Genen. Viele dieser Beziehungen sind f"ur
das Wachstum der Pflanze jedoch irrelevant, weil weder das aktivierende noch das aktivierte Gen
jemals tats"achlich aktiviert werden. Bei der Vielzahl regulatorischer Beziehungen ist es
schwierig, zu erkennen, welche dieser Beziehungen bei der Entstehung des Ph"anotyps der Pflanze
eine Rolle gespielt haben. (f) zeigt dasselbe Genom wie (e), der Unterschied ist, da"s hier die unbenutzten
Gene ausgeblendet wurden. Auf diese Weise werden die relevanten regulatorischen Wechselwirkungen
aus der Vielzahl der Aktivierungsbeziehungen herausgearbeitet, und der Zusammenhang zwischen dem Entwicklungsprogramm
und dem Ph"anotyp wird klar erkennbar.
}
\end{figure}

Die Aktivierung eines Gens erfolgt in den LindEvol-Modellen durch \verb|divide|-Aktionen.
Die lokale Struktur, in der sich eine neu gebildete Zelle befindet, entspricht in der
Regel nicht der lokalen Struktur der Mutterzelle. Somit findet eine Differenzierung in
Abh"angigkeit von lokaler, positionaler Information statt.
Weiterhin ver"andert sich bei einer Teilung der Zustand der Mutterzelle und weiterer
Zellen, die sich in der Nachbarschaft der neu entstandenen Zelle befinden.
Durch diese Ver"anderungen der lokalen Strukturen von Zellen erfolgt die Aktivierung
von Genen. Somit wird die
Aktivierung eines Gens durch die Aktivit"aten von Genen reguliert.
Nur ein Gen, dessen linke Seite eine leere lokale Struktur spezifiziert, wird nicht
durch Zellteilungen aktiviert, sondern es kann ausschlie"slich in der Keimzelle einer
Pflanze aktiv werden, weshalb es als \introdef{Keimzellgen} bezeichnet wird.
Insgesamt bilden die Gene, die im Laufe der Morphogenese einer LindEvol-Pflanze aktiviert
werden, ein regulatorisches Netzwerk aus, wie es auch bei molekulargenetischen
Systemen, z.B. f"ur die MADS-Box-Gene beschrieben wurde \cite{Theissen95}.

Bei der blockorientierten Genominterpretation kommen keine unspezifizierten Bits
in der linken Seite einer Regel vor. 
Wird die blockorientierte Genominterpretation also in einem Modell eingesetzt,
bei dem die Pflanzenzellen nicht "uber interne Zust"ande verf"ugen (s.\ \ref{celldef}),
wird ein Gen durch genau eine lokale Struktur
aktiviert. Wenn die Aktivierung des Gens zur Produktion einer neuen Zelle f"uhrt,
werden dadurch die lokalen Strukturen an neun Zellpositionen, n"amlich an
der Position der Tochterzelle und an ihren acht Nachbarpositionen, beeinflu"st.

Die lokale Struktur der Mutterzelle nach einer erfolgreichen Teilung kann aus
der Regel, die die Teilung gesteuert hat, vollst"andig abgeleitet werden, indem
einfach das der Position der Tochterzelle zugeordnete Bit in der linken Seite
der Regel zus"atzlich gesetzt wird. F"ur die restlichen Zellpositionen, deren
lokale Struktur durch die Entstehung der Tochterzelle ver"andert wird, kann die
resultierende Struktur hingegen nicht vollst"andig abgeleitet werden, weil
Teile der Nachbarschaft der betroffenen Zelle nicht zu der Nachbarschaft der
Mutterzelle geh"oren. Abb.\ \ref{neighbor-inf} zeigt die Positionen, 
deren lokale Struktur durch die Produktion einer neuen Zelle modifiziert wird.
Zwischen den modifizierten Nachbarschaften und der Nachbarschaft der Mutterzelle
existiert jeweils ein "Uberlappungsbereich. Der Bereich einer lokalen Struktur, der in einem
solchen "Uberlappungsbereich enthalten ist, wird \introdef{partielle lokale Struktur}
genannt.

Wenn die aus einer Teilung resultierende lokale Struktur derart mit der in der
linken Seite einer Regel spezifizierten lokalen Struktur "uberlagert werden
kann, da"s die partiellen Strukturen im "Uberlappungsbereich "ubereinstimmen,
bedeutet dies, da"s die betreffende Regel durch die Teilung aktiviert werden
kann. Diejenige Regel, die die Teilung ausgel"ost hat, wird in diesem Fall
\introdef{Aktivator} genannt.

Wenn eine Regel \regel{A} Aktivator einer Regel \regel{R} ist, bedeutet dies
nicht, da"s die Ausl"osung einer Zellteilung durch \regel{A} zwangsl"aufig
die Aktivierung von \regel{R} nach sich zieht. Neben der durch \regel{A}
produzierten Zelle k"onnen weitere, au"serhalb des "Uberlappungsbereichs liegende
Zellen erforderlich sein, um die in der
linken Seite von \regel{R} spezifizierte lokale Struktur zu erzeugen.
In diesem Fall sind weitere Zellteilungen erforderlich,
um \regel{R} zu aktivieren. Zur Ausf"uhrung dieser weiteren Zellteilungen k"onnen
weitere Aktivatoren au"ser \regel{A} erforderlich sein.

Die durch eine Zellteilung beeinflu"sten Nachbarschaften zerfallen, wie Abb.\ \ref{neighbor-inf} zeigt,
in zwei Klassen: Die zentrale Position der beeinflu"sten Nachbarschaft kann innerhalb
oder au"serhalb des "Uberlappungsbereichs liegen. Liegt die zentrale Position innerhalb
des "Uberlappungsbereichs, bedeutet dies, da"s aus der nach der Teilung resultierenden Struktur
entnommen werden kann, ob sich an dieser Position eine Zelle befindet. Zu der lokalen Struktur
dieser Zelle wird dann durch die vom Aktivator \regel{A} herbeigef"uhrte Zellteilung eine Zelle
hinzugef"ugt, die zur Aktivierung von \regel{R} erforderlich ist. In diesem Fall wird \regel{A}
als \introdef{starker Aktivator} von \regel{R} bezeichnet.

Liegt die zentrale Zelle der durch die von dem Aktivator \regel{A} herbeigef"uhrten Zellteilung
beeinflu"sten Nachbarschaft dagegen au"serhalb der Nachbarschaft der Mutterzelle, welche \regel{A}
exprimiert hat, kann dagegen aus der resultierenden Nachbarschaft der Mutterzelle nicht abgeleitet
werden, ob sich an der zentralen Position der modifizierten Nachbarschaft "uberhaupt eine Zelle befindet.
Wenn dort eine Zelle existiert, wird durch die von \regel{A} gesteuerte Zellteilung ein Beitrag
zur Aktivierung von \regel{R} in dieser Zelle geleistet. Es ist jedoch auch m"oglich, da"s an
der fraglichen Position "uberhaupt keine Zelle exisitiert. Daher wird ein Aktivator dieser Art
als \introdef{schwacher Aktivator} von \regel{R} bezeichnet.

Das von den Genen eines Genoms gebildete regulatorische Netzwerk kann nun graphisch
dargestellt werden, indem von jedem Gen eine gerichtete Kante zu den Genen, dessen
Aktivator es ist, gezeichnet wird. Bei schwachen Aktivatoren werden graue, bei starken
Aktivatoren schwarze Pfeile gezeichnet. Abb.\ \ref{lnd1-graphgenom} demonstriert diese
Technik zur Visualisierung regulatorischer Netzwerke anhand einiger Beispiele.


\subsubsection{Promotororientierte Genominterpretation}
\label{prominterdef}

Ein Gen ist ein definierter Sequenzabschnitt in einem Genom. Bei der blockorientierten
Genominterpretation werden die Abschnitte des Genoms, die jeweils ein Gen bilden, aufgrund
ihrer absoluten Position im Genom definiert. In der molekularen Natur werden Gene, oder
genauer, transkriptionelle Einheiten (vgl.\ \cite{Watson}, s.\ 233), jedoch nicht aufgrund
ihrer absoluten Position erkannt; vielmehr sind sie dadurch festgelegt, da"s der Transkriptionsmechanismus
bestimmte Sequenzmotive in der DNA, Promotoren und Terminatoren, zu erkennen vermag, bei denen die
Transkription initiiert bzw.\ terminiert wird. Dieses Prinzip der Markenorientierung wurde in
der promotororientierten Genominterpretation umgesetzt.

Gene sind bei der promotororientierten Genominterpretation als Abschnitte des Genoms definiert,
die durch Bytes mit speziellen Werten begrenzt sind. Die L"ange von Genen kann daher bei der
promotororientierten Genominterpretation unterschiedliche Werte annehmen. Die Bytes, die Anfang
und Ende eines Gens festlegen, werden entsprechend ihrer molekularen Vorbilder als Promotoren
und Terminatoren bezeichnet. Ein \introdef{Promotor} ist ein Byte, in dem das 7.\ Bit
gesetzt ist, dessen Wert also zwischen 128 und 255 liegt. Ein \introdef{Terminator} ist
ein Byte, in dem das 6.\ Bit gesetzt ist, dessen Wert dementsprechend entweder
zwischen 64 und 127 oder zwischen 192 und 255 liegt.

Ein \introdef{terminiertes Gen} ist ein Abschnitt von einem Promotor bis
zu einem Terminator, in dem sich keine Promotoren oder Terminatoren befinden.
Ein \introdef{nicht terminiertes Gen} ist ein Abschnitt zwischen zwei
Promotoren, in dem keine Promotoren oder Terminatoren liegen, oder ein
von Promotoren und Terminatoren freier Abschnitt zwischen einem Promotor
und dem Ende des Genoms.

Ein Byte, in dem sowohl
das 6.\ als auch das 7.\ Bit gesetzt sind, ist gleichzeitig Terminator und Promotor
des folgenden Gens. Ein Gen, das durch einen solchen Terminator-Promotor
beendet wird, ist terminiert.

Bytes zwischen einem Promotor und einem Terminator, in denen weder das
6.\ noch das 7.\ Bit gesetzt sind, werden \introdef{Operatoren} genannt.
Durch jeden Operator wird ein Bit der linken Regelseite spezifiziert. Die
Bits 0--4 des Operators geben den Index des spezifizierten Bits an. Damit ist
die Spezifikation von maximal 32 Bits m"oglich. Bei Modellen mit Zellzust"anden
mit einer Breite von 16 Bit werden nur die Bits 0--3 der Operatoren verwendet,
das 4.\ Bit wird ignoriert. Das 5.\ Bit eines Operators gibt den Zustand
des spezifizierten Zustands an. Wenn in einem Gen mehrere Operatoren dasselbe
Bit in der Zustandsklasse spezifizieren, wird nur der letzte dieser Operatoren
wirksam. Ein Bit in der Zustandsklasse, das durch keinen Operator spezifiziert
wird, bleibt unspezifiziert. Somit werden Gene, die keine Operatoren besitzen,
also nur aus einem Promotor und einem Terminator bestehen, konstitutiv exprimiert.

Bei terminierten Genen steht in den Bits 0--5 des Terminators der Sechsbit-Aktionscode
der rechten Seite der Regel. Dessen Codierung wird in (\ref{actioncodedef}) beschrieben.
Nicht terminierte Gene l"osen keine Aktion aus.

\begin{figure}[t]

\unitlength1cm
\begin{picture}(16,5)
{\scriptsize
\put(0,2.6){\makebox(1.3,0.3)[l]{Index:}}
\put(1.3,2.6){\makebox(0.7,0.3){0}}
\put(2.0,2.6){\makebox(0.7,0.3){1}}
\put(2.7,2.6){\makebox(0.7,0.3){2}}
\put(3.4,2.6){\makebox(0.7,0.3){3}}
\put(4.1,2.6){\makebox(0.7,0.3){4}}
\put(4.8,2.6){\makebox(0.7,0.3){5}}
\put(5.5,2.6){\makebox(0.7,0.3){6}}
\put(6.2,2.6){\makebox(0.7,0.3){7}}
\put(6.9,2.6){\makebox(0.7,0.3){8}}
\put(7.6,2.6){\makebox(0.7,0.3){9}}
\put(8.3,2.6){\makebox(0.7,0.3){10}}
\put(9.0,2.6){\makebox(0.7,0.3){11}}
\put(9.7,2.6){\makebox(0.7,0.3){12}}
\put(10.4,2.6){\makebox(0.7,0.3){13}}
\put(11.1,2.6){\makebox(0.7,0.3){14}}
\put(11.8,2.6){\makebox(0.7,0.3){15}}
\put(12.5,2.6){\makebox(0.7,0.3){16}}
\put(13.2,2.6){\makebox(0.7,0.3){17}}
\put(13.9,2.6){\makebox(0.7,0.3){18}}
\put(14.6,2.6){\makebox(0.7,0.3){19}}
\put(15.3,2.6){\makebox(0.7,0.3){20}}
}
\put(0,2){\makebox(1.3,0.5)[l]{Genom:}}
{\ttfamily
% \put(1,2){\line(1,0){5.5}}
% \put(1,2.5){\line(1,0){5.5}}
\put(1.3,2){\framebox(0.7,0.5){\$01}}
\put(2.0,2){\framebox(0.7,0.5){\$80}} % gen 0
\put(2.7,2){\framebox(0.7,0.5){\$00}}
\put(3.4,2){\framebox(0.7,0.5){\$22}}
\put(4.1,2){\framebox(0.7,0.5){\$01}}
\put(4.8,2){\framebox(0.7,0.5){\$40}} %
\put(5.5,2){\framebox(0.7,0.5){\$1B}}
\put(6.2,2){\framebox(0.7,0.5){\$2F}}
\put(6.9,2){\framebox(0.7,0.5){\$97}} % gen 1, unterminiert
\put(7.6,2){\framebox(0.7,0.5){\$10}}
\put(8.3,2){\framebox(0.7,0.5){\$24}}
\put(9.0,2){\framebox(0.7,0.5){\$BC}} % gen 2
\put(9.7,2){\framebox(0.7,0.5){\$31}}
\put(10.4,2){\framebox(0.7,0.5){\$01}}
\put(11.1,2){\framebox(0.7,0.5){\$47}} %
\put(11.8,2){\framebox(0.7,0.5){\$10}}
\put(12.5,2){\framebox(0.7,0.5){\$A7}} % gen 3
\put(13.2,2){\framebox(0.7,0.5){\$1C}}
\put(13.9,2){\framebox(0.7,0.5){\$F9}} %% gen 4
\put(14.6,2){\framebox(0.7,0.5){\$7B}} %
\put(15.3,2){\framebox(0.7,0.5){\$6D}}
}
\thicklines
% \put(2.0,1.5){\line(0,1){0.2}}
\put(2.0,1.35){\framebox(3.5,0.35){Gen 0}}
% \put(5.5,1.5){\line(0,1){0.2}}
% \put(6.9,1.5){\line(0,1){0.2}}
\put(6.9,1.35){\framebox(2.1,0.35){Gen 1}}
% \put(9.0,1.5){\line(0,1){0.2}}
% \put(9.0,1.5){\line(0,1){0.2}}
\put(9.0,1.35){\framebox(2.8,0.35){Gen 2}}
% \put(11.8,1.5){\line(0,1){0.2}}
% \put(12.5,1.5){\line(0,1){0.2}}
\put(12.5,1.35){\framebox(2.1,0.35){Gen 3}}
% \put(14.6,1.5){\line(0,1){0.2}}
% \put(13.9,1.0){\line(0,1){0.2}}
\put(13.9,0.85){\framebox(1.4,0.35){Gen 4}}
% \put(15.3,1.0){\line(0,1){0.2}}
\thinlines
\end{picture}

\begin{picture}(16,7.5)

\put(0,7.0){\makebox(2,0.5)[l]{Gen 0:}}
{\ttfamily
\put(2.0,7.0){\framebox(0.7,0.5){\$80}} % gen 0
\put(2.7,7.0){\framebox(0.7,0.5){\$00}}
\put(3.4,7.0){\framebox(0.7,0.5){\$22}}
\put(4.1,7.0){\framebox(0.7,0.5){\$01}}
\put(4.8,7.0){\framebox(0.7,0.5){\$40}} %

\put(6.00,7.0){\makebox(0.25,0.5){*}}
\put(6.25,7.0){\makebox(0.25,0.5){*}}
\put(6.50,7.0){\makebox(0.25,0.5){*}}
\put(6.75,7.0){\makebox(0.25,0.5){*}}

\put(7.00,7.0){\makebox(0.25,0.5){*}}
\put(7.25,7.0){\makebox(0.25,0.5){*}}
\put(7.50,7.0){\makebox(0.25,0.5){*}}
\put(7.75,7.0){\makebox(0.25,0.5){*}}

\put(8.00,7.0){\makebox(0.25,0.5){*}}
\put(8.25,7.0){\makebox(0.25,0.5){*}}
\put(8.50,7.0){\makebox(0.25,0.5){*}}
\put(8.75,7.0){\makebox(0.25,0.5){*}}

\put(9.00,7.0){\makebox(0.25,0.5){*}}
\put(9.25,7.0){\makebox(0.25,0.5){1}}
\put(9.50,7.0){\makebox(0.25,0.5){0}}
\put(9.75,7.0){\makebox(0.25,0.5){0}}

\put(10.00,7.0){\makebox(5,0.5)[l]{-> action code \$00}}

\put(3.05,7.0){\line(0,-1){0.2}}
\put(3.05,6.8){\line(1,0){6.775}}
\put(9.875,6.8){\vector(0,1){0.2}}

\put(3.75,7.0){\line(0,-1){0.4}}
\put(3.75,6.6){\line(1,0){5.625}}
\put(9.375,6.6){\vector(0,1){0.4}}

\put(4.45,7.0){\line(0,-1){0.6}}
\put(4.45,6.4){\line(1,0){5.175}}
\put(9.625,6.4){\vector(0,1){0.6}}

\put(5.15,7.5){\line(0,1){0.2}}
\put(5.15,7.7){\line(1,0){7.85}}
\put(13.0,7.7){\vector(0,-1){0.2}}
}

\put(0,5.5){\makebox(2,0.5)[l]{Gen 1:}}
{\ttfamily
\put(2.0,5.5){\framebox(0.7,0.5){\$97}} % gen 1, unterminiert
\put(2.7,5.5){\framebox(0.7,0.5){\$10}}
\put(3.4,5.5){\framebox(0.7,0.5){\$24}}

\put(6.00,5.5){\makebox(0.25,0.5){*}}
\put(6.25,5.5){\makebox(0.25,0.5){*}}
\put(6.50,5.5){\makebox(0.25,0.5){*}}
\put(6.75,5.5){\makebox(0.25,0.5){*}}

\put(7.00,5.5){\makebox(0.25,0.5){*}}
\put(7.25,5.5){\makebox(0.25,0.5){*}}
\put(7.50,5.5){\makebox(0.25,0.5){*}}
\put(7.75,5.5){\makebox(0.25,0.5){*}}

\put(8.00,5.5){\makebox(0.25,0.5){*}}
\put(8.25,5.5){\makebox(0.25,0.5){*}}
\put(8.50,5.5){\makebox(0.25,0.5){*}}
\put(8.75,5.5){\makebox(0.25,0.5){1}}

\put(9.00,5.5){\makebox(0.25,0.5){*}}
\put(9.25,5.5){\makebox(0.25,0.5){*}}
\put(9.50,5.5){\makebox(0.25,0.5){*}}
\put(9.75,5.5){\makebox(0.25,0.5){0}}

\put(10.00,5.5){\makebox(5,0.5)[l]{-> no action}}

\put(3.05,5.5){\line(0,-1){0.2}}
\put(3.05,5.3){\line(1,0){6.775}}
\put(9.875,5.3){\vector(0,1){0.2}}

\put(3.75,5.5){\line(0,-1){0.4}}
\put(3.75,5.1){\line(1,0){5.125}}
\put(8.875,5.1){\vector(0,1){0.4}}
}

\put(0,4.0){\makebox(2,0.5)[l]{Gen 2:}}
{\ttfamily

\put(2.0,4.0){\framebox(0.7,0.5){\$BC}} % gen 2
\put(2.7,4.0){\framebox(0.7,0.5){\$31}}
\put(3.4,4.0){\framebox(0.7,0.5){\$01}}
\put(4.1,4.0){\framebox(0.7,0.5){\$47}} %

\put(6.00,4.0){\makebox(0.25,0.5){*}}
\put(6.25,4.0){\makebox(0.25,0.5){*}}
\put(6.50,4.0){\makebox(0.25,0.5){*}}
\put(6.75,4.0){\makebox(0.25,0.5){*}}

\put(7.00,4.0){\makebox(0.25,0.5){*}}
\put(7.25,4.0){\makebox(0.25,0.5){*}}
\put(7.50,4.0){\makebox(0.25,0.5){*}}
\put(7.75,4.0){\makebox(0.25,0.5){*}}

\put(8.00,4.0){\makebox(0.25,0.5){*}}
\put(8.25,4.0){\makebox(0.25,0.5){*}}
\put(8.50,4.0){\makebox(0.25,0.5){*}}
\put(8.75,4.0){\makebox(0.25,0.5){*}}

\put(9.00,4.0){\makebox(0.25,0.5){*}}
\put(9.25,4.0){\makebox(0.25,0.5){*}}
\put(9.50,4.0){\makebox(0.25,0.5){0}}
\put(9.75,4.0){\makebox(0.25,0.5){*}}

\put(10.00,4.0){\makebox(5,0.5)[l]{-> action code \$07}}

\put(3.05,4.0){\line(0,-1){0.2}}
\put(3.05,3.8){\line(1,0){6.575}}
\put(9.625,3.8){\vector(0,1){0.2}}

\put(3.75,4.0){\line(0,-1){0.4}}
\put(3.75,3.6){\line(1,0){5.875}}
\put(9.625,3.6){\vector(0,1){0.4}}

\put(4.45,4.5){\line(0,1){0.2}}
\put(4.45,4.7){\line(1,0){8.55}}
\put(13.0,4.7){\vector(0,-1){0.2}}
}

\put(0,2.5){\makebox(2,0.5)[l]{Gen 3:}}
{\ttfamily

\put(2.0,2.5){\framebox(0.7,0.5){\$A7}} % gen 3
\put(2.7,2.5){\framebox(0.7,0.5){\$1C}}
\put(3.4,2.5){\framebox(0.7,0.5){\$F9}} %% gen 4

\put(6.00,2.5){\makebox(0.25,0.5){*}}
\put(6.25,2.5){\makebox(0.25,0.5){*}}
\put(6.50,2.5){\makebox(0.25,0.5){*}}
\put(6.75,2.5){\makebox(0.25,0.5){0}}

\put(7.00,2.5){\makebox(0.25,0.5){*}}
\put(7.25,2.5){\makebox(0.25,0.5){*}}
\put(7.50,2.5){\makebox(0.25,0.5){*}}
\put(7.75,2.5){\makebox(0.25,0.5){*}}

\put(8.00,2.5){\makebox(0.25,0.5){*}}
\put(8.25,2.5){\makebox(0.25,0.5){*}}
\put(8.50,2.5){\makebox(0.25,0.5){*}}
\put(8.75,2.5){\makebox(0.25,0.5){*}}

\put(9.00,2.5){\makebox(0.25,0.5){*}}
\put(9.25,2.5){\makebox(0.25,0.5){*}}
\put(9.50,2.5){\makebox(0.25,0.5){*}}
\put(9.75,2.5){\makebox(0.25,0.5){*}}

\put(10.00,2.5){\makebox(5,0.5)[l]{-> action code \$39}}

\put(3.05,2.5){\line(0,-1){0.2}}
\put(3.05,2.3){\line(1,0){3.825}}
\put(6.875,2.3){\vector(0,1){0.2}}

\put(3.75,3.0){\line(0,1){0.2}}
\put(3.75,3.2){\line(1,0){9.25}}
\put(13.0,3.2){\vector(0,-1){0.2}}
}

\put(0,1.0){\makebox(2,0.5)[l]{Gen 4:}}
{\ttfamily

\put(2.0,1.0){\framebox(0.7,0.5){\$F9}} %% gen 4
\put(2.7,1.0){\framebox(0.7,0.5){\$7B}} %

\put(6.00,1.0){\makebox(0.25,0.5){*}}
\put(6.25,1.0){\makebox(0.25,0.5){*}}
\put(6.50,1.0){\makebox(0.25,0.5){*}}
\put(6.75,1.0){\makebox(0.25,0.5){*}}

\put(7.00,1.0){\makebox(0.25,0.5){*}}
\put(7.25,1.0){\makebox(0.25,0.5){*}}
\put(7.50,1.0){\makebox(0.25,0.5){*}}
\put(7.75,1.0){\makebox(0.25,0.5){*}}

\put(8.00,1.0){\makebox(0.25,0.5){*}}
\put(8.25,1.0){\makebox(0.25,0.5){*}}
\put(8.50,1.0){\makebox(0.25,0.5){*}}
\put(8.75,1.0){\makebox(0.25,0.5){*}}

\put(9.00,1.0){\makebox(0.25,0.5){*}}
\put(9.25,1.0){\makebox(0.25,0.5){*}}
\put(9.50,1.0){\makebox(0.25,0.5){*}}
\put(9.75,1.0){\makebox(0.25,0.5){*}}

\put(10.00,1.0){\makebox(5,0.5)[l]{-> action code \$3B}}

\put(3.05,1.5){\line(0,1){0.2}}
\put(3.05,1.7){\line(1,0){9.95}}
\put(13.0,1.7){\vector(0,-1){0.2}}
}


\end{picture}

\caption{\label{prominterfig}
Schema der promotororientierten Genominterpretation.
}
\end{figure}

Abb.~\ref{prominterfig} zeigt die Arbeitsweise der promotororientierten
Genominterpretation an einem Beispiel, in dem die oben beschriebenen
Mechanismen erl"autert werden. Der Pseudocode f"ur die Decodierung bei der
promotororientierten Genominterpretation ist:

\begin{verbatimcmd}
r = 0
i = 0
WHILE i < genome_length
  WHILE (i < genome_length) AND (\pcodemeta{Bit 7 im i-ten Byte des Genoms gel{\"o}scht})
    i = i + 1
  WEND
  IF i < genome_length
    i = i + 1
    \pcodemeta{Initialisiere Zustandsteil der r-ten Regel mit unspezifizierten Bits}
    WHILE (i < genome_length) AND (\pcodemeta{Bits 6 und 7 im i-ten Byte des Genoms gel{\"o}scht})
      j = \pcodemeta{Bits 0 bis 4 des i-tes Bytes des Genoms}
      IF \pcodemeta{Bit 5 im i-ten Byte des Genoms gesetzt}
        \pcodemeta{Spezifiziere j-tes Bit im Zustandsteil der r-ten Regel als} 1
      ELSE
        \pcodemeta{Spezifiziere j-tes Bit im Zustandsteil der r-ten Regel als} 0
      ENDIF
      i = i + 1
    WEND
    IF ((i < genome_length) AND (\pcodemeta{Bit 6 im i-ten Byte des Genoms gesetzt})
      \pcodemeta{Aktionscode der r-ten Regel} = \pcodemeta{i-tes Byte des Genoms}
      \pcodemeta{L{\"o}sche Bits 6 und 7 im Aktionscode der r-ten Regel}
    ELSE
      \pcodemeta{r-te Regel ist nicht terminiert und hat keinen Aktionscode}
    ENDIF
    r = r + 1
  ENDIF
WEND
\end{verbatimcmd}


\subsubsection{Codierung der rechten Seite}
\label{actioncodedef}

Wie in den beiden vorangehenden Abschnitten dargestellt wurde, wird
bei allen Interpretersystemen die rechte Seite der Regeln durch
acht oder durch sechs Bit breite \introdef{Aktionscodes} codiert. Die Decodierung
dieser Aktionscodes wird durch die 
die Kontrollparameter des genetischen Systems \verb|num_divide|,
\verb|num_flyingseed|, \verb|num_localseed|, \verb|num_mutminus|, \verb|num_mutplus|
und \verb|num_statebit| gesteuert. Diese Kontrollparameter geben jeweils an, wievielen
verschiedenen Aktionscodes die entsprechende Aktion zugeordnet wird.
F"ur die Decodierung werden zun"achst folgende Hilfswerte berechnet:

\medskip
\begin{tabular}{l@{ = }l}
\verb|lnd_divide|,     & \verb|num_divide|, \\
\verb|lnd_flyingseed|, & \verb|lnd_divide| + \verb|num_flyingseed| \\
\verb|lnd_localseed|,  & \verb|lnd_flyingseed| + \verb|num_localseed| \\
\verb|lnd_mutminus|,   & \verb|lnd_localseed| + \verb|num_mutminus| \\
\verb|lnd_mutplus|,    & \verb|lnd_mutminus| + \verb|num_mutplus| \\
\verb|lnd_statebit|,   & \verb|lnd_mutplus| + \verb|num_statebit| \\
\end{tabular}
\medskip

Den Sechs- oder Achtbitcodes $x$ werden nun entsprechend der folgenden
Tabelle Aktionen zugeordnet:

\medskip
\begin{tabular}{l@{ $\rightarrow$ }l}
$0 \leq x <$ \verb|lnd_divide|                        & \verb|divide|, $pos = x \wedge \mathit{divmask}$ \\
$\verb|lnd_divide| \leq x <$ \verb|lnd_flyingseed|    & \verb|flyingseed| \\
$\verb|lnd_flyingseed| \leq x <$ \verb|lnd_localseed| & \verb|localseed| \\
$\verb|lnd_localseed| \leq x <$ \verb|lnd_mutminus|   & \verb|mutminus| \\
$\verb|lnd_mutminus| \leq x <$ \verb|lnd_mutplus|     & \verb|mutplus| \\
$\verb|lnd_mutplus| \leq x <$ \verb|lnd_statebit|     & \verb|statebit|, $num = x \wedge \mathit{sbmask}$ \\
\end{tabular}
\medskip

Der Wert f"ur $\mathit{divmask}$ ist bei allen Modellen fest vorgegeben. Bei
den Modellen mit zweidimensionaler Welt ist $\mathit{divmask} = 7$, in dreidimensionalen Modellen
ist $\mathit{divmask} = 31$. Diese Werte gew"ahrleisten, da"s der Wertebereich des Ausdrucks $x \wedge \mathit{divmask}$
die jeweilige Indexmenge der Nachbarpositionen vollst"andig umfa"st, wenn f"ur \verb|num_divide| ein
Wert gr"o"ser als $\mathit{divmask}$ gew"ahlt wird. Im dreidimensionalen Fall kann der Ausdruck
$x \wedge \mathit{divmask}$ die Werte 18 bis 31 liefern, die keinen g"ultigen Index einer Nachbarposition
f"ur die Tochterzelle darstellen. Eine Zelle, die einen solchen Aktionscode ausf"uhrt, verbraucht ihre
Energie, ohne da"s eine Zellteilung stattfindet.

Ebenfalls fest vorgegeben ist der Wert von $\mathit{sbmask}$. Bei Modellen mit 16 Bit breitem internen Zellzustand
ist $\mathit{sbmask} = 15$, bei 32 Bit breitem internem Zellzustand ist $\mathit{sbmask} = 31$. Durch diese Auswahl
wird sichergestellt, da"s durch den Ausdruck $x \wedge \mathit{sbmask}$ s"amtliche Bits des internen Zustands
angesprochen werden k"onnen, sofern \verb|num_statebit| gr"o"ser als $\mathit{sbmask}$ ist.

Die Decodierung ist nur dann wohldefiniert, wenn die Summe der Kontrollparameter des genetischen Systems
(sie entspricht \verb|lnd_statebit|) 256 bei Modellen mit Achtbitcodes und 64 bei
Modellen mit Sechsbitcodes betr"agt.

\medskip
\noindent\begin{tabularx}{\linewidth}{|c|l|X|} \hline
\multicolumn{3}{|c|}{\bfseries Kontrollparameter der Genominterpretation} \\ \hline
Formel- & Variablenname & Beschreibung \\
buchstabe &              & \\ \hline
- & \verb|num_divide|     & Anzahl Aktionscodes f"ur \verb|divide| \\
- & \verb|num_flyingseed| & Anzahl Aktionscodes f"ur \verb|flyingseed| \\
- & \verb|num_localseed|  & Anzahl Aktionscodes f"ur \verb|localseed| \\
- & \verb|num_mutminus|   & Anzahl Aktionscodes f"ur \verb|mutminus| \\
- & \verb|num_mutplus|    & Anzahl Aktionscodes f"ur \verb|mutplus| \\
- & \verb|num_statebit|   & Anzahl Aktionscodes f"ur \verb|statebit| \\ \hline
\end{tabularx}
\medskip


\subsection{Mutation}
\label{mutationdef}

Mutation ist als zufallsgesteuerte, ungerichtete Anwendung von Mutationsoperatoren
auf ein Genom implementiert. Es gibt vier Typen von Mutationsereignissen,
die Austauschmutation, die Insertion, die Deletion und die Genduplikation.

Von der \introdef{Austauschmutation} existieren zwei Varianten. Bei der byteweisen
Austauschmutation wird der Wert eines Bytes im Genom durch einen Zufallswert
ausgetauscht. Bei einer bitweisen Austauschmutation wird dagegen ein durch Zufall
bestimmtes Bit im betroffenen Byte invertiert.

Eine \introdef{Insertion} wird durch das Einf"ugen von Blocks aus Bytes
mit Zufallswerten in das Genom realisiert. Eine \introdef{Deletion} ist
die Entfernung eines derartigen Blocks. Die Blockgr"o"se ist zwei
Bytes bei Modellen mit blockorientierter Genominterpretation (s.\ \ref{blockinterdef}) , bei
Modellen mit promotororientierter (s.\ \ref{prominterdef}) Genominterpretation werden einzelne Bytes
eingef"ugt bzw.\ entfernt.

Die \introdef{Genduplikation} wurde nur bei den Modellen mit promotororientierter
Genominterpretation implementiert. Bei ihr
wird eine exakte Kopie eines Gens an das Genom angeh"angt.

Zu den Kontrollparametern einer Simulation geh"oren die Basiswerte f"ur die
Austauschrate $M_r$, die Insertionsrate $M_i$ und die Deletionsrate
$M_d$. Bei einigen Modellen kommt die Duplikationsrate $M_{\mathit{dup}}$ hinzu.
Diese Raten werden durch die Variablennamen \verb|m_replacement|, \verb|m_insertion|,
\verb|m_deletion| bzw.\ \verb|m_duplication| bezeichnet.

Der \introdef{Mutationsfaktor} $m_f$ (\verb|m_factor|) steuert das Ausma"s, in dem die Basisraten
durch die Zellaktionen \verb|mut-| und \verb|mut+| "uber den
Mutationsexponenten $\mu$ (\verb|m_counter|) beeinflu"st werden k"onnen. Die effektiven Raten,
die die Wahrscheinlichkeit f"ur das Eintreten eines Mutationsereignisses
pro Position und Zeitschritt angeben, sind:

\begin{equation}
m_r = M_r \cdot {m_f}^{\mu}
\end{equation}
\begin{equation}
m_i = M_i \cdot {m_f}^{\mu}
\end{equation}
\begin{equation}
m_d = M_d \cdot {m_f}^{\mu}
\end{equation}
\begin{equation}
m_{\mathit{dup}} = M_{\mathit{dup}} \cdot {m_f}^{\mu}
\end{equation}

Nach der Mutation eines Genoms wird der Mutationsexponent $\mu$ des
mutierten Genoms stets wieder auf 0 zur"uckgesetzt.

Die algorithmischen Details der verschiedenen Mutationstypen werden
im Folgenden durch Pseudocode verdeutlicht. \pcodemeta{Zufallszahl}
bedeutet dabei eine Flie"skommazahl zwischen 0 und 1. Das Genom
wird durch den Variablennamen \verb|genome| bezeichnet. Der
Pseudocode f"ur die byteweise Austauschmutation ist:

\begin{verbatimcmd}
FOR i = 0 TO genome_length
  IF \pcodemeta{Zufallszahl} < \pcodemeta{\begin{math}m\sb{r}\end{math}}
    genome[i] = \pcodemeta{Zufallswert zwischen 0 und 255}
  ENDIF
NEXT i
\end{verbatimcmd}

Bei der bitweisen Austauschmutation unterscheidet sich nur die Art der
Ver"anderung der durch ein Mutationsereignis betroffenen Bytes:

\begin{verbatimcmd}
FOR i = 0 TO genome_length
  IF \pcodemeta{Zufallszahl} < \pcodemeta{\begin{math}m\sb{r}\end{math}}
    r = \pcodemeta{Zufallswert zwischen 0 und 7}
    flipmask = \pcodemeta{r-tes Bit gesetzt, alle anderen Bits gel{\"o}scht}
    genome[i] = genome[i] XOR flipmask
  ENDIF
NEXT i
\end{verbatimcmd}

Die Arbeitsweise der Insertionsmutation wird folgenderma"sen in Pseudocode beschrieben,
wobei \verb|blocksize| die Gr"o"se eines durch ein Insertionsereignis eingef"ugten
Blocks angibt:

\begin{verbatimcmd}
i = 0
WHILE i <= genome_length
  IF \pcodemeta{Zufallszahl} < \pcodemeta{\begin{math}m\sb{i}\end{math}}
    FOR j = genome_length - 1 DOWNTO i
      genome[j + blocksize] = genome[j]
    NEXT j
    FOR j = i TO blocksize - 1
      genome[j] = \pcodemeta{Zufallswert zwischen 0 und 255}
    NEXT j
    genome_length = genome_length + blocksize
    i = i + blocksize
  ENDIF
  i = i + 1
WEND
\end{verbatimcmd}

Die Deletion wirkt in "ahnlicher Weise:

\begin{verbatimcmd}
i = 0
WHILE i + blocksize < genome_length
  IF \pcodemeta{Zufallszahl} < \pcodemeta{\begin{math}m\sb{d}\end{math}}
    FOR j = i TO genome_length - blocksize
      genome[j] = genome[j + blocksize]
    NEXT j
    genome_length = genome_length - blocksize
    i = i - 1
  ENDIF
  i = i + 1
WEND
\end{verbatimcmd}

Bei der Genduplikation werden nicht Zeichen oder Bl"ocke des Genoms,
sondern individuelle Gene dupliziert:

\begin{verbatimcmd}
FOR i = 0 TO num_genes - 1
  IF \pcodemeta{Zufallszahl} < \pcodemeta{\begin{math}m\sb{\mathit{dup}}\end{math}}
    l = \pcodemeta{L{\"a}nge des i-ten Gens}
    FOR j = 0 TO l - 1
      genome[i + genome_length] = genome[i + j]
    NEXT j
    genome_length = genome_length + l
  ENDIF
NEXT i
\end{verbatimcmd}

Bei der Simulation des gesamten Mutationsvorgangs werden die einzelnen
Mutationstypen nacheinander durchgef"uhrt, nachdem die effektiven Mutationsraten
berechnet wurden.
Soweit ein Mutationstyp in einem konkreten Modell nicht enthalten ist,
entf"allt nat"urlich der entsprechende Schritt. Insgesamt ergibt sich folgender
Ablauf:

\begin{verbatimcmd}
\pcodemeta{Berechne \begin{math}m\sb{r}, m\sb{i}, m\sb{d}\end{math} und \begin{math}m\sb{\mathit{dup}}\end{math}}
\pcodemeta{F{\"u}hre Austauschmutationen durch}
\pcodemeta{F{\"u}hre Insertionen durch}
\pcodemeta{F{\"u}hre Deletionen durch}
\pcodemeta{F{\"u}hre Genduplikationen durch}
\end{verbatimcmd}

Dies bedeutet, da"s die sp"ateren Mutationsoperationen jeweils an
Genomen erfolgen, die bereits den vorangegangenen Mutationsoperatoren
unterworfen wurden. Unter anderem hat dies zur Folge, da"s bei gleicher
Insertions- und Deletionsrate der Erwartungswert f"ur die L"ange eines
Genoms nach Insertion und Deletion nicht seiner urspr"unglichen L"ange
entspricht.

\medskip
\noindent\begin{tabularx}{\linewidth}{|c|l|X|} \hline
\multicolumn{3}{|c|}{\bfseries Kontrollparameter der Mutation} \\ \hline
Formel- & Variablenname & Beschreibung \\
buchstabe &              & \\ \hline
$M_r$     & \verb|m_replacement| & Basisrate f"ur Austauschmutationen \\
$M_i$     & \verb|m_insertion|   & Basisrate f"ur Insertionen \\
$M_d$     & \verb|m_deletion|    & Basisrate f"ur Deletionen \\
$M_{\mathit{dup}}$ & \verb|m_duplication| & Basisrate f"ur Genuplikationen \\
$m_f$     & \verb|m_factor|      & Faktor zur Modifikation der Basisraten pro \verb|mut+|- bzw.\ \verb|mut+|-Aktion \\ \hline
\end{tabularx}
\medskip


% \subsection{Zusammenfassung der Kontrollparameter}
% 
% \medskip
% \noindent\begin{tabularx}{\linewidth}{|c|l|X|} \hline
% Formel- & Bezeichner & Beschreibung \\
% buchstabe & & \\ \hline
% $p$ & \verb|psize| & Populationsgr"o"se \\
% $p_s$ & \verb|psize_init| & Populationsgr"o"se beim Simulationsstart \\
% $l_s$ & \verb|glen_init| & L"ange der am Simulationsstart zuf"allig generierten Genome \\
% $s$ & \verb|srate| & Selektionsrate \\
% $w$ & \verb|world_width| & Breite der Welt \\
% $h$ & \verb|world_height| & H"ohe der Welt \\
% $D$ & \verb|p_random_death| & Basiswahrscheinlichkeit f"ur Absterben \\
% $d_n$ & \verb|rdeath_f_numcells| & Modifikator f"ur Absterbewahrscheinlichkeit in Abh"angigkeit von Anzahl Zellen \\
% $d_e$ & \verb|rdeath_f_energy| & Modifikator f"ur Absterbewahrscheinlichkeit in Abh"angigkeit von Gesamtenergie \\
% $d_l$ & \verb|leanover_penalty| & Modifikator f"ur Absterbewahrscheinlichkeit in Abh"angigkeit vom "Uberhang \\
% $M_r$ & \verb|m_replacement| & Basiswert der Austauschrate \\
% $M_i$ & \verb|m_insertion| & Basiswert der Insertionsrate \\
% $M_d$ & \verb|m_deletion| & Basiswert der Deletionsrate \\
% $M_{\mathit{dup}}$ & \verb|m_duplication| & Basiswert der Duplikationsrate \\
% $m_f$ & \verb|m_factor| & Modifikationsfaktor f"ur Mutationsraten \\
%  & \verb|num_divide| & Anzahl Codewerte f"ur \verb|divide| \\
%  & \verb|num_flyingseed| & Anzahl Codewerte f"ur \verb|flyingseed| \\
%  & \verb|num_localseed| & Anzahl Codewerte f"ur \verb|localseed| \\
%  & \verb|num_mutminus| & Anzahl Codewerte f"ur \verb|mut-| \\
%  & \verb|num_mutplus| & Anzahl Codewerte f"ur \verb|mut+| \\
%  & \verb|num_statebit| & Anzahl Codewerte f"ur \verb|statebit| \\
%  & \verb|random_seed| & {\slshape seed} zur Initialisierung des Zufallsgenerators \\ \hline
% \end{tabularx}
% \medskip


\subsection{Technische Ausf"uhrung}
\label{technicaldef}

\subsubsection{Organisation der Simulationsl"aufe}

Alle Simulationsexperimente, die in dieser Arbeit besprochen werden, haben einen eindeutigen
Namen, der sich aus Buchstaben und Ziffern zusammensetzt. Zu jedem Experiment geh"ort
eine Parameterdatei, deren Name dem Experimentnamen entspricht. In dieser Parameterdatei
sind s"amtliche Kontrollparameter f"ur das Simulationsexperiment spezifiziert.
% Im Anhang \ref{paramfile-appendix}
% sind die Parameterdateien zu allen in dieser Arbeit angesprochenen Experimenten abgedruckt.
Da ein Simulationsexperiment durch den Satz der verwendeten Kontrollparameter eindeutig identifiziert
werden kann, ist somit die exakte Reproduktion der hier diskutierten Ergebnisse m"oglich.

Die Namen der Simulationsl"aufe entstanden im Verlauf der Entwicklung der Simulationsprogramme
und -experimente und haben keinen durchg"angigen systematischen Aufbau. Sie werden im Text dort
verwendet, wo die eindeutige Bezeichnung eines Laufs erforderlich ist.


\subsubsection{Graphische Darstellung der Pflanzen}

Pflanzenzellen werden als quadratische Boxen dargestellt. Ihr Energiezustand wird dabei durch die
Farbgebung angezeigt, dunkle Farben zeigen energiereiche Zellen an. Wird nur eine Pflanze dargestellt,
werden f"ur energiereiche Zellen schwarz gef"ullte, f"ur energielose Zellen wei"se oder hellgraue
Boxen verwendet. Diese Darstellung wurde bereits in Abb.\ \ref{lnd1-graphgenom} verwendet.
Bei der Darstellung mehrerer Pflanzen werden die Zellen verschiedener Pflanzen in
unterschiedlichen Farben dargestellt, damit die individuellen Pflanzen voneinander unterschieden
werden k"onnen. Die dunkleren Farbt"one zeigen dabei jeweils energiereiche Zellen an.
Die farbigen Abbildungen sind aus drucktechnischen Gr"unden im Anhang \ref{colorplates}
zusammengefa"st.


\chapter{Arbeitsmaterial}

\section{Hard- und Firmware}
\label{material-hardware}

Zur Entwicklung der Programme und zur Durchf"uhrung der Simulationen wurden folgende Systeme eingesetzt:

\bigskip
\begin{tabular}{lll}
{\bfseries Hardware} & {\bfseries Betriebssystem} & {\bfseries Hersteller} \\
Atari TT030 & TOS 3.06 / MiNT 1.12 & Atari Corp.  (ATC) \\
Alpha AXP 3000, Modell 800 & OSF1 3.0 & Digital Equipment Corp. \\
			   &          & (DEC) \\
Indigo XS   & IRIX 4.0.5F & Silicon Graphics, Inc. \\
	    &             & (SGI) \\
\end{tabular}


\section{Software}
\label{material-software}

\subsection{Programmentwicklung}

S"amtliche im Rahmen dieser Arbeit erstellte Software wurde in der Programmiersprache
ANSI C \cite{Harbison91} erstellt. Als Compiler diente GNU C (\verb|gcc|), der C-Compiler des
GNU-Projekts, in der Version 2.7.0 und in fr"uheren Versionen. Zur Entwicklung wurden weiterhin
der GNU Debugger (\verb|gdb|, Version 4.13 und fr"uhere) und GNU Make (\verb|make|, Version 3.71
und fr"uhere) eingesetzt.
Die Entwicklungsinstrumente wurden, nebst zugeh"origer Dokumentation, in elektronischer Form "uber
{\slshape ftp} bezogen.


\subsection{Datenanalyse und Visualisierung}

Zur Visualisierung numerischer Ergebnisse aus Simulationsl"aufen wurde
das Programm \verb|gnuplot| von den Autoren Thomas Williams und Colin Kelley
verwendet. \verb|gnuplot| kann von vielen {\slshape ftp}-Servern
bezogen werden (s.\ Anhang \ref{electronicsources}). Es dient zur interaktiven Darstellung numerischer Daten
in unterschiedlichen Formen sowie zur Anfertigung entsprechender
Computergraphiken.
Zur Stammbaumanalyse wurde das Programmpaket PHYLIP \cite{PHYLIP} verwendet.

Zur weitergehenden graphischen Darstellung von Daten aus Simulationsl"aufen
wurde das Programm \verb|proplot| erstellt, zu dem sich weitere Informationen
im Anhang \ref{technicalstuff} finden.



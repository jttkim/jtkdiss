\chapter[LindEvol-GA]{LindEvol-GA: Modell auf der Basis eines genetischen Algorithmus}
\label{lindevol-ga}


\section{Definition von LindEvol-GA}
\label{lndga-def}

\subsection{Modellkomponenten}

In LindEvol-GA werden die folgenden in (\ref{modeldef}) definierten Komponenten
eingesetzt:

\begin{itemize}
\item Topologie: Zweidimensionales Gitter (\protect\ref{topo2d}).
\item Zellparameter: Nur Energieparameter (\protect\ref{celldef}). Der Zustand eine
    Zelle ist also durch die sie umgebende lokale Struktur vollst"andig bestimmt
    (Gl.\ \ref{stateq-nointernal}).
\item Zellaktionen: \verb|divide|, \verb|mut-| und \verb|mut+| (\protect\ref{cellactiondef}).
    Aktionen zur Samenproduktion werden nicht ben"otigt, weil die Vermehrung durch den
    Selektionsmechanismus des genetischen Algorithmus und nicht unmittelbar
    durch Pflanzenaktivit"at erfolgt.
    Die Aktion \verb|divide| scheitert in LindEvol-GA immer, wenn die Zielposition f"ur die
    Tochterzelle bereits besetzt ist.
\item Genominterpretation: LindEvol-GA arbeitet mit blockorientierter
    Genominterpretation (\protect\ref{blockinterdef}).
\item Mutation: In LindEvol-GA kommen byteweise Austauschmutationen sowie
    Insertionen und Deletionen von Bl"ocken aus zwei Bytes vor (\protect\ref{mutationdef}).
\end{itemize}


\subsection{Vegetationsperiode}

Eine Generation von Pflanzen w"achst jeweils eine \introdef{Vegetationsperiode} lang
gemeinsam in der Welt. Eine solche Vegetationsperiode besteht aus
einer Anzahl von Tagen (s.\ \ref{timedef}).
Die Anzahl Tage pro Vegetationsperiode wird durch den Kontrollpamamter $V$ (\verb|num_days|)
bestimmt. Zu Beginn der
Vegetationsperiode bestehen die Pflanzen aus einzelnen, energielosen Zellen,
die mit randomisierter Reihenfolge "aquidistant am Boden der Welt plaziert werden.
Nach dem Ablauf der Vegetationsperiode wird jedem Genom die Anzahl der energiereichen
Zellen der entsprechenden Pflanze (s.\ \ref{plantdef}) als Fitnesswert
zugewiesen.


\subsection{Der genetische Algorithmus}

Der in LindEvol-GA verwendete genetische Algorithmus besteht im Kern aus einer
Schleife, in der nacheinander Fitnessbewertung, Selektion (Tod und Reproduktion)
und Mutation stattfinden. Abb.\ \ref{lndga-flowchart} verdeutlicht die Abl"aufe.
Zu Beginn einer Simulation wird die eine Population
von Genomen zufallsgesteuert generiert. Die Gr"o"se der Population wird dabei
durch den Kontrollparameter \verb|psize| festgelegt, der Kontrollparameter
\verb|glen_init| bestimmt die L"ange der generierten Genome.

Die Ermittlung der Fitnesswerte erfolgt, wie oben beschrieben, durch die Simulation
einer Vegetationsperiode.
Beim Selektionsschritt werden die Genome in der Population
zun"achst nach ihren Fitnesswerten absteigend geordnet. Dann werden die Genome mit 
geringen Fitnesswerten, die im hinteren Bereich dieser Liste stehen, durch zuf"allig aus 
dem "uberlebenden Teil der Population ausgew"ahlte Genome ersetzt. Die Selektionsrate 
\verb|srate| ist der Kontrollparameter, der dabei die Gr"o"se des zu ersetzenden Anteils
der Population angibt. Dieser Selektionsvorgang simuliert eine rein vegetative (asexuelle)
Vermehrung.

Nach dem Selektionsschritt werden alle Genome mutiert. Mutation findet in LindEvol-GA also
nur nach jeder Vegetationsperiode, nicht nach jedem Tag statt. Aus diesem Grund haben
nur die \verb|mut-|- und \verb|mut+|-Aktionen, die am letzten Tag einer Vegetationsperiode
ausgef"uhrt werden, einen Einflu"s auf die effektiven Mutationsraten.


\begin{figure}

\unitlength0.5cm
\begin{picture}(16,17)
\thicklines

\put(8,16){\framebox(14,1){Initialisierung der Population}}
\put(15,16){\line(0,-1){2}}

\put(8,11){\framebox(14,3){
\parbox{14cm}{
\centerline{Ermittlung der Fitnesswerte}
\centerline{der Genome durch Simulation}
\centerline{der Vegetationsperiode}}}}
\put(15,11){\vector(0,-1){1}}

\put(8,5){\framebox(14,5){
\parbox{14cm}{
\centerline{Selektion: $p*s$ Genome mit den}
\centerline{niedrigsten Fitnesswerten werden}
\centerline{entfernt und durch Kopien von}
\centerline{Genomen des "uberlebenden Teils}
\centerline{der Population ersetzt}}}}
\put(15,5){\vector(0,-1){1}}

\put(8,3){\framebox(14,1){Mutation der Genome}}
\put(15,3){\vector(0,-1){1}}

\put(8,1){\framebox(14,1){Inkrementieren der Generation}}
\put(15,1){\line(0,-1){0.5}}
\put(19,0.5){\oval(8,1)[b]}
\put(23,14.5){\line(0,-1){14}}
\put(19,14.5){\oval(8,1)[t]}
\put(15,14.5){\vector(0,-1){0.5}}
\end{picture}

\caption{\label{lndga-flowchart}
Flu"sdiagramm f"ur LindEvol-GA
}
\end{figure}

% Pflanzen ueberleben nur eine Generation, nur Genome werden
% zwischen Generationen vererbt.

% 

Das Konzept von LindEvol-GA entspricht in weiten Teilen den Entw"urfen von
Wilson \cite{Wilson89}. Anders als bei einem traditionellen genetischen Algorithmus,
bei dem der Fitnesswert eines Genoms nicht durch die restliche Population
beeinflu"st wird, haben bei LindEvol-GA r"aumliche und zeitliche Interaktionen
zwischen den Pflanzen erhebliche Auswirkungen auf die Fitness der Genome.
Daher findet bei LindEvol-GA keine Adaptation an eine von au"sen definierte
Fitnessfunktion, sondern vielmehr eine intrinsische Adaptation \cite{Packard89} statt.
Tabelle \ref{lndga-controlparams} zeigt eine Zusammenfassung der Kontrollparameter von
LindEvol-GA.

% \subsection{Kontrollparameter von LindEvol-GA}
% \label{lndga-controlparams}

\begin{table}[tb]
\noindent\begin{tabularx}{\linewidth}{|c|l|X|} \hline
Formel- & Bezeichner & Beschreibung \\
buchstabe & & \\ \hline
$p$ & \verb|psize| & Populationsgr"o"se \\
$l_s$ & \verb|glen_init| & L"ange der am Simulationsstart zuf"allig generierten Genome \\
$s$ & \verb|srate| & Selektionsrate \\
$M_r$ & \verb|m_replacement| & Basiswert der Austauschrate \\
$M_i$ & \verb|m_insertion| & Basiswert der Insertionsrate \\
$M_d$ & \verb|m_deletion| & Basiswert der Deletionsrate \\
$m_f$ & \verb|m_factor| & Modifikationsfaktor f"ur Mutationsraten \\
$w$ & \verb|world_width| & Breite der Welt \\
$h$ & \verb|world_height| & H"ohe der Welt \\
$V$ & \verb|num_days| & Anzahl Tage pro Vegetationsperiode \\
 & \verb|num_divide| & Anzahl Codewerte f"ur \verb|divide| \\
 & \verb|num_mutminus| & Anzahl Codewerte f"ur \verb|mut-| \\
 & \verb|num_mutplus| & Anzahl Codewerte f"ur \verb|mut+| \\
 & \verb|random_seed| & {\slshape seed} zur Initialisierung des Zufallsgenerators \\ \hline
\end{tabularx}

\caption{\label{lndga-controlparams}
Tabelle der Kontrollparameter von LindEvol-GA.
}
\end{table}


Der folgende Pseudocode verdeutlicht die Abl"aufe bei einer LindEvol-GA--Simulation:

\begin{verbatimcmd}
\pcodemeta{Generiere zufallsgesteuert eine Anfangspopulation}
WHILE \pcodemeta{Abbruchkriterien nicht erf{\"u}llt}
  \pcodemeta{Erzeuge f{\"u}r jedes Genom eine einzellige Pflanze}
  \pcodemeta{Plaziere Pflanzen in randomisierter Reihenfolge {\"a}quidistant am Boden der Welt}
  FOR d = 0 TO num_days - 1
    \pcodemeta{Simuliere Lichtbestrahlung}
    \pcodemeta{Erzeuge randomisierte Abarbeitungsreihenfolge \begin{math}r\sb{0}, r\sb{1}, \ldots, r\sb{p - 1}\end{math}  der Pflanzen}
    FOR i = 0 TO psize - 1
      \pcodemeta{Simuliere Wachstum der Pflanze \begin{math}r\sb{i}\end{math}}
    NEXT i
  NEXT d
  FOR i = 0 TO psize
    \pcodemeta{Fitness \begin{math}f\sb{i}\end{math} des i-ten Genoms} = \pcodemeta{Anzahl energiereicher Zellen der i-ten Pflanze}
  NEXT i
  \pcodemeta{Sortiere Genome nach absteigenden Fitnesswerten}
  FOR i = psize - 1 DOWNTO psize - psize * srate
    q = \pcodemeta{Ganzzahlige Zufallszahl zwischen 0 und} psize * srate - 1
    \pcodemeta{Ersetze i-tes Genom durch eine Kopie des q-ten Genoms}
  NEXT i
  FOR i = 0 TO psize - 1
    \pcodemeta{Mutiere i-tes Genom}
  NEXT i
WEND
\end{verbatimcmd}


\subsection{Kontrollsimulationen mit randomisierten Fitnesswerten}
\label{lndga-neutral}

Die in (\ref{modellansatz}) beschriebenen Ansatzpunkte zur Modellierung der Evolution
morphogenetischer Systeme sind bei LindEvol-GA ausschlie"slich in der Fitnessbewertung
durch die Simulation einer Vegetationsperiode enthalten. Die Verfahren zur Selektion
und Mutation sind dagegen Standardkomponenten der genetischen Algorithmen. Somit ist
k"onnen Kontrollmodelle konstruiert werden, bei denen andere Fitnessfunktionen
verwendet werden. Anhand solcher Kontrollen kann festgestellt werden, ob ein in
LindEvol-GA beobachtetes Ph"anomen auf die komplexe Form der Fitnessberechnung
zur"uckgeht, oder ob es sich um eine generelle Eigenschaft des verwendeten
genetischen Algorithmus handelt.

Zur Durchf"uhrung solcher Kontrollen wurde das LindEvol-GA--Simulationsprogramm
so ver"andert, da"s nach der Simulation einer Vegetationsperiode Zufallszahlen
als Fitnesswerte der Genome verwendet werden. Wachstum und Gesamtenergie einer
Pflanze bleiben also ohne jede Auswirkung auf die Fitness ihres Genoms. Alle anderen
Komponenten der Simulation blieben unver"andert. Dieses Kontrollmodell mit
randomisierten Fitnesswerten simuliert eine vollst"andig neutrale Evolution,
weil jede Mutation des Genoms keinerlei Folgen f"ur den Fitnesswert des
betroffenen Genoms hat. Daher wird dieses Modell auch kurz als \introdef{neutrale
Kontrolle} bezeichnet.


% \section{Vergleich mit anderen Simulationsmodellen}

\section{Mutationsfolgenabsch"atzung}
\label{mutconsequences}

In LindEvol-GA werden beim Selektionsschritt Kopien von Genomen aus dem "uberlebenden
Teil der Population erzeugt. Diese werden im anschlie"senden Mutationsschritt in ungerichteter
Weise ver"andert. Somit haben sowohl die Selektionsrate als auch die Mutationsraten ihren
Einflu"s auf die Anzahl der unver"anderten Kopien eines Genoms, das beim Selektionsschritt
"uberlebt hat.

Der Erwartungswert dieser Anzahl gibt eine Obergrenze f"ur die Geschwindigkeit an, mit der
sich der Bestand eines Genotyps in der Population ausbreiten kann. Er wird daher als
\introdef{Ausbreitungszahl} bezeichnet. Nur wenn die Ausbreitungszahl gr"o"ser als 1 ist,
kann sich ein Genotyp stabil in der Population etablieren. Eine Ausbreitungszahl, die kleiner
als 1 ist, zeigt an, da"s der Genotyp sich, auch wenn er "uberlegene Fitnesswerte aufweist,
nicht in der Population ausbreiten kann, weil beim Selektionsschritt weniger Kopien des
entsprechenden Genotyps erzeugt werden, als beim Mutationsschritt ver"andert werden.
Im Folgenden wird eine Formel f"ur die Ausbreitungszahl abgeleitet.

Von einem Genom, das beim Selektionsschritt eliminiert wird, existieren in der Population
der n"achsten Generation keine Kopien. "Uberlebt ein Genom dagegen bei der Selektion,
verbleibt das Genom in der Population. Zus"atzlich besteht die M"oglichkeit, da"s Kopien
des Genoms erzeugt werden. Der Erwartungswert f"ur die Anzahl der erzeugten Kopien ist
$s/(1-s)$, dabei ist $s$ die Selektionsrate. Der Erwartungswert f"ur die Gesamtzahl von
Kopien eines Genoms, welches beim Selektionsschritt "uberlebt hat, betr"agt somit:

\begin{equation}
\label{numoffspring-eq}
N(s) = 1 + \frac{s}{1-s} = \frac{1}{1-s}
\end{equation}

Sei nun $l$ die L"ange des Genoms, dann existieren $l$ verschiedene Stellen, an denen eine
Austauschmutation erfolgen kann, Insertionen k"onnen an $l+1$ verschiedenen Stellen und
Deletionen an $l-1$ verschiedenen Stellen stattfinden. Die Wahrscheinlichkeit, da"s das Genom
beim Mutationsschritt von keiner Mutation betroffen wird, betr"agt also:

\begin{equation}
P_u(m_r, m_i, m_d, l) = (1-m_r)^l \cdot (1-m_i)^{l+1} \cdot (1-m_d)^{l-1}
\end{equation}

Wenn Austausch- Insertions- und Deletionsrate den gleichen Wert $m_r = m_i = m_d = m$ haben,
kann diese Formel vereinfacht werden. Es gilt dann:

\begin{equation}
\label{unchanged-eq}
P_u(m, l) = (1-m)^{3l}
\end{equation}

F"ur die Ausbreitungsrate $A$ gilt nun: $A = N \cdot P_u$, wenn alle Mutationsraten den gleichen
Wert haben, erh"alt man mit den Gleichungen \ref{numoffspring-eq} und \ref{unchanged-eq}:

\begin{equation}
\label{genomespreadm-eq}
A(m, s, l) = \frac{(1-m)^{3l}}{1-s}
\end{equation}

In L"aufen ohne Insertion und Deletion ($m_i = m_d = 0$) gilt dagegen $P_u(m_r, l) = (1-m)^l$
und f"ur die Ausbreitungszahl erh"alt man:

\begin{equation}
\label{genomespreadr-eq}
A(m_r, s, l) = \frac{(1-m_r)^l}{1-s}
\end{equation}

F"ur die Evolutionsprozesse auf ph"anotypischer Ebene ist in LindEvol-GA jedoch die Ausbreitung
von Genotypen nicht so interessant wie die Ausbreitung von Entwicklungsprogrammen.
Genome k"onnen in LindEvol-GA eine gro"se Zahl von Genen enthalten, die bei der Entwicklung
der Pflanze niemals aktiviert werden. Diese Gene sind f"ur das im Genom codierte
Entwicklungsprogramm irrelevant, sie k"onnen wegfallen oder auf vielerlei Art variiert
werden, ohne da"s das Entwicklungsprogramm dadurch ver"andert wird. Die
\introdef{Ausbreitungszahl eines Entwicklungsprogramms} kann auf der Grundlage der Wahrscheinlichkeit daf"ur,
da"s kein aktives Gen des Entwicklungsprogramms von einer Mutation betroffen wird,
abgesch"atzt werden. Ein Gen, das bei der blockorientierten Genominterpretation stets aus
zwei Bytes besteht, kann ver"andert werden durch:

\begin{itemize}
\item eine Austauschmutation eines seiner beiden Bytes,
\item eine Insertion zwischen seinen beiden Bytes,
\item eine Deletion, wobei drei verschiedene M"oglichkeiten bestehen, es kann entweder
das gesamte Gen (Deletion bei Position 0 bez"uglich des Gens), sein erstes Byte
(Deletion bei Position -1) oder sein zweites Byte (Deletion bei Position 1) entfernt
werden.
\end{itemize}

Sei $r$ die Anzahl der aktiven Gene des Entwicklungsprogramms.
Die Wahrscheinlichkeit, da"s kein an der Codierung des Entwicklungsprogramms beteiligtes Gen
durch eine Mutation der o.g.\ Arten betroffen wird, kann abgesch"atzt werden durch:

\begin{equation}
\label{prgunchanged-eq}
P_E(m_r, m_i, m_d, r) \approx (1-m_r)^{2r} \cdot (1-m_i)^{r} \cdot (1-m_d)^{3r}
\end{equation}

Hierbei handelt es sich um eine Absch"atzung, denn verschiedene Effekte bleiben unber"ucksichtigt.
Unter anderem ist eine Deletion an Position -1 bez"uglich des ersten Gens im Genom nicht
m"oglich, ebenso wie eine Deletion an Position +1 bez"uglich des letzten Gens. Weiterhin
ist die Gesamtanzahl der Deletionsm"oglichkeiten, bei denen das codierte Entwicklungsprogramm
ver"andert wird, kleiner als $3l$, wenn mehrere Gene des Programms unmittelbar hintereinander
liegen; in diesem Fall stimmt die Deletion an Position +1 bez"uglich des Gens $n$ "uberein
mit der Deletion an Position -1 bez"uglich des Gens $n+1$. Die aktiven Gene der Entwicklungsprogramme
sind in den Genomen, die im Ergebnisteil diskutiert werden, jedoch nicht in derartigen dichten
Clustern angeordnet, daher ist dieser Effekt in der Praxis unerheblich.

Falls $m_r = m_i = m_d = m$ gilt, vereinfacht sich diese Formel zu:

\begin{equation}
\label{prgunchangedm-eq}
P_E(m, r) = (1-m)^{6r} = P_u(m, 2r)
\end{equation}

Die Ausbreitungszahl eines Entwicklungsprogramms aus $r$ aktiven Genen kann somit abgesch"atzt werden durch:

\begin{equation}
\label{programspreadm-eq}
A_E(m, s, r) \approx \frac{(1-m)^{6r}}{1-s}
\end{equation}

Beim Fehlen von Insertion und Deletion ergibt sich entsprechend:

\begin{equation}
\label{programspreadr-eq}
A_E(m_r, s, r) \approx \frac{(1-m_r)^{2r}}{1-s}
\end{equation}

Auch hier mu"s der Absch"atzungscharakter betont werden. Zum einen wurde 
vernachl"assigt, da"s ein Gen, welches von einer Mutation betroffen wird, mit einer kleinen
Wahrscheinlichkeit dabei nicht ver"andert wird. Zum anderen wurde auch die M"oglichkeit, da"s
ein Wachstumsprogramm durch Mutationen im inaktiven Bereich des Genoms ver"andert werden
kann, au"ser acht gelassen. Die Wahrscheinlichkeiten f"ur diese Prozesse sind jedoch klein
im Vergleich zu den Wahrscheinlichkeiten daf"ur, da"s durch die bei der Ableitung ber"ucksichtigten
Mutationsvorg"ange "Anderungen des Entwicklungsprogramms zustandekommen.
Nur bei sehr langen Genomen k"onnen Mutationen im inaktiven Bereich eine signifikante Rolle
spielen.

Auf der Grundlage der hier hergeleiteten Gleichungen f"ur die Ausbreitungszahl eines Genoms (bzw.\ eines
Entwicklungsprogramms) kann eine Absch"atzung der Obergrenze der Mutationsraten, bis zu der die Ausbreitung
eines Genoms fester L"ange bei gegebener Selektionsrate m"oglich ist, vorgenommen werden.
Entsprechend der Definition der Ausbreitungszahl mu"s, damit ein Genom sich ausbreiten
kann, seine Ausbreitungszahl gr"o"ser als 1.0 sein.
Die Mutationsrate, bei der ein Genom fester L"ange $l$ bei gegebener Selektionsrate $s$ gerade eine Ausbreitungszahl
von 1.0 hat, stellt die \introdef{kritische Fehlergrenze} $t(s,l)$ ({\slshape critical error threshold}) dieses Genoms dar
(vgl.\ \cite{Maynard89}). Es gilt also $A(t(s,l), s, l) = 1$. Wenn $m_r = m_i = m_d$ gilt, kann
Gl.\ \ref{genomespreadm-eq} hier eingesetzt werden.
Nach Aufl"osung nach $t(s,l)$ erh"alt man f"ur die kritische Fehlergrenze:

\begin{equation}
\label{errthresholdm-eq}
t(s,l) = 1-\sqrt[3l]{1-s}
\end{equation}

F"ur Simulationsl"aufe, in denen nur Austauschmutationen vorkommen, erh"alt man unter Verwendung von Gl.\ \ref{genomespreadr-eq}
auf analogem Wege:

\begin{equation}
\label{errthresholdr-eq}
t(s,l) = 1-\sqrt[l]{1-s}
\end{equation}

In entsprechender Weise kann auch die kritische Fehlergrenze eines Entwicklungsprogramms der L"ange $r$ im Falle $m_r = m_i = m_d$
errechnet werden:

\begin{equation}
\label{errthresholdprgm-eq}
t_E(s,r) = 1-\sqrt[6r]{1-s}
\end{equation}

und, falls $m_i = m_d = 0$ gilt:

\begin{equation}
\label{errthresholdprgr-eq}
t_E(s,r) = 1-\sqrt[2r]{1-s}
\end{equation}


\section{Ergebnisse und Diskussion}
\label{lndga-results}

Der Ergebnisteil ist in drei Abschnitte unterteilt. Zun"achst werden in (\ref{lndga-individualsims}) sechs
Simulationsl"aufe detailliert beschrieben und diskutiert. Damit wird ein Eindruck von den Pflanzenformen,
die in LindEvol-GA evolvieren, vermittelt, ein "Uberblick "uber die in LindEvol-GA auftretenden
evolution"aren Prozesse gegeben, und schlie"slich wird die Bedeutung der Mutationsraten und der Selektionsrate
f"ur den Evolutionsvorgang schlaglichtartig beleuchtet.

Im zweiten Abschnitt (\ref{lndga-parameterscan}) wird das Zusammenspiel von Mutation und Selektion systematisch anhand globaler,
einen Simulationslauf charakterisierender Me"sgr"o"sen untersucht. Der Schwerpunkt bei dieser Analyse ist
die Entstehung morphologischer und taxonomischer Komplexit"at.

Der dritte Abschnitt (\ref{lndga-phylogeny}) besch"aftigt sich mit den M"oglichkeiten zur Verwendung von LindEvol-GA als
System zum Testen der Leistungsf"ahigkeit von Stammbaumrekonstruktionsmethoden. Hier werden verschiedene
Experimente beschrieben und diskutiert, mit denen die Auswirkungen verschiedener Evolutionsprozesse,
die in LindEvol-GA auftreten, auf die Stammbaumrekonstruktion untersucht wurden.


\subsection{Einzelsimulationen}
\label{lndga-individualsims}

Zur Beleuchtung der evolution"aren Prozesse in LindEvol-GA und ihrer Abh"angigkeit von den
Mutationsraten und der Selektionsrate wurde ein Satz von L"aufen durchgef"uhrt. Bei den Kontrollparametern
der einzelnen L"aufe unterscheiden sich nur Selektionsrate und Mutationsraten, die anderen
Kontrollparameter wurden konstant auf folgende Werte gesetzt:

\medskip
\begin{tabular}{ll}
Populationsgr"o"se: & 50 \\
Mutationsfaktor: & 2.0 \\
Breite der Welt: & 150 \\
H"ohe der Welt: & 30  \\
Genoml"ange der Startpopulation: & 20 Gene (40 Bytes) \\
L"ange einer Vegetationsperiode: & 30 Tage \\
Anzahl Generationen:             & 5000 \\
\verb|num_divide| & 224 \\
\verb|num_mutminus| & 16 \\
\verb|num_mutplus| & 16 \\
\verb|random_seed| & 12345 \\
\end{tabular}
\medskip

Die Werte f"ur die Populationsgr"o"se und die Ausdehnung der Welt erlauben einerseits die Ausbildung
interessanter morphologischer und phylogenetischer Strukturen. Andererseits sind sie so klein, da"s
die Durchf"uhrung von Serien mit vielen L"aufen und von aufwendigen Analysen wie der Distanzverteilungsanalyse
und der Stammbaumrekonstruktion mit vertretbarem Rechenzeitaufwand m"oglich sind.

Durch den Mutationsfaktor ist die Halbierung der Mutationsraten durch \verb|mut-| und ihre
Verdopplung durch \verb|mut+| m"oglich, was eine signifikante "Anderung bedeutet. Erheblich
gr"o"sere Werte f"ur den Mutationsfaktor sind problematisch, weil dann bei extremen Werten
des Mutationsexponenten der Wertebereich der Flie"skommaarithmetik bei der Berechnung des
effektiven Faktors zur Modifikation der Mutationsraten "uber- bzw.\ unterschritten werden kann.

Alle drei Mutationsraten (Austausch-, Insertions- und Deletionsrate) wurden stets auf den
gleichen Wert gesetzt, wobei die Werte 0.1 (hohe Mutationsraten), 0.01 (moderate Mutationsraten)
und 0.001 (niedrige Mutationsraten) verwendet wurden. F"ur die Selektionsrate wurden die
Werte 0.5 (moderate Selektion) und 0.8 (scharfe Selektion) verwendet. Insgesamt wurden sechs
L"aufe durchgef"uhrt, deren Namen aus der folgenden Tabelle zu entnehmen sind:

\medskip
\begin{tabular}{|c||c|c|} \hline
 & moderate & scharfe \\
 & Selektion & Selektion \\ \hline \hline
niedrige Mutationsraten & \runname{xlong00105} & \runname{xlong00108} \\ \hline
moderate Mutationsraten & \runname{xlong0105}  & \runname{xlong0108}  \\ \hline
hohe Mutationsraten     & \runname{xlong1005}  & \runname{xlong1008}  \\ \hline
\end{tabular}
\medskip



\subsubsection{Lauf \runname{xlong0105}: Moderate Mutationsraten und moderate Selektion}
\label{xlong0105section}

\begin{figure}

\unitlength1cm
\begin{picture}(16,17)
\put(0,0){\makebox(16,17){\epsfxsize=16cm \epsffile{xlong0105.eps}}}
\end{picture}
\caption{\label{xlong0105results}
LindEvol-GA Lauf \runname{xlong0105} mit moderaten Mutationsraten (0.01) und moderater Selektion (0.5).
Die mit "`Distanzverteilungen"' beschriftete Box zeigt die Abfolge der Distanzverteilungen in
der in (\protect\ref{ddistr-method}) beschriebenen Grauwertdarstellung.
}
\end{figure}


Dieser Lauf wurde mit Mutationsraten von 0.01 und einer Selektionsrate von 0.5 durchgef"uhrt.
Abb.\ \ref{xlong0105results} zeigt die Entwicklung verschiedener numerischer Werte in
diesem Lauf.


% \begin{figure}[p]
% 
% \unitlength1cm
% \begin{picture}(16,21)
% \put(0,20.5){\makebox(16,2)[b]{\epsfxsize=10cm \epsffile{xlong0105-0000004w.eps}}}
% \put(0,20){\makebox(16,0.5){{\bfseries a:} Generation 4}}
% \put(0,18.0){\makebox(16,2)[b]{\epsfxsize=10cm \epsffile{xlong0105-0000040w.eps}}}
% \put(0,17.5){\makebox(16,0.5){{\bfseries b:} Generation 40}}
% \put(0,15.5){\makebox(16,2)[b]{\epsfxsize=10cm \epsffile{xlong0105-0000120w.eps}}}
% \put(0,15){\makebox(16,0.5){{\bfseries c:} Generation 120}}
% \put(0,13.0){\makebox(16,2)[b]{\epsfxsize=10cm \epsffile{xlong0105-0000500w.eps}}}
% \put(0,12.5){\makebox(16,0.5){{\bfseries d:} Generation 500}}
% \put(0,10.5){\makebox(16,2)[b]{\epsfxsize=10cm \epsffile{xlong0105-0002200w.eps}}}
% \put(0,10){\makebox(16,0.5){{\bfseries e:} Generation 2200}}
% \put(0,8.0){\makebox(16,2)[b]{\epsfxsize=10cm \epsffile{xlong0105-0002207w.eps}}}
% \put(0,7.5){\makebox(16,0.5){{\bfseries f:} Generation 2207}}
% \put(0,5.5){\makebox(16,2)[b]{\epsfxsize=10cm \epsffile{xlong0105-0002400w.eps}}}
% \put(0,5){\makebox(16,0.5){{\bfseries g:} Generation 2400}}
% \put(0,3.0){\makebox(16,2)[b]{\epsfxsize=10cm \epsffile{xlong0105-0003608w.eps}}}
% \put(0,2.5){\makebox(16,0.5){{\bfseries h:} Generation 3608}}
% \put(0,0.5){\makebox(16,2)[b]{\epsfxsize=10cm \epsffile{xlong0105-0004999w.eps}}}
% \put(0,0){\makebox(16,0.5){{\bfseries i:} Generation 4999}}
% \end{picture}
% 
% \caption{\label{xlong0105worlds}
% Gesamtaufnahmen der Welt in verschiedenen Abschnitten des Laufs \runname{xlong0105}. Es wurde jeweils
% die gesamte Welt am Ende der Vegetationsperiode dargestellt.
% }
% \end{figure}


\begin{figure}

\unitlength1cm
\begin{picture}(16,20)
\put(4,17.5){\makebox(7,4)[b]{\epsfysize=4cm \epsffile{xlong0105-0000004p.eps}}}
\put(4,17.0){\makebox(7,0.5){{\bfseries a:} Generation 4}}
\put(0,14.0){\makebox(7,4)[b]{\epsfysize=4cm \epsffile{xlong0105-0000040p.eps}}}
\put(0,13.5){\makebox(7,0.5){{\bfseries b:} Generation 40}}
\put(8,14.0){\makebox(7,4)[b]{\epsfysize=4cm \epsffile{xlong0105-0000120p.eps}}}
\put(8,13.5){\makebox(7,0.5){{\bfseries c:} Generation 120}}
\put(0,9.5){\makebox(7,4)[b]{\epsfysize=4cm \epsffile{xlong0105-0000500p.eps}}}
\put(0,9.0){\makebox(7,0.5){{\bfseries d:} Generation 500}}
\put(8,9.5){\makebox(7,4)[b]{\epsfysize=4cm \epsffile{xlong0105-0002200p.eps}}}
\put(8,9.0){\makebox(7,0.5){{\bfseries e:} Generation 2200}}
\put(0,5.0){\makebox(7,4)[b]{\epsfysize=4cm \epsffile{xlong0105-0002207p.eps}}}
\put(0,4.5){\makebox(7,0.5){{\bfseries f:} Generation 2207}}
\put(8,5.0){\makebox(7,4)[b]{\epsfysize=4cm \epsffile{xlong0105-0002400p.eps}}}
\put(8,4.5){\makebox(7,0.5){{\bfseries g:} Generation 2400}}
\put(0,0.5){\makebox(7,4)[b]{\epsfysize=4cm \epsffile{xlong0105-0003608p.eps}}}
\put(0,0.0){\makebox(7,0.5){{\bfseries h:} Generation 3608}}
\put(8,0.5){\makebox(7,4)[b]{\epsfysize=4cm \epsffile{xlong0105-0004999p.eps}}}
\put(8,0.0){\makebox(7,0.5){{\bfseries i:} Generation 4999}}
\end{picture}

\caption{\label{xlong0105genomes}
Genome von Pflanzen aus den in Abb.\ \protect\ref{xlong0105worlds} dargestellten Welten. Oben ist jeweils
der Ph"anotyp der Pflanze dargestellt, darunter ihr Genom in graphischer Form. Schwarze Pfeile bedeuten
starke, graue Pfeile schwache Aktivierung. Es sind nur die aktiven Gene dargestellt
(vgl.\ Abb.\ \protect\ref{lnd1-graphgenom})
}
\end{figure}


\paragraph{Evolution von Fitness und Ph"anotypen}
\label{xlong0105-pheno}

Der Verlauf der Fitnesswerte zeigt deutlich verschiedene Stadien des Evolutionsprozesses an.
Anhand der Darstellung der Wuchsformen in repr"asentativen Generationen der verschiedenen
Phasen, die in Abb.\ \ref{xlong0105worlds} (Seite \pageref{xlong0105worlds})
dargestellt sind, k"onnen die einzelnen Phasen
folgenderma"sen charakterisiert werden:

In der Anfangsphase sind die Fitnesswerte niedrig, und die maximale
Fitness hat einen Wert, der jeweils w"ahrend einer Reihe von Generationen strikt konstant bleibt.
Die Pflanzen wachsen nur zu einer genetisch festgelegten Gr"o"se heran, sie
entwickeln hier nur wenige Zellen. Die Zellteilungen finden zu Anfang der Vegetationsperiode
statt. Am Ende der Vegetationsperiode sind fast alle Zellen energiereich. Die erste Form dieser
Art, die sich aus der Anfangspopulation heraus durchsetzt, ist eine zweizellige Pflanze
(Abb.\ \ref{xlong0105worlds}a), sp"ater entwickeln sich aus dieser Form drei- und vierzellige
Pflanzen.

Etwa bei Generation 30 findet ein steiler Anstieg der Fitnesswerte statt. Dieser evolution"are
Sprung kennzeichnet den "Ubergang vom beschr"ankten zum unbeschr"ankten Wachstum. Beim unbeschr"ankten
Wachstum stellt die Pflanze ihre Wachstumsaktivit"at w"ahrend der gesamten Vegetationsperiode
nicht ein. Typisch f"ur die Wachstumsprogramme solcher Pflanzen ist ein apikales Wachstum: Zellen
in lokalen Strukturen, die am oberen oder "au"seren Bereich der Pflanze liegen, teilen sich
und produzieren Zellen noch weiter oben oder au"sen.
Zellen, die rundherum von weiteren Zellen umgeben sind, deren lokale Struktur also typisch f"ur eine
im Inneren der Pflanze gelegene Zelle ist, teilen sich nicht, sondern sie erhalten ihre Energie
zur Erh"ohung des Fitnesswerts der Pflanze. Abb.\ \ref{xlong0105worlds}b zeigt, da"s die Pflanzen
hier senkrecht nach oben wachsende Formen entwickelt haben.

Bei Generation 100
kommt es zu einem weiteren Anstieg der Fitnesswerte. Wie die Abbildung \ref{xlong0105worlds}c
zeigt, r"uhrt dieser Anstieg daher, da"s von den senkrechten Sprossen aus seitlich Zellen
gebildet werden, die zus"atzlich Energie absorbieren und speichern k"onnen.

Die n"achste auffallendere "Anderung der Fitnesswerte ist eine kleine Spitze der Maximalwerte
bei Generation 500. Anstelle der einzelnen seitlichen Zellen entwickeln in dieser Phase
einzelne Pflanzen unbeschr"ankt wachsende, diagonale Seitentriebe (Abb.\ \ref{xlong0105worlds}d).
Mit dieser Wuchsform k"onnen h"ohere Fitnesswerte als die Form mit seitlichen Einzelzellen erreicht
werden. Anders als bei den bisher beobachteten Formen besteht bei der Wuchsform mit seitlichen
Trieben jedoch die Gefahr der Kollision mit anderen Pflanzen. Hier tritt also erstmals eine echte
Konkurrenz und eine gegenseitige Beeinflussung der Fitnesswerte zwischen den Pflanzen ein.
Kollisionen f"uhren zur Verschwendung von Energie, da Teilungsversuche scheitern, wenn die
Zielposition f"ur die Tochterzelle bereits besetzt ist. Weiterhin schr"anken sich die Pflanzen
nun gegenseitig in ihrem Wachstum ein, indem sie einander "uberwuchern. Aus diesem Grund geht der
hier beobachtete Anstieg der maximalen Fitnesswerte nicht, wie bei den bisher betrachteten
evolution"aren Spr"ungen, mit einem Anstieg der durchschnittlichen Fitness einher. Die nachteiligen
Auswirkungen der Kollisionen erkl"art auch, weshalb die neue Wuchsform sich nicht dauerhaft
durchsetzt. Nach weniger als hundert Generationen verdr"angen Formen, bei denen nur das vertikale,
nicht aber das horizontale Wachstum unbeschr"ankt ist, die Formen mit diagonalen Seitentrieben
vollst"andig aus der Population.

Nach diesem Vorgang ist eine mehr als 1000 Generationen lange Phase relativer Stasis zu beobachten.
Lediglich leichte Ver"anderungen in der Dynamik der Fluktuation der maximalen Fitnesswerte weisen
darauf hin, da"s es Ver"anderungen in Details des Wachstumsverhaltens gibt. Nach dieser Periode
der Stasis ist, etwa bei Generation 2200, wieder ein sprunghafter Anstieg sowohl der maximalen als
auch durchschnittlichen Fitness zu verzeichnen. Die Abbildungen \ref{xlong0105worlds}e und
\ref{xlong0105worlds}f zeigen, da"s diesem Sprung eine grundlegene "Anderung des Bauplans der
Pflanzen zugrundeliegt: Statt eines vertikal wachsenden Hauptsprosses bilden sie nun einen
diagonalen Hauptspro"s aus.

Bei den diagonalen
Formen ist die Anzahl der Zellen, die an einem Tag ein Photon absorbieren k"onnen, gr"o"ser
als bei den vertikalen Formen. Daher k"onnen sie erheblich effektiver als vertikal wachsende Pflanzen
Licht absorbieren. Da"s die diagonal wachsenden Formen sich nicht schon wesentlich fr"uher
durchgesetzt haben, liegt daran, da"s nicht die prinzipielle Effektivit"at, sondern die
Effektivit"at im Kontext der Gesamtpopulation letztlich "uber die Vermehrungschancen einer
Wuchsform entscheidet. Bei einer diagonal wachsenden
Pflanze in einer Umgebung vertikal wachsender Pflanzen ist die Wahrscheinlichkeit, da"s ihr Wachstum
vorzeitig durch Kollisionen zum Erliegen kommt, recht hoch. Abb.\ \ref{xlong0105worlds}e zeigt
einige diagonal wachsende Pflanzen, deren Wachstum durch Kollisionen mit vertikal wachsenden
Pflanzen vorzeitig zum Stillstand kommt. Nur eine diagonal wachsende Pflanze hat alle ihre
vertikal wachsenden Nachbarn "uberwuchert. Da"s sich die diagonalen Formen in dieser Phase "uberhaupt
durchsetzen k"onnen, liegt auch daran, da"s die dominierenden vertikalen Formen hier eine k"urzere,
gedrungene Gestalt haben. Mit der Zunahme der H"aufigkeit
diagonaler Formen verringert sich ihr Kollisionsrisiko, w"ahrend es f"ur die vertikal wachsenden
Formen gleichzeitig steigt (s.\ Abb.\ \ref{xlong0105worlds}f)

% Drop der Anzahl benutzter Gene

Weiterhin ist die vertikale Form bereits durch eine lange Evolution
seitlicher Verzweigungen gut an die Wachstumsbedingungen in einer von vertikal wachsenden Pflanzen
dominierten Welt angepa"st. Diese auf dem vertikalen Grundbauplan aufbauenden Modifikationen
gehen beim "Ubergang zum diagonalen Grundbauplan verloren, weil die Aktivierungsketten
zwischen dem Keimzellgen und den f"ur die Modifikationen verantwortlichen Genen unterbrochen 
werden.

In dieser Weise sind die ersten diagonal wachsenden Pflanzen primitive und schlecht adaptierte
Realisierungen des neuen Bauplans. Offenbar "uberwiegt jedoch der Vorteil der verbesserten
Effizienz bei der Lichtabsorption, so da"s der neue, diagonale Bauplan sich gegen die vertikal
wachsenden Formen durchsetzt und letztere dauerhaft verdr"angt.

Nach seiner Etablierung erf"ahrt auch der diagonale Bauplan verschiedene Modifikationen.
Die erste Adaptation ist die Verdickung des Hauptsprosses durch die Auflagerung weiterer
Zellschichten (Abb.\ \ref{xlong0105worlds}g). Sp"ater ist die Evolution von Seitentrieben zu beobachten,
die unbeschr"ankt diagonal nach rechts von der diagonal nach links wachsenden Hauptachse auswachsen.
In einem Abschnitt, der sich etwa zwischen den Generationen 3500 bis 3800 erstreckt, treten Pflanzen
auf, die durch die Ausbildung vieler diagonaler Seitentriebe eine stark buschf"ormige Erscheinung
haben (Abb.\ \ref{xlong0105worlds}h).
In allzu gro"ser Zahl sind diese Seitentriebe jedoch unvorteilhaft, weil ihre Ausbildung stets mit dem
Risiko von Kollisionen und daraus resultierender Energievergeudung verbunden ist. Die buschf"ormigen
Wuchsformen verdr"angen die haupts"achlich diagonal nach links wachsenden Formen daher nicht dauerhaft.
Zum Ende des Laufs \verb|xlong0105| wird die Population durch Formen, die nur wenige seitliche
Verzweigungen bilden, dominiert (Abb.\ \ref{xlong0105worlds}i).


\paragraph{Evolution der Genome und Wachstumsprogramme}
\label{xlong0105-geno}
% Genomlaenge

Die Genoml"angen schwanken relativ stark im Laufe der Simulation, diese Schwankungen weisen jedoch
eine geringe Korrelation zu der Evolution der Fitness und der Gestalt der Pflanzen auf. Die Genoml"ange
ist jedoch deutlich mit der Anzahl verschiedener Genotypen korreliert. Dies liegt daran, da"s die
Mutationsraten pro Zeichen und Generation definiert sind. Ein kurzes Genom wird daher mit geringerer
Wahrscheinlichkeit beim Mutationsschritt ver"andert als ein langes Genom. Eine Verringerung der
Anzahl unterschiedlicher Genotypen f"uhrt auch zu einem R"uckgang der genetischen Diversit"at.

% Anzahl aktiver Gene

\begin{sloppypar}
Abb.\ \ref{xlong0105genomes} zeigt einige f"ur die verschiedenen Phasen des Evolutionsprozesses
in \verb|xlong0105| charakteristische regulatorische Netzwerke. Beim Wachstum einer zweizelligen
Pflanze wird nur ein Gen aktiviert, das eine Teilung der Keimzelle ausl"ost
(Abb.\ \ref{xlong0105genomes}a). Mit der Evolution von 
Pflanzen mit genetisch beschr"anktem Wachstum in den ersten 30 Generationen nimmt die Anzahl der
beim Wachstum der Pflanzen aktivierten Gene zun"achst zu. Bei der Ausbreitung der unbeschr"ankt
wachsenden Pflanzen sinkt die Anzahl aktivierter Gene jedoch wieder ab. Der Grund daf"ur ist,
da"s die Grundform der vertikal wachsenden Pflanze durch ein regulatorisches Netzwerk von nur
zwei Genen codiert wird (Abb.\ \ref{xlong0105genomes}b), bei dem ein Gen
sein eigener Aktivator ist. Dieses Gen (Gen 20 in der Abbildung) veranla"st die Produktion einer
Tochterzelle, in der wiederum dasselbe Gen aktiviert wird. Mit diesem Selbstregulationsmechanismus
kann so mit einer geringen Anzahl von Genen das Wachstumsprogramm f"ur eine unbeschr"ankt wachsende
Pflanze codiert werden, deren Fitness derjenigen beschr"ankt wachsender Formen, deren Wachstum durch gr"o"sere
regulatorische Netzwerke gesteuert wird, "uberlegen ist.
\end{sloppypar}

Das Auftreten von Formen mit seitlichen Zellen f"uhrt zum n"achsten evolution"aren Schritt. Diese
Formen kommen durch eine Erweiterung des Grundprogramms f"ur das unbeschr"ankte Wachstum eines vertikalen
Sprosses zustande, entsprechend nimmt auch der Umfang des regulatorischen Netzwerks wieder zu
(Abb.\ \ref{xlong0105genomes}c). Diese Abbildung zeigt, da"s die Evolution der vertikalen
Formen mit seitlich auswachsenden Zellen einige Gene anf"anglich zur Aktivierung von Genen f"uhrt,
die Teilungsversuche ausl"osen, die zum Scheitern verurteilt sind. In Abb.\ \ref{xlong0105genomes}c
steuert Gen 11 den Versuch der Produktion einer Tochterzelle rechts "uber der Mutterzelle bei einer
lokalen Struktur, in der diese Position bereits durch eine Zelle derselben Pflanze besetzt ist.
Durch die Aktivierung dieses Gens wird Energie verbraucht, ohne da"s eine neue Zelle produziert werden
kann, dem Nachteil des Energieverbrauchs steht also kein Vorteil gegen"uber.

Die Formen mit einem prim"aren, vertikalen Spross und sekund"aren, diagonalen Trieben weisen "uberraschend
komplexe regulatorische Netzwerke auf (Abb.\ \ref{xlong0105genomes}d). Das Keimzellgen 55 und Gen 36
codieren in bereits bekannter Weise das Wachstum des vertikalen Hauptsprosses. Die Ausbildung des seitlichen
Triebes wird durch die Aktivierung von Gen 37, das die Produktion einer Zelle seitlich des Hauptsprosses
ausl"ost, initiiert. Dieses aktiviert Gen 25, was schlie"slich das selbstaktivierende Gen 58 aktiviert,
das die Elongation des seitlichen Triebes steuert. Durch die Aktivierung von Gen 22 wird dem seitlichen
Trieb eine weitere Zellschicht aufgelagert. Schlie"slich kommt es durch die Initiierung des seitlichen
Triebes zur Aktivierung von Gen 3, was durch die Aktivierung der Gene 27 und 30 zur Beendigung des
Wachstums des Hauptsprosses f"uhrt. Im Unterschied zu den zuvor beobachteten vertikal wachsenden Formen (Abbildungen
\ref{xlong0105genomes}b und \ref{xlong0105genomes}c) kann so im oberen Bereich des Hauptsprosses Energie
gesammelt werden.

Der hier beobachtete "Ubergang vom prim"aren, vertikalen zum sekund"aren, diagonalen Wachstum zeigt,
wie die stochastische Natur der Lichtabsorption durch die deterministischen Entwicklungsprogramme
genutzt werden kann, um "Uberg"ange der beschriebenen Art mit einer bestimmten Wahrscheinlichkeit
zu initiieren. Zu dem "Ubergang vom vertikalen zum diagonalen Wachstum kommt es bei dem in
Abb.\ \ref{xlong0105genomes}d gezeigten Entwicklungsprogramm dann, wenn das Photon in der Spalte,
in der der vertikale Spro"s w"achst, nicht von der apikalen Zelle, sondern von einer darunterliegenden
Zelle absorbiert wird. Dies f"uhrt zur Aktivierung des Gens 37 und damit zu der Initiierung des
diagonalen Seitentriebs.

Am oberen Ende des terminierten Hauptsprosses befindet sich, um
eine Position gegen"uber dem Hauptspro"s nach links verschoben, eine Zelle, welche die Aktion \verb|mut-|
ausf"uhrt. Die dadurch erreichte Senkung dient der Verbesserung der Chance, das komplexe Entwicklungsprogramm
ohne Beeintr"achtigung oder Zerst"orung durch Mutationen in die n"achste Generation zu vererben.
Das in Abb.\ \ref{xlong0105genomes}d gezeigte Wachstumsprogramm umfa"st neun Gene. Ohne das Gen 30 w"aren es acht
Gene, und die Ausbreitungszahl (Gl.\ \ref{programspreadm-eq}) betr"uge
$A_E(0.01, 0.5, 8) = 1/(1-0.5) \cdot (1-0.01)^{3 \cdot 16} \approx 1.23$. Mit dem Gen 30 ist das Wachstumsprogramm ein Gen
l"anger. Bei rund der H"alfte der Genome mit diesem Entwicklungsprogramm werden jedoch durch die Aktivierung des
Gens 30 am letzten Tag der Vegetationsperiode die effektiven Mutationsraten halbiert. Die Ausbreitungszahl
dieses Entwicklungsprogramms betr"agt somit $(A_E(0.01, 0.5, 9) + A_E(0.005, 0.5, 9)) / 2 \approx 1.34$.
Die Expression von Gen 30 f"uhrt somit zu einem signifikanten Reproduktionsvorteil.

Die hohe Anf"alligkeit des komplexen Wachstumsprogramms in Abb.\ \ref{xlong0105genomes}d ist m"oglicherweise
einer der Faktoren, die eine dauerhafte Etablierung dieser Wuchsform verhindern. In Generation 2200
dominieren gedrungene Formen, die seitlich mehrere Zellschichten ausbilden k"onnen
(Abb.\ \ref{xlong0105genomes}e). Dieses seitliche Wachstum wird in der Regel durch Kollision mit dem
linken Nachbarn begrenzt. In der Generation 2200 treten bereits die ersten diagonal wachsenden
Formen auf, die in weniger als 100 Generationen die vertikalen Formen verdr"angen. Abb.\ \ref{xlong0105genomes}f
zeigt, da"s das Entwicklungsprogramm der diagonal wachsenden Pflanzen extrem einfach ist, es ist analog
zu dem der einfachen vertikalen Pflanzen (Abb.\ \ref{xlong0105genomes}b). Da keine Zelle einer diagonal
wachsenden Pflanze eine andere Zelle derselben Pflanze beschattet, haben diagonal wachsende Pflanzen jedoch
typischerweise erheblich mehr energiereiche Zellen als vertikal wachsende Pflanzen, und somit haben sie
auch einen h"oheren Fitnesswert.

Auch das Grundprogramm der diagonal wachsenden Pflanze wird im weiteren Verlauf der Evolution erweitert.
Abb.\ \ref{xlong0105genomes}g zeigt ein Genom, in dem neben den beiden Genen 24 und 16, die das Wachstum
des prim"aren, diagonalen Sprosses steueren, drei weitere Gene zur Produktion von Zellen in einer diesem
Hauptspro"s aufgelagerten Schicht aktiviert werden.

Abb.\ \ref{xlong0105genomes}h zeigt das regulatorische Netzwerk im Genom einer buschf"ormigen Pflanze.
Die Gene 9 und 18 steuern in der zuvor beschriebenen Weise das Wachstum eines diagonalen Prim"arsprosses,
daher kann diese Form genetisch als auf der diagonalen Grundform basierend charakterisiert werden.
Zur Entwicklung der buschf"ormigen Gestalt werden insgesamt sieben Gene aktiviert, das Entwicklungsprogramm
ist also, "ahnlich wie das in Abb.\ \ref{xlong0105genomes}d Dargestellte, relativ komplex.
Wie bereits bei der Diskussion von Abb.\ \ref{xlong0105genomes}d ausgef"uhrt wurde, werden die
Reproduktionserfolge bei komplexen Wuchsformen durch die Mutation st"arker geschm"alert als bei
einfachen Entwicklungsprogrammen; die Verringerung der effektiven Mutationsraten bringt bei komplexen
Entwicklungsprogrammen dementsprechend einen gr"o"seren Vorteil. Dies erkl"art den Ausschlag des
durchschnittlichen Mutationsexponenten in den negativen Bereich bei Generation 3600, gegen Ende
der Periode des Auftretens komplexerer, buschf"ormiger Formen. Trotz dieser Senkung der effektiven
Mutationsraten sterben die buschf"ormigen Formen aus. Ein Faktor, der dabei eine Rolle spielt, ist,
wie bereits in \ref{xlong0105-pheno} besprochen, das gr"o"sere Kollisionsrisiko bei buschf"ormigem
Wachstum. Ein weiteres Problem liegt in der Etablierung einer mutationsratenver"andernden 
Aktivit"at mit ausreichend feiner Regulierbarkeit. Um optimale Wirkung zu entfalten, mu"s die
Aktion \verb|mut-| von einer oder von wenigen Zellen am letzten Tag der Vegetationsperiode
ausgef"uhrt werden, wie dies bei dem in Abb.\ \ref{xlong0105genomes}d gezeigten Wachstumsprogramm
der Fall ist. Wie Abb.\ \ref{xlong0105genomes}h zeigt, finden
bei dem buschf"ormigen Wachstum nur Zellteilungen in diagonale Richtungen statt. Dementsprechend
k"onnen nur die vier diagonalen Nachbarpositionen in einer lokalen Struktur besetzt werden.
Keine der lokalen Strukturen aus dieser eingeschr"ankten Menge tritt gleichzeitig ausreichend
zuverl"assig und ausreichend selten auf, um als Ausl"oser f"ur die Aktion \verb|mut-| geeignet zu sein.

Nach dem Aussterben der buschf"ormigen Pflanzen wird die Population von Formen dominiert, die einen
verdickten, diagonalen Hauptspro"s mit wenigen Verzweigungen ausbilden. Abb.\ \ref{xlong0105genomes}i
zeigt eine typische Pflanze dieser Art und ihr Genom mit dem regulatorischen Netzwerk. Die Abbildung
zeigt, da"s bei diesem Entwicklungsprogramm wiederum die Verzweigungsh"aufigkeit unter Ausnutzung der stochastischen
Natur der Lichtabsorption reguliert wird. Das Gen 1 wird nur dann aktiviert, wenn in der dem Hauptspro"s
aufgelagerten Zellkette, die durch die Aktivit"at des Gens 22 erzeugt wird, eine Zelle isoliert ist.
Wenn eine Zelle, die durch Gen 22 produziert wurde, bei ihrer ersten Lichtabsorption bereits einen
linksoberen oder einen rechtsunteren Nachbarn hat, wird kein diagonal nach rechts wachsender Seitentrieb
initiiert, sondern die Zelle teilt sich nicht mehr und dient so als Energiespeicher, der der Erh"ohung
des Fitnesswerts dient. Das Wachstum dieses recht erfolgreichen Ph"anotyps wird von nur vier Genen
gesteuert, was das entsprechende Entwicklungsprogramm robust gegen Mutationen macht.


\paragraph{Evolution der Mutationsraten}

In \runname{xlong0105} kommt es zu keiner Evolution signifikant ver"anderter Mutationsraten. Der
Durchschnittswert des Mutationsexponenten fluktuiert, ohne das eine Tendenz erkennbar wird. Lediglich
bei Generation 3600 kommt es zu einer kurzen Episode, in der die Mutationsexponenten deutlich erniedrigt
sind. Dies steht im Zusammenhang mit dem Auftreten buschiger Pflanzen, wie in (\ref{xlong0105-geno})
beschrieben wurde.


\paragraph{Distanzverteilungen und DVK}

Die Editierabst"ande der Genome sind w"ahrend des gesamten Laufs relativ gleichm"a"sig verteilt, was
in der Grauwertdarstellung zu einem entsprechend einheitlichen Grauton in dem Intervall, in dem die
Werte verteilt sind, f"uhrt. Die Gr"o"se dieses Intervalls wird durch die Genoml"angen bestimmt.
Die Distanzwerte h"aufen sich etwas an der oberen Grenze dieses Intervalls, dies macht sich in einem
etwas dunkleren Band bemerkbar, das in etwa der Kurve der Genoml"angen, multipliziert mit zwei (weil
ein Gen aus zwei Bytes besteht) entspricht. Die Feststellung, da"s die Distanzwerte bis hinauf
in den Maximalbereich relativ gleichm"a"sig verteilt sind, zeigt, da"s die Genome sich stets in weit auseinanderliegenden
Bereichen im Sequenzraum verteilen. Dies ist m"oglich, weil in den meisten Abschnitten von \runname{xlong0105}
nur ein kleiner Anteil der Gene eines Genoms w"ahrend der Pflanzenentwicklung aktiviert wird. Beispielsweise
liegt w"ahrend der gesamten zweiten H"alfte von \runname{xlong0105} die durchschnittliche Genoml"ange stets um 70,
die durchschnittliche Anzahl der aktivierten Gene bleibt jedoch stets unter 7. Somit bleiben mehr als 90\%
eines Genoms bei der Pflanzenentwicklung inaktiv.

Die DVK hat in Generation 0 einen Wert von ca.\ 0.5, dieser ist typisch f"ur vollst"andig randomisierte
Populationen, die bei LindEvol-GA als Startpopulationen verwendet werden. Innerhalb der ersten zehn
Generationen schnellt die DVK auf Werte im Bereich von 4 hoch. W"ahrend des gesamten Laufs bleiben die
DVK-Werte in diesem Bereich. Die verschiedenen Evolutionsprozesse, wie evolution"are Spr"unge und "Anderungen
der Genoml"angen, haben keine signifikanten Auswirkungen auf die DVK der relativen Editierabst"ande.
Da bei der Diskretisierung des relativen Editierabstands in 100 Intervalle der Maximalwert der DVK
$-\log(0.01) \approx 4.6$ betr"agt, sind die beobachteten DVK-Werte im Bereich von 4 als hoch anzusehen.


% Einzellerei -> beschaenktes Wachstum -> unbeschraenktes Wachstum
% Genomlaenge: starke Schwankungen bei gleichbleibender Fitness
% Anzahl benutzter Gene: Tendenz steigend, Einbruch bei Uebergang zu diagonaler Form.
% Anzahl Arten: Wechselwirkung mit Genomlaenge wg. Mutationsrate pro Zeichen
% Distanzverteilung: Relativ gleichmaessig
% DVK: Sprunghafter Anstieg bei Uebergang von randomisierter zu strukturierter Population
%     Leichter Anstieg bei evolutionaerem Sprung
% Genetische Diversitaet: Hauptsaechlich abhaengig von Genomlaenge -> Anzahl Arten
% Mutationsexponent: Keine nennenswerten Entwicklungen
% Selbstaktivierung -> MADS-zeugs
% Durchsetzung am ehesten dauerhaft, wenn mit anstieg mittlerer Fitness verbunden.


\subsubsection{Lauf \runname{c-xlong0105}: Neutrale Kontrollsimulation mit moderaten Mutationsraten und
    moderater Selektion}
\label{c-xlong0105section}

\begin{figure}

\unitlength1cm
\begin{picture}(16,17)
\put(0,0){\makebox(16,17){\epsfxsize=16cm \epsffile{c-xlong0105.eps}}}
\end{picture}
\caption{\label{c-xlong0105results}
Lauf der neutralen Kontrollsimulation \runname{c-xlong0105}. Dieser Lauf wurde mit 
denselben Kontrollparametern wie Lauf \runname{xlong0105} durchgef"uhrt.
}
\end{figure}

Abb.\ \ref{c-xlong0105results} zeigt die Ergebnisse der neutralen Kontrolle zum
Lauf \runname{xlong0105}. Diese Kontrollsimulation zeigt, da"s es in dem Simulationszeitraum von
5000 Generationen zu erheblichen Schwankungen der Genoml"angen allein aufgrund von {\slshape random
drift} kommen kann. Die Genome bleiben jedoch k"urzer als in Lauf \runname{xlong0105}, weil
bei der neutralen Kontrolle keine f"ur die Selektion relevante genetische Information gebildet
werden kann.

In Abb.\ \ref{c-xlong0105results} kann der Zusammenhang zwischen der Genoml"ange und der
Anzahl verschiedener Genotypen, die sich wiederum auf die genetische Diversit"at auswirkt,
deutlich beobachtet werden. In den Generationen 2000 bis 3500, in denen die Genome relativ
lang sind, ist die Anzahl verschiedener Genotypen maximal oder fast maximal, und die genetische
Diversit"at ist hoch. Nach Generation 4000 werden die Genoml"angen teils extrem niedrig.
Kurze Genome werden durch die Mutation mit geringerer Wahrscheinlichkeit ver"andert als
lange Genome. Entsprechend erh"oht sich bei kurzen Genomen die Wahrscheinlichkeit daf"ur, da"s
die Nachkommen eines Individuums in der folgenden Generation identische Genotypen haben.
Dies erkl"art, weshalb es in den Abschnitten mit extrem kurzen Genomen zu starken R"uckg"angen
bei der Anzahl unterschiedlicher Genotypen und der genetischen Diversit"at kommt.

Die DVK-Werte sind in der neutralen Kontrolle ungef"ahr genauso gro"s wie im Lauf \runname{xlong0105}.
Dies zeigt, da"s die Bildung einer strukturierten Vielfalt durch den Mechanismus des genetischen
Algorithmus nicht in entscheidender Weise von der verwendeten Fitnessfunktion abh"angt. Das ausbalancierte
Zusammenspiel von Mutation und Selektion reicht zur Entstehung der strukturierten Vielfalt aus.

Die R"uckg"ange der DVK in Abschnitten, in denen die Genome extrem kurz werden, liegt an der
diskreten Natur des relativen Editierabstandes. Die Anzahl der Werte, die der relative Editierabstand
bei einer maximalen Genoml"ange von $l_{\mathit{max}}$ annehmen kann, ist nach oben beschr"ankt durch die
Anzahl aller rationaler Zahlen $i/j$ mit $i, j \in \mbox{\bfseries N}_0, i \leq j, 1 \leq j \leq l_{\mathit{max}}$.
Diese ist nach oben beschr"ankt durch die Anzahl aller Br"uche der Form $i/j$. Mit jedem festen $j$ k"onnen
$j+1$ Br"uche mit $0 \leq i/j \leq 1$ gebildet werden. Die Gesamtzahl aller Br"uche mit $1 \leq j \leq l_{\mathit{max}}$
und $0 \leq i/j \leq 1$ betr"agt daher $\sum_{j=1}^{l_{\mathit{max}}}j+1 = l_{\mathit{max}} \cdot (l_{\mathit{max}}+3) / 2$.
(Die tats"achliche Anzahl unterschiedlicher rationaler
Zahlen der Form $i/j$ ist kleiner, weil bei dieser Z"ahlung Br"uche, von denen eine gek"urzte Form bereits
gez"ahlt wurde, nicht ausgelassen werden.) Sinkt die Anzahl der m"oglichen Werte, die
der relative Editierabstand annehmen kann, unter die Anzahl der Intervalle, mit der bei der Berechnung
der DVK diskretisiert wird, k"onnen einige Intervalle keine Distanzwerte mehr enthalten, und die entsprechenden
H"aufikgeitswerte sind 0. Auf diese Weise kommt zu
einer Ungleichm"a"sigkeit der Verteilung der Werte auf die Diskretisierungsintervalle, die zu einer kleineren
DVK f"uhrt. Bei $l_{\mathit{max}}=10$ kann der relative Editierabstand nach der obigen Absch"atzung h"ochstens
$(10 \cdot 13)/2 = 65$ verschiedene Werte annehmen, die DVK kann damit den Maximalwert von $-\log(1/65) \approx 4.17$
nicht "uberschreiten. Dieser Wert liegt bereits deutlich unterhalb des durch die Anzahl der Diskretisierungsintervalle
gegebenen Limits von $-\log(0.01) \approx 4.6$.
Der in Abschnitten mit sehr kurzen Genomen beobachtete starke R"uckgang
der DVK kann daher nicht allein auf eine Abnahme der Diversit"at infolge der geringen Mutationsaktivit"at zur"uckgef"uhrt
werden; auch der beschriebene Diskretisierungseffekt tr"agt zu den beobachteten R"uckg"angen bei.


% -> neutrale Kontrolle


\subsubsection{Lauf \runname{xlong0108}: Moderate Mutationsraten und scharfe Selektion}

\begin{figure}

\unitlength1cm
\begin{picture}(16,17)
\put(0,0){\makebox(16,17){\epsfxsize=16cm \epsffile{xlong0108.eps}}}
\end{picture}
\caption{\label{xlong0108results}
LindEvol-GA Lauf \runname{xlong0108} mit moderaten Mutationsraten (0.01) und scharfer Selektion (0.8).
}
\end{figure}

\begin{figure}[t]

\unitlength1cm
\begin{picture}(16,14)
\put(0,9.5){\makebox(7,4)[b]{\epsfysize=4cm \epsffile{xlong0108-0000400p.eps}}}
\put(0,9.0){\makebox(7,0.5){{\bfseries a:} Generation 400}}
\put(8,9.5){\makebox(7,4)[b]{\epsfysize=4cm \epsffile{xlong0108-0000600p.eps}}}
\put(8,9.0){\makebox(7,0.5){{\bfseries b:} Generation 600}}
\put(0,5.0){\makebox(7,4)[b]{\epsfysize=4cm \epsffile{xlong0108-0001000p.eps}}}
\put(0,4.5){\makebox(7,0.5){{\bfseries c:} Generation 1000}}
\put(8,5.0){\makebox(7,4)[b]{\epsfysize=4cm \epsffile{xlong0108-0002000p.eps}}}
\put(8,4.5){\makebox(7,0.5){{\bfseries d:} Generation 2000}}
\put(0,0.5){\makebox(7,4)[b]{\epsfysize=4cm \epsffile{xlong0108-0004000p.eps}}}
\put(0,0.0){\makebox(7,0.5){{\bfseries e:} Generation 4000}}
\put(8,0.5){\makebox(7,4)[b]{\epsfysize=4cm \epsffile{xlong0108-0004999p.eps}}}
\put(8,0.0){\makebox(7,0.5){{\bfseries f:} Generation 4999}}
\end{picture}

\caption{\label{xlong0108genomes}
Genome von typischen Pflanzen aus verschiedenen Generationen des Laufs \runname{xlong0108}. Oben ist jeweils
der Ph"anotyp der Pflanze dargestellt, darunter ihr Genom in graphischer Form mit regulatorischem Netzwerk.
}
\end{figure}

Abb.\ \ref{xlong0108results} zeigt die numerischen Ergebnisse des Laufs \runname{xlong0108}, der
mit der scharfen Selektionsrate von 0.8 und moderaten Mutationsraten (0.01) durchgef"uhrt wurde.
Die Fitnesswerte erreichen in diesem Lauf h"ohere Maxima als im Lauf \runname{xlong0105}. Die
durchschnittlichen Fitnesswerte sind hingegen nicht h"oher als in \runname{xlong0105}.
Abb.\ \ref{xlongworlds} (Seite \pageref{xlongworlds}) zeigt die Ursache f"ur diesen Befund. W"ahrend in \runname{xlong0105}
Formen mit nur wenigen Verzweigungen dominieren, sind in Generation 4999 von Lauf \runname{xlong0108}
buschartige Pflanzen vorherrschend. Die Selektionsvorteile derartiger Formen sind bei scharfer Selektion
st"arker als bei moderater Selektion. Bei scharfer Selektion ist ein Fitnesswert im Spitzenbereich
zur erfolgreichen Vermehrung erforderlich. Dieser kann nur durch die "Uberwucherung von Nachbarpflanzen
erreicht werden. Pflanzen, die "uberwachsen werden, haben kaum eine Fortpflanzungschance, auch wenn
sie die Energieverluste durch Kollisionen durch eine sparsamere Wuchsform begrenzen.

Wie in \runname{xlong0105} lassen sich anhand des Verlaufs der Fitnesswerte einige evolution"are
Spr"unge identifizieren. Auch in \runname{xlong0108} entwickeln sich zun"achst beschr"ankt wachsende
Formen. Die ersten unbeschr"ankt wachsenden Formen wachsen vertikal (Abb.\ \ref{xlong0108genomes}a).
Kurz vor der Generation 500 erfolgt der evolution"are Schritt vom vertikalen zum diagonalen Wachstum
(Abb.\ \ref{xlong0108genomes}b). Aus der diagonal wachsenden Grundform entwickeln sich in wenigen
hundert Generationen verzweigte Formen (Abb.\ \ref{xlong0108genomes}c). Gelegentlich treten Phasen
auf, in denen die buschigen Formen durch stark verzweigte Pflanzen verdr"angt werden (Abb.\ \ref{xlong0108genomes}d).
Formen mit wenigen Verzweigungen setzen sich jedoch zu keinem Zeitpunkt durch. Am Ende des Laufs dominieren
wieder buschartige Formen (Abb.\ \ref{xlong0108genomes}e). 

Die Genoml"angen sowie die Anzahl aktivierter Gene erreichen h"ohere Spitzenwerte als in \runname{xlong0105}.
Dies liegt u.a.\ an den Beschr"ankungen des Genominterpretationssystems. Beim Wachstum einer buschf"ormigen
Pflanze kommt es an der Wachstumsfront zu vielen "ahnlichen lokalen Strukturen. Um eine stabil fortschreitendes,
buschf"ormiges Wachstum zu gew"ahrleisten, mu"s f"ur jede dieser lokalen Strukturen eine Regel, die eine die
Buschform erg"anzende Zellteilung verursacht, im Genom codiert sein. Daher sind z.B.\ zur Codierung der in
Generation 4999 von \runname{xlong0108} typischen Form (Abb.\ \ref{xlong0108genomes}f) sieben Gene erforderlich, w"ahrend
die in Generation 4999 von \runname{xlong0105} verbreitete, gering verzweigte Form (Abb.\ \ref{xlong0105genomes}i)
mit nur vier Genen codiert wird.

\begin{sloppypar}
Eine dauerhafte Evolution einer aktiven Modifikation der Mutationsraten ist in \runname{xlong0108} nicht
zu beobachten, wie in \runname{xlong0105} weist die Kurve des durchschnittlichen Mutationsexponenten lediglich
einige episodische Ausschl"age auf.
\end{sloppypar}

Auch der Verlauf der DVK entspricht qualitativ demjenigen, der in \runname{xlong0105} beobachtet wurde: Nach einem
anf"anglichen Anstieg ver"andert sich die DVK w"ahrend des gesamten Verlaufs nicht mehr in signifikanter Weise.
Die DVK pendelt sich in einem Bereich hoher Werte ein, sie
ist jedoch etwas niedriger als in \runname{xlong0105}. Aus den Distanzverteilungen
ist ersichtlich, da"s die H"aufigkeiten der gr"o"seren Distanzwerte geringer sind als die der kleinen Distanzwerte;
die h"oheren taxonomischen Kategorien sind also unterrepr"asentiert. Dies ist dadurch, da"s bei der scharfen
Selektionsrate von 0.8 ein Cluster sehr schnell aus der gesamten Population verdr"angt werden kann, zu erkl"aren.
Durch die scharfe Selektion ist also die Ausbildung einer breitgef"acherten Taxonomie erschwert, das System tendiert
in Richtung Konvergenz (vgl.\ \ref{ddc-method}). Diese Tendenz ist jedoch nur schwach ausgepr"agt, dies zeigt sich
darin, da"s die Anzahl der verschiedenen Genotypen und die genetische Diversit"at w"ahrend des gesamten Laufs
im Maximalbereich liegen, und da"s auch die DVK in einem im Vergleich zu \runname{xlong0105} nur wenig kleinere
Werte annimmt.




\subsubsection{Lauf \runname{xlong1005}: Hohe Mutationsraten und moderate Selektion}
\label{xlong1005section}

\begin{figure}

\unitlength1cm
\begin{picture}(16,17)
\put(0,0){\makebox(16,17){\epsfxsize=16cm \epsffile{xlong1005.eps}}}
\end{picture}
\caption{\label{xlong1005results}
LindEvol-GA Lauf \runname{xlong1005} mit hohen Mutationsraten (0.1) und moderater Selektion (0.5).
}
\end{figure}

% \begin{figure}
% 
% \unitlength1cm
% \begin{picture}(16,5)
% \put(0,0.5){\makebox(7,4)[b]{\epsfysize=4cm \epsffile{xlong1005-0004168p.eps}}}
% \put(0,0.0){\makebox(7,0.5){{\bfseries a:} Generation 4168}}
% \put(8,0.5){\makebox(7,4)[b]{\epsfysize=4cm \epsffile{xlong1005-0004999p.eps}}}
% \put(8,0.0){\makebox(7,0.5){{\bfseries b:} Generation 4999}}
% \end{picture}
% 
% \caption{\label{xlong1005genomes}
% Genome von Pflanzen aus verschiedenen Generationen des Laufs \runname{xlong1005}.
% Da sich infolge der zu hohen Mutationsraten keine Spezies dauerhaft etablieren kann,
% stellen diese Beispiele keine charakteristischen Pflanzen dar.
% }
% \end{figure}

Bei diesem Lauf, dessen Verlaufsdiagramm in Abb.\ \ref{xlong1005results}
gezeigt wird, verl"auft der Evolutionsproze"s qualitativ anders als in den beiden zuvor diskutierten
L"aufen. Obwohl es bei der maximalen Fitness immer wieder zu signifikant erh"ohten Werten kommt, k"onnen
die entsprechenden Pflanzenformen sich nicht dauerhaft in der Population etablieren. Die stark erh"ohten
Fitnesswerte verschwinden stets nach einigen Generationen wieder, und es kommt zu keiner Erh"ohung der
durchschnittlichen Fitness.

Abb.\ \ref{xlongworlds}d zeigt, da"s auch bei den Ph"anotypen keinerlei Entwicklung von Formen mit komplexerer
Morphologie festzustellen ist. Auch nach 4999 Generationen existieren etliche Pflanzen, die "uberhaupt nicht
wachsen. Keine der Pflanzen in Abb.\ \ref{xlongworlds}d zeigt unbeschr"anktes Wachstum. Dieses tritt zwar
immer wieder auf, was jeweils mit einem Peak bei der maximalen Fitness
einhergeht, die unbeschr"ankt wachsenden Formen verschwinden jedoch trotz ihres Selektionsvorteils wieder
aus der Population.

Diese Beobachtungen weisen darauf hin, da"s 
bei den hohen Mutationsraten von 0.1 und der moderaten Selektionsrate von 0.5 
die kritische Fehlergrenze selbst f"ur elementare Entwicklungsprogramme "uberschritten ist.
Dies kann auch anhand von Gl.\ \ref{errthresholdprgm-eq} nachgewiesen werden.
Zur Realisierung einer unbeschr"ankt
wachsenden Pflanze sind mindestens zwei Gene erforderlich, ein Keimzellgen und ein Gen f"ur das apikale
Wachstum. Abb.\ \ref{xlong0105genomes}b zeigt ein Beispiel hierf"ur. Die kritische Fehlergrenze f"ur ein solches
Entwicklungsprogramm betr"agt $t_E(0.5,2) = \sqrt[6 \cdot 2]{1-0.5} \approx 0.056$, dieser Wert wird durch die Mutationsraten
von 0.1 deutlich "uberschritten.
Selbst elementarste Entwicklungsprogramme
f"ur unbeschr"ankt wachsende Pflanzenformen werden also durch die hohen Mutationsraten so schnell wieder zerst"ort,
da"s ihre Ausbreitung und dauerhafte Etablierung in der Population trotz "uberlegener Fitnesswerte unm"oglich
ist. Somit ist eine signifikante Zunahme an morphologischer Komplexit"at gegen"uber dem
vollst"andig randomisierten Startzustand nicht m"oglich. Lediglich eine gewisse H"aufung zweizelliger Pflanzen
kann festgestellt werden; ihre kritische Fehlergrenze liegt bei $t_E(0.5,1) = \sqrt[6 \cdot 1]{1-0.5} \approx 0.109$ und ist
somit durch die Mutationsraten von 0.1 nicht "uberschritten.

Wie bei den beiden zuvor besprochenen L"aufen pendelt sich die DVK von Anfang an auf einen Wert ein, auf dem
sie w"ahrend des gesamten Laufs, abgesehen von statistischen Schwankungen, verbleibt. Dieser Wert liegt mit ca.\ 2.5
deutlich unterhalb der hohen DVK-Werte um 4, die bei den L"aufen \runname{xlong0105} und \runname{xlong0108}
beobachtet wurden. Die Distanzverteilungen von \runname{xlong1005} zeigen, da"s kleine Distanzwerte v"ollig
fehlen, die Werte h"aufen sich im Bereich gro"ser Distanzen, in der N"ahe des Bereichs der Erwartungswerte
f"ur Distanzen zwischen vollkommen unkorrelierten Genomen. Das System tendiert also stark in Richtung Randomisierung,
was sich in der Gr"o"se der gemessenen DVK-Werte deutlich niederschl"agt. Derselbe Effekt, der die Evolution
von Pflanzenformen mit erh"ohten Fitnesswerten verhindert, f"uhrt also auch zu deutlich erniedrigten DVK-Werten.


\subsubsection{Lauf \runname{xlong1008}: Hohe Mutationsraten und scharfe Selektion}
\label{xlong1008section}

\begin{figure}

\unitlength1cm
\begin{picture}(16,20)
\put(0,0){\makebox(16,20){\epsfxsize=16cm \epsffile{xlong1008mx.eps}}}
\end{picture}
\caption{\label{xlong1008results}
LindEvol-GA Lauf \runname{xlong1008} mit hohen Mutationsraten (0.1) und scharfer Selektion (0.8).
}
\end{figure}

% \begin{figure}[p]
% 
% \unitlength1cm
% \begin{picture}(16,20)
% \put(0,18.0){\makebox(16,2)[b]{\epsfxsize=10cm \epsffile{xlong1008-0000050w.eps}}}
% \put(0,17.5){\makebox(16,0.5){{\bfseries a:} Generation 50}}
% \put(0,15.5){\makebox(16,2)[b]{\epsfxsize=10cm \epsffile{xlong1008-0000500w.eps}}}
% \put(0,15){\makebox(16,0.5){{\bfseries b:} Generation 500}}
% \put(0,13.0){\makebox(16,2)[b]{\epsfxsize=10cm \epsffile{xlong1008-0001400w.eps}}}
% \put(0,12.5){\makebox(16,0.5){{\bfseries c:} Generation 1400}}
% \put(0,10.5){\makebox(16,2)[b]{\epsfxsize=10cm \epsffile{xlong1008-0001700w.eps}}}
% \put(0,10){\makebox(16,0.5){{\bfseries d:} Generation 1700}}
% \put(0,8.0){\makebox(16,2)[b]{\epsfxsize=10cm \epsffile{xlong1008-0002200w.eps}}}
% \put(0,7.5){\makebox(16,0.5){{\bfseries e:} Generation 2200}}
% \put(0,5.5){\makebox(16,2)[b]{\epsfxsize=10cm \epsffile{xlong1008-0002400w.eps}}}
% \put(0,5){\makebox(16,0.5){{\bfseries f:} Generation 2400}}
% \put(0,3.0){\makebox(16,2)[b]{\epsfxsize=10cm \epsffile{xlong1008-0003000w.eps}}}
% \put(0,2.5){\makebox(16,0.5){{\bfseries g:} Generation 3000}}
% \put(0,0.5){\makebox(16,2)[b]{\epsfxsize=10cm \epsffile{xlong1008-0004999w.eps}}}
% \put(0,0){\makebox(16,0.5){{\bfseries h:} Generation 4999}}
% \end{picture}
% 
% \caption{\label{xlong1008worlds}
% Gesamtaufnahmen der Welt in verschiedenen Abschnitten des Laufs \runname{xlong1008}. Es wurde jeweils
% die gesamte Welt am Ende der Vegetationsperiode dargestellt.
% }
% \end{figure}

\begin{figure}

\unitlength1cm
\begin{picture}(16,20)
\put(0,14.0){\makebox(7,4)[b]{\epsfysize=4cm \epsffile{xlong1008-0000050p.eps}}}
\put(0,13.5){\makebox(7,0.5){{\bfseries a:} Generation 50}}
\put(8,14.0){\makebox(7,4)[b]{\epsfysize=4cm \epsffile{xlong1008-0000500p.eps}}}
\put(8,13.5){\makebox(7,0.5){{\bfseries b:} Generation 500}}
\put(0,9.5){\makebox(7,4)[b]{\epsfysize=4cm \epsffile{xlong1008-0001400p.eps}}}
\put(0,9.0){\makebox(7,0.5){{\bfseries c:} Generation 1400}}
\put(8,9.5){\makebox(7,4)[b]{\epsfysize=4cm \epsffile{xlong1008-0001700p.eps}}}
\put(8,9.0){\makebox(7,0.5){{\bfseries d:} Generation 1700}}
\put(0,5.0){\makebox(7,4)[b]{\epsfysize=4cm \epsffile{xlong1008-0002200p.eps}}}
\put(0,4.5){\makebox(7,0.5){{\bfseries e:} Generation 2200}}
\put(8,5.0){\makebox(7,4)[b]{\epsfysize=4cm \epsffile{xlong1008-0002400p.eps}}}
\put(8,4.5){\makebox(7,0.5){{\bfseries f:} Generation 2400}}
\put(0,0.5){\makebox(7,4)[b]{\epsfysize=4cm \epsffile{xlong1008-0003000p.eps}}}
\put(0,0.0){\makebox(7,0.5){{\bfseries g:} Generation 3000}}
\put(8,0.5){\makebox(7,4)[b]{\epsfysize=4cm \epsffile{xlong1008-0004999p.eps}}}
\put(8,0.0){\makebox(7,0.5){{\bfseries h:} Generation 4999}}
\end{picture}

\caption{\label{xlong1008genomes}
Genome von typischen Pflanzen aus verschiedenen Generationen in Lauf \runname{xlong1008}.  Oben ist jeweils
der Ph"anotyp der Pflanze dargestellt, darunter ihr Genom in graphischer Form.
}
\end{figure}

Lauf \runname{xlong1008} wurde mit hohen Mutationsraten von 0.1 und der scharfen Selektionsrate von 0.8
durchgef"uhrt. Bei diesem Lauf kommt es trotz der hohen Mutationsraten zur dauerhaften Etablierung
unbeschr"ankt wachsender Formen mit komplexer Morphologie (vgl.\ Abb.\ \ref{xlong1008worlds},
Seite \pageref{xlong1008worlds}). Im Vergleich
zu den bisher besprochenen L"aufen kommt es in \runname{xlong1008} zu einer deutlichen Evolution negativer
Mutationsexponenten, die Mutationsraten werden von den Pflanzen also durch Ausf"uhrung des
Befehls \verb|mut-| aktiv gesenkt.

Anhand der Abbildungen \ref{xlong1008worlds} und \ref{xlong1008genomes} k"onnen die einzelnen evolution"aren
Schritte in \runname{xlong1008}
verfolgt werden. Zun"achst treten vertikal wachsende Formen auf (Abb.\ \ref{xlong1008worlds}a), die etwa
bei Generation 400 durch diagonal nach links wachsende Formen verdr"angt werden (Abb.\ \ref{xlong1008worlds}b).
Sowohl bei den vertikal wachsenden als auch die diagonal wachsenden Formen ist stets ein gro"ser Anteil
von einzelligen oder beschr"ankt wachsenden Pflanzen in der Population zu beobachten. Die hohen Mutationsraten
f"uhren dazu, da"s bei einem gro"sen Teil der Nachkommenschaft der unbeschr"ankt wachsenden Formen das im
Genom codierte Wachstumsprogramm durch Mutationen ver"andert wird. Aus Abb\ \ref{xlong1008genomes}
ist ersichtlich, da"s sowohl das Entwicklungsprogramm der vertikal wachsenden als auch dasjenige
der diagonal wachsenden Form nur zwei Gene umfa"st. Nach Gl.\ \ref{prgunchangedm-eq}
ist bei den Mutationsraten von 0.1 die Wahrscheinlichkeit, da"s ein solches aus zwei Genen bestehendes
Entwicklungsprogramm beim Mutationsschritt nicht von mindestens einer Mutation betroffen wird,
$P_E(0.1,2) = (1 - 0.1)^{6 \cdot 2} \approx 0.282$. Dieser Wert stimmt gut mit der Beobachtung "uberein, da"s in
Abb.\ \ref{xlong1008worlds}a und \ref{xlong1008worlds}b jeweils nur rund 30\% der Population den
unbeschr"ankt wachsenden Ph"anotyp aufweisen. Diese "Ubereinstimmung ist ein Beleg f"ur die Praxisg"ultigkeit
der in (\ref{mutconsequences}) abgeleiteten Absch"atzungen. Der geringe Prozentsatz an unbeschr"ankt wachsenden Pflanzen
erkl"art auch den ausgesprochen niedrigen Wert der durchschnittlichen
Fitness.

\begin{sloppypar}
Die Ausbreitungszahl eines Entwicklungsprogramms aus zwei Genen betr"agt in \runname{xlong1008}
$A_E(0.1,0.8,2) \approx 1.41$, daher ist die Ausbreitung und Etablierung der elementaren unbeschr"ankt
wachsenden Formen m"oglich. Bei einem Entwicklungsprogramm mit drei Genen ist die Ausbreitungszahl dagegen
mit $A_E(0.1, 0.8, 3) \approx 0.75$ kleiner als 1.0; dies erkl"art, weshalb es nicht, wie in den L"aufen mit
moderaten Mutationsraten, nach dem Auftreten der ersten unbeschr"ankt wachsenden Formen bald die Evolution
komplexerer, unbeschr"ankt wachsender Formen einsetzt, obwohl
in Abb.\ \ref{xlong1008worlds}a ein Ansatz hierzu erkennbar ist.
\end{sloppypar}

Die einfachen, diagonalen Formen dominieren "uber eine ausgedehnte Periode bis etwa zur Generation 1400.
Hier kommt es zu einem signifikanten Anstieg der mittleren Fitness, der mit einem deutlichen Ausschlag
der Durchschnittswerte der Mutationsexponenten in den negativen Bereich verbunden ist.
Abb.\ \ref{xlong1008worlds}c zeigt, da"s die morphologische Komplexit"at der Pflanzen im Vergleich zu
der elementaren, diagonalen Form zugenommen hat. Gleichzeitig ist auch die Dichte der unbeschr"ankt
wachsenden Pflanzen gr"o"ser. Abb.\ \ref{xlong1008genomes}c zeigt die bei diesem evolution"aren Sprung
neu aufgetretene Strategie. Die Gene 15 und 8 sind f"ur die Ausbildung des prim"aren, diagonalen
Sprosses zust"andig. Gen 27 bewirkt die Auflagerung einer zweiten Zellschicht. Gen 29 dient dazu, diese
aufgelagerte Zellschicht durch Teilungen von Zellen dieser Schicht selbst zu vervollst"andigen.

Wenn die Zelle, die Gen 29 exprimiert, auf gleicher
H"ohe mit der apikalen Zelle des diagonal wachsenden Sprosses ist, ver"andert dies die Nachbarschaft
der apikalen Zelle, so da"s das Wachstum des Prim"arsprosses zum Erliegen kommt. In diesem Fall wird
jedoch in der bei diesem Vorgang erzeugten Tochterzelle wiederum Gen 29 exprimiert, was dann zur
Entstehung einer neuen apikalen Zelle f"uhrt. Auf diese Weise kommt der Knick im oberen Bereich
der in Abb.\ \ref{xlong1008genomes}c dargestellten Pflanze zustande. Das Gen 29 hat also zwei
unterschiedliche Funktionen, es spielt sowohl beim Wachstum der aufgelagerten Zellschicht als auch
bei einem recht komplexen Proze"s der Verlagerung der Hauptspro"sachse eine Rolle. Derartige
Mehrfachfunktionen wurden auch bei verschiedenen molekularen Genen, die die Morphogenese
bei Pflanzen steuern, festgestellt. F"ur die in der Simulation beobachteten Pflanzen hat die
Wuchsform mit gelegentlichen Knicken den Vorteil, da"s bei einem Knick jeweils die unterste Zelle,
die dem Hauptspro"s aufgelagert wird, Energie speichert, anstatt die Aktion \verb|mut-| auszuf"uhren.
Dies bringt einen geringen Fitnessvorteil, der jedoch bei der scharfen Selektionsrate von 0.8
entscheidend sein kann.

Die Zellen in der aufgelagerten Zellschicht sind nicht inaktive Energiespeicher, wie dies in
\runname{xlong0105} beobachtet wurde. Stattdessen f"uhrt Gen 23 dazu, da"s diese Zellen die
Aktion \verb|mut-| ausf"uhren. Wie aus der Kurve der minimalen Mutationsexponenten in Abb.\ \ref{xlong1008results}
ersichtlich ist, liegen die mit dieser Strategie erreichten Minimalwerte des Mutationsexponenten bei
Werten um -5, die effektiven Mutationsraten von Genomen, die diesen Wert erzielen, betragen nur
$0.1 \cdot 2^{-5} = 0.003125$. Da die Zellen der unteren Zellkette Energie speichern, liegt der
Fitnesswert bei dieser Strategie in derselben Gr"o"senordnung wie bei der einfachen diagonalen
Form. Der Pflanzentyp mit einer aufgelagerten Zellschicht, deren Energiegewinn zur Senkung der
effektiven Mutationsraten eingesetzt wird, kann also mit den einfachen diagonalen
Formen bei der Selektion konkurrieren, er besitzt aber keinen relevanten Selektionsvorteil.
Bei den stark verringerten effektiven Mutationsraten von 0.003125 ist die Ausbreitungszahl
$A_E(0.003125, 0.8, 5) \approx 4.55$, also erheblich gr"o"ser als die der einfachen diagonalen
Form, die 1.41 betr"agt (s.o.).
Obwohl die mutationsratensenkenden Formen also keinen Selektionsvorteil im Sinne "uberlegener
Fitnesswerte aufweisen, erzeugen sie in der Folgegeneration eine gr"o"sere Anzahl von
Nachkommen, bei denen das Entwicklungsprogramm erhalten ist, und setzen sich daher gegen
die einfachen diagonalen Formen durch.

Die Phase der Dominanz mutationsratensenkender Formen, die bei Generation 1400 beginnt, geht
etwa bei Generation 1650 zu Ende. Die Ursachen f"ur diese Entwicklung konnten nicht gekl"art 
werden. Bei Generation 1700 sind keine Pflanzen mit komplexerer Form mehr in der Population
vorhanden, es dominieren, wie vor Generation 1400, elementare, diagonal wachsende Pflanzen
(Abb.\ \ref{xlong1008worlds}d und \ref{xlong1008genomes}d).

Bei Generation 2200 kommt es zu einem erneuten Auftreten von komplexen Wuchsformen
(Abb.\ \ref{xlong1008worlds}e), die aktiv die effektive Mutationsrate erniedrigen. Die zun"achst
auftretenden Formen (Abb.\ \ref{xlong1008genomes}e) "ahneln vom Prinzip her den bei Generation
1400 beobachteten Formen, sie bilden jedoch eine unmittelbar dem Hauptspro"s aufgelagerte
Zellschicht aus, und sie erzeugen etwas mehr zus"atzliche energiespeichernde Zellen, denn der
Vorgang, der zur Bildung einer zus"atzlichen energiespeichernden Zelle f"uhrt, kann 
nicht nur an der Spitze, sondern "uberall am Spro"s ablaufen.

Die mutationsratensenkenden Formen etablieren sich von Generation 2200 an stabil in der Population.
Sie entwickeln sich weiter zu buschigen Formen (Abb.\ \ref{xlong1008worlds}f und \ref{xlong1008genomes}f),
bei denen nur die am Rand gelegenen Zellen Energie speichern, w"ahrend s"amtliche im Innern der
Pflanze gelegenen Zellen der Senkung der effektiven Mutationsraten dienen (Gen 8). Eine andere
Spielart dieser Strategie kann bei Generation 3000 beobachtet werden (Abb.\ \ref{xlong1008worlds}g und \ref{xlong1008genomes}g),
hier werden zus"atzlich durch die Zellen an der Oberfl"ache der buschartigen Zellmasse durch die
Aktivit"at des Gens 5 die effektiven Mutationsraten gesenkt, w"ahrend das Gen 3 zur Ausbildung
diagonaler Seitentriebe aus energiespeichernden Zellen dient. Bei Generation 4999
(Abb.\ \ref{xlong1008worlds}h und \ref{xlong1008genomes}h) schlie"slich kann eine verzweigt wachsende
Form beobachtet werden, in der Triebe mit mehrfacher Zelldicke der Energiespeicherung und Seitentriebe
einfacher Dicke der Senkung der effektiven Mutationsraten dienen.

In allen Phasen, in denen komplexe, mutationsratensenkende Pflanzenformen auftreten, kommt
es zu einem deutlichen Anstieg der DVK. Auch bei den Distanzverteilungen k"onnen die
Auswirkungen der Absenkung der effektiven Mutationsraten beobachtet werden. In den Phasen,
in denen der durchschnittliche Mutationsexponent um 0 schwankt, sind die Distanzwerte massiv
im oberen Bereich konzentriert, die Distanzverteilungen "ahneln denen, die in \runname{xlong1005}
beobachtet wurden. Bei den Ausschl"agen des durchschnittlichen Mutationsexponenten in den negativen
Bereich ist jeweils eine erheblich gleichm"a"sigere Verteilung der Distanzwerte zu beobachten.
Bei Generation 3500 wurden die niedrigsten Werte des durchschnittlichen Mutationsexponenten gemessen.
Hier kommt es zu einem deutlichen R"uckgang der DVK, der, wie die Distanzverteilungen zeigen,
auf eine starke H"aufung der Distanzwerte im kleinen Bereich zur"uckgeht. Diese Beobachtungen
zeigen, da"s die Evolution komplexer Wuchsformen und die Absenkung der effektiven Mutationsraten
zuverl"assig anhand des Auftretens signifikant erh"ohter DVK-Werte charakterisiert werden k"onnen.



\subsubsection{Lauf \runname{xlong00105}: Geringe Mutationsraten und moderate Selektion}
\label{xlong00105section}

\begin{figure}

\unitlength1cm
\begin{picture}(16,17)
\put(0,0){\makebox(16,17){\epsfxsize=16cm \epsffile{xlong00105.eps}}}
\end{picture}
\caption{\label{xlong00105results}
LindEvol-GA Lauf \runname{xlong00105} mit geringen Mutationsraten (0.001) und moderater Selektion (0.5).
}
\end{figure}

Der Verlauf der numerischen Werte, die bei dem mit den geringen Mutationsraten von 0.001 und der moderaten
Selektionsrate von 0.5 durchgef"uhrten Lauf \runname{xlong00105} ermittelt wurden, sind in Abb.\ \ref{xlong00105results}
dargestellt. Der Verlauf der Fitnesswerte zeigt zwei deutliche evolution"are Spr"unge. Der erste Sprung kommt durch
das Auftreten unbeschr"ankt wachsender, diagonaler Formen zustande. Der zweite Sprung, etwa bei Generation 750,
r"uhrt von einem "Ubergang von einfachen diagonalen Formen zu Formen mit aufgelagerten Zellschichten her.
Vergleichbare evolution"are Spr"unge
waren auch in \runname{xlong0105} zu beobachten, insbesondere der "Ubergang zu Formen mit aufgelagerten Zellschichten
vollzog sich bei den zehnfach h"oheren Mutationsraten erheblich schneller.

Die diagonal wachsenden Formen mit aufgelagerten
Zellschichten dominieren bis zum Ende des Laufs bei Generation 4999. Wie aus
Abb.\ \ref{xlongworlds}a (Seite \pageref{xlongworlds}) ersichtlich
ist, entsprechen diese Formen im Prinzip denjenigen, die bei Generation 2400 in \runname{xlong0105} beobachtet
wurden. Die Fitnesswerte pendeln sich in "ahnlichen Bereichen wie in \runname{xlong0105} ein, wobei die Schwankungsbreite
geringer ist. Zu keinem Zeitpunkt treten buschf"ormig wachsende oder verzweigte Formen auf. Eine Evolution der
aktiven Beeinflussung der effektiven Mutationsraten ist nicht zu beobachten.

Bei den geringen Mutationsraten von 0.001 haben nicht nur Entwicklungsprogramme, sondern auch Genome der in
\runname{xlong00105} beobachteten L"angen Ausbreitungszahlen, die "uber 1 liegen; bei einer Genoml"ange von
50 Genen betr"agt die Ausbreitungszahl $A(0.001, 0.5, 100) \approx 1.48$. Dies f"uhrt dazu, da"s von vielen
"uberlebenden Genotypen einer Generation mehr als eine unver"anderte Kopie in der n"achsten Generation exisitieren,
so da"s die Anzahl verschiedener Genotypen stets deutlich kleiner als die Populationsgr"o"se ist. Dementsprechend
bleibt auch die genetische Diversit"at unterhalb des Maximalwerts von 3.91. Mit dem Anstieg der Genoml"angen
im Verlauf der Simulation steigen die Anzahl verschiedener Genotypen und die genetische Diversit"at, was durch
ein entsprechendes Absinken der Ausbreitungszahl erkl"art werden kann.

Wie auch in den anderen L"aufen, bei denen keine aktive Mutationsratenver"anderung beobachtet wurde, pendelt
sich auch in \runname{xlong00105} die DVK in einem Bereich ein, der sich w"ahrend des Laufs nicht ver"andert.
Die DVK liegt deutlich niedriger als in \runname{xlong0105}, die Schwankungen sind jedoch st"arker als in \runname{xlong0105}.
In der Abfolge der Distanzverteilungen ist eine ausgepr"agte Wellenstruktur erkennbar, die bei den bisher
diskutierten Ergebnissen nicht beobachtet wurde. Der Wellenverlauf ist deutlich mit den DVK-Schwankungen
korreliert.

Die Wellen kommen durch die Divergenz von taxonomischen Clustern zustande. Die Genome eines Clusters codieren
typischerweise dasselbe Entwicklungsprogramm, sie unterscheiden sich jedoch in den inaktiven Bereichen. Mit
fortschreitendem Alter eines solchen Clusters w"achst die Anzahl der Unterschiede in den inaktiven Bereichen,
und dementsprechend nimmt der Maximalwert der Distanz von Genomen innerhalb des Clusters zu. Somit wandert
der Peak, der durch die Distanzen zwischen Genomen eines Clusters zustandekommt (vgl.\ \ref{ddistr-method}),
immer weiter in den Bereich h"oherer Distanzwerte. Wenn schlie"slich ein Subcluster innerhalb des divergierenden
Clusters s"amtliche Schwestergruppen vollst"andig verdr"angt hat, verschwindet der entsprechende Peak, und die
Welle endet. Damit verschwindet auch ein St"uck Vielfalt aus dem Spektrum der Distanzwerte, was einen R"uckgang
der DVK nach sich zieht.

Anders als bei den bisher besprochenen L"aufen treten Distanzwerte in der N"ahe des aufgrund der Genoml"ange zu
erwartenden Maximums in \runname{xlong00105} nur ganz selten auf. Die meisten Wellen enden, bevor sie den Bereich
der H"alfte dieses Maximalwerts erreichen. Aus dieser Beobachtung ist zu folgern, da"s die Population in
einem kleinen Teilraum des Sequenzraums konzentriert ist. Zu jedem Zeitpunkt ist somit ein erheblicher Teil
des gesamten Sequenzraums von allen Genomen der Population aus nur nach einer gr"o"seren Zahl von Mutationsschritten
erreichbar, so da"s dieser Teil der Exploration durch den genetischen Algorithmus nicht unmittelbar zug"anglich ist.


\subsubsection{Lauf \runname{xlong00108}: Geringe Mutationsraten und scharfe Selektion}
\label{xlong00108section}

\begin{figure}

\unitlength1cm
\begin{picture}(16,17)
\put(0,0){\makebox(16,17){\epsfxsize=16cm \epsffile{xlong00108.eps}}}
\end{picture}
\caption{\label{xlong00108results}
LindEvol-GA Lauf \runname{xlong00108} mit moderaten Mutationsraten (0.001) und scharfer Selektion (0.8).
}
\end{figure}

\begin{figure}

\unitlength1cm
\begin{picture}(16,5)
\put(4,0.5){\makebox(7,4)[b]{\epsfysize=4cm \epsffile{xlong00108-0004999p.eps}}}
\put(4,0.0){\makebox(7,0.5){{\bfseries b:} \runname{xlong00108}, Generation 4999}}
\end{picture}

\caption{\label{xlong00108genomes}
Eine typische Pflanze aus \runname{xlong00108} mit ihrem Genom.
}
\end{figure}

Als letzte Einzelsimulation wird nun der Lauf \runname{xlong00108} diskutiert, der mit geringen Mutationsraten von
0.001 und einer scharfen Selektionsrate von 0.8 durchgef"uhrt wurde. Abb.\ \ref{xlong00108results} zeigt die Verlaufsdaten
dieses Laufs. Im Vergleich zu den restlichen Einzell"aufen f"allt in Abb.\ \ref{xlong00108results} die lange Anfangsperiode
beschr"ankten Wachstums ins Auge. Auch in \runname{xlong00105} wurde beobachtet, da"s die Perioden evolution"arer Stasis
im Vergleich zu L"aufen mit h"oheren Mutationsraten stark verl"angert sind. Die Absenkung der Mutationsraten f"uhrt zu
einer starken Verl"angerung der Zeit, die bis zum Auftreten einer Schl"usselmutation, die einen evolution"aren Sprung
verursacht, verstreicht.

Trotz der geringen Mutationsraten kann es jedoch auch zu einer schnellen Abfolge evolution"arer Spr"unge kommen.
Beispielsweise erscheinen fast unmittelbar nach dem "Ubergang zu unbeschr"anktem Wachstum Pflanzen mit aufgelagerten
Zellen. Im weiteren Verlauf der Simulation entwickeln sich auffallende buschartige Formen mit einem Hauptspro"s, der
sich aus diagonal aneinandergesetzten Teilst"ucken aus jeweils zwei vertikal "ubereinanderliegenden Zellen zusammensetzt
(Abb.\ \ref{xlong00108genomes} und \ref{xlongworlds}b, Seite \pageref{xlongworlds}). Dieser Spro"styp ist auf genetischer Ebene bemerkenswert, weil
zu seiner Codierung zwei Gene, die alternierend das apikale Wachstum steuern (Gene 39 und 40) erforderlich sind. Somit
erfordert dieser Spro"styp ein h"oheres Ma"s genetischer Komplexit"at als die bisher beobachteten vertikalen oder diagonalen
Spro"stypen, bei denen jeweils nur ein Gen zur Steuerung des apikalen Wachstums erforderlich ist.

Weiterhin sind an der in Abb.\ \ref{xlong00108genomes} gezeigten Pflanze die aus dem Hauptspro"s nach rechts unten
ausgewachsenen Zellen auffallend. Sie sind s"amtlich energielos, sie spielen also keine Rolle als
Energiespeicher. Es ist jedoch denkbar, da"s sie verhindern, da"s der Hauptspro"s der Pflanze von diagonalen Seitentrieben
der rechten Nachbarpflanze durchdrungen wird. Auch die vertikal-diagonale Mischform des Hauptspro"swachstums kann als
Strategie zur Verbindung des Schutzes gegen Durchdringung durch andere Pflanzen, den ein vertikaler Spro"s bietet, und der
Vergr"o"serung der Anzahl der dem Licht ausgesetzten Zellen, die durch diagonales Wachstum erm"oglicht wird, gedeutet
werden.

Auch in \runname{xlong00108} sind, wie in \runname{xlong00105}, die Ausbreitungszahlen von Genomen gr"o"ser als 1.
Daher sind in keiner Generation der Simulation alle Genome unterschiedlich. In der Anfangsphase liegt die Anzahl
unterschiedlicher Genotypen unter 10, sp"ater steigt sie mit anwachsenden Genoml"angen deutlich an. In entsprechender
Weise nimmt auch die genetische Diversit"at zu.

Ein dauerhafter evolution"arer Trend zur aktiven Modifikation der effektiven Mutationsraten ist nicht zu beobachten.
Wie auch in den anderen L"aufen, in denen keine signifikante Mutationsratenmodifikation auftritt, pendelt sich auch
in \runname{xlong00108} die DVK auf einem Wert ein, der sich im Laufe der Simulation nicht wesentlich ver"andert.
In der Anfangsphase sind allerdings ein leicht verringerter charakteristischer DVK-Wert und erheblich st"arkere
Schwankumgen zu beobachten als w"ahrend der restlichen Simulation. Dies ist auf Diskretisierungseffekte zur"uckzuf"uhren.
W"ahrend der ersten 500 Generationen bewegen sich die Genoml"angen im Bereich der initialen Genoml"ange von 20 Genen.
Dar"uberhinaus liegen minimale und maximale Genoml"ange sehr dicht beieinander. Dies hat, wie bereits in
(\ref{c-xlong0105section}) beschrieben wurde, zur Folge, da"s die Anzahl der insgesamt m"oglichen Werte des
relativen Editierabstands kleiner wird als die Anzahl der Diskretisierungsintervalle, was zu einem R"uckgang
der DVK f"uhrt. Aus der Abfolge der Distanzverteilungen ist ersichtlich, da"s gerade in der Anfangsphase oft nur
wenige Editierabstandswerte in der Distanzmatrix vertreten sind. Weiterhin existieren in der Anfangsphase durchg"angig weniger als
zehn verschiedene Genotypen in der Population. Dementsprechend ist ein gr"o"serer Anteil der Distanzwerte 0. Statistische
Fluktuationen der Gr"o"sen der Best"ande der verschiedenen Genotypen k"onnen die Verteilung der Distanzwerte,
die gr"o"ser als 0 sind, erheblich beeinflussen. Dies erkl"art die starken Schwankungen der DVK, die in der
Anfangsphase von \runname{xlong00108} auftreten.


\subsubsection{Vergleichende Diskussion der Einzelsimulationen}
\label{lndga-comparesingle}

% \begin{figure}[p]
% 
% \unitlength1cm
% \begin{picture}(16,15)
% \put(0,13.0){\makebox(16,2)[b]{\epsfxsize=10cm \epsffile{xlong00105-0004999w.eps}}}
% \put(0,12.5){\makebox(16,0.5){{\bfseries a:} Generation 4999 von Lauf \runname{xlong00105}}}
% \put(0,10.5){\makebox(16,2)[b]{\epsfxsize=10cm \epsffile{xlong00108-0004999w.eps}}}
% \put(0,10.0){\makebox(16,0.5){{\bfseries b:} Generation 4999 von Lauf \runname{xlong00108}}}
% \put(0,8.0){\makebox(16,2)[b]{\epsfxsize=10cm \epsffile{xlong0105-0004999w.eps}}}
% \put(0,7.5){\makebox(16,0.5){{\bfseries c:} Generation 4999 von Lauf \runname{xlong0105}}}
% \put(0,5.5){\makebox(16,2)[b]{\epsfxsize=10cm \epsffile{xlong0108-0004999w.eps}}}
% \put(0,5.0){\makebox(16,0.5){{\bfseries d:} Generation 4999 von Lauf \runname{xlong0108}}}
% \put(0,3.0){\makebox(16,2)[b]{\epsfxsize=10cm \epsffile{xlong1005-0004999w.eps}}}
% \put(0,2.5){\makebox(16,0.5){{\bfseries e:} Generation 4999 von Lauf \runname{xlong1005}}}
% \put(0,0.5){\makebox(16,2)[b]{\epsfxsize=10cm \epsffile{xlong1008-0004999w.eps}}}
% \put(0,0){\makebox(16,0.5){{\bfseries f:} Generation 4999 von Lauf \runname{xlong1008}}}
% \end{picture}
% 
% \caption{\label{xlongworlds}
% Aufnahmen der Welt jeweils am Ende der Simulationsl"aufe.
% }
% \end{figure}

\paragraph{Emergente Evolution kooperativen Wachstumsverhaltens}
In den Simulationsl"aufen, die in den Abschnitten \ref{xlong0105section} bis \ref{xlong00108section}
besprochen wurden, wurde die Evolution einer Vielzahl unterschiedlicher Pflanzenformen und Entwicklungsstrategien
beobachtet. Die Entstehung und die evolution"are Stabilisierung dieser Pflanzenformen wurde durch vielf"altige,
komplexe Interaktionen zwischen den Pflanzen in der Welt beeinflu"st. Durch den Ansatz, die Fitness auf
der Grundlage der Simulation einer Vegetationsperiode zu bewerten, ist es gelungen, intrinsische Adaptationsvorg"ange
\cite{Packard89} in einem genetischen Algorithmus zu simulieren. Die Umwelt, die die "Uberlebens- und Vermehrungschancen
eines Individuums bestimmt, hat in LindEvol-GA keinen statischen Charakter, sondern sie wird durch die Pflanzen in
der Population gestaltet. Mit einer Fitnessfunktion im traditionellen Sinne, die die
Fitness eines Individuums unabh"angig von den restlichen Individuen in der Population bewertet, kann dies
nicht simuliert werden.

Bei der intrinsischen Adaptation in LindEvol-GA ist nicht nur die H"ohe der Fitness, die eine Wuchsform
erzielen kann, entscheidend. Vielmehr spielt die F"ahigkeit einer Pflanze, diesen Fitnesswert im gemeinsamen Wachstum
mit den anderen Pflanzen in der Welt zu erreichen, eine ma"sgebliche Rolle. Dies f"uhrt zu einer evolution"aren
Stabilisierung kooperativen Wachstumsverhaltens. Dieser Effekt ist bei geringeren Selektionsraten st"arker,
wie auch bei der vergleichenden Betrachtung der am Ende der besprochenen L"aufe evolvierten Gemeinschaften
(Abb.\ \ref{xlongworlds}, Seite \pageref{xlongworlds}) erkennbar wird. Die Abbildung zeigt, da"s sich in allen drei L"aufen mit scharfer
Selektion buschartige Pflanzenformen durchgesetzt haben. Bei moderater Selektion bewiesen die diagonal
wachsenden Formen mit mehreren Zellschichten die gr"o"sere evolution"are Stabilit"at. In \runname{xlong0105}
(\ref{xlong0105section}) traten buschf"ormig wachsende Pflanzen auf, die sich nicht durchsetzen konnten
und wieder verschwanden. Ebenso konnten sich w"ahrend einer ausgedehnten Phase der Dominanz vertikaler
Formen keine diagonalen Formen durchsetzen; die diagonalen Formen hatten in einer von vertikal wachsenden
Pflanzen dominierten Welt ein hohes Kollisionsrisiko und schlechte Wachstumschancen, obwohl sie in einer
diagonal wachsenden Gemeinschaft h"ohere Fitnesswerte erreichen als die vertikal wachsenden Pflanzen.

Die gr"o"sere Bedeutung kooperativen Wachstumsverhaltens bei niedrigerer Selektionsrate kann durch 
Unterschiede in der Ausbreitungsdynamik erkl"art werden. Bei der hohen Selektionsrate von 0.8 ist 
ein herausragender Fitnesswert erforderlich, um das "Uberleben eines Genoms zu sichern. Dies kann durch
eine buschf"ormige Pflanze am besten erreicht werden. Bei niedrigen und moderaten Mutationsraten
sind die Ausbreitunszahlen typischer Entwicklungsprogramme gr"o"ser als 4. Daher kann ein Entwicklungsprogramm
evolution"ar stabil sein, wenn bis zu
drei von vier Pflanzen mit diesem Entwicklungsprogramm im statistischen Mittel w"ahrend einer
Vegetationsperiode durch Kollisionen soviel Energie verlieren und in ihrem Wachstum eingeengt
werden, da"s ihre Genome bei der Selektion nicht "uberleben.

Bei der moderaten Selektionsrate von 0.5 bringt ein Fitnesswert im Spitzenbereich keine signifikanten
Vorteile bei der Selektion; ein leicht "uberdurchschnittlicher Fitnesswert reicht zur Sicherung des
"Uberlebens aus. Da von einem "uberlebenden Genom nach der Selektion im statistischen Mittel
zwei Kopien existieren (Gl.\ \ref{numoffspring-eq}), sind bei einer Selektionsrate von 0.5
alle Ausbreitungszahlen kleiner als 2. Nun kann in einer Gemeinschaft diagonaler Pflanzen eine
buschf"ormig wachsende Pflanze nur dann einen "uberdurchschnittlichen Fitnesswert erzielen,
wenn es ihr gelingt, mindestens einen ihrer Nachbarn zu "uberwuchern. Die Chance, da"s dies
gelingt, betr"agt ungef"ahr 1:1. W"achst ihr Nachbar
schneller, kann sich die Pflanze nicht buschf"ormig entfalten. Sie wird in einem diagonalen Bereich
eingeengt und vergeudet in signifikantem Ausma"s Energie, da es bei Teilungsversuchen
zur buschf"ormigen Ausdehnung zu Kollisionen mit ihrem Nachbarn kommt. Der Fitnesswert einer
solchen "`eingeklemmten"' Pflanze ist daher unterdurchschnittlich. Somit "uberleben rund die H"alfte
der Kopien des Entwicklungsprogramms buschf"ormiger Pflanzen den Selektionsschritt nicht. Der statistische
Mittelwert der Anzahl "uberlebender Kopien dieses Entwicklungsprogramms liegt damit bei der H"alfte
der Ausbreitungszahl, und somit unter 1, weil die Ausbreitungszahl kleiner als 2 ist. Ein buschf"ormiges
Entwicklungsprogramm kann sich also bei moderater Selektion nicht ausbreiten, weil die Nachteile
bei Kollisionen die Vorteile bei freier Entfaltung der Buschform "uberwiegen. Diese evolution"are
Stabilisierung kooperativen Wachstumsverhaltens stellt ein emergentes Ph"anomen dar. Da es anhand von
Absch"atzungen erkl"art, nicht jedoch aus der Spezifikation von LindEvol-GA abgeleitet werden kann,
kann dieses Ph"anomen nach \cite{Assad92} als stark emergent ({\slshape "`strongly emergent"'}) eingestuft werden.
"Ahnliche Formen emergenter Evolution kooperativen Verhaltens wurden in der Vergangenheit bei verschiedenen Simulationen
auf der Basis des \textsl{prisoner's dilemma} beobachtet \cite{Lindgren92,May95}.

\paragraph{Evolution regulatorischer Netzwerke}
Die Aktivierung von Genen erfolgt im Modell LindEvol-GA durch Aktivit"aten anderer Gene. Diese Aktivierungsbeziehungen
zwischen Genen bilden in ihrer Gesamtheit ein regulatorisches Netzwerk, das mithilfe der in (\ref{blockinterdef})
entwickelten Darstellungsmethode visualisiert werden kann. Auch die Enstehung dieser Netzwerke ist ein
emergentes Ph"anomen. Regulatorische Netzwerke werden derzeit
auch in der Molekulargenetik als Schl"usselmechanismen der genetischen Steuerung der Morphogenese beschrieben
\cite{Theissen95}. In den molekulargenetischen Netzwerken wurde in mehreren F"allen beobachtet, da"s ein
Faktor bei mehreren morphogenetischen Prozessen eine Rolle spielt. Auch in LindEvol-GA k"onnen Gene beobachtet
werden, die an unterschiedlichen Wachstumsvorg"angen aktiv beteiligt sind.
Ein Beispiel hierf"ur wurde in der Diskussion der Generation 1400 von \runname{xlong1008}
beschrieben. Diese Parallelen zwischen molekulargenetischen Systemen und den Modellsystemen in LindEvol auf
einer emergenten Ebene belegen, da"s die Repr"asentation entscheidender Faktoren der Morphogenese in dem
Modell LindEvol-GA gelungen ist. In LindEvol-GA entstehen komplexe regulatorische Netzwerke zur Steuerung
der Morphogenese in einem Evolutionsproze"s. Dies ist ein deutliches Indiz daf"ur, da"s die derzeit diskutierten
Systeme der genetischen Steuerung der Morphogenese gut mit der Annahme der evolution"aren Genese dieser Systeme
vereinbar sind.


\paragraph{Distanzverteilungsanalyse}
Beim "Uberblick "uber die L"aufe zeigt sich eine Korrelation zwischen der Evolution komplexer Wuchsformen
und dem Auftreten hoher Werte der Distanzverteilungskomplexit"at.
Die niedrigsten DVK-Werte wurden bei moderater Selektion und hohen Mutationsraten (\runname{xlong1005})
sowie bei scharfer Selektion und geringen Mutationsraten (\runname{xlong00108}) gemessen. In
\runname{xlong1005} fand keine relevante evolution"are Entwicklung statt, weil
die kritische Fehlergrenze selbst f"ur elementare Entwicklungsprogramme deutlich "uberschritten war.
In \runname{xlong00108} ist andererseits
eine starke Tendenz zur Konvergenz vorhanden, die sich in einer im Vergleich zu anderen L"aufen stark
verl"angerten Anfangsphase niederschl"agt, in der wenig komplexe, beschr"ankt wachsende Formen dominieren.

In beiden L"aufen mit moderaten Mutationsraten
wurde eine Vielzahl interessanter und komplexer evolution"arer Prozesse beobachtet. Gleichzeitig
wurden in diesen L"aufen konstant hohe DVK-Werte gemessen. In \runname{xlong1008} traten Phasen mit
signifikant unterschiedlichen DVK-Werten auf. In den Phasen niedriger DVK existieren nur elementare,
unbeschr"ankt wachsende Formen mit Entwicklungsprogrammen geringer Komplexit"at. In den Phasen mit
hoher DVK ist hingegen eine durch die aktive Absenkung der Mutationsraten erm"oglichte
Entwicklung komplexer Wuchsformen zu beobachten. Sowohl bei der Betrachtung der DVK-Werte in
\runname{xlong1008} als auch bei der vergleichenden Betrachtung der DVK-Werte in den verschiedenen
L"aufen kann also eine Korrelation zwischen effizienter Evolution komplexer Formen und hohen Werten
der DVK festgestellt werden.

Auch die Untersuchung der Distanzverteilungen erm"oglichte interessante Einblicke in die
evolution"are Dynamik. Starke Konvergenz und starke Randomisierung f"uhren zu einer Konzentration
der Distanzwerte in einem engen Bereich, somit k"onnen diese Zust"ande unmittelbar aus der
Distanzverteilung ersehen werden. Eine dauerhafte Konvergenz fand jedoch selbst bei niedrigen
Mutationsraten nicht statt. Der zeitliche Verlauf der Distanzverteilungen zeigte, da"s die
Distanzen zwischen divergierenden Clustern erhebliche Werte erreichen. Die Aufspaltung in zwei
Cluster, deren Koexistenz sich "uber einen ausreichend langen Zeitraum erstreckt, um die Entwicklung
einer signifikanten Distanz zwischen den Clustern zu erlauben, ist jedoch bei niedrigen
Mutationsraten ein seltenes Ereignis, daher f"uhrt jeder dieser Divergenzprozesse zu einer
charakteristischen Welle in der Abfolge der Distanzverteilungen. Bei den zehnfach h"oheren
moderaten Mutationsraten entfernen sich divergierende Cluster erheblich schneller voneinander.
Auch zwischen nur kurze Zeit koexistierenden Clustern entwickelt sich daher eine signifikante
Distanz. Insgesamt koexistiert daher bei moderaten Mutationsraten stets eine gro"se Menge
signifikant unterschiedlicher Cluster, was eine gleichm"a"sige Verteilung der Distanzwerte
zur Folge hat.

In allen L"aufen, in denen komplexe Pflanzenformen entstanden, wurde stets nur ein kleiner Teil der
Gene in den Genomen beim Wachstum aktiviert. Der restliche Teil des Genoms bleibt inaktiv. Sequenzbereiche
ohne biologische Funktion sind auch in der Molekulargenetik bekannt (sog.\ {\slshape junk DNA}). In diesen
Bereichen sind die meisten Mutationen neutral, d.h.\ sie bleiben ohne Auswirkungen auf den Ph"anotyp.
Somit existiert zu jedem Wachstumsprogramm eine Menge von Genomen, die dieses Programm codieren
und sich nur in den inaktiven Bereichen unterscheiden. In Analogie zu neutralen Netzen \cite{Forst95} wird eine solche
Menge als \introdef{neutrale Menge} des entsprechenden Wachstumsprogramms bezeichnet.

Die Distanzverteilungen sind in diesem Zusammenhang als Methode zu verstehen, mit der die Dynamik der
Ausbreitung der Population in einer neutralen Menge verfolgt werden kann. Eine Wellenstruktur in den
Distanzverteilungsfolgen, die bei geringen Mutationsraten auftrat,
zeigt an, da"s die Population in punktuell konzentrierten Clustern innerhalb
der neutralen Menge verteilt ist. Eine gleichm"a"sigere Verteilung der Distanzwerte, wie sie bei moderaten
Mutationsraten beobachtet wurde, zeigt dagegen eine
st"arker diffuse Verteilung der Population in der neutralen Menge an.
In \cite{Forst95} wird darauf hingewiesen, da"s nur von bestimmten Teilbereichen eines neutralen Netzes
ein "Ubergang auf ein anderes Netz m"oglich ist. "Ubertragen auf LindEvol-GA bedeutet dies, das
der "Ubergang von einem Wachstumsprogramm zu einem anderen nur aus einer bestimmten Teilmenge
der neutralen Menge des Wachstumsprogramms heraus m"oglich ist. Bei einer st"arkeren Diffusion
der Population innerhalb der neutralen Menge werden auch diese Teilmengen, von denen aus evolution"are
Schritte erfolgen k"onnen, schneller erreicht. Dies erkl"art, weshalb komplexe, unbeschr"ankt wachsende
Formen in den L"aufen mit moderaten Mutationsraten erheblich schneller auftraten als bei geringen
Mutationsraten.  Da gleichm"a"siger verteilte Distanzwerte auch zu einer h"oheren DVK f"uhren, 
beleuchtet diese Betrachtung des Evolutionsprozesses in LindEvol-GA auch die Korrelation zwischen
hohen DVK-Werten und komplexen Evolutionsprozessen.


% DVK-Ergebnisse koennen durch Diskretisierungseffekte verfaelscht werden.
% Hohe DVK notwendig, aber nicht hinreichend fuer effektive Evolution (tritt auch bei ungerichteter
%     Evolution in neutraler Kontrolle auf.
% Vergleich mit Theissen95 - regulatorische Netzwerke
% Hinweis auf beschraenkte Leistungsfaehigkeit der blockorientierten Genominterpretation.
% Multiple Genfunktionen (xlong1008genomes-c)
% Distanzverteilungen zeigen Perkolation des "neutralen Netzwerks".
% Korrelation hohe DVK / komplexe Formen.


\subsection{Systematische Untersuchung der Eigenschaften von \\
LindEvol-GA in Abh"angigkeit von Mutationsraten und Selektionsraten}
\label{lndga-parameterscan}

Bei der Diskussion der L"aufe in (\ref{lndga-individualsims}) wurde die zentrale Bedeutung der Selektionsrate und der
Mutationsraten f"ur die Dynamik des Evolutionsprozesses in LindEvol-GA an vielen Stellen deutlich. Um die
Bedeutung dieser Parameter f"ur die Evolution komplexer Strukturen systematisch zu charakterisieren, wurde
eine Serie von L"aufen durchgef"uhrt, bei der die Selektionsrate von 0.0 bis 0.9 in Schritten von 0.1
und die Mutationsraten von 0.0 bis 0.1 in Schritten von 0.01 gew"ahlt wurden.
Alle drei Mutationsraten hatten, wie auch bei den bereits besprochenen L"aufen, den gleichen Wert.
Es wurden s"amtliche
Kombinationen der 10 Werte f"ur die Selektionsrate und der 11 Werte f"ur die Mutationsraten verwendet,
insgesamt wurden somit 110 verschiedene L"aufe durchgef"uhrt.
Die restlichen Kontrollparameter
blieben dabei konstant auf denselben Werten, die auch in (\ref{lndga-individualsims}) verwendet wurden.

Jeder Lauf erstreckte sich "uber 5000 Generationen. In jedem Lauf wurden die Mittelwerte der
durchschnittlichen Fitness, der DVK, der genetischen Diversit"at, des durchschnittlichen Mutationsexponenten,
der durchschnittlichen Genoml"ange und der durchschnittlichen Anzahl aktivierter Gene jeweils "uber die
letzten 50 Generationen ermittelt. Die Ergebnisse dieses Experiments, das den Namen \runname{xlong}
tr"agt, sind in den Abbildungen \ref{xlong-fit} bis \ref{xlong-gdv} in Abh"angigkeit von den Mutationsraten ($m$)
und der Selektionsrate ($s$) dargestellt.


\begin{figure}[p]

\unitlength1cm
\begin{picture}(16,9)
\put(0,0){\makebox(16,10)[b]{\epsfxsize=14cm \epsffile{rxlong-fit.eps}}}
\end{picture}
\caption{\label{xlong-fit}
Mittelwert der durchschnittlichen Fitness in Abh"angigkeit von den Mutationsraten (m) und der Selektionsrate (s).
F"ur jedes Parameterpaar wurde in den Generationen 4950--4999 durchschnittliche Fitness der
Gesamtpopulation ermittelt. Die Graphik zeigt die Mittelwerte der 50 gemessenen Durchschnittswerte
in Abh"angigkeit von den Mutationsraten und der Selektionsrate des entsprechenden Laufs.
Die zehn gr"o"sten Werte sind zus"atzlich durch ein Rautensymbol gekennzeichnet
}
\end{figure}


\begin{figure}[p]

\unitlength1cm
\begin{picture}(16,10)
\put(0,0){\makebox(16,10)[b]{\epsfxsize=14cm \epsffile{rxlong-ent.eps}}}
\end{picture}
\caption{\label{xlong-ent}
Mittelwert der DVK in Abh"angigkeit von den Mutationsraten (m) und der Selektionsrate (s).
Die Rauten kennzeichnen die L"aufe mit den h"ochsten DVK-Werten.
}
\end{figure}


\begin{figure}

\unitlength1cm
\begin{picture}(16,10)
\put(0,0){\makebox(16,10)[b]{\epsfxsize=14cm \epsffile{rxlong-nus.eps}}}
\end{picture}
\caption{\label{xlong-nus}
Mittelwert der durchschnittlichen Anzahl aktivierter Gene in Abh"angigkeit von den Mutationsraten (m) und der Selektionsrate (s).
Die Rauten kennzeichnen die L"aufe mit den H"ochstwerten.
}
\end{figure}


\begin{figure}

\unitlength1cm
\begin{picture}(16,10)
\put(0,0){\makebox(16,10)[b]{\epsfxsize=14cm \epsffile{xlong-gdv.eps}}}
\end{picture}
\caption{\label{xlong-gdv}
Mittelwert der genetischen Diversit"at in Abh"angigkeit von den Mutationsraten (m) und der Selektionsrate (s).
}
\end{figure}



\begin{figure}

\unitlength1cm
\begin{picture}(16,10)
\put(0,0){\makebox(16,10)[b]{\epsfxsize=14cm \epsffile{rxlong-mflag.eps}}}
\end{picture}
\caption{\label{xlong-mflag}
Mittelwert des durchschnittlichen Mutationsexponenten in Abh"angigkeit von den Mutationsraten (m) und der Selektionsrate (s).
Die zehn L"aufe mit den niedrigsten Mittelwerten des durchschnittlichen Mutationsexponenten sind durch
eine Raute hervorgehoben.
}
\end{figure}


\begin{figure}

\unitlength1cm
\begin{picture}(16,9)
\put(0,0){\makebox(16,10)[b]{\epsfxsize=12cm \epsffile{xlong-fitent.eps}}}
\end{picture}
\caption{\label{xlong-fitent}
Auftragung der Mittelwerte der DVK in den 110 durchgef"uhrten L"aufen in Abh"angigkeit
von den mittleren Durchschnittswerten der Fitness.
}
\end{figure}


\begin{figure}

\unitlength1cm
\begin{picture}(16,9)
\put(0,0){\makebox(16,10)[b]{\epsfxsize=12cm \epsffile{xlong-nusent.eps}}}
\end{picture}
\caption{\label{xlong-nusent}
Auftragung der Mittelwerte der DVK in den 110 durchgef"uhrten L"aufen in Abh"angigkeit
von den mittleren Durchschnittswerten der Anzahl aktivierter Gene.
}
\end{figure}


\begin{figure}

\unitlength1cm
\begin{picture}(16,9)
\put(0,0){\makebox(16,10)[b]{\epsfxsize=12cm \epsffile{xlong-maxent.eps}}}
\end{picture}
\caption{\label{xlong-maxent}
Auftragung der Mutationsraten ($m$) und Selektionsraten ($s$) der zwanzig L"aufe, in denen die
h"ochsten Werte der DVK gemessen wurden. Die Kurve zeigt den Graphen von $m=t_E(s,7)$, der kritischen Fehlergrenze von
Entwicklungsprogrammen aus sieben Genen, Oberhalb der Kurve k"onnen sich solche Entwicklungsprogramme
nur bei aktiver Absenkung der effektiven Mutationsraten etablieren (s.\ Text). Wertepaare von L"aufen, in denen eine signifikante
Absenkung des Mutationsexponenten ermittelt wurde, sind zus"atzlich durch ein Kreuz gekennzeichnet.
}
\end{figure}


\begin{figure}

\unitlength1cm
\begin{picture}(16,9)
\put(0,0){\makebox(16,10)[b]{\epsfxsize=12cm \epsffile{ldetailc00-maxent.eps}}}
\end{picture}
\caption{\label{ldetailc00-maxent}
Auftragung der Mutationsraten ($m$) und Selektionsraten ($s$) der achtzig L"aufe des Experiments \runname{ldetailc00}
(s.\ Text), in denen die h"ochsten DVK-Werte gemessen wurden.
In \runname{ldetailc00} war keine aktive Senkung der effektiven Mutationsraten m"oglich.
Die Kurve zeigt die kritische Fehlergrenze f"ur Entwicklungsprogrammen aus sieben Genen.
}
\end{figure}

% LindEvol-spezifischer Effekt: buschf"ormige Pflanzen aktivieren mehr Gene, daher
%     Zunahme der Anzahl aktivierter Gene bei hoeheren Selektionsraten.


\subsubsection{Fitnesswerte}

Abb.\ \ref{xlong-fit} zeigt, da"s die durch die in \runname{xlong} variierten Kontrollparameter
$m$ und $s$ aufgespannte Ebene im Kontrollparameterraum von LindEvol-GA in zwei Phasen zerf"allt.
In einer Phase mit hohen Mutationsraten und geringen Selektionsraten liegen die Mittelwerte der
durchschnittlichen Fitness bei 1. In diesem Bereich kommt keine Evolution von Pflanzenformen mit
erh"ohten Fitnesswerten zustande, weil die Mutationsraten die kritische Fehlergrenze selbst f"ur
elementare Entwicklungsprogramme deutlich "uberschreiten. Der zuvor
besprochene Lauf \runname{xlong1005} (\ref{xlong1005section}) f"allt in diesen Bereich.

Abb.\ \ref{xlong-fit} zeigt auch, da"s ohne Mutation ($m=0$) keine Evolution komplexer Wuchsformen mit
hohen Fitnesswerten m"oglich ist. Ohne Mutation setzt sich eine in der Startpopulation vorhandene
zweizellige Pflanzenform vollst"andig durch. Ohne Mutationen k"onnen sich aus dieser Form keine
komplexeren Formen entwickeln.

Die h"ochsten Mittelwerte der durchschnittlichen Fitness wurden bei gem"a"sigten Mutationsraten von
0.01 und 0.02 und moderater bis scharfer Selektionsrate zwischen 0.4 und 0.9 beobachtet. Bei h"oheren
Mutationsraten wurden etwas niedrigere mittlere Durchschnittsfitnesswerte gemessen, dies liegt vor
allem an der zunehmenden H"aufigkeit von Mutanten mit extrem niedrigen Fitnesswerten. Im Bereich der
Phasengrenze zwischen hohen und marginalen Fitnesswerten sind einige L"aufe mit nur leicht erh"ohten Fitnesswerten
zu beobachten. Bei diesen L"aufen entwickeln sich komplexe Pflanzenformen, die einen gro"sen Teil
ihrer Energie zur Senkung der effektiven Mutationsraten einsetzen. In Abb.\ \ref{xlong-mflag} k"onnen
die L"aufe, bei denen eine starke Senkung der effektiven Mutationsraten stattfand, identifiziert werden.


\subsubsection{Komplexit"at}
\label{lndgascan-complexity}

Die Fl"ache der mittleren DVK-Werte in Abh"angigkeit von den Mutationsraten und der Selektionsrate
weist eine "ahnliche Struktur wie die der mittleren Fitnesswerte auf. In einem Bereich mit hohen
Mutationsrate und geringen Selektionsraten liegen die Mittelwerte der DVK unter 2.0. Ohne Mutation
betr"agt die DVK 0 in allen L"aufen, in denen die Selektionsrate gr"o"ser als 0.0 ist. In einer
Population, in der alle Genome identisch sind, ist $f(0) = 1$, und f"ur alle $d>0$ ist $f(d) = 0$,
somit betr"agt die Shannon-Entropie der Distanzverteilung 0.

Hohe DVK-Werte sind im Bereich gem"a"sigter bis mittlerer Mutationsraten und moderater bis scharfer
Selektionsrate zu beobachten. Aber auch bei ausgesprochen hohen Mutationsraten und scharfer Selektion
existiert eine Gruppe von vier L"aufen mit hohen Mittelwerten der DVK. Der Vergleich mit Abb.\ \ref{xlong-mflag}
zeigt, da"s bei allen vier L"aufen dieser Gruppe eine starke Absenkung der effektiven Mutationsraten stattfand.
Zu diesen vier L"aufen z"ahlt auch der in (\ref{xlong1008section}) besprochene Lauf \runname{xlong1008}.

Die Korrelation zwischen hohen Fitnesswerten und hohen DVK-Werten wird durch eine Graphik, in der die in den
110 durchgef"uhrten L"aufen gemessenen mittleren Fitnesswerte und die Mittelwerte der DVK als Wertepaare
aufgetragen werden, besonders deutlich erkennbar.
Durch dieses Vorgehen wird die Korrelation ohne die willk"urliche Anordnung der
L"aufe aufgrund der Kontrollparameterwerte herausgearbeitet (vgl.\ \cite{Langton92}). Abb.\ \ref{xlong-fitent}
zeigt diese Auftragung. Der Me"spunkt bei einer Fitness von 2 und einer DVK von 0 r"uhrt von den L"aufen
mit $m=0.0$ und $s>0$ her. Bei DVK-Werten bis etwas mehr als 2 liegen die mittleren Fitnesswerte bei
1. Diese Me"spunkte geh"oren zu L"aufen, in denen keine Evolution von Pflanzen mit erh"ohten Fitnesswerten
m"oglich war, weil die kritische Fehlergrenze f"ur elementare Entwicklungsprogramme "uberschritten war.
Bei h"oheren DVK-Werten werden auch
erh"ohte Fitnesswerte beobachtet. Dabei ist eine Korrelation zwischen der H"ohe der Fitness und der H"ohe
der DVK deutlich erkennbar, obwohl die DVK-Werte im Bereich h"oherer Fitnesswerte st"arker streuen.

Ein weiteres Indiz f"ur die Komplexit"at eines Entwicklungsprogramms ist die Anzahl der Gene, in denen
es codiert ist, und die infolgedessen w"ahrend der Auspr"agung des Programms aktiviert werden.
Abb.\ \ref{xlong-nus} zeigt die Mittelwerte der durchschnittlichen Anzahl der aktivierten
Gene. Die Graphik zeigt, da"s  die komplexesten Entwicklungsprogramme bei scharfer Selektion evolvieren.
Dieser Effekt ist zumindest teilweise auf die Genominterpretation zur"uckzuf"uhren, die dazu f"uhrt,
da"s die bei scharfer Selektion bevorzugten buschartigen Pflanzenformen zu ihrer Codierung eine gr"o"sere
Anzahl von Genen ben"otigen. Die H"ochstwerte der mittleren Anzahl aktivierter Gene lagen im Bereich um sieben.

Hinsichtlich der Mutationsraten verteilen sich die Spitzenwerte der Entwicklungsprogramml"angen auf
zwei Bereiche. Die meisten Spitzenwerte wurden bei L"aufen mit scharfer Selektion und gem"a"sigten
Mutationsraten gemessen. Ein Bereich hoher Entwicklungsprogramml"angen ist jedoch auch bei scharfer
Selektion und hohen Mutationsraten zu beobachten. In diesem Bereich kommt es zur Evolution von
Pflanzenformen, die eine aktive Senkung der effektiven Mutationsraten betreiben, wozu mindestens ein
weiteres Gen zus"atzlich zu den die Morphogenese steuernden Genen erforderlich ist. Auch die hohen
Werte der Entwicklungsprogramml"angen, die bei moderaten Selektionsraten und intermedi"aren Mutationsraten
zustandekommen, gehen auf L"aufe zur"uck, in denen eine Evolution aktiver Reduktion der effektiven
Mutationsraten abl"auft (vgl.\ Abb.\ \ref{xlong-mflag}).

Abb.\ \ref{xlong-nusent} zeigt eine Auftragung der Mittelwerte der durchschnittlichen Anzahl der aktivierten
Gene jeweils w"ahrend der letzten 50 Generationen eines Laufs gegen den Mittelwert der DVK. Auch hier wird
eine Korrelation zwischen hohen DVK-Werten und einer gro"sen Anzahl aktiver Gene in den evolvierten Wachstumsprogrammen
deutlich. Die Beobachtung, da"s bei einer Evolution von Pflanzen mit komplexer Morphologie und entsprechend
komplexen Entwicklungsprogrammen auch eine hohe DVK auftritt, wird durch diese Korrelationsanalysen in systematischer
Weise best"atigt.

Abb.\ \ref{xlong-maxent} zeigt ein Diagramm der Wertepaare der Mutations- und Selektionsrate
der L"aufe mit den 20 h"ochsten Mittelwerten der DVK. In dem Diagramm ist weiterhin die Kurve der kritischen
Fehlergrenze von Entwicklungsprogrammen mit sieben aktiven Genen als Funktion der Selektionsrate, $t_E(s,7)$, eingezeichnet
(Gl.\ \ref{errthresholdprgm-eq}), weil sieben der Bereich der h"ochsten beobachteten L"angen von Entwicklungsprogrammen
war. Im Bereich unterhalb dieser Kurve ist die Evolution von Entwicklungsprogrammen maximaler Komplexit"at m"oglich,
w"ahrend oberhalb der Kurve nur bei aktiver Verringerung der effektiven Mutationsraten die Evolution von Entwicklungsprogrammen
dieser Komplexit"at m"oglich ist. Die Funktion $t_E(s,7)$ kann somit als \textsl{edge of chaos} \cite{Gutowitz95,Kauffman92,Langton92}
angesehen werden.

Die L"aufe mit hohen DVK-Werten zerfallen bez"uglich der Mutationsraten
und der Selektionsrate deutlich in zwei Gruppen. Bei einer Gruppe von L"aufen liegen die Mutationsraten
in der N"ahe der Kurve oder deutlich darunter. Die Basiswerte der Mutationsraten sind also niedrig genug,
um die Ausbreitung komplexer Entwicklungsprogramme zuzulassen. Es ist eine gewisse H"aufung der L"aufe in
dieser Gruppe unterhalb der Kurve der kritischen Fehlergrenze, jedoch in ihrer N"ahe zu beobachten.

Bei einer weiteren Gruppe von L"aufen liegen die
Mutationsraten deutlich oberhalb der Kurve. Bei allen
L"aufen in dieser Gruppe fand eine signifikante Absenkung des Mutationsexponenten statt, wie auch der
Vergleich mit Abb.\ \ref{xlong-mflag} deutlich macht. 

Die Evolution komplexer Entwicklungsprogramme bei hohen Mutationsraten durch aktive Verringerung der effektiven
Mutationsraten ist nicht m"oglich, wenn die Kontrollparameter \verb|num_mutminus| und \verb|num_mutplus| auf
0 gesetzt werden. Dies war bei der Serie von L"aufen \runname{ldetailc00} der Fall. Sie wurde mit folgenden
Kontrollparametern durchgef"uhrt:

\medskip
\begin{tabular}{ll}
Selektionsrate: & 0.0 bis 0.95, Schrittweite 0.05 \\
Austauschrate: & 0.0 bis 0.1, Schrittweite 0.005 \\
Insertionsrate: & 0.0 \\
Deletionsrate: & 0.0 \\
Populationsgr"o"se: & 50 \\
Mutationsfaktor: & 1.0 \\
Breite der Welt: & 150 \\
H"ohe der Welt: & 30  \\
Genoml"ange der Startpopulation: & 20 Gene (40 Bytes) \\
L"ange einer Vegetationsperiode: & 30 Tage \\
Anzahl Generationen:             & 350 \\
\end{tabular}

\begin{tabular}{ll}
\verb|num_divide| & 256 \\
\verb|num_mutminus| & 0 \\
\verb|num_mutplus| & 0 \\
\verb|random_seed| & 12345 \\
\end{tabular}
\medskip

Verschiedene Ergebnisse des Experiments \runname{ldetailc00} wurden in \cite{Kim95} ver"offentlicht.
Abb.\ \ref{ldetailc00-maxent} zeigt ein zu Abb.\ \ref{xlong-maxent} analoges Diagramm. Da in \runname{ldetailc00}
nur Austauschmutationen stattfanden, wurde die Kurve der kritischen Fehlergrenze eines Entwicklungsprogramms
von sieben Genen anhand von Gl.\ \ref{errthresholdprgr-eq} berechnet. Abb.\ \ref{ldetailc00-maxent} zeigt,
da"s bei fehlender M"oglichkeit zur aktiven Verringerung der effektiven Mutationsraten keine L"aufe mit hoher
DVK auftreten, bei denen die Austauschrate deutlich "uber dem Wert liegt, ab dem die Ausbreitung eines Entwicklungsprogramms
aus sieben Genen unm"oglich ist.

In Abb.\ \ref{ldetailc00-maxent} tritt das Ph"anomen der H"aufung von L"aufen mit hoher DVK unterhalb der 
kritischen Fehlergrenze deutlich zutage, bei L"aufen, in denen die Austauschrate die kritische Fehlergrenze
weit unterschreitet, wurden submaximale DVK-Werte gemessen. Dieses Ph"anomen trat auch in den L"aufen \runname{xlong00105}
(\ref{xlong00105section}) und \runname{xlong00108} (\ref{xlong00108section}) deutlich zutage.
Durch eine hohe DVK werden also solche L"aufe
charakterisiert, bei denen das Verh"altnis von Selektionsrate und Mutationsraten einerseits die Evolution
komplexer Entwicklungsprogramme zul"a"st, andererseits aber auch eine m"oglichst schnelle Diffusion innerhalb
neutraler Mengen erm"oglicht. Dieser Befund, wie auch die gute Korrelation zwischen dem Graphen der Funktion $t_E(s,7)$
und dem Bereich maximaler DVK-Werte, belegen einen hochsignifikanten Zusammenhang zwischen hohen DVK-Werten
und Evolution im Bereich der \textsl{edge of chaos} in LindEvol-GA.

Abb.\ \ref{xlong-gdv} zeigt, da"s die genetische Diversit"at in den L"aufen mit Selektion und ohne Mutation
stets 0 betr"agt, wie dies bei Populationen identischer Individuen zu erwarten ist. In allen anderen L"aufen
ist die genetische Diversit"at jedoch im Maximalbereich, sowohl bei L"aufen, in denen die h"ochsten Fitnesswerte
erreicht wurden als auch bei L"aufen, in denen infolge der "Uberschreitung der kritischen Fehlergrenze keinerlei
Evolution komplexer Pflanzenformen stattfand. Die genetische Diversit"at ist damit zur Charakterisierung der
evolution"aren Dynamik in L"aufen von LindEvol-GA nicht verwendbar.

% Problem genetische Diversitaet: Kein Rueckgang bei hoher Randomisierung.


\subsubsection{Mutationsexponenten}

Auswirkungen der Beeinflussung der effektiven Mutationsraten durch die evolvierenden Individuen wurden bereits
in verschiedenen Systemen untersucht \cite{Maley94}. Die f"ur LindEvol entwickelte Modellierungsmethode unterscheidet
sich von anderen Ans"atzen dadurch, da"s die Modifikation der effektiven Mutationsraten durch den Energiebedarf
der Aktionen \verb|mut-| und \verb|mut+| mit der Fitness gekoppelt ist. Auf diese Weise wird ein intrinsischer
Mechanismus realisiert, der einer Evolution beliebig kleiner Mutationsraten entgegenwirkt, die ohne diesen
Mechanismus m"oglicherweise ablaufenen w"urde, weil geringere Mutationsraten eine h"ohere Ausbreitungszahl
zur Folge haben (s.\ Gl.\ \ref{programspreadm-eq}).

Die Mittelwerte der durchschnittlichen Mutationsexponenten (Abb.\ \ref{xlong-mflag}) zeigen, da"s eine Evolution
von Pflanzenformen, die ihre effektiven Mutationsraten aktiv absenken, nur in L"aufen stattfindet, bei denen
die Mutationsraten oberhalb von $t_E(s,3)$ liegen. Durch die aktive Verringerung der effektiven Mutationsraten
kann in diesem Bereich sowohl eine erhebliche Vergr"o"serung der Ausbreitungszahl als auch eine drastische
Erh"ohung der Fitness erreicht werden, wie dies in (\ref{xlong1008section}) f"ur den Lauf \runname{xlong1008}
beschrieben wurde. Im Bereich von Mutationsraten, die im Verh"altnis zu $t_E(s,7)$, dem Bereich der \textsl{edge of chaos},
hoch sind, kann also eine Evolution hin zur \textsl{edge of chaos} \cite{Kauffman92} beobachtet werden.


\subsection{LindEvol-GA als Testsystem f"ur Verfahren zur Rekonstruktion der Phylogenie}
\label{lndga-phylogeny}

Die am Anfang eines Laufs von LindEvol-GA zufallsgesteuert generierten Genome haben keinerlei Verwandtschaft zueinander,
sie stellen Anfangsknoten von miteinander nicht verbundenen Stammb"aumen dar. Im Laufe einer Anfangsphase breiten sich
die Nachkommen eines Genoms der Startpopulation in der ganzen Population aus, w"ahrend alle anderen Linien aussterben.
Von diesem Zeitpunkt an geh"oren s"amtliche Genome der Population zu einem gemeinsamen Stammbaum. Der Stammbaum 
(bzw.\ die Stammb"aume in der Anfangsphase) der Genome in einem LindEvol-GA--Lauf k"onnen jederzeit ermittelt werden.

Bei der molekularen Evolution sind dagegen nur die Sequenzen rezenter Lebensformen zug"anglich. Zur Rekonstruktion der
Phylogenie aus dieser Sequenzinformation existieren verschiedene Verfahren \cite{Felsenstein88}.
Bei allen Rekonstruktionsmethoden mu"s zun"achst ein \textsl{multiple alignment} \cite{GCG} der Sequenzen erstellt
werden. Durch dieses wird festgelegt, welche Zeichen in den verschiedenen Sequenzen als zueinander homolog
angenommen werden sollen. In weiteren Schritten wird dann unter Auswertung der Information "uber Gemeinsamkeiten und
Unterschiede in den homologen Positionen ein Stammbaum rekonstruiert.

Eine Standardmethode zur Charakterisierung der Leistungsf"ahigkeit und
Genauigkeit von Verfahren zur Stammbaumrekonstruktion ist es, eine Sequenzmenge durch die Simulation eines Evolutionsprozesses
zu erzeugen. Dazu wird h"aufig die Simulation eines neutralen Evolutionsprozesses mit fest vorgegebenem Stammbaum
eingesetzt. Diese Sequenzmenge wird dann mit dem zu testenden Rekonstruktionsverfahren bearbeitet, und die resultierenden,
rekonstruierten Stammb"aume werden mit dem aus der Simulation bekannten, korrekten Stammbaum verglichen. Eine solche
Untersuchung wird in \cite{KuhnerFelsenstein} beschrieben.

Die Evolution der Lebensformen auf der Erde ist keine strikt neutrale Evolution. Mutationen k"onnen zwar neutral sein,
sie k"onnen aber auch drastische Auswirkungen auf die "Uberlebens- und Vermehrungschancen eines Organismus haben.
In LindEvol-GA kommen sowohl neutrale Mutationen als auch Mutationen, die Auswirkungen unterschiedlichsten Ausma"ses
auf die Morphologie der simulierten Pflanzen haben, vor. Anders als neutrale Evolutionsvorg"ange stimmt LindEvol-GA also
in dieser Eigenschaft qualitativ mit der molekularen Evolution "uberein. Insofern ist LindEvol-GA ein realistischeres
Modell der Evolution als Simulationen neutraler Evolutionsprozesse. Es bietet sich an, die Bedeutung von Abweichungen
von der Annahme neutraler Evolution auf die Genauigkeit der Stammbaumrekonstruktion zu untersuchen, indem LindEvol-GA
als Simulation zur Generierung von Testdaten f"ur die Rekonstruktionsverfahren verwendet wird.

Zu diesem Zweck wurden verschiedene L"aufe mit LindEvol-GA durchgef"uhrt.
In regelm"a"sigen Abst"anden, jeweils alle 20 Generationen, wurden aus den Simulationsl"aufen jeweils gleichzeitig
der Stammbaum der Population und
alle Genome der Population abgespeichert.
Zur Verarbeitung durch Stammbaumanalyseprogramme wurden die Bytesequenzen der Genome in Sequenzen des
Nucleotidalphabets $\mathrm\{A,C,G,T\}$ (Adenosin, Cytidin, Guanosin und Thymidin)
"ubertragen. Jedes Byte wurde wurde dazu in vier Bitgruppen zu je zwei
Bits eingeteilt, jede Bitgruppe wurde entsprechend der folgenden Tabelle in ein Nucleotidzeichen "ubersetzt:

\medskip
\begin{tabular}{ccc}
{\ttfamily 00} & $\rightarrow$ & A \\
{\ttfamily 01} & $\rightarrow$ & C \\
{\ttfamily 10} & $\rightarrow$ & G \\
{\ttfamily 11} & $\rightarrow$ & T \\
\end{tabular}
\medskip

Der bei einer Generation extrahierte tats"achliche Baum und die bei dieser Generation erhaltene Sequenzmenge
werden zusammen im Folgenden als ein \introdef{Datensatz} bezeichnet.
Nur Populationen, bei denen alle Genome von einen gemeinsamen Vorfahren abstammen,
wurden f"ur die Untersuchungen verwendet; Datens"atze aus der Anfangsphase der Simulationen, bei denen die Genome
von von mehr als einem Genom aus der Startpopulation abstammten, wurden ignoriert.

Bei den L"aufen wurden jeweils die Insertions- und die Deletionsrate auf Null gesetzt; die Genoml"ange bleibt
also in den L"aufen konstant.
Auf diese Weise wird ein Mutationsproze"s simuliert, der dem Jukes-Cantor-Modell der molekularen Uhr
(vgl.\ Dokumentation zu PHYLIP) entspricht. Ferner sind bei der Beschr"ankung auf Austauschmutationen
die Zeichen aller Genome, die einen gemeinsamen Vorfahr haben, positionsweise homolog. Ein aufwendiges
\textsl{alignment}-Verfahren ist somit nicht erforderlich; die Sequenzen k"onnen einfach untereinander
geschrieben werden. Unzutreffende Homologieannahmen k"onnen daher das Rekonstruktionsergebnis nicht
beeintr"achtigen.

Die Sequenzdaten wurden mit den Rekonstruktionsverfahren \textsl{parsimony} und \textsl{neighbor joining} untersucht.
Hierf"ur wurden die entsprechenden Programme aus dem Programmpaket
PHYLIP \cite{PHYLIP} eingesetzt. Die \textsl{parsimony}-Analyse erfolgte mittels des Programms \prgname{dnapars}.
Dieses Programm erzeugt unter Umst"anden mehrere Stammb"aume mit maximaler \textsl{parsimony}.

Zur Rekonstruktion
mit dem \textsl{neighbor joining} Verfahren wurden aus den Sequenzmengen zun"achst mit dem Programm \prgname{dnadist}
eine Distanzmatrix errechnet, wobei das Jukes-Cantor-Modell der Sequenzevolution als Grundlage der Distanzberechnung
ausgew"ahlt wurde. Eine Verf"alschung der Rekonstruktion aufgrund von Unterschieden zwischen dem tats"achlichen und
dem bei der Rekonstruktion angenommenen Mutationsproze"s ist damit ausgeschlossen.
Die mittels \prgname{dnadist} erhaltenen Distanzmatrizen wurden zur
Stammbaumrekonstruktion mit dem Programm \prgname{neighbor} verwendet.

Die Distanzberechnung nach dem Jukes-Cantor-Modell ist nur dann m"oglich, wenn die betrachteten Sequenzen mehr
als 25\% "Ubereinstimmungen haben. Das Jukes-Cantor-Modell sagt voraus, da"s die  Anzahl der Unterschiede zwischen
zwei verwandten DNA-Sequenzen mit der Zeit exponentiell asymptotisch gegen 75\% konvergiert. Daher kann mithilfe
des Jukes-Cantor-Modells nur f"ur Sequenzen, die weniger als 75\% Unterschiede aufweisen, die seit ihrer Abstammung
von einem gemeinsamen Vorfahren vergangene Zeit abgesch"atzt werden. Wenn das Programm \prgname{dnadist} ein
Paar von Sequenzen findet, die weniger als 25\% "Ubereinstimmungen haben, bricht es ab. In solchen F"allen war die
Rekonstruktion mittels \textsl{neighbor joining} nicht m"oglich. Die Auswahl einer geringen Mutationsrate hilft,
dieses Problem zu vermeiden.

Als Ma"s f"ur die Rekonstruktionsgenauigkeit wurde der \textsl{dT score}
(vgl.\ \ref{glossary} zwischen den jeweils rekonstruierten B"aumen und dem tats"achlichen Stammbaum berechnet;
diese Methode zur Evaluierung der Rekonstruktionsergebnisse wurde von Kuhner und Felsenstein \cite{KuhnerFelsenstein}
"ubernommen. Von allen auf der Basis einer Sequenzmenge rekonstruierten B"aumen wurde der \textsl{dT score} zu
dem tats"achlichen Stammbaum berechnet. Wenn mit einer Sequenzmenge mehrere \textsl{most parsimonious trees}
gefunden wurden, wurde der Durchschnittswert der \textsl{dT scores} zwischen diesen B"aumen und dem tats"achlichen
Stammbaum berechnet.

Zu allen LindEvol-GA--L"aufen wurde eine neutrale Kontrolle (\ref{lndga-neutral}) durchgef"uhrt.
Bei diesen L"aufen wurden, wie bei den LindEvol-GA--L"aufen, alle 20 Generationen der tats"achliche Stammbaum
und die Genome gespeichert. Mit diesen \introdef{Kontrolldatens"atzen} wurden dieselben Rekonstruktionsverfahren 
wie mit den aus LindEvol-GA entnommenen Datens"atzen durchgef"uhrt. Die Differenzen zwischen den bei einem
LindEvol-GA--Lauf und dem entsprechenden Lauf der neutralen Kontrolle gemessenen \textsl{dT scores} gibt
Aufschlu"s "uber Auswirkungen nichtneutraler Evolutionsprozesse in LindEvol-GA auf die Stammbaumrekonstruktion.
Abb.\ \ref{phylo-setup} fa"st das experimentelle Verfahren zusammen und verdeutlicht den Datenflu"s.


\begin{figure}[t]
\unitlength1cm
{
\begin{picture}(16,10)
\ttfamily
\put(0,9.5){\framebox(16,0.5){Kontrollparameter}}
\put(3.5,9.5){\vector(0,-1){0.5}}
\put(12.5,9.5){\vector(0,-1){0.5}}

\put(0,8.5){\makebox(7,0.5){\centerline{LindEvol-GA}}}
\put(3.5,8.75){\oval(7,0.5)}
\put(1.5,8.5){\vector(0,-1){0.5}}
\put(5.5,8.5){\vector(0,-1){0.5}}

\put(0,7){\framebox(3,1){
\parbox{3cm}{
\centerline{tats"achl.}
\centerline{Baum}}}}
\put(1.5,7){\vector(0,-1){3.5}}

\put(4,7){\framebox(3,1){
\parbox{3cm}{
\centerline{Sequenz-}
\centerline{menge}}}}
\put(5.5,7){\vector(0,-1){0.5}}

\put(4,5.5){\makebox(3,1){
\parbox{3cm}{
\centerline{Rekon-}
\centerline{struktion}}}}
\put(5.5,6){\oval(3,1)}
\put(5.5,5.5){\vector(0,-1){0.5}}

\put(4,4){\framebox(3,1){
\parbox{3cm}{
\centerline{Rekonst.}
\centerline{B"aume}}}}
\put(5.5,4){\vector(0,-1){0.5}}

\put(0,3){\makebox(7,0.5){Baumvergleich}}
\put(3.5,3.25){\oval(7,0.5)}
\put(3.5,3){\vector(0,-1){0.5}}

\put(0,2){\framebox(7,0.5){dT scores}}
\put(3.5,2){\vector(0,-1){0.5}}

%neutrale Kontrolle
\put(9,8.5){\makebox(7,0.5){\centerline{neutrale Kontrolle}}}
\put(12.5,8.75){\oval(7,0.5)}
\put(10.5,8.5){\vector(0,-1){0.5}}
\put(14.5,8.5){\vector(0,-1){0.5}}

\put(9,7){\framebox(3,1){
\parbox{3cm}{
\centerline{tats"achl.}
\centerline{Baum}}}}
\put(10.5,7){\vector(0,-1){3.5}}

\put(13,7){\framebox(3,1){
\parbox{3cm}{
\centerline{Sequenz-}
\centerline{menge}}}}
\put(14.5,7){\vector(0,-1){0.5}}

\put(13,5.5){\makebox(3,1){
\parbox{3cm}{
\centerline{Rekon-}
\centerline{struktion}}}}
\put(14.5,6){\oval(3,1)}
\put(14.5,5.5){\vector(0,-1){0.5}}

\put(13,4){\framebox(3,1){
\parbox{3cm}{
\centerline{Rekonst.}
\centerline{B"aume}}}}
\put(14.5,4){\vector(0,-1){0.5}}

\put(9,3){\makebox(7,0.5){Baumvergleich}}
\put(12.5,3.25){\oval(7,0.5)}
\put(12.5,3){\vector(0,-1){0.5}}

\put(9,2){\framebox(7,0.5){dT scores}}
\put(12.5,2){\vector(0,-1){0.5}}

\put(0,1){\makebox(16,0.5){Vergleich, Differenzbildung}}
\put(8,1.25){\oval(16,0.5)}
\put(8,1){\vector(0,-1){0.5}}

\put(0,0){\framebox(16,0.5){Unterschiede zwischen LindEvol-GA und der neutralen Kontrolle}}

\end{picture}
}
\caption["Uberblick "uber den Aufbau der Rekonstruktionsexperimente]
{\label{phylo-setup}
"Uberblick "uber den Aufbau der Rekonstruktionsexperimente. Daten werden durch eckige Boxen
dargestellt, Boxen mit abgerundeten Ecken symbolisieren Rechenverfahren.
Ein LindEvol-GA--Lauf und ein
Lauf der neutralen Kontrolle werden mit demselben Kontrollparametersatz durchgef"uhrt.
Zu verschiedenen Zeiten werden aus beiden L"aufen jeweils der tats"achliche Stammbaum der
Population und die Population der Genome entnommen. Auf der Basis der Genompopulationen werden
rekonstruierte B"aume erstellt. Diese werden mit dem tats"achlichen Baum verglichen, indem
der \textsl{dT score} berechnet wird. Unterschiede zwischen den mit LindEvol-GA und der
neutralen Kontrolle gemessenen \textsl{dT scores} weisen auf nichtneutrale Prozesse, die
die Rekonstruktionsgenauigkeit beeinflussen, hin.
}
\end{figure}

Wenn von einem Genom bei einem Selektionsschritt mehr als eine Kopie erzeugt wird, f"uhrt dies zu einer Multifurkation
im tats"achlichen Stammbaum. Multifurkationen f"uhren zu Verzerrungen bei der \textsl{dT score} Analyse,
weil die Rekonstruktionsprogramme stets vollst"andig aufgel"oste, bin"are B"aume erzeugen, und somit der tats"achliche
Baum mit Multifurkationen au"serhalb des L"osungsbereichs der Rekonstruktionsverfahren liegt. Ferner hat ein
Baum mit Multifurkationen weniger Kanten als ein vollst"andig aufgel"oster Baum. Der Maximalwert des
\textsl{dT score} zwischen dem tats"achlichen Baum und rekonstruierten B"aumen verkleinert sich daher,
wenn der tats"achliche Baum Multifurkationen enth"alt. Die Absolutwerte dieser \textsl{dT scores} sind
somit bei Datens"atzen mit unterschiedlich vielen Multifurkationen im tats"achlichen Baum nicht vergleichbar.
Die Wahrscheinlichkeit des Auftretens einer Multifurkation ist umso gr"o"ser, je h"oher die Selektionsrate
ist. Aus diesem Grund konnten die oben beschriebenen Schwierigkeiten vermieden werden, indem die niedrige
Selektionsrate 0.2 gew"ahlt wurde.


\subsubsection{Rekonstruktion und Evolution komplexer Entwicklungsprogramme}
\label{slrecsection}

\begin{figure}[t]

\unitlength1cm
\begin{picture}(16,14)
\put(0,0){\makebox(16,14)[b]{\epsfxsize=16cm \epsffile{shortgenomes.eps}}}
\end{picture}
\caption[Ergebnisse shortgenomes]
{\label{shortgenomesresults}
Verlaufsdiagramm des LindEvol-GA--Laufs \runname{shortgenomes}.
}
\end{figure}

\begin{figure}
\unitlength1cm
\begin{picture}(16,20.5)
{
\small
\put(0,15.5){\makebox(16,5)[b]{\epsfxsize=14cm \epsffile{shortgenomes-nj.eps}}}
\put(0,15){\makebox(16,0.5){\textbf{a:} \textsl{dT scores} mit \textsl{neighbor joining}
bei \runname{shortgenomes} und der neutralen Kontrolle}}
\put(0,10){\makebox(16,5)[b]{\epsfxsize=14cm \epsffile{shortgenomes-dp.eps}}}
\put(0,9.5){\makebox(16,0.5){\textbf{b:} Durchschnittliche \textsl{dT scores} mit \textsl{parsimony}
bei \runname{shortgenomes} und der neutralen Kontrolle}}
\put(0,4.5){\makebox(16,5)[b]{\epsfxsize=14cm \epsffile{shortgenomes-dpdiff.eps}}}
\put(1,3.5){\makebox(14,1){\parbox{14cm}{\textbf{c:} Differenz der durchschnittlichen \textsl{dT scores} mit \textsl{parsimony}
bei \runname{shortgenomes} und der neutralen Kontrolle}}}
\put(0,0.5){\makebox(16,3)[b]{\epsfxsize=14cm \epsffile{shortgenomes-ug.eps}}}
\put(0,0){\makebox(16,0.5){\textbf{d:} Evolution der durchschnittlichen Anzahl aktivierter Gene in \runname{shortgenomes}}}
}
\end{picture}
\caption[Stammbaumrekonstruktion shortgenomes]
{\label{shortgenomes-phylo}
\textsl{dT score} Ergebnisse aus den Datens"atzen aus \runname{shortgenomes} (Boxen) und aus den zugeh"origen neutralen Kontrolle
(Rauten). Bei einigen Datens"atzen der neutralen Kontrolle war die Erstellung einer Distanzmatrix mit \prgname{dnadist}
nicht m"oglich (s.\ Text), hier fehlen die entsprechenden Me"spunkte in (a).
(c) zeigt die Differenz zwischen den jeweils in \runname{shortgenomes} und in der neutralen Kontrolle gemessenen
\textsl{dT scores} mit \textsl{parsimony}.
Zum Vergleich die durchschnittliche Anzahl aktivierter Gene (d).
% Die Zunahme der Komplexit"at der Entwicklungsprogramme
% ist mit einem deutlichen R"uckgang der Rekonstruktionsgenauigkeit korreliert.
}
\end{figure}

\begin{figure}[t]

\unitlength1cm
\begin{picture}(16,14)
\put(0,0){\makebox(16,14)[b]{\epsfxsize=16cm \epsffile{medgenomes.eps}}}
\end{picture}
\caption[Ergebnisse medgenomes]
{\label{medgenomesresults}
Verlaufsdiagramm des LindEvol-GA--Laufs \runname{medgenomes}.
}
\end{figure}

\begin{figure}
\unitlength1cm
\begin{picture}(16,20.5)
{
\small
\put(0,15.5){\makebox(16,5)[b]{\epsfxsize=14cm \epsffile{medgenomes-nj.eps}}}
\put(0,15){\makebox(16,0.5){\textbf{a:} \textsl{dT scores} mit \textsl{neighbor joining}
bei \runname{medgenomes} und der neutralen Kontrolle}}
\put(0,10){\makebox(16,5)[b]{\epsfxsize=14cm \epsffile{medgenomes-dp.eps}}}
\put(0,9.5){\makebox(16,0.5){\textbf{b:} Durchschnittliche \textsl{dT scores} mit \textsl{parsimony}
bei \runname{medgenomes} und der neutralen Kontrolle}}
\put(0,4.5){\makebox(16,5)[b]{\epsfxsize=14cm \epsffile{medgenomes-dpdiff.eps}}}
\put(1,3.5){\makebox(14,1){\parbox{14cm}{\textbf{c:} Differenz der durchschnittlichen \textsl{dT scores} mit \textsl{parsimony}
bei \runname{medgenomes} und der neutralen Kontrolle}}}
\put(0,0.5){\makebox(16,3)[b]{\epsfxsize=14cm \epsffile{medgenomes-ug.eps}}}
\put(0,0){\makebox(16,0.5){\textbf{d:} Evolution der durchschnittlichen Anzahl aktivierter Gene in \runname{medgenomes}}}
}
\end{picture}
\caption[Stammbaumrekonstruktion medgenomes]
{\label{medgenomes-phylo}
\textsl{dT score} Ergebnisse aus den Datens"atzen aus \runname{medgenomes} (Boxen) und aus den zugeh"origen neutralen Kontrolle
(Rauten). Wo die Berechnung der Distanzmatrix nicht m"oglich war, fehlen die Me"spunkte f"ur die neutrale Kontrolle in
(a). (c) zeigt die Differenzen der mit \textsl{parsimony} beobachteten Werte in \runname{medgenomes} und in der
neutralen Kontrolle. Zum Vergleich die durchschnittliche Anzahl aktivierter Gene (d).
% Die Evolution komplexer Entwicklungsprogramme
% bleibt ohne signifikante Auswirkungen auf die Rekonstruktionsgenauigkeit.
}
\end{figure}

Bei neutraler Evolution bleibt jede Mutation ohne Auswirkungen auf die Fitness des betroffenen Genoms.
Diese Annahme gilt nicht f"ur LindEvol-GA. Mutationen in Genen, die beim Wachstum einer Pflanze aktiviert
werden, f"uhren in aller Regel zu "Anderungen in der Morphologie und der Fitness. Bei inaktiven Genen
sind dagegen alle Mutationen, die die rechte Seite der codierten Regel ver"andern, neutral, und Mutationen
an der linken Seite sind neutral, sofern die linke Seite nach der Mutation nicht eine lokale Struktur
spezifiziert, die beim Wachstum der Pflanze auftritt.

Die Anzahl der beim Wachstum der Pflanzen aktivierten Gene steigt typischerweise im Verlauf eines LindEvol-GA--Laufs,
die M"oglichkeiten zu neutralen Mutationen nehmen somit ab. Der Raum f"ur neutrale Mutationen wird umso st"arker
eingeschr"ankt, desto h"oher der Anteil der aktiven Gene an der Gesamtmenge der Gene eines Genoms ist. Bei kurzen
Genomen ist also eine st"arkere Abweichung von der Annahme der neutralen Evolution zu erwarten. Um die Auswirkungen
dieser Annahme auf die Rekonstruktionsgenauigkeit zu untersuchen, wurden die
L"aufe \runname{shortgenomes} mit einer Genoml"ange von 10 und \runname{medgenomes} mit einer Genoml"ange von 20
durchgef"uhrt.

\medskip
\begin{tabular}{ll}
Populationsgr"o"se: & 50 \\
Selektionsrate: & 0.2 \\
Austauschrate: & 0.003 \\
Mutationsfaktor: & 2.0 \\
Breite der Welt: & 150 \\
H"ohe der Welt: & 30  \\
Genoml"ange der Startpopulation: & 10 Gene in \runname{shortgenomes} \\
				 & 20 Gene in \runname{medgenomes} \\
L"ange einer Vegetationsperiode: & 30 Tage \\
Anzahl Generationen:             & 3000 \\
\verb|num_divide| & 224 \\
\verb|num_mutminus| & 16 \\
\verb|num_mutplus| & 16 \\
\verb|random_seed| & 1 \\
\end{tabular}
\medskip

Die L"aufe der neutralen Kontrollen hei"sen \runname{c-shortgenomes} bzw.\ \runname{c-medgenomes}
Die Abbildungen \ref{shortgenomesresults} und \ref{medgenomesresults} zeigen die Verlaufsdiagramme der L"aufe
\runname{shortgenomes} und \runname{medgenomes}. Viele der typischen Prozesse von LindEvol-GA--L"aufen mit
niedrigen Mutationsraten sind auch in diesen Diagrammen zu sehen (vgl.\ \ref{xlong00105section}).

Abb.\ \ref{shortgenomes-phylo} zeigt die Ergebnisse der Rekonstruktionsanalysen mit \runname{shortgenomes}
und der dazugeh"origen neutralen Kontrolle. Der durchschnittliche \textsl{dT score} zwischen den mit
\textsl{parsimony} rekonstruierten B"aumen und dem tats"achlichen Stammbaum ist bei den meisten Datens"atzen
aus \runname{shortgenomes} etwas gr"o"ser als in den entsprechenden Generationen in der neutralen Kontrolle.
Die Rekonstruktion ist also bei der neutralen Kontrolle genauer. Mit der Zunahme der Anzahl der aktivierten
Gene vergr"o"sert sich die Differenz zwischen dem bei \runname{shortgenomes} und dem bei der neutralen
Kontrolle gemessenen Durchschnittswert des \textsl{dT score}. Bei den Messungen mit \textsl{neighbor joining}
ist ein "ahnlicher Trend erkennbar, wegen der zahlreichen fehlenden Me"spunkte ist er jedoch nicht sehr
deutlich.

\begin{sloppypar}
Der Durchschnittswert der Anzahl der aktiven Gene steigt im Verlauf von \runname{shortgenomes} von drei auf acht.
Die Anzahl der inaktiven Gene sinkt somit von sieben auf zwei, also von 70\% auf 20\% der
Gesamtgenoml"ange.
\end{sloppypar}

Ein anderes Bild ergibt sich bei \runname{medgenomes} und \runname{c-medgenomes}, dem entsprechenden Lauf der
neutralen Kontrolle. Die Rekonstruktionsgenauigkeit mit \textsl{parsimony} ist sowohl in \runname{medgenomes}
als auch in \runname{c-medgenomes} erheblich besser als in den entsprechenden L"aufen mit kurzen Genomen.
Der gr"o"sere Informationsgehalt der l"angeren Sequenzen macht eine akkuratere Rekonstruktion m"oglich.
Weiterhin zeigt die Zunahme der Komplexit"at der Entwicklungsprogramme
keine signifikante Auswirkungen auf die Rekonstruktionsgenauigkeit. Der Grund hierf"ur wird deutlich, wenn man
die Anzahl der aktivierten Gene in Relation zur Gesamtl"ange der Genome betrachtet (s.\ Abb.\ \ref{medgenomes-phylo}d).
Nur rund 25\% der Gene eines Genoms werden aktiviert, die restlichen
75\% der Gene bleiben inaktiv.

Bei einer nicht aktiven Regel kann der
Aktionsteil beliebig ver"andert werden, ohne da"s die Gestalt der Pflanze ver"andert wird. Auch der Zustandsteil
einer nicht aktiven Regel kann ohne Auswirkungen auf das Wachstum der Pflanze ver"andert werden, solange der Zustandsteil
in der ver"anderten Form nicht einen Zustand, der in der Pflanze tats"achlich auftritt, spezifiziert. Bei einer aktiven
Regel sind dagegen nur wenige Mutationen des Bytes, das den Aktionsteil codiert, neutral; dies ergibt sich aus der
Redundanz der Aktionscodierung. Bei fester Genoml"ange ist daher die neutrale Menge (vgl.\ \ref{lndga-comparesingle}),
zu der ein Genom geh"ort, umso gr"o"ser, je weniger Gene das in dem Genom codierte Entwicklungsprogramm umfa"st.

Aus dem Vergleich der Ergebnisse der L"aufe \runname{shortgenomes} und \runname{medgenomes} geht somit hervor,
da"s die Stammbaumrekonstruktion umso st"arker beeintr"achtigt wird, desto kleiner die neutrale Menge,
zu der die Genome geh"oren,
ist. Bei kurzen Genomen sind die neutralen Mengen kleiner als bei langen Genomen, was zu einer generell schlechteren
Rekonstruktion bei kurzen Genomen f"uhrt. Das Anwachsen des Anteils aktivierter Gene auf 80\% aller Gene eines Genoms
in \runname{shortgenomes} hat einen signifikanten R"uckgang der Rekonstruktionsgenauigkeit zur Folge. Eine Beeintr"achtigung
der Stammbaumrekonstruktion durch nichtneutrale Evolutionseinfl"usse auf bis zu 25\% der Gene eines Genoms in
\runname{medgenomes} ist dagegen mit der hier verwendeten Methode nicht nachweisbar.

Die Abnahme der Rekonstruktionsgenauigkeit bei wachsender Komplexit"at der Entwicklungsprogramme kann
als Folge der "Uberlagerung zweier informationsbildender Prozesse interpretiert werden.
Neutrale Evolution kann als ein Proze"s gesehen werden, bei dem durch die Replikation von Sequenzen
und durch neutrale Selektion ein Stammbaum als Zufallsgraph generiert wird. Ein Mutationsproze"s bewirkt
dabei, da"s in den Sequenzen Information, die eine Rekonstruktion der Abstammung erlaubt, gebildet wird.
Diese Information, die auch als phylogenetisches Signal bezeichnet wird,
ist umso ausgepr"agter und deutlicher, je l"anger die evolvierenden Sequenzen sind.
Bei nichtneutraler Selektion bestehen Korrelationen zwischen der genomischen Sequenz eines Individuums
und seinen Vermehrungschancen. Einige Sequenzen werden dementsprechend systematisch h"aufiger repliziert
als andere Sequenzen. Dies stellt ebenfalls einen informationsbildenden Proze"s dar, bei dem es sich um die
evolution"are Optimierung handelt. Sowohl das phylogenetische Signal als auch die durch evolution"are
Optimierung generierte Information sind in Sequenzen codiert, wozu jeweils bestimmte Sequenzl"angen ben"otigt
werden. Bei fester Sequenzl"ange f"uhrt die Bildung
eines ausgepr"agten Signals durch den Proze"s evolution"arer Optimierung daher zu einem R"uckgang der zur
Codierung des phylogenetischen Signals effektiv verf"ugbaren Sequenzl"ange, was einen Verlust an Qualit"at
des phylogenetischen Signals nach sich zieht.



% selektionsrelevante Information verdraengt phylogentisches Signal aus Kanal der Informations"ubertragung
% Anteil junk entscheidend, Aenderungen der Dynamik bei komplexeren Entwicklungsprogrammen bleiben ohne
%     Auswirkungen auf Rekonstruktion.
% phylogenetisches Signal als Effekt der Diffusion in neutraler Menge

% Note: Neutrale Kontrolle ist eine Simulation, bei der alle Genome zu einer gemeinsamen neutralen Menge geh"oren.


\subsubsection{Rekonstruktion und Evolution der Mutationsrate}
\label{mutrecsection}

\begin{figure}[t]

\unitlength1cm
\begin{picture}(16,14)
\put(0,0){\makebox(16,14)[b]{\epsfxsize=16cm \epsffile{mutfgenomes.eps}}}
\end{picture}
\caption[Ergebnisse mutfgenomes]
{\label{mutfgenomesresults}
LindEvol-GA--Lauf \runname{mutfgenomes}, Selektionsrate 0.2, Austauschrate 0.07.
}
\end{figure}

\begin{figure}
\unitlength1cm
\begin{picture}(16,18.5)
\put(0,13.5){\makebox(16,5)[b]{\epsfxsize=14cm \epsffile{mutfgenomes-dp.eps}}}
\put(0,13){\makebox(16,0.5){\textbf{a:} \textsl{dT score} Differenz bei \textsl{parsimony}
zwischen \runname{mutfgenomes} und neutraler Kontrolle}}
\put(0,8){\makebox(16,5)[b]{\epsfxsize=14cm \epsffile{mutfgenomes-dpdiff.eps}}}
\put(1,7){\makebox(14,1){\parbox{14cm}{\textbf{b:} Differenz der durchschnittlichen \textsl{dT scores} mit \textsl{parsimony}
bei \runname{mutfgenomes} und der neutralen Kontrolle}}}
\put(0,4){\makebox(16,3)[b]{\epsfxsize=14cm \epsffile{mutfgenomes-avm.eps}}}
\put(0,3.5){\makebox(16,0.5){\textbf{c:} Evolution des durchschnittlichen Mutationsexponenten in \runname{mutfgenomes}}}
\put(0,0.5){\makebox(16,3)[b]{\epsfxsize=14cm \epsffile{mutfgenomes-ug.eps}}}
\put(0,0){\makebox(16,0.5){\textbf{d:} Evolution der durchschnittlichen Anzahl aktivierter Gene in \runname{mutfgenomes}}}
\end{picture}
\caption[Stammbaumrekonstruktion mutfgenomes]
{\label{mutfgenomes-phylo}
\textsl{dT score} Ergebnisse (a) aus \runname{mutfgenomes} (Boxen) und der zugeh"origen neutralen Kontrolle (Rauten).
Infolge der hohen Mutationsraten war die Analyse mit \textsl{neighbor joining} nicht m"oglich.
Zum Vergleich der durchschnittliche Mutationsexponent (c) und die durchschnittliche Anzahl aktivierter
Gene (d). Trotz der starken Schwankungen der \textsl{dT scores} ist eine Korrelation zwischen
dem evolution"aren Sprung zur aktiven Senkung der effektiven Mutationsraten und einem Anwachsen der
\textsl{dT scores} bei den Datens"atzen aus \runname{mutfgenomes} erkennbar.
}
\end{figure}

Um die Auswirkungen einer aktiven Absenkung der effektiven Mutationsraten auf die Stammbaumrekonstruktion zu untersuchen,
wurden an dem Lauf \runname{mutfgenomes} die entsprechenden Untersuchungen durchgef"uhrt. Bei diesem Lauf wurden
folgende Kontrollparameter gew"ahlt:

\medskip
\begin{tabular}{ll}
Populationsgr"o"se: & 50 \\
Selektionsrate: & 0.2 \\
Austauschrate: & 0.07 \\
Mutationsfaktor: & 4.0 \\
Breite der Welt: & 150 \\
H"ohe der Welt: & 30  \\
Genoml"ange der Startpopulation: & 20 Gene \\
L"ange einer Vegetationsperiode: & 30 Tage \\
Anzahl Generationen:             & 5000 \\
\verb|num_divide| & 192 \\
\verb|num_mutminus| & 32 \\
\verb|num_mutplus| & 32 \\
\verb|random_seed| & 12345 \\
\end{tabular}
\medskip

Abweichend von den bisher besprochenen L"aufen wurden ein gr"o"serer Mutationsfaktor und gr"o"sere Werte f"ur
\verb|num_mutminus| und \verb|num_mutplus| gew"ahlt. Der Grund hierf"ur ist, da"s mit den in den anderen L"aufen
verwendeten Werten und einer konstanten Genoml"ange von 20 Genen keine L"aufe mit ausgepr"agter Evolution einer
aktiven Absenkung der effektiven Mutationsraten beobachtet werden konnten. Die Auswahl eines gr"o"seren Mutationsfaktors
verursacht in diesem Lauf keine Probleme durch Extremwerte au"serhalb des Wertebereichs der Flie"skommaarithmetik,
weil bei der geringen Selektionsrate stark kooperationsorientierte Formen evolvieren (vgl.\ \ref{lndga-comparesingle}),
und somit keine extrem gro"sen Pflanzen auftreten. Au"ser dem Mutationsfaktor, der Austauschrate und der Konfiguration
des Genominterpreters wurden die Kontrollparameter so wie in \runname{medgenomes} gew"ahlt, um diese beiden L"aufe
m"oglichst gut vergleichen zu k"onnen.

Abb.\ \ref{mutfgenomesresults} zeigt das Verlaufsdiagramm von \runname{mutfgenomes}. Der evolution"are Sprung,
bei dem die aktive Verringerung der effektiven Mutationsraten auftritt ist kurz nach Generation 3500 deutlich zu
sehen. Er ist, wie dies auch f"ur \runname{xlong1008} beschrieben wurde (vgl.\ \ref{xlong1008section}), mit einem
Anstieg der DVK und einer qualitativen "Anderung der Distanzverteilungscharakteristik verbunden.

Abb.\ \ref{mutfgenomes-phylo} zeigt die Ergebnisse der Stammbaumrekonstruktion bei \runname{mutfgenomes} und bei der
zugeh"origen neutralen Kontrolle \runname{c-mutfgenomes}. Wegen der hohen Mutationsraten war eine Rekonstruktion
mit \textsl{neighbor joining} nur bei wenigen Datens"atzen aus \runname{mutfgenomes} und bei keinem Datensatz aus
\runname{c-mutfgenomes} m"oglich, daher k"onnen nur Ergebnisse mit \textsl{parsimony} pr"asentiert werden. Die Abbildung
zeigt, da"s mit dem evolution"aren Sprung, bei dem komplexe Entwicklungsprogramme auftreten, deren evolution"are
Stabilit"at durch eine aktive Senkung der effektiven Austauschrate erreicht wird, eine deutliche Verschlechterung
der Rekonstruktionsgenauigkeit einhergeht. Dies stellt einen Unterschied zu den mit \runname{medgenomes} erhaltenen
Ergebnissen dar.

\begin{sloppypar}
Der Anteil der aktivierten Gene ist in \runname{medgenomes} sogar etwas gr"o"ser als in \runname{mutfgenomes}. Die
Rekonstruktion verschlechtert sich also nicht in \runname{mutfgenomes} mehr als in \runname{medgenomes}, weil in
\runname{mutfgenomes} komplexere Entwicklungsprogramme mit entsprechend kleineren neutralen Mengen evolvieren.
Es ist somit anzunehmen, da"s der beobachtete R"uckgang auf die aktive Beeinflussung der effektiven Mutationsraten
zur"uckzuf"uhren ist. Dies wird umso einleuchtender, wenn man bedenkt, da"s das Ausma"s der Absenkung der effektiven
Mutationsraten auch bei Pflanzen mit identischen Entwicklungsprogrammen individuell verschieden sein kann, weil es
von der stochastisch gestalteten Lichtabsorption abh"angt. Es kommt somit nicht zu einer gleichm"a"sigen Erniedrigung der
Austauschrate, sondern vielmehr sind den einzelnen Kanten des tats"achlichen Stammbaums der Population unterschiedliche
effektive Mutationsraten zuzuordnen. Es ist bekannt, da"s dies zu Beeintr"achtigungen der Stammbaumrekonstruktion
f"uhren kann \cite{Felsenstein88}.
\end{sloppypar}

Der beobachtete Anstieg der Differenz der in \runname{mutfgenomes} und in der neutralen Kontrolle gemessenen \textsl{dT scores} 
kann jedoch nicht quantitativ mit der Evolution der Verringerung der effektiven Mutationsraten korreliert werden. Das Problem
hierbei ist, da"s die Rekonstruktion in \runname{mutfgenomes} global besser ist als in \runname{medgenomes}, was sich in
absolut niedrigeren \textsl{dT scores} bei \runname{mutfgenomes} niederschl"agt. Bei absolut niedrigeren \textsl{dT scores}
ist aber ein st"arkerer Anstieg bei einer Verschlechterung der Rekonstruktionsqualit"at zu erwarten. Zur Kl"arung
dieser Zusammenh"ange sind weitere Untersuchungen erforderlich.




% \subsubsection{Rekonstruktionsgenauigkeit in Abh"angigkeit von Austausch- und Selektionsrate}

% Baeume mit Multifurkationen bei hoeheren Selektionsraten -> dT-scores bei verschiedenen Selektionsraten
%     nicht vergleichbar.
% Aehnlichkeit parsimony - neighbor gutes Qualitaetskriterium.

\subsubsection{Absch"atzung der Rekonstruktionsgenauigkeit durch Vergleich verschiedener Verfahren}

\begin{figure}

\unitlength1cm
\begin{picture}(16,10)
\put(0,0){\makebox(16,10)[b]{\epsfxsize=14cm \epsffile{dpvsnjdp.eps}}}
\end{picture}
\caption{\label{phylo-dpnj}
Auftragung der durchschnittlichen \textsl{dT scores} zwischen den \textsl{parsimony}-B"aumen und dem
\textsl{neighbor joining}-Baum gegen die durchschnittlichen \textsl{dT scores} zwischen den \textsl{parsimony}-B"aumen und
dem tats"achlichen Stammbaum.
}
\end{figure}

Zur Untermauerung der Signifikanz von Rekonstruktionsergebnissen werden die fraglichen Daten regelm"a"sig mit verschiedenen
Verfahren zur Stammbaumrekonstruktion untersucht. H"aufig werden dabei \textsl{parsimony} und \textsl{neighbor joining}
eingesetzt (z.B.\ \cite{Doyle94, Purugganan95}). Erh"alt man mit verschiedenen Verfahren "ahnliche oder identische B"aume,
deutet dies auf eine gute Rekonstruktion hin. Dieses Verfahren kann anhand der in (\ref{slrecsection}) und
(\ref{mutrecsection}) diskutierten Ergebnisse quantitativ untersucht werden.

Zu diesem Zweck wurden f"ur alle Datens"atze die \textsl{dT scores} zwischen dem mit \textsl{neighbor joining} erhaltenen
Baum und allen mit \textsl{parsimony} rekonstruierten Stammb"aumen berechnet. Von diesen \textsl{dT scores} wurde der
Durchschnittswert berechnet. Weiterhin wurde in entsprechender Weise der durchschnittliche \textsl{dT score} zwischen
dem tats"achlichen Baum und den mit \textsl{parsimony} rekonstruierten Stammb"aumen berechnet. Auf diese Weise wurde von
jedem Datensatz, bei dem eine Rekonstruktion mit \textsl{neighbor joining} m"oglich war, ein Wertepaar aus dem
durchschnittlichen \textsl{dT score} zwischen dem tats"achlichen Baum und den \textsl{parsimony}-B"aumen und dem
durchschnittlichen \textsl{dT score} zwischen dem mit \textsl{neighbor joining} rekonstruierten Baum und den
\textsl{parsimony}-B"aumen erhalten. S"amtliche Wertepaare dieser Art aus den in den vorangegangenen Abschnitten
diskutierten L"aufen sind in Abb.\ \ref{phylo-dpnj} zusammengefa"st. Die Abbildung zeigt eine deutliche, lineare Korrelation
zwischen den beiden gemessenen Gr"o"sen. Eine lineare Regressionsanalyse der 545 Wertepaare  ergibt einen
Korrelationskoeffizienten von 0.756, die Korrelation ist also signifikant (vgl.\ \cite{Hainzl}, Seite 250). Die Steigung betr"agt 0.658, der
Achsenabschnitt ist 18.0. Der relativ gro"se Achsenabschnitt deutet darauf hin, da"s der beobachtete lineare Zusammenhang
bei kleineren \textsl{dT scores} m"oglicherweise nicht besteht. Das Ergebnis zeigt jedoch, da"s der \textsl{dT score}
zwischen dem mit \textsl{neighbor joining} berechneten Baum und den mit \textsl{parsimony} erhaltenen B"aumen zur
Absch"atzung der durch den \textsl{dT score} quantifizierten Abweichung zwischen den \textsl{parsimony}-B"aumen 
und dem tats"achlichen Stammbaum geeignet ist.


% deutliche Signale durch Verhinderung von Verzerrungen durch fehlerhaftes Alignment oder Abweichungen zwischen Mutationsmodell
%     und tatsaechlichem Mutationsprozess.
% relativ triviale Ergebnisse. Jedoch Demonstration eines grossen Einsatzfeldes fuer LindEvol.
% Uebertragung auf molekulare Evolution schwierig: Genome sind nicht durchg"angig homolog,
%     Orthologie-Paralogie-Problem, usw.
% Neu an Mutationsratenevolution in LindEvol: Energieabhaengigkeit


\section{Zusammenfassung der Ergebnisse von LindEvol-GA}

LindEvol-GA ist ein einfaches Evolutionsmodell, bei dem die Evolutionsbedingungen durch die Selektionsrate und die Mutationsraten,
die extern definierte Kontrollparameter sind, bestimmt werden. Die Bedeutung dieser Kontrollparameter f"ur die Ausbreitung
von Genomen und Entwicklungsprogrammen in der Population konnte mit mathematischen Methoden charakterisiert werden. Die
experimentellen Resultate stimmen mit den Ergebnissen der mathematischen Analyse gut "uberein. 

Die Evolution komplexer Entwicklungsprogramme und komplexer Pflanzenformen ist in einem ausgedehnten Parameterbereich von
LindEvol-GA m"oglich. Das durch die blockorientierte Genominterpretation implementierte System zur Codierung von Entwicklungsprogrammen
hat sich damit als geeignet f"ur die evolution"are Entwicklung von Wachstumsprogrammen erwiesen. Der Vergleich mit
\textsl{Tierran} \cite{Ray92}, einem System zur Codierung von Computerprogrammen, das speziell unter dem Gesichtspunkt
der Eignung f"ur evolution"are Optimierung entwickelt wurde, zeigt, da"s einige der von \textsc{Ray} idendifizierten Schl"usselprinzipien
zur Gew"ahrleistung der "`Evolvierbarkeit"' eines Codierungssystems auch bei der blockorientierten Genominterpretation
realisiert sind. Der Befehlssatz von \textsl{Tierran} wurde bewu"st klein gew"ahlt, er umfa"st 32 Befehle. Bei LindEvol-GA
umfa"st die Menge der m"oglichen Aktionen \verb|divide| mit acht verschiedenen Operanden sowie \verb|mut-| und \verb|mut+|,
also insgesamt zehn Befehle. Eine weitere Gemeinsamkeit zwischen \textsl{Tierran} und der blockorientierten Genominterpretation
ist, da"s beide Interpretersysteme keine ung"ultigen Programme kennen.

Die blockorientierte Genominterpretation wurde als Modell f"ur die genetisch gesteuerte Differenzierung in Abh"angigkeit
der Position einer Zelle im Organismus entwickelt. Als emergentes Ph"anomen tritt eine Evolution regulatorischer Netzwerke
auf. Derartige Netzwerke wurden auch f"ur molekulare Systeme beschrieben (vgl.\ \cite{Theissen95}). Diese Gemeinsamkeiten
zwischen LindEvol-GA und molekularbiologischen Systemen im expliziten und im emergenten Bereich berechtigt zu der Annahme,
da"s weitere emergente Evolutionsprozesse Entsprechungen in der molekularen Evolution haben.

Ein Ph"anomen, bei dem die Vermutung von Parallelen zur molekularen Evolution naheliegt, ist die in LindEvol-GA beobachtete
emergente Evolution kooperativen Wachstumsverhaltens. Die Tatsache, da"s dieses Ph"anomen aufgrund der
intrinsischen Gestaltung der Fitnessfunktion in LindEvol-GA zustandekommt (vgl.\ \ref{lndga-comparesingle}), weist
auf die Schl"usselbedeutung der intrinsischen Adaptation f"ur die Evolution des Lebens auf der Erde hin (vgl.\ \cite{Packard89}).

Ein weiteres Ph"anomen in der Evolution in LindEvol-GA, das auch bei molekularen Organismen beobachtet wird, ist die
aktive, energieabh"angige Absenkung der effektiven Mutationsraten (vgl.\ \cite{Watson}, Kapitel 12). Es konnten zwei
f"ur die Evolution der energieabh"angigen Verringerung der Mutationsraten relevante Mechanismen identifiziert werden.
Zum einen ist der Anteil von Nachkommen, die ein unver"andertes Entwicklungsprogramm erben, bei niedrigeren Mutationsraten
gr"o"ser, somit k"onnen sich Genome mit kleineren effektiven Mutationsraten bei konstanter Anzahl von Nachkommen schneller
ausbreiten. Zum anderen erm"oglichen geringere effektive Mutationsraten die Ausbreitung komplexerer Entwicklungsprogramme
und somit die Evolution von Pflanzenformen mit h"oherer Fitness.

Durch die Analyse von Serien von LindEvol-GA--Simulationen konnte ein Bereich charakterisiert werden, in dem sich durch
das Zusammenspiel von Mutation und Selektion komplexe taxonomische Strukturen ausbilden, die durch eine hohe DVK charakterisiert
sind. Es zeigte sich eine "uberraschend deutliche Korrelation zwischen der Ausbildung komplexer taxonomischer Strukturen und
der Evolution von Pflanzen mit hoher Fitness, die sich durch eine hohe morphologische Komplexit"at auszeichnen.
Mithilfe der mathematisch abgeleiteten Absch"atzung der kritischen Fehlergrenze f"ur Entwicklungsprogramme wurde
gezeigt, da"s im Bereich maximaler taxonomischer Komplexit"at eine \textsl{edge of chaos} verl"auft. Die Evolution
der aktiven Senkung der effektiven Mutationsraten konnte als \textsl{"`evolution to the edge of chaos"'} charakterisiert
werden. Sie ist stets mit einer deutlichen Zunahme der DVK verbunden.

Schlie"slich wurden die Standardmethoden \textsl{parsimony} und \textsl{neigbor joining} zur Stammbaumrekonstruktion mit
LindEvol-GA getestet. Dabei wurde eine Beeintr"achtigung der Rekonstruktion durch Selektionsdruck auf einen signifikanten
Anteil der Gene eines Genoms sowie durch die aktive Verringerung der effektiven Mutationsraten demonstriert. Weiterhin
zeigte sich, da"s die Anzahl der Unterschiede zwischen den mit \textsl{parsimony} und \textsl{neigbor joining} rekonstruierten
B"aumen eine gute Absch"atzung f"ur die Anzahl der Unterschiede zwischen den rekonstruierten B"aumen und dem tats"achlichen
Stammbaum ist.

Die in LindEvol-GA explizit als Kontrollparameter vorgegebene Selektionsrate spielte sowohl bei der emergenten
Evolution kooperativen Verhaltens als auch bei der Charakterisierung der Bedingungen f"ur die Entstehung komplexer
taxonomischer Strukturen eine entscheidende Rolle. Als Repr"asentation der biologischen Selektionsvorg"ange ist
das Verfahren der Simulation von Vegetationsperioden fester L"ange und der Reproduktion von Individuen in Abh"angigkeit
von ihrem Energiegehalt unbefriedigend. Die starre Modellierung der Selektion la"st weder die Evolution von Pflanzen,
die unterschiedlich alt, werden zu, noch erm"oglicht sie die Entstehung "okologischer Nischen f"ur kleine Pflanzenformen.
Zur Aufhebung dieser Beschr"ankungen wurde das Modell LindEvol-B entwickelt, das Gegenstand des n"achsten Kapitels ist.


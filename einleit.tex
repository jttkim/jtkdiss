\chapter{Einleitung}

Die gro"se Vielfalt der Lebensformen ist f"ur jeden Betrachter auch ohne wissenschaftlichen
Hintergrund offensichtlich. Die systematische Bestandsaufnahme und wissenschaftliche Betrachtung
dieser Vielfalt f"uhrte zur Entwicklung der Taxonomie, welche die Lebensformen aufgrund von
"Ahnlichkeiten und gemeinsamen Merkmalen in Taxons unterschiedlicher Kategorien klassifiziert.
Die Beobachtung der taxonomischen Struktur der Vielfalt des Lebens gab den Anla"s zur Entwicklung
der Evolutionstheorie durch \textsc{Darwin} \cite{Darwin}. Die Untersuchung der vielf"altigen Formen und Funktionen lebender
Systeme sowie ihre Entstehung durch die  Evolution bilden bis heute zentrale Schwerpunkte biologischer Forschung.

Die Komplexit"at der heutigen Lebewesen ist ein beeindruckendes Ergebnis der Evolution. Ebenso
faszinierend ist die Vielfalt der Lebensformen, die wir heute auf der Erde beobachten. Es liegt nahe, 
anzunehmen, da"s zwischen der Vielfalt und der Komplexit"at des Lebens Zusammenh"ange bestehen, da"s sie
sich gegenseitig bedingen (vgl.\ \cite{Stanley73}). Auf genetischer Ebene spielen Ver"anderungen und
Neuentwicklungen von Genen, welche die Morphogenese eines Organismus steuern, eine Schl"usselrolle bei
der Entstehung neuer Lebensformen (s.\ \cite{Carroll95,Theissen95}).
Die Untersuchung und Charakterisierung des Zusammenspiels dieser Faktoren
ist der Gegenstand der vorliegenden Arbeit. 

Es ist nicht m"oglich, diese Aspekte des Evolutionsgeschehens in formaler, quantitativer Weise zu beschreiben und somit
ein mathematisches Modell der Evolution zu erhalten, anhand dessen die Beziehungen zwischen Komplexit"at und Vielfalt
der Lebensformen untersucht werden k"onnen. Wie viele andere Systeme und Vorg"ange in der Biologie ist die Evolution
"`zu verwickelt, um der Mathematik zug"anglich zu sein"' (\textsc{Schr"odinger}, \cite{Schroedinger44}). Zur Untersuchung von komplexen
Systemen, die sich der mathematischen Modellierung entziehen, haben sich in den letzten Jahrzehnten Computermodelle
in vielen F"allen bew"ahrt. Daher wurde auch in der vorliegenden Arbeit die Computermodellierung als Untersuchungsmethode
gew"ahlt.


\section{Computermodelle komplexer biologischer \\ Systeme}

Praktisch jedes Objekt biologischer Forschung kann als Komponente eines komplexen Systems beschrieben
werden. Auf der molekularen Ebene sind biologische Makromolek"ule wie Proteine und Ribonucleins"auren komplexe
Objekte, die aus ihren jeweiligen monomeren Komponenten aufgebaut sind. Diese Makromolek"ule sind wiederum Komponenten
subzellul"arer Strukturen wie Membranen, Ribosomen, Filamenten und Chromosomen, die ihrerseits die Komponenten einer
Zelle bilden. Aus Zellen leitet sich eine Vielzahl organismischer Strukturen ab, neben der physikalischen Gestalt
eines Lebewesens sind auch verschiedene Subsysteme wie das neuronale System und das Immunsystem aus zellul"aren
Komponenten aufgebaut. Ensembles von Lebewesen k"onnen komplexe soziale Strukturen bilden. Viele verschiedene
Lebensformen bilden schlie"slich gemeinsam ein "Okosystem; die Gesamtheit aller "Okosysteme wird als Biosph"are
bezeichnet.

Von vielen dieser komplexen biologischen Systeme existieren bereits unterschiedliche Computermodelle. Viele dieser
Modelle wurden mit dem Ziel entwickelt, bestimmte Leistungen biologischer Systeme in technischen Systemen zu
realisieren, indem man die Arbeitsweise der biologischen Vorlage imitiert. Die folgenden Abschnitte geben einen
"Uberblick "uber die Computermodelle, die f"ur die vorliegende Arbeit von Bedeutung sind.


\subsection{Evolution"are Algorithmen}

Die heutigen Lebensformen sind durch eine Vielzahl teils hochkomplexer biologischer Systeme (z.B.\ Rezeptorsysteme
wie Augen, Ohren, das Nervensystem, motorische Systeme) sehr gut an das Leben
in ihrer nat"urlichen Umwelt angepa"st. Betrachtet man die Anpassung an die Umweltbedingungen als eine Aufgabe,
so kann man sagen, da"s viele Aufgaben dieser Art durch die Evolution in beeindruckender Weise gel"ost wurden. 
Die Grundidee evolution"arer Algorithmen ist es, einen Evolutionsproze"s zu simulieren und dabei als Adaptationskriterium
die Qualit"at der L"osung eines technischen Problems einzusetzen, um so
die F"ahigkeiten der Evolution zur L"osung komplexer Optimierungsprobleme zu nutzen.

Die ersten evolution"aren Algorithmen waren die Evolutionsstrategie von \textsc{Rechenberg} \cite{Rechenberg} und der
genetische Algorithmus von \textsc{Holland} \cite{Goldberg89}. Seither wurde eine gro"se Zahl evolution"arer Algorithmen
entwickelt.
Ein evolution"arer Algorithmus umfa"st typischerweise die folgenden Komponenten:

\begin{itemize}

\item Eine Repr"asentation f"ur L"osungen zu dem zu behandelnden Problem in endlichen Objekten.
    Bei der Evolutionsstrategie \cite{Rechenberg} wurden Vektoren reeller Zahlen verwendet, in
    genetischen Algorithmen \cite{Goldberg89} werden Bitstrings eingesetzt.

\item Eine Funktion, die die Qualit"at einer L"osung angibt. Diese Funktion wird als Fitnessfunktion
    bezeichnet.

\item Ein Selektionsverfahren, mit dem aus einer Menge von L"osungen eine Teilmenge ausgew"ahlt wird.
    Die Wahrscheinlichkeit der Auswahl einer L"osung h"angt dabei von ihrem Fitnesswert ab.

\item Genetische Operatoren, die neue L"osungen auf der Grundlage bestehender L"osungen generieren. Hierbei werden
    mehrere Typen von Operatoren unterschieden. Mutationsoperatoren erstellen neue L"osungen auf der
    Grundlage einer bestehenden L"osung, indem sie einzelne Komponenten zufallsgesteuert ver"andern.
    Rekombinationsoperatoren erzeugen neue L"osungen durch die Kombination von Teilen mehrerer bestehender L"osungen.
    Rekombinationsoperatoren, die zwei bestehende L"osungen verarbeiten, werden als \textsl{crossover}-Operatoren
    bezeichnet.

\end{itemize}

Die Arbeitsweise eines evolution"aren Algorithmus ist die iterative Generierung von Populationen von L"osungen
durch einen zyklischen Proze"s. Ein Zyklus besteht aus einer Fitnessbewertung der L"osungen in der Population und
einem Selektions- und Reproduktionsvorgang, bei dem mit dem Selektionsverfahren L"osungen ausgew"ahlt werden,
aus denen mithilfe der genetischen Operatoren eine neue Population zusammengestellt wird. Bei erfolgreichem
Ablauf dieses simulierten Evolutionsprozesses steigt die Qualit"at der L"osungen in der Population von Generation
zu Generation.


\subsection{Lindenmayer-Systeme}

Die von \textsc{Lindenmayer} entwickelten
Lindenmayer-Systeme \cite{Prusinkiewicz90,Prusinkiewicz94,RozenbergSalomaa86}, auch kurz L-Systeme genannt, sind formale Modelle
des Pflanzenwachstums. Die r"aumliche Struktur einer Pflanze wird bei L-Systemen in einer Zeichenkette codiert.
Das Entwicklungsprogramm, welches das Wachstum der Pflanze steuert, wird durch einen Satz von Ersetzungsregeln
repr"asentiert.
Das Wachstum einer Pflanze wird durch die Anwendung dieser Ersetzungsregeln auf diese Zeichenkette simuliert.
Eine solche Regel bewirkt die Ersetzung einer bestimmten Abfolge von Zeichen durch eine andere Zeichenfolge.
Ein Satz von Ersetzungsregeln wird als Regelwerk bezeichnet. Zur Simulation des Wachstums in unterschiedlichen
Entwicklungsphasen k"onnen mehrere Regelwerke bereitgestellt werden. Bestehen die zu ersetzenden Zeichenfolgen
stets nur aus einem Zeichen, spricht man von einem kontextfreien L-System.

Die Darstellung der r"aumlichen Gestalt der Pflanze erfolgt durch eine graphische Interpretation der entsprechenden
Zeichenkette nach dem \textsl{turtlegraphics}-Prinzip. Die eine Pflanze repr"asentierende Zeichenkette wird dabei
als eine Folge von Anweisungen f"ur einen Zeichenautomaten interpretiert.
Der Zeichenautomat besitzt einen \textsl{stack}, der es erm"oglicht, seine Position im Raum abzuspeichern und sp"ater
wieder aufzusuchen. Das Ablegen von Positionen auf dem \textsl{stack} und das Zur"uckholen dieser Positionen
erfolgt dabei durch spezielle Zeichen in der Zeichenkette, die als Metasymbole bezeichnet werden. Bei der Gestaltung
der Regelwerke ist darauf zu achten, da"s diese Metasymbole nur in der zul"assigen, paarweisen Art auftreten.
Dies ist gew"ahrleistet, wenn Metasymbole bei der Ersetzung stets nur paarweise erzeugt werden, und wenn sie nicht
durch weitere Ersetzungen wieder entfernt werden.

Die Anwendung von Ersetzungsregeln simuliert den Proze"s der Differenzierung. Werde beispielsweise ein apikales
Meristem durch das Symbol $M$ und ein Blattprimordium durch das Symbol $P$ in der Zeichenkette repr"asentiert,
dann kann die Entstehung eines solchen Primordiums durch die Ersetzungsregel $M \rightarrow PM$
simuliert werden. Durch die iterative Anwendung der Ersetzungsregeln entstehen
fraktale Strukturen. Selbst mit einfachen L-Systemen lassen sich "uberraschend komplexe Pflanzenformen
generieren (vgl.\ \cite{Tragut89}). Aufwendige L-Systeme erlauben die Erstellung von Computergraphiken simulierter
Pflanzen, die von Photographien realer Pflanzen kaum zu unterscheiden sind \cite{Prusinkiewicz90}.


\subsection{Zellularautomaten}

Zellularautomaten sind in der Physik inzwischen eine etablierte Technik zur Simulation von Syst\-emen in ausgedehnten
R"aumen. Auch zur Simulation biologischer Szenarien k"onnen Zellularautomaten eingesetzt werden. Ein einfaches, recht
gut bekanntes Beispiel ist das \textsl{Game of Life} von \textsc{Conway} \cite{Gardner70}.

Ein Zellularautomat besteht aus Elementarautomaten, die in einem Gitter angeordnet sind. Dieses Gitter kann ein- oder
mehrdimensional sein. Jeder Elementarautomat hat einen individuellen Zustand. Die Eingabe eines Elementarautomaten
besteht aus den Zust"anden der Automaten in seiner Nachbarschaft, zu der auch der Elementarautomat selbst geh"ort.
Typische Nachbarschaften bei zweidimensionalen, orthogonalen Gittern sind die von-Neumann- oder F"unfzell-Nachbarschaft, die
aus dem Elementarautomaten selbst und seinen unmittelbar in vertikale und horizontale Richtung angrenzenden Nachbarn
besteht, und die Moore- oder Neunzell-Nachbarschaft, die zus"atzlich die vier diagonal angrenzenden Nachbarn
einschlie"st \cite{Langton90}. In Abh"angigkeit von seiner Eingabe ermittelt der Elementarautomat als Ausgabe seinen
neuen Zustand mithilfe der Zustands"ubergangsfunktion, die f"ur alle Elementarautomaten gleich ist. Die Berechnung
des Zustands eines Elementarautomaten wird als \textsl{update} bezeichnet.

Die Zeit verl"auft bei Zellularautomaten in diskreten Zeitschritten. In einem Zeitschritt wird jeder Elementarautomat
einem \textsl{update} unterworfen. Dabei ist zwischen parallelem und asynchronem \textsl{update}
zu unterscheiden. Beim parallelen \textsl{update} wird der neue Zustand der Automaten in Abh"angigkeit von der
Konfiguration ihrer Nachbarschaft zu Beginn des Zeitschritts ermittelt. Die Zustands"anderung eines Elementarautomaten
wirkt sich erst im n"achsten Zeitschritt auf seine Nachbarn aus. Beim asynchronen \textsl{update} steuert dagegen stets die
jeweils aktuelle Umgebung den Zustands"ubergang, Automaten, die sp"ater bearbeitet werden, "`sehen"' also schon die
"Anderungen, die bei ihren Nachbarn zuvor stattgefunden haben.

Zellularautomaten werden zur Untersuchung von Ph"anomenen eingesetzt, die durch lokale Interaktionen zwischen
gleichartigen Agenten auf globalem Ma"sstab auftreten und einer mathematischen Modellierung nicht zug"anglich sind.
Ph"anomene dieser Art treten auch im bereits erw"ahnten \textsl{Game of Life} auf, beispielsweise kann der Verlauf
der Populationsgr"o"se im \textsl{Game of Life} einen chaotischen Verlauf haben, der mit Differentialgleichungsmodellen
der Populationsdynamik nicht erkl"art werden kann.


\section{\textsl{Artificial Life}}

In den vorangehenden Abschnitten wurden nur einige Ans"atze zur Umsetzung komplexer biologischer Systeme in Computermodelle
und Algorithmen angesprochen. Auch in anderen Disziplinen dienten in der j"ungeren Vergangenheit biologische Systeme
als Vorbild bei der Entwicklung unterschiedlichster Techniken. Ein geradezu klassisches Beispiel hierf"ur sind die
nach dem Vorbild des Gehirns entwickelten neuronalen Netze, die in der k"unstlichen Intelligenz intensiv eingesetzt
werden. In der Robotertechnik versucht man, autonom agierende Einheiten zu entwickeln, indem man sich bei der Gestaltung
verschiedener Subsysteme an biologischen Vorlagen orientiert. Schlie"slich erm"oglichten neue Entwicklungen im Bereich
der Biotechnologie und die Entdeckung der Ribozyme die Realisierung molekularer Evolutionsprozesse \textsl{in vitro}.
Alle diese Str"omungen flossen in den letzten Jahren des vergangenen Jahrzehnts zu einem neuen, interdisziplin"aren
Forschungsfeld zusammen, f"ur das sich die Bezeichnung \textsl{Artificial Life} etabliert hat. Im Vorwort von \cite{ALifeII}
definiert \textsc{Langton} \textsl{Artificial Life} wie folgt:

\begin{quotation}
"`Artificial Life is a field of study devoted to understanding life by attempting to abstract
the fundamental dynamical principles underlying biological phenomena, and recreating these dynamics in
other physical media -- such as computers -- making them accessible to new kinds of experimental manipulation and testing."'
\end{quotation}

Ein Kernbereich der \textsl{Artificial Life} Forschung ist demnach die Umsetzung biologischer Systeme in Computermodelle.
Ein Schwerpunkt in diesem Bereich ist die Modellierung der Evolution. F"ur die Informatik sind solche Modelle interessant,
weil evolution"are Algorithmen als universelle Optimierungsverfahren eingesetzt werden k"onnen. F"ur die Biologie
sind Evolutionsmodelle als Grundlage f"ur vergleichende Untersuchungen der Evolution von Bedeutung
(vgl. \cite{Keller94,Maley94,Ray92}).

Alle Ph"anomene des Lebens k"onnen als komplexe Ph"anomene klassifiziert werden. Eine zentrale Arbeitshypothese des
\textsl{Artificial Life} ist, da"s biologische Komplexit"at ein emergentes \cite{Assad92,Baas94,Cariani92}
Ph"anomen ist, das durch ein von einfachen
Regeln bestimmtes Zusammenspiel vieler elementarer Komponenten, die einen einfachen Aufbau haben, zustandekommt.
Da keine ausreichend allgemeine Definition von Komplexit"at existiert, bereitet die Charakterisierung der emergenten
Komplexit"at jedoch Schwierigkeiten. Dieses Problem wird dadurch versch"arft, da"s komplexe Ph"anomene typischerweise
in einem Bereich zwischen monotoner Ordnung und beliebigem Chaos zu beobachten sind \cite{CampbellLNCS88},
der h"aufig als \textsl{edge of chaos}
bezeichnet wird \cite{Gutowitz95,Kauffman92,Langton92}.

\textsl{Artificial Life} hat bereits eine Vielzahl bedeutender Ergebnisse hervorgebracht. So wurde das Dilemma der Hyperzyklentheorie
\cite{Eigen71}, da"s ein Hyperzyklus durch einen Parasiten unweigerlich ausgel"oscht wird, durch die Umsetzung des Hyperzyklenmodells
in ein Zellularautomatenmodell gel"ost \cite{Boerlijst92}. \textsc{Ray} gelang mit \textsl{Tierra} die Entwicklung einer
Programmiersprache, mit der
die evolution"are Entwicklung von Programmen m"oglich ist \cite{Ray92}. Die Theorie der RNA-basierten Evolution wurde sowohl mit
biochemischen Methoden \cite{Joyce92} als auch mittels mathematischer und computergest"utzter Modellbildung \cite{Forst95,Schuster94}
erheblich weiter entwickelt. Lindenmayer-Systeme und verwandte Verfahren wurden erfolgreich als Modelle der Morphogenese
eingesetzt \cite{deBoer92,Hamilton93,Prusinkiewicz94}. Die Vielfalt der M"oglichkeiten zur Anwendung von
\textsl{Artificial Life} Methoden
zur Untersuchung biologischer Fragestellungen \cite{Taylor94} motivierte zur Anfertigung der vorliegenden Arbeit.

    
\section{Zielsetzung und Vorgehensweise}

Gegenstand der vorliegenden Arbeit ist die Untersuchung der Evolution von Pflanzen mit komplexer Morphologie.
Zu diesem Zweck wurde eine Familie von Computermodellen der Evolution pflanzlicher Wuchsformen entwickelt.
Die Repr"asentation der Pflanzen erfolgt dabei in einem zwei- oder dreidimensionalen, diskreten, orthogonalen Gitter. Das
Pflanzenwachstum findet durch Teilungen einzelner Pflanzenzellen statt. Zur Steuerung einer Zellteilung werden dabei, wie
bei Zellularautomaten, nur Informationen aus der lokalen Umgebung der Zelle herangezogen. Anders als bei Zellularautomaten
gibt es keine global einheitliche Zustands"ubergangsfunktion. Jede Pflanze besitzt vielmehr ein individuelles Genom, das
als Lindenmayer-Regelwerk zur Steuerung der Aktivit"aten ihrer Zelle interpretiert wird. Bei der Gestaltung der
Interpretationsmechanismen wurden die derzeit bekannten Schl"usselmechanismen der genetischen Steuerung der Morphogenese so
weit wie m"oglich umgesetzt. Die Modelle werden kurz als LindEvol-Modelle bezeichnet.

In den Modellen traten verschiedene emergente Ph"anomene auf, die auch in der nat"urlichen Evolution beobachtet
werden. Dazu z"ahlen die Evolution kooperativen Verhaltens und die Evolution der aktiven Senkung der Mutationsraten.
F"ur beide Ph"anomene wurden auch auf der Grundlage einer einfachen mathematischen Analyse Erkl"arungen gefunden.

Als Vorschlag f"ur ein allgemeines Ma"s f"ur die Komplexit"at evolvierender Systeme wird in der vorliegenden Arbeit
die Distanzverteilungskomplexit"at eingef"uhrt. Anhand der LindEvol-Modelle wurde die Eignung der Distanzverteilungskomplexit"at
durch Vergleiche mit anderen Indikatoren und Ma"sen f"ur Komplexit"at demonstriert.


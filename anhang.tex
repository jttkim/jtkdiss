\begin{appendix}


\chapter{Elektronisches Quellenverzeichnis}
\label{electronicsources}

Neben der im Literaturverzeichnis aufgelisteten Literatur wurden bei
der Erstellung der vorliegenden Arbeit verschiedene durch das
Internet zug"angliche Informationsquellen intensiv genutzt.
Die wichtigsten elektronischen Informationsquellen werden in diesem Anhang
aufgelf"uhrt.


\section{\textsl{ftp}}
\label{ftp-app}

{\slshape ftp} ist die Abk"urzung f"ur engl.\ {\slshape file transfer protocol}, einem
Protokoll zur "Ubertragung von Dateien "uber das Internet. Im Allgemeinen geh"ort zu {\slshape ftp}
auch die Identifikation des Benutzers durch die Eingabe seiner
Kennung und seines Passworts ({\slshape login}). Bei der speziellen Form des {\slshape anonymous ftp}
l"a"st der Server jedoch den Zugriff auf daf"ur freigegebene Daten durch
jeden Benutzer zu. Der Benutzer hat dabei \verb|anonymous| oder
\verb|ftp| als Kennung und seine Email-Adresse als Passwort anzugeben.
{\slshape Anonymous ftp} ist einer der Hauptverbreitungswege f"ur
nichtkommerzielle Software. Viele Softwareprodukte, die im Rahmen der
vorliegenden Arbeit zum Einsatz kamen, wurden "uber {\slshape ftp}
bezogen. In der folgenden Tabelle sind einige {\slshape ftp}-Server
aufgef"uhrt, die im Rahmen dieser Arbeit als Bezugsquellen f"ur
nichtkommerzielle Software und Dokukmentation genutzt wurden.

\medskip
\noindent\begin{tabularx}{\linewidth}{|l|X|} \hline
Servername & Angebot \\ \hline
\verb|alife.santafe.edu|       & Ver"offentlichungen wissenschaftlicher Artikel im Bereich
                                 des {\slshape Artificial Life} \\
\verb|atari.archive.umich.edu| & Software f"ur Atari-Computer \\
\verb|ftp.uni-muenster.de|     & GNU Software f"ur Atari \\ 
\verb|ftp.uni-paderborn.de|    & Unix-Systemsoftware und Dokumentation, Atari-Soft\-ware \\
\verb|life.slhs.udel.edu|      & Dokumentation zu Tierra \protect\cite{Ray92} \\
\verb|prep.ai.mit.edu|         & Prim"arer Server f"ur GNU-Software und Dokumentation \\
\verb|src.doc.ic.ac.uk|        & Software f"ur Atari-Computer \\ \hline
\end{tabularx}
\medskip


\section{\textsl{World Wide Web}}

Das \textsl{World Wide Web} (WWW) ist eine Sammlung von Hypertext-Dokumenten, die in HTML
(\textsl{hypertext markup language}) angelegt sind. HTML-Dokumente werden auf Servern,
die mit dem Internet verbunden sind, zur Verf"ugung gestellt.
Mithilfe entsprechender Programme,
die als \textsl{WWW-Browser} bezeichnet werden, k"onnen die HTML-Dokumente von diesen
Servern abgerufen und betrachtet werden.  Ein HTML-Dokument kann
Verweise, die als Hyperlinks bezeichnet werden, auf andere HTML-Dokumente enthalten.
Dokumente werden dabei durch einen URL (\textsl{universal resource locator}) spezifiziert.
Damit kann nicht nur auf andere Dokumente, die auf demselben Server sind, verwiesen
werden, vielmehr sind Verweise auf Dokumente auf beliebigen Servern m"oglich. 
Im Folgenden sind die URLs einiger Dokumente aufgelistet, die f"ur die vorliegende Arbeit
relevante Informationen enthalten.

\begin{itemize}

\item \verb|http://alife.santafe.edu/|: Software, Dokumentation und wissenschaftliche Manuskripte
zum Themenbereich \textsl{Artificial Life}, vom Santa Fe Institute Studies in the Sciences of Complexity (SFI).

\item \verb|http://www.cpsc.ucalgary.ca/projects/bmv/vmm/title.html|: Einf"uhrung in L-Systeme und
verwandte Modelle der Morphogenese von \textsc{P.\ Prusinkiewicz} und \textsc{M.\ Hammel}.

\item \verb|http://www-personal.engin.umich.edu/~streak/bib/|: Bibliographie zum Themenbereich komplexe
Systeme von \textsc{T.\ Belding}.

\end{itemize}


\section{\textsl{Usenet}}

Das Usenet besteht aus derzeit rund 6000 Diskussionsforen, die {\slshape newsgroups}
genannt werden. Mithilfe entsprechender Software kann jeder Benutzer eines
Rechners mit Zugang zum Internet die Beitr"age in den {\slshape newsgroups}
lesen und selbst Beitr"age verfassen. \textsl{Newsgroups}, in denen Themen mit Bezug
zu der vorliegenden Arbeit diskutiert werden, sind \verb|comp.ai.alife|, \verb|comp.ai.genetic|,
\verb|bionet.molbio.evolution|, \verb|bionet.info-theory|, \verb|sci.bio.evolution|, 
\verb|sci.chaos| und \verb|sci.nonlinear|.



\chapter{Arbeitsbeschreibung der erstellten Software}
\label{technicalstuff}

Die Quelltexte s"amtlicher Software, die im Rahmen dieser Arbeit zur Durchf"uhrung von
Simulationen und ihrer Auswertung erstellt wurde, werden gleichzeitig mit der Ver"offentlichung
dieser Arbeit im Internet zur Verf"ugung gestellt. Sie kann herabgeladen werden:

\begin{itemize}
\item "uber \prgname{ftp} von \verb|ftp.mpiz-koeln.mpg.de|, Verzeichnis \verb|/pub/zwdv/kim/diss/|,
\item "uber das \textsl{world wide web} (WWW) von \\
    \verb|http://www.mpiz-koeln.mpg.de/~kim/kim.html|.
\end{itemize}

Mit den Quelltexten werden vollst"andige Anleitungen zur Bedienung der Programme
zur Verf"ugung gestellt. Zweck dieser Dokumentation ist es, zus"atzliche Einblicke in
die Implementation der Systeme zu geben, detailliertere Angaben zur Benutzung der Programme
sind den o.a.\ elektronischen Quellen zu entnehmen.

Alle Programme arbeiten kommandozeilenorientiert, um eine automatische Durchf"uhrung
von Simulationsprojekten optimal zu unterst"utzen.
Zur Durchf"uhrung von Simulationsl"aufen dient jeweils ein \introdef{Simulationsprogramm}.
Die Simulationsprogramme zu den Modellen sind:

\medskip
\begin{tabular}{ll}
LindEvol-GA: & \prgname{lnd1v20} \\
Neutrale Kontrolle zu LindEvol-GA: & \prgname{lnd1c20} \\
Serien von LindEvol-GA--L"aufen: & \prgname{l1v20loop} \\
LindEvol-B: & \prgname{lnd2v03} \\
LindEvol-P: & \prgname{lnd5v00} \\
LindEvol-P/3D: & \prgname{lnd5x00} \\
\end{tabular}
\medskip

Die Kontrollparameter f"ur einen Simulationslauf entnehmen die Simulationsprogramme einer
\introdef{Parameterdatei}. Der Name der Parameterdatei wird dabei mittels der Option
\texttt{-f} an das Simulationsprogramm "ubergeben. Die Standardsyntax zur Durchf"uhrung einer
Simulation lautet somit:

\begin{verbatimcmd}
\pcodemeta{Programmname} -f \pcodemeta{Parameterdateiname}
\end{verbatimcmd}

In der Parameterdatei sind s"amtliche Kontrollparameter, die f"ur eine eindeutige Spezifikation
des Simulationslaufs erforderlich sind, festgelegt. Das Parameterdatei f"ur den Lauf \runname{xlong0105}
(\ref{xlong0105section}) zeigt beispielhaft das Format der Parameterdateien:

\begin{verbatim}
    simname = xlong0105
    srate = 0.5
    m_replacement = 0.01
    m_insertion = 0.01
    m_deletion = 0.01
    psize = 50
    recombination = 0
    m_factor = 2.00
    world_width = 150
    world_height = 30
    num_days = 30
    glen_init = 20
    num_generations = 5000
    dmt_savefreq = 0
    ddistr_freq = 10
    phhist_freq = 0
    num_divide = 224
    num_mutminus = 16
    num_mutplus = 16
    random_seed = 12345
\end{verbatim}

Neben den jeweils bei der Definition der Modelle angegebenen Kontrollparametern k"onnen die Parameterdateien
zus"atzliche Parameter enthalten, die Art und Menge der w"ahrend des Laufs zu speichernden Information betreffen.
Beispielsweise bewirkt die Zeile \verb|ddistr_freq = 10|, da"s alle 10 Generationen eine Distanzverteilungsanalyse
durchgef"uhrt wird. Die Zeile \verb|phhist_freq = 0| besagt, da"s keine Datens"atze f"ur die Stammbaumrekonstruktion
angelegt werden sollen.

Die Simulationsprogramme legen ihre Ergebnisse in Dateien unterschiedlichen Typs ab. Die Dateinamen
werden durch das Anh"angen verschiedener Erweiterungen an den Namen des Simulationslaufs erzeugt.
Folgende Dateitypen werden dabei verwendet:

\medskip
\begin{tabular}{ll}
\verb|*.pro|: & Protokoldatei der numerischen Werte wie Fitness, Populationsgr"o"se usw. \\
\verb|*.dst|: & Distanzverteilungsdatei. \\
\verb|*.dat|: & Simulationszustand zu einem bestimmten Zeitschritt \\
\verb|*.pix|: & "`Pixeldatei"'. \\
\verb|*.stb|: & Stammbaumdatei, enth"alt die tats"achlichen Stammb"aume. \\
\end{tabular}
\medskip

Zur weiteren Verarbeitung dieser Dateien wurden folgende Programme erstellt:

\begin{itemize}

\item Das Programm \prgname{proplot} dient zur Erstellung von Verlaufsgraphiken auf der Grundlage
von Protokolldateien. Die Protokolldatei enth"alt einen Kopf, in dem die Typen und Bezeichnungen
der gespeicherten Daten angegeben werden. In einer Layoutdatei (Dateityp \verb|*.lay|) werden die
Anordnung der Boxen des Verlaufsdiagramm und die in den Boxen darzustellenden Daten beschrieben.
Durch einen Aufruf der Form

\begin{verbatimcmd}
proplot -i \pcodemeta{Protokolldatei} -o \pcodemeta{PostScript-Datei} -l \pcodemeta{Layoutdatei}
\end{verbatimcmd}

wird ein Verlaufsdiagramm in der Seitenbeschreibungssprache PostScript \cite{PostScript} erstellt.
Der Kopf der Protokolldatei enth"alt einen Verweis auf ggfs.\ angelegte Distanzverteilungsdateien,
damit k"onnen auch Distanzverteilungen in Verlaufsdiagrammen dargestellt werden.

\item Aus den Simulationszustandsdateien k"onnen mithilfe eines \prgname{stplot}-Programms Darstellungen
der Welt, Genomlistings und Darstellungen einzelner Pflanzen erstellt werden. Zu jedem Simulationsprogramm
gibt es ein eigenes Programm dieser Art. F"ur LindEvol-GA ist \prgname{stplot1}, f"ur LindEvol-B \prgname{stplot2},
f"ur LindEvol-P \prgname{stplot5} und f"ur LindEvol-P/3D \prgname{stplot5x} zu verwenden. Ferner unterst"utzen
die Simulationsprogramme die Option \verb|-r| zur Fortsetzung von Simulationen auf der Basis von Zustandsdateien
mit einem Aufruf der Form:

\begin{verbatimcmd}
lnd1v20 -r \pcodemeta{Simulationslaufname}
\end{verbatimcmd}

\item Mit den Pixeldateien ist die Betrachtung der Abl"aufe der ph"anotypischen Prozesse in der Welt
in einer Animation m"oglich. Zur Darstellung dieser Animationen dient das Programm \prgname{xlmovie},
das unter X-Windows l"auft.

\item \begin{sloppypar} Das Programm \prgname{treechk} dient dazu, die \textsl{dT scores} zwischen den tats"achlichen Stammb"aumen
in der Stammbaumdatei und rekonstruierten Stammb"aumen zu berechnen.
\end{sloppypar}

\end{itemize}

% \section{Programm \prgname{lnd1v20}: Simulationsprogramm zum Modell LindEvol-GA}
% 
% Das Programm \prgname{lnd1v20} dient zur Durchf"uhrung von Simulationen des Modells
% LindEvol-GA (Kap.\ \ref{lindevol-ga}). Das Programm entnimmt seine Eingabeparameter
% aus der Kommandozeile, insbesondere wird hier auch der Name der Parameterdatei, aus
% der die Kontrollparameter gelesen werden, angegeben. Die Kommandozeilenschnittstelle
% entspricht den bei UNIX "ublichen Standards, Optionen werden durch Buchstaben mit
% einem vorangestellten Minuszeichen angegeben; bei manchen Optionen mu"s ein weiteres
% Argument folgen. Die Optionen sind:
% 
% \noindent\begin{tabularx}{\linewidth}{|l|X|} \hline
% {\bfseries Option} & {\bfseries Bedeutung} \\ \hline
% \mycodestyle{-q} & Aktiviert den {\slshape quiet}-Modus. Diese Option unterdr"uckt
%     die Ausgabe einer Darstellung der Welt in ASCII-Zeichen nach jeder Generation
%     auf dem Terminal. \\
% \mycodestyle{-t} & Eine Testsimulation mit fest in das Programm eingebauten
%     Kontrollparametern wird ausgef"uhrt. Dies dient zur Kontrolle von Installationen
%     auf neuen Plattformen. \\
% \mycodestyle{-w} & Eine "`Welt"'-Datei, in der die Welt am Ende jeder Vegetationsperiode
%     in ASCII-Zeichen dargestellt wird, wird erzeugt. \\
% \mycodestyle{-p} & Eine Pixeldatei wird beim Simulationslauf erzeugt. In dieser ist
%     eine Beschreibung der Welt am Ende jeden Tages enthalten. Anhand der Pixeldatei
%     kann das Wachstum der Pflanzen mit dem Programm \prgname{xlmovie} als Animation
%     angezeigt werden. \\
% \mycodestyle{-f} \pcodemeta{file} & Die Kontrollparameter des Simulationslaufs werden
%     aus der Datei \pcodemeta{file} gelesen. \\
% \mycodestyle{-r} \pcodemeta{name} & Der Simulationslauf \pcodemeta{name} wird fortgesetzt.
%     Dies hat zur Voraussetzung, da"s der fortzusetzende Simulationslauf ordnungsgem"a"s
%     beendet wurde, wobei eine Zustandsdatei erzeugt wurde. \\
% \mycodestyle{-r} \pcodemeta{file} & Aus der Datei \pcodemeta{file} werden Generationen
%     gelesen, bei denen eine Zustandsdatei erzeugt werden soll. Diese Zustandsdateien
%     dienen zur weitergehenden Analyse einzelnger Genotypen und Ph"anotypen. \\ \hline
% \end{tabularx}
% 
% Neben den in (\ref{lndga-controlparams}) zusammengefa"sten Kontrollparametern k"onnen
% in den Kontrollparameterdateien einige weitere Parameter spezifiziert werden, die zur
% Steuerung der Art und der Menge der erzeugten Daten dienen. Der Verlauf der Simulation
% selbst wird durch diese Parameter nicht beeinflu"st:
% 
% \begin{tabularx}{\linewidth}{|l|X|} \hline
% {\bfseries Parametername}  & {\bfseries Funktion} \\ \hline
% \verb|dmt_savefreq| & H"aufigkeit (in Generationen), mit der die Matrix der Editierabst"ande
%     der Population abgespeichert wird. Wird hier 0 angegeben, findet keine
%     Speicherung von Distanzmatrizen statt. \\
% \verb|ddistr_freq|  & H"aufigkeit, mit der Distanzverteilungen und DVK berechnet
%     werden. Bei 0 unterbleibt die Distanzverteilungsanalyse. \\
% \verb|phhist_freq|  & H"aufigkeit, mit der die Genome der Population und der
%     korrekte Stammbaum der Population abgespeichert werden. Wird dieser Parameter auf
%     0 gesetzt, werden keine Genome und B"aume gespeichert. \\ \hline
% \end{tabularx}
% 
% \section{Programm \prgname{lnd2v03}: Simulationsprogramm zum Modell LindEvol-B}
% 
% \section{Programm \prgname{lnd5v00}: Simulationsprogramm zum Modell LindEvol-P}
% 
% \section{Programm \prgname{lnd5x00}: Simulationsprogramm zum Modell LindEvol-P/3D}
% 
% 
% \section{Programm \prgname{proplot} zur graphischen Ausgabe von Simulationsergebnissen}
% \label{proplot-doc}
% 
% Das Programm \prgname{proplot} dient zur Erstellung von Graphiken aus Verlaufsdaten von Simulationen
% 
% 
% \section{Programm \prgname{xlmovie} zur Anzeige von Animationen des Pflanzenwachstums}
% 
% % \section{Funktionsbibliothek {\tt gnlib} zur Protokollierung der Abstammung}
% 
% % \section{Funktionsbibliothek {\tt ptlib} zur Analyse und Ausgabe von Stammb"aumen}


\chapter{Referenz der Symbole und Formelbuchstaben}
\label{symbolref}

\noindent\begin{tabularx}{\linewidth}{|c|l|X|} \hline
Formel-      & Bezeichner               & Beschreibung \\
buchstabe    &                          & \\ \hline
$A(m, s, l)$ & -                        & Ausbreitungszahl eines Genoms der L"ange $l$ bei der Mutationsrate $m$
                                          und der Selektionsrate $s$ (\ref{mutconsequences}) \\
$A_E(m, s, r)$ & -                      & Ausbreitungszahl eines Entwicklungsprogramms aus $r$ aktiven Genen bei der Mutationsrate $m$
                                          und der Selektionsrate $s$ (\ref{mutconsequences}) \\
$C_d$        & -                        & Distanzverteilungskomplexit"at (DVK) (\ref{ddc-method}) \\
$C_i$        & -                        & $i$-te Zelle einer Pflanze (\ref{celldef}) \\
$D(g_i,g_j)$ & -                        & Distanz zwischen den Sequenzen $g_i$ und $g_j$ \\
$d$          & \verb|p_random_death|    & Basiswahrscheinlichkeit f"ur Absterben (\ref{lnd2-def}) \\
$d_e$        & \verb|rdeath_f_energy|   & Modifikator f"ur Absterbewahrscheinlichkeit in Abh"angigkeit
                                          von der Gesamtenergie einer Pflanze (\ref{lnd2-def}) \\
$d_n$        & \verb|rdeath_f_numcells| & Modifikator f"ur Absterbewahrscheinlichkeit in Abh"angigkeit von Anzahl Zellen 
					  einer Pflanze (\ref{lnd2-def}) \\
$d_l$        & \verb|leanover_penalty|  & Modifikator f"ur Absterbewahrscheinlichkeit in Abh"angigkeit vom "Uberhang 
					  einer Pflanze (\ref{lnd2-def}) \\
$E(C)$       & -                        & Energieparameter einer Zelle $C$, energielos (0) oder energiereich (1)
					  (\ref{celldef}) \\
$F(d)$       & -                        & H"aufigkeit des Distanzwerts $d$ in der Distanzmatrix einer
					  Population (\ref{ddistr-method})\\
$f(d)$       & -                        & relative H"aufigkeit des Distanzwerts $d$ in der Distanzmatrix
					  einer Population (\ref{ddc-method}) \\
$h$          & \verb|world_height|      & H"ohe der Welt (\ref{topodef}) \\
$L$          & -                        & "Uberhangskoeffizient (\ref{lnd2-def}) \\
$l$          & \verb|genome_length|     & L"ange eines Genoms in Bytes (\ref{genomedef}) \\
$l_s$        & \verb|glen_init|         & L"ange der am Simulationsstart zuf"allig generierten Genome (\ref{genomedef}) \\ \hline
\end{tabularx}

\noindent\begin{tabularx}{\linewidth}{|c|l|X|} \hline
Formel-      & Bezeichner               & Beschreibung \\
buchstabe    &                          & \\ \hline
$M_r$        & \verb|m_replacement|     & Basiswert der Austauschrate (\ref{mutationdef}) \\
$M_i$        & \verb|m_insertion|       & Basiswert der Insertionsrate (\ref{mutationdef}) \\
$M_d$        & \verb|m_deletion|        & Basiswert der Deletionsrate (\ref{mutationdef}) \\
$M_{\mathit{dup}}$  & \verb|m_duplication|     & Basiswert der Duplikationsrate (\ref{mutationdef}) \\
$m_f$        & \verb|m_factor|          & Modifikationsfaktor f"ur Mutationsraten (\ref{mutationdef}) \\
$N_C$        & -                        & Nachbarzellfunktion der Zelle $C$ (\ref{celldef}) \\
$n_c$        & \verb|num_cells|         & Anzahl Zellen einer Pflanze, ihre Gr"o"se (\ref{plantdef}) \\
$n_e$        & \verb|total_energy|      & Anzahl energiereicher Zellen einer Pflanze, ihre Gesamtenergie \\
$P_d$        & -                        & effektive Absterbewahrscheinlichkeit pro Zeitschritt (\ref{lnd2-def}) \\
$P_{\mathit{kill}}$ & -                        & Erfolgswahrscheinlichkeit eines Angriffs (\ref{lnd2-def}) \\
$p$          & \verb|psize|             & Populationsgr"o"se \\
$p_s$        & \verb|psize_init|        & Populationsgr"o"se beim Simulationsstart \\
$S(C)$       & -                        & Abbildung des lokalen Zustands einer Zelle $C$ auf eine ganze Zahl (\ref{celldef}) \\
$s$          & \verb|srate|             & Selektionsrate (\ref{lndga-def}) \\
$t(s,l)$     & -                        & kritische Fehlergrenze f"ur die Ausbreitung eines Genoms der L"ange $l$
                                          bei der Selektionsrate $s$ (\ref{mutconsequences}) \\
$t_E(s,r)$   & -                        & kritische Fehlergrenze f"ur die Ausbreitung eines Entwicklungsprogramms aus $r$ aktiven Genen
                                          bei der Selektionsrate $s$ (\ref{mutconsequences}) \\
$w$          & \verb|world_width|       & Breite der Welt (\ref{topodef}) \\
$x(C)$       & -                        & X-Koordinate der Zelle $C$ (\ref{celldef}) \\
$y(C)$       & -                        & Y-Koordinate der Zelle $C$ (\ref{celldef}) \\
$Z(C)$       & -                        & Zustand einer Zelle $C$ (\ref{celldef}) \\
$Z_i(C)$     & -                        & interner Zustand einer Zelle $C$ (\ref{celldef}) \\
$z(C)$       & -                        & Z-Koordinate der Zelle $C$ (\ref{celldef}) \\
$\mu$        & \verb|mut_counter|       & Mutationsexponent einer Pflanze (\ref{mutationdef}) \\
-            & \verb|num_days|          & Anzahl Tage pro Vegetationsperiode in LindEvol-GA (\ref{lndga-def}) \\
-            & \verb|num_divide|        & Anzahl Codewerte f"ur \verb|divide| (\ref{interpreterdef}) \\
-            & \verb|num_flyingseed|    & Anzahl Codewerte f"ur \verb|flyingseed| (\ref{interpreterdef}) \\
-            & \verb|num_localseed|     & Anzahl Codewerte f"ur \verb|localseed| (\ref{interpreterdef}) \\
-            & \verb|num_mutminus|      & Anzahl Codewerte f"ur \verb|mut-| (\ref{interpreterdef}) \\
-            & \verb|num_mutplus|       & Anzahl Codewerte f"ur \verb|mut+| (\ref{interpreterdef}) \\
-            & \verb|num_statebit|      & Anzahl Codewerte f"ur \verb|statebit| (\ref{interpreterdef}) \\
-            & \verb|random_seed|       & {\slshape seed} zur Initialisierung des Zufallsgenerators (\ref{rndgeneratordef}) \\ \hline
\end{tabularx}


% \chapter{Parameterdateien}
% \label{paramfile-appendix}

\chapter{Erkl"arung}

Ich versichere, da"s ich die von mir vorgelegte Dissertation selbst"andig angefertigt, die benutzten Quellen
und Hilfsmittel vollst"andig angegeben und die Stellen der Arbeit -- einschlie"slich Tabellen, Karten und
Abbildungen --, die anderen Werken im Wortlaut oder dem Sinn nach entnommen sind, in jedem Einzelfall
als Entlehnung kenntlich gemacht habe; da"s diese Dissertation noch keiner anderen Fakult"at oder Universit"at
zur Pr"ufung vorgelegen hat; da"s sie -- abgesehen von unten angegebenen Teilpublikationen --
noch nicht ver"offentlicht worden ist sowie, da"s ich eine solche Ver"offentlichung vor Abschlu"s des
Promotionsverfahrens nicht vornehmen werde.

Die Bestimmungen dieser Promotionsordnung sind mir bekannt. Die von mir vorgelegte Dissertation ist von
Prof.\ Dr.\ Heinz Saedler betreut worden.

\vspace*{3cm}
\noindent (Jan Tai Tsung Kim)

\vspace*{1cm}

\noindent Teilpublikationen:

\begin{itemize}

\item Kim, J.T. 1995. "`Emergente taxonomische Strukturen in einem Modell koevolvierender Pflanzen."'
    In \textsl{GWDG-Bericht} \textbf{40:} 91-113. Gesellschaft f"ur wissenschaftliche Datenverarbeitung, G"ottingen.

\item Kim, J.T. 1995. "`Using Distance Distributions to Measure Complexity of Populations."'
    In \textsl{1993 Lectures in Complex Systems,} herausgegeben von D.L.\ Stein und L.\ Nadel.
    Santa Fe Institute Studies in the Sciences of Complexity, Lectures Volume VI.
    Addison-Wesley, Redwood City, CA.

\item Kim, J.T. 1995. "`Distanzverteilungskomplexit"at: Ein Ma"s f"ur die strukturierte Diversit"at
    evolvierender Populationen."' In \textsl{Proceedings des Workshops "`Artificial Life "'},
    herausgegeben von K.\ Dautenhahn, T.\ Christaller, J.\ Heistermann, R.\ Hofest"adt, J.\ Kaiser,
    H.-G.\ Lipinski, C.\ M"uller-Schloer, W.\ Schiffmann und D.\ Sch"utt. GMD-Studien, Nr.\ 271, Sankt Augustin.

\end{itemize}


\chapter{Lebenslauf}

\begin{description}

\item[Name:] Kim.

\item[Vornamen:] \underline{Jan} Tai Tsung.

\item[Familienstand:] ledig.

\item[Geburtsdatum und -ort:] 27.\ Apr.\ 1965, Holzlar, Rhein-Sieg-Kreis (heute Stadt Bonn).

\item[Staatsangeh"origkeit:] deutsch.

\item[Grundschule:] 1971--1974, Martin-Luther-Schule, Br"uhl.

\item[Gymnasium:] \begin{sloppypar} 1974--1981, St"adtisches Gymnasium Br"uhl;
    1981--1984, K"o\-nig\-in-Lu\-ise-Schu\-le in K"oln.
    \end{sloppypar}

\item[Abitur:] 14.\ Jun.\ 1984, Abitur an der K"onigin-Luise-Schule, K"oln.

\item[Universit"at:] 1984--1991, Studium der Biologie an der Universit"at
    zu K"oln.

\item[Vordiplom:] 18.\ Apr.\ 1986 im Fach Biologie an der Universit"at zu K"oln.

\item[Diplom:] 26.\ Aug.\ 1991 im Fach Biologie an der Universit"at zu K"oln.
%    Hauptfach: Genetik, Nebenf"acher: Biochemie und Organische Chemie.
%    Diplomarbeit am Max-Planck-Institut f"ur Z"uchtungsforschung, K"oln,
%    "uber Computersimulationen der Evolution.

\item[Promotionsstudium:] 1991--heute, an der Universit"at zu K"oln.
%    Arbeit am Max-Planck-Institut f"ur Z"uchtungsforschung, Fortsetzung
%    der Arbeit an Computersimulationen der Evolution. Voraussichtlicher
%    Abschlu"s im Herbst 1995.

\item[Derzeitige] {\bf Anschrift:}
{\flushleft
  Jan T.\ Kim \\
  Moltkestra"se 91-95 \\
  50674 K"oln

  Tel.: (0221) 51 34 54 \\
%   \vspace{1ex}
%   {\bf Institutsadresse:} \\
%   Jan T.\ Kim \\
%   Max-Planck-Institut f"ur Z"uchtungsforschung \\
%   Carl-von-Linn{\'e}-Weg 10 \\
%   50829 K"oln \\
% 
%   Tel.:: +49-221-50 62-538 \\
%   Fax: +49-221-5062-513 \\
%   \vspace{1ex}
%   email: {\tt kim@mpiz-koeln.mpg.de}, {\tt a1624091@smail.rrz.uni-koeln.de} \\
%   WWW: \verb|http://www.mpiz-koeln.mpg.de/~kim/kim.html|
}

\end{description}

\chapter{Farbige Abbildungen}
\label{colorplates}

% \cleardoublepage

\begin{figure}[p]

\unitlength1cm
\begin{picture}(16,21)
\put(0,20.5){\makebox(16,2)[b]{\epsfxsize=10cm \epsffile{xlong0105-0000004w.eps}}}
\put(0,20){\makebox(16,0.5){{\bfseries a:} Generation 4}}
\put(0,18.0){\makebox(16,2)[b]{\epsfxsize=10cm \epsffile{xlong0105-0000040w.eps}}}
\put(0,17.5){\makebox(16,0.5){{\bfseries b:} Generation 40}}
\put(0,15.5){\makebox(16,2)[b]{\epsfxsize=10cm \epsffile{xlong0105-0000120w.eps}}}
\put(0,15){\makebox(16,0.5){{\bfseries c:} Generation 120}}
\put(0,13.0){\makebox(16,2)[b]{\epsfxsize=10cm \epsffile{xlong0105-0000500w.eps}}}
\put(0,12.5){\makebox(16,0.5){{\bfseries d:} Generation 500}}
\put(0,10.5){\makebox(16,2)[b]{\epsfxsize=10cm \epsffile{xlong0105-0002200w.eps}}}
\put(0,10){\makebox(16,0.5){{\bfseries e:} Generation 2200}}
\put(0,8.0){\makebox(16,2)[b]{\epsfxsize=10cm \epsffile{xlong0105-0002207w.eps}}}
\put(0,7.5){\makebox(16,0.5){{\bfseries f:} Generation 2207}}
\put(0,5.5){\makebox(16,2)[b]{\epsfxsize=10cm \epsffile{xlong0105-0002400w.eps}}}
\put(0,5){\makebox(16,0.5){{\bfseries g:} Generation 2400}}
\put(0,3.0){\makebox(16,2)[b]{\epsfxsize=10cm \epsffile{xlong0105-0003608w.eps}}}
\put(0,2.5){\makebox(16,0.5){{\bfseries h:} Generation 3608}}
\put(0,0.5){\makebox(16,2)[b]{\epsfxsize=10cm \epsffile{xlong0105-0004999w.eps}}}
\put(0,0){\makebox(16,0.5){{\bfseries i:} Generation 4999}}
\end{picture}

\caption{\label{xlong0105worlds}
Gesamtaufnahmen der Welt in verschiedenen Abschnitten des Laufs \runname{xlong0105}. Es wurde jeweils
die gesamte Welt am Ende der Vegetationsperiode dargestellt.
}
\end{figure}

% \cleardoublepage

\begin{figure}[p]

\unitlength1cm
\begin{picture}(16,20)
\put(0,18.0){\makebox(16,2)[b]{\epsfxsize=10cm \epsffile{xlong1008-0000050w.eps}}}
\put(0,17.5){\makebox(16,0.5){{\bfseries a:} Generation 50}}
\put(0,15.5){\makebox(16,2)[b]{\epsfxsize=10cm \epsffile{xlong1008-0000500w.eps}}}
\put(0,15){\makebox(16,0.5){{\bfseries b:} Generation 500}}
\put(0,13.0){\makebox(16,2)[b]{\epsfxsize=10cm \epsffile{xlong1008-0001400w.eps}}}
\put(0,12.5){\makebox(16,0.5){{\bfseries c:} Generation 1400}}
\put(0,10.5){\makebox(16,2)[b]{\epsfxsize=10cm \epsffile{xlong1008-0001700w.eps}}}
\put(0,10){\makebox(16,0.5){{\bfseries d:} Generation 1700}}
\put(0,8.0){\makebox(16,2)[b]{\epsfxsize=10cm \epsffile{xlong1008-0002200w.eps}}}
\put(0,7.5){\makebox(16,0.5){{\bfseries e:} Generation 2200}}
\put(0,5.5){\makebox(16,2)[b]{\epsfxsize=10cm \epsffile{xlong1008-0002400w.eps}}}
\put(0,5){\makebox(16,0.5){{\bfseries f:} Generation 2400}}
\put(0,3.0){\makebox(16,2)[b]{\epsfxsize=10cm \epsffile{xlong1008-0003000w.eps}}}
\put(0,2.5){\makebox(16,0.5){{\bfseries g:} Generation 3000}}
\put(0,0.5){\makebox(16,2)[b]{\epsfxsize=10cm \epsffile{xlong1008-0004999w.eps}}}
\put(0,0){\makebox(16,0.5){{\bfseries h:} Generation 4999}}
\end{picture}

\caption{\label{xlong1008worlds}
Gesamtaufnahmen der Welt in verschiedenen Abschnitten des Laufs \runname{xlong1008}. Es wurde jeweils
die gesamte Welt am Ende der Vegetationsperiode dargestellt.
}
\end{figure}

% \cleardoublepage

\begin{figure}[p]

\unitlength1cm
\begin{picture}(16,15)
\put(0,13.0){\makebox(16,2)[b]{\epsfxsize=10cm \epsffile{xlong00105-0004999w.eps}}}
\put(0,12.5){\makebox(16,0.5){{\bfseries a:} Generation 4999 von Lauf \runname{xlong00105}}}
\put(0,10.5){\makebox(16,2)[b]{\epsfxsize=10cm \epsffile{xlong00108-0004999w.eps}}}
\put(0,10.0){\makebox(16,0.5){{\bfseries b:} Generation 4999 von Lauf \runname{xlong00108}}}
\put(0,8.0){\makebox(16,2)[b]{\epsfxsize=10cm \epsffile{xlong0105-0004999w.eps}}}
\put(0,7.5){\makebox(16,0.5){{\bfseries c:} Generation 4999 von Lauf \runname{xlong0105}}}
\put(0,5.5){\makebox(16,2)[b]{\epsfxsize=10cm \epsffile{xlong0108-0004999w.eps}}}
\put(0,5.0){\makebox(16,0.5){{\bfseries d:} Generation 4999 von Lauf \runname{xlong0108}}}
\put(0,3.0){\makebox(16,2)[b]{\epsfxsize=10cm \epsffile{xlong1005-0004999w.eps}}}
\put(0,2.5){\makebox(16,0.5){{\bfseries e:} Generation 4999 von Lauf \runname{xlong1005}}}
\put(0,0.5){\makebox(16,2)[b]{\epsfxsize=10cm \epsffile{xlong1008-0004999w.eps}}}
\put(0,0){\makebox(16,0.5){{\bfseries f:} Generation 4999 von Lauf \runname{xlong1008}}}
\end{picture}

\caption{\label{xlongworlds}
Aufnahmen der Welt jeweils am Ende der Simulationsl"aufe.
}
\end{figure}

% \cleardoublepage

\begin{figure}[p]

\unitlength1cm
\begin{picture}(16,13)
\put(0,10.5){\makebox(16,2)[b]{\epsfxsize=16cm \epsffile{lnd2-w01000.eps}}}
\put(0,10){\makebox(16,0.5){Generation 1000}}
\put(0,8.0){\makebox(16,2)[b]{\epsfxsize=16cm \epsffile{lnd2-w10000.eps}}}
\put(0,7.5){\makebox(16,0.5){Generation 10000}}
\put(0,5.5){\makebox(16,2)[b]{\epsfxsize=16cm \epsffile{lnd2-w20000.eps}}}
\put(0,5){\makebox(16,0.5){Generation 20000}}
\put(0,3.0){\makebox(16,2)[b]{\epsfxsize=16cm \epsffile{lnd2-w30000.eps}}}
\put(0,2.5){\makebox(16,0.5){Generation 30000}}
\put(0,0.5){\makebox(16,2)[b]{\epsfxsize=16cm \epsffile{lnd2-w45000.eps}}}
\put(0,0){\makebox(16,0.5){Generation 45000}}
\end{picture}

\caption[Weltdarstellungen aus einer LindEvol-B--Simulation]
{\label{lnd2-worlds}
Weltdarstellungen aus dem LindEvol-B--Lauf \runname{m00007n000e25l15}
}
\end{figure}


\begin{figure}[p]

\unitlength1cm
\begin{picture}(16,5)
\put(0,3.0){\makebox(16,2)[b]{\epsfxsize=16cm \epsffile{m02n000e25l02w.eps}}}
\put(0,2.5){\makebox(16,0.5){\textbf{a:} $d_l = 0.02$}}
\put(0,0.5){\makebox(16,2)[b]{\epsfxsize=16cm \epsffile{m02n000e25l10w.eps}}}
\put(0,0){\makebox(16,0.5){\textbf{b:} $d_l = 0.1$}}
\end{picture}

\caption[Weltdarstellungen aus einer LindEvol-B--Simulation]
{\label{lnd2-otherworlds}
Weltdarstellungen aus LindEvol-B--L"aufen, jeweils nach dem Zeitschritt 49999. Bei $d_l = 0.02$
entsteht eine Pflanze, die die gesamte Welt "uberspannt. Bei $d_l = 0.1$ weisen diagonal
wachsende Pflanzen bereits eine ausreichende Stabilit"at auf, um das "Uberleben als Art
zu gew"ahrleisten. Buschf"ormig verzweigte Pflanzen sind jedoch erheblich stabiler und k"onnen
somit bis zur vollen H"ohe der Welt heranwachsen.
}
\end{figure}

% \cleardoublepage

\begin{figure}[p]

\unitlength1cm
\begin{picture}(16,21)

\put(0,20.5){\makebox(16,2)[b]{\epsfxsize=16cm \epsffile{m02n000e25l15-0000100w.eps}}}
\put(0,20.0){\makebox(16,0.5){\textbf{a:} Zeitschritt 100}}
\put(0,18.0){\makebox(16,2)[b]{\epsfxsize=16cm \epsffile{m02n000e25l15-0001200w.eps}}}
\put(0,17.5){\makebox(16,0.5){\textbf{b:} Zeitschritt 1200}}
\put(0,15.5){\makebox(16,2)[b]{\epsfxsize=16cm \epsffile{m02n000e25l15-0003800w.eps}}}
\put(0,15.0){\makebox(16,0.5){\textbf{c:} Zeitschritt 3800}}
\put(0,13.0){\makebox(16,2)[b]{\epsfxsize=16cm \epsffile{m02n000e25l15-0009995w.eps}}}
\put(0,12.5){\makebox(16,0.5){\textbf{d:} Zeitschritt 9995}}
\put(0,10.5){\makebox(16,2)[b]{\epsfxsize=16cm \epsffile{m02n000e25l15-0010110w.eps}}}
\put(0,10){\makebox(16,0.5){\textbf{e:} Zeitschritt 10110}}
\put(0,8.0){\makebox(16,2)[b]{\epsfxsize=16cm \epsffile{m02n000e25l15-0010265w.eps}}}
\put(0,7.5){\makebox(16,0.5){\textbf{f:} Zeitschritt 10265}}
\put(0,5.5){\makebox(16,2)[b]{\epsfxsize=16cm \epsffile{m02n000e25l15-0028250w.eps}}}
\put(0,5){\makebox(16,0.5){\textbf{g:} Zeitschritt 28250}}
\put(0,3.0){\makebox(16,2)[b]{\epsfxsize=16cm \epsffile{m02n000e25l15-0036585w.eps}}}
\put(0,2.5){\makebox(16,0.5){\textbf{h:} Zeitschritt 36585}}
\put(0,0.5){\makebox(16,2)[b]{\epsfxsize=16cm \epsffile{m02n000e25l15-0039040w.eps}}}
\put(0,0){\makebox(16,0.5){\textbf{i:} Zeitschritt 39040}}
\end{picture}

\caption[Weltdarstellungen aus einer LindEvol-P--Simulation]
{\label{lnd5-worlds}
Weltdarstellungen aus dem LindEvol-P--Lauf \runname{m02n000e25l15}
}
\end{figure}

% \cleardoublepage

\begin{figure}[p]
\unitlength1cm
\begin{picture}(16,20)
\put(0,15.5){\makebox(8,4.5)[b]{\epsfysize=3.7cm \epsffile{m02n000e25l30i-0008000w.eps}}}
\put(0,15){\makebox(8,0.5){\textbf{a:} Zeitschritt 8000}}
\put(8,15.5){\makebox(8,4.5)[b]{\epsfysize=3.7cm \epsffile{m02n000e25l30i-0016000w.eps}}}
\put(8,15){\makebox(8,0.5){\textbf{b:} Zeitschritt 16000}}
\put(0,10.5){\makebox(8,4.5)[b]{\epsfysize=4cm \epsffile{m02n000e25l30i-0020000w.eps}}}
\put(8,10.5){\makebox(8,4.5)[b]{\epsfysize=4cm \epsffile{m02n000e25l30i-0020000wt.eps}}}
\put(0,10){\makebox(16,0.5){\textbf{c:} Zeitschritt 20000, Vorderansicht und Vogelperspektive}}
\put(0,5.5){\makebox(8,4.5)[b]{\epsfysize=4cm \epsffile{m02n000e25l30i-0024000w.eps}}}
\put(8,5.5){\makebox(8,4.5)[b]{\epsfysize=4cm \epsffile{m02n000e25l30i-0024000wt.eps}}}
\put(0,5){\makebox(16,0.5){\textbf{d:} Zeitschritt 24000, Vorderansicht und Vogelperspektive}}
\put(0,0.5){\makebox(8,4.5)[b]{\epsfysize=4cm \epsffile{m02n000e25l30i-0040000w.eps}}}
\put(8,0.5){\makebox(8,4.5)[b]{\epsfysize=4cm \epsffile{m02n000e25l30i-0040000wt.eps}}}
\put(0,0){\makebox(16,0.5){\textbf{e:} Zeitschritt 40000, Vorderansicht und Vogelperspektive}}
\end{picture}

\caption[Weltdarstellungen des LindEvol-P/3D--Laufs \runname{m02n000e25l30i}]
{\label{lnd53d-worlds}
Darstellungen der Welt in \runname{m02n000e25l30i} bei verschiedenen Zeitschritten. Welten mit komplexeren Pflanzen
sind sowohl in der Ansicht von der Seite als auch in der Ansicht von oben (Vogelperspektive) dargestellt.
}
\end{figure}

% \cleardoublepage

\end{appendix}


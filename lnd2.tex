\chapter[LindEvol-B]{LindEvol-B: Modell mit asynchronen Generationen}
\label{lindevol-2}


\section{Definition von LindEvol-B}
\label{lnd2-def}

\subsection{Modellkomponenten}

LindEvol-B ist ein Modell, in dem das Wachstum der Pflanzen nicht in einer durch Vegetationsperioden
synchronisierten Weise abl"auft. Wie in LindEvol-GA findet eine blockorientierte Genominterpretation
statt, was durch das nachgestellte "`B"' angedeutet wird. LindEvol-B wurde mit den folgenden Komponenten aufgebaut:

\begin{itemize}
\item Topologie: Zweidimensionales Gitter (\protect\ref{topo2d}).
\item Zellparameter: Nur Energieparameter (\protect\ref{celldef}), kein interner Zustand.
    Der Zustand einer Zelle ist also durch die sie umgebende lokale Struktur vollst"andig bestimmt
    (Gl.\ \ref{stateq-nointernal}).
\item Zellaktionen: \verb|divide|, \verb|flyingseed|, \verb|localseed|,
    \verb|mut-| und \verb|mut+| (\protect\ref{cellactiondef}).
\item Genominterpretation: LindEvol-B arbeitet mit blockorientierter
    Genominterpretation (\protect\ref{blockinterdef}).
\item Mutation: In LindEvol-B kommen bitweise Austauschmutationen sowie
    Insertionen und Deletionen von Bl"ocken aus zwei Bytes vor (\protect\ref{mutationdef}).
\end{itemize}


\subsection{Absterbewahrscheinlichkeit und Angriffe}

Die evolution"are Selektion wird in LindEvol-B durch mehrere Komponenten
bestimmt, die wesentliche biologische und physikalische Faktoren, welche
in der Natur die "Uberlebens- und Fortpflanzungschancen mitbestimmen,
repr"asentieren.

Die Vermehrung der Pflanzen erfolgt durch die Aktionen \verb|flyingseed|
und \verb|localseed|, nicht, wie in LindEvol-GA, durch einen starren
Selektionsmechanismus. Damit ist der Fortpflanzungserfolg einer Pflanze
nicht in unmittelbarer Weise von ihrer Gr"o"se abh"angig.

Pflanzen sterben in LindEvol-B entweder durch einen zufallsabh"angigen
Tod oder durch einen Angriff. Durch den zufallsabh"angigen Tod werden
verschiedene physikalische und biologische Einfl"usse, die zum Absterben
einer Pflanze f"uhren k"onnen, in stark generalisierter Weise repr"asentiert.
Die Berechnung der Wahrscheinlichkeit f"ur den zufallsabh"angigen Tod $P_d$
einer Pflanze wird in LindEvol-B durch die Kontrollparameter $d$, $d_n$, $d_e$
und $d_l$ bestimmt. Zur Berechnung von $P_d$ wird zun"achst der "Uberhangkoeffizient
$L$ berechnet:

\begin{equation}
\label{leanover-eq}
L = \frac{2}{n_c(n_c - 1)} \left| \sum_{i=0}^{n_c-1} \frac{w}{2} - ((\frac{3w}{2} + x(C_i) - x(C_0)) \mbox{ mod } w) \right|
\end{equation}

Dieser Ausdruck liefert Zahlen zwischen 0 und 1, wobei gr"o"sere Werte eine st"arkere
Abweichung von einer um die Keimzelle ausbalancierten Anordnung der Zellen der Pflanze
anzeigen. Die Form des Summanden dient zur Gew"ahrleistung der Unabh"angigkeit
des "Uberhangskoeffizienten von der absoluten Position der Pflanze in der Welt:
Der Term $x(C_i) - x(C_0)$ gibt im Prinzip die horizontale Abweichung der Zelle $C_i$
von der Keimzelle $C_0$ an; wenn allerdings die Pflanze von der X-Koordinate 0
zur X-Koordinate $w - 1$ (oder in umgekehrter Richtung) gewachsen ist, liefert
dieser Term einen Wert des Betrags $w - 1$, obwohl die Pflanze nur eine Gitterposition
weitergewachsen ist. Durch die Addition von $3w / 2$ und die Modulo-Division durch
$w$ erh"alt man einen Term, der f"ur Zellen oberhalb der Keimzelle $w/2$ liefert,
f"ur links der Keimzelle gewachsene Zellen liefert er kleinere, f"ur rechts gewachsene
Zellen hingegen gr"o"sere Werte. Durch die Subtraktion dieses Werts von $w / 2$
bekommt man schlie"slich negative Werte f"ur Zellen rechts von der Keimzelle und
positive Werte f"ur Zellen links von der Keimzelle.

Weil Zellen immer nur in der Moore-Nachbarschaft bereits bestehender Zellen gebildet
werden k"onnen, mu"s auf allen X-Koordinaten zwischen der Keimzelle und der
links- bzw.\ rechts"au"sersten Zelle mindestens eine Zelle existieren. Daher ist
$n_c(n_c - 1) / 2$ der Maximalwert der Summe in Gl.\ \ref{leanover-eq}, und die
Division durch diesen Maximalwert gew"ahrleistet eine Normierung auf das Intervall
$[0, 1]$. Dieser Kehrwert ist nicht definiert falls $n_c = 1$, in diesem Fall
wird $L = 0$ gesetzt -- eine Pflanze, die nur aus einer Zelle besteht, weist
auch keinen "Uberhang auf.

Die Absterbewahrscheinlichkeit wird nun in Abh"angigkeit von ihrer Gr"o"se,
ihrer Gesamtenergie und ihrem "Uberhangskoeffizienten folgenderma"sen berechnet:

\begin{equation}
\label{deathprob-eq}
P_d = d \cdot n_c^{d_n} \cdot (n_e + 1)^{d_e} + d_l \cdot L
\end{equation}

Durch $d_n$ werden die generalisierten Auswirkungen der Gr"o"se einer Pflanze
auf ihre "Uberlebenswahrscheinlichkeit festgelegt. Bei einem negativen Wert
ist die "Uberlebenswahrscheinlichkeit umso besser, je gr"o"ser die Pflanze
ist, ein positiver Wert f"ur $d_n$ hat den gegenteiligen Effekt. Somit kann
zum Beispiel der nat"urliche Effekt eines besseren Schutzes vor Fra"s durch
die Wahl eines negativen Wertes f"ur $d_n$, die gr"o"sere Anf"alligkeit gegen
Wind und andere mechanische Belastungen durch einen positiven Wert simuliert
werden.

$d_e$ implementiert den gleichen Mechanismus f"ur die Gesamtenergie der Pflanze.
Hier liegt die Annahme, da"s Vitalit"at und
Widerstandskraft bei einer energiereichen Pflanze h"oher sind als bei einer
Pflanze mit geringer Gesamtenergie, nahe. 
Daher ist die Auswahl negativer Werte
f"ur $d_e$ sinnvoll.

Durch $d_l$ schlie"slich wird das Ausma"s der Destabilisierung einer Pflanze
durch mechanisches Ungleichgewicht bestimmt. Da die "uberhangsabh"angige
Komponente ein Summand von $P_d$ ist, kann ein starker "Uberhang nicht
durch Gr"o"se oder Energiegehalt kompensiert werden.

\introdef{Angriffe} finden in LindEvol-B statt, wenn bei \verb|divide|
die Zielposition der Tochterzelle bereits besetzt ist. In diesem Fall wird
die Pflanze zu der die bereits vorhandene Zelle geh"ort, angegriffen.
Die Wahrscheinlichkeit $P_{\mathit{kill}}$
f"ur den Erfolg eines Angriffs ist umgekehrt proportional zu der Gesamtenergie
der angegriffenen Pflanze $A$:

\begin{equation}
P_{\mathit{kill}} = \frac{1}{n_e(A)}
\end{equation}

Ist ein Angriff erfolgreich, wird die angegriffene Pflanze aus der Welt
entfernt. Sofern die Pflanze nicht sich selbst erfolgreich angegriffen
hat, verl"auft anschlie"send die Teilung erfolgreich. Bei einem erfolglosen
Angriff scheitert die Teilung.


\subsection{Simulationsablauf}

Wie auch in LindEvol-GA wird in LindEvol-B zu Beginn eines Simulationslaufs
eine Startpopulation von Genomen vorgegebener L"ange zufallsgesteuert
erzeugt. Mit dieser Startpopulation wird in eine Schleife eingetreten,
bei der jeweils ein Tag pro Durchlauf simuliert wird. 
Dieser Simulationsablauf ist in Abb.\ \ref{lnd2-flowchart} graphisch
dargestellt.

Innerhalb der Schleife "uber die Tage wird zun"achst die Bestrahlung der
Welt mit Licht simuliert (vgl.\ \ref{energydef}). Danach werden die Pflanze in einer
Reihenfolge abgearbeitet, die f"ur jeden Durchlauf erneut zufallsgesteuert
festgelegt wird. Bei der Abarbeitung einer Pflanze wird diese zun"achst
zufallsgesteuert mit der durch Gleichung \ref{deathprob-eq}
bestimmten Wahrscheinlichkeit entfernt. "Uberlebt die Pflanze diesen
Schritt, wird durch die Interpretation des Genoms der Pflanze f"ur jede Zelle der Pflanze
die auszuf"uhrenden Aktionen ermittelt und ausgef"uhrt. Im Anschlu"s daran
wird das Genom der Pflanze mutiert, wobei der w"ahrend der Abarbeitung
der Pflanzenzellen durch \verb|mut-| und \verb|mut+| ver"anderte Stand
des Mutationsexponenten ber"ucksichtigt wird.

\begin{figure}
\unitlength0.5cm
\begin{picture}(15,18)
\thicklines

\put(8,17){\framebox(14,1){Initialisierung der Population}}
\put(15,17){\vector(0,-1){2}}

\put(9,13.5){\makebox(12,1){F"ur jede Pflanze:}}

\put(9,10){\framebox(12,3){
\parbox{12cm}{
\centerline{Absterben mit Wahrscheinlichkeit}
\centerline{$P_d = d \cdot n_c^{d_n} \cdot (n_e + 1)^{d_e} + d_l \cdot L$}
}}}
\put(15,10){\vector(0,-1){1}}

\put(8,3){\framebox(14,12){}}
\put(15,3){\vector(0,-1){1}}

\put(9,6){\framebox(12,3){
\parbox{12cm}{
\centerline{Abarbeitung der Pflanze: Zellen}
\centerline{k"onnen sich teilen, Samen}
\centerline{produzieren etc.}}}}
\put(15,6){\vector(0,-1){1}}

\put(9,4){\framebox(12,1){Mutation des Genoms}}

\put(8,1){\framebox(14,1){Inkrementieren des Zeitschritts}}
\put(15,1){\line(0,-1){0.5}}
\put(19,0.5){\oval(8,1)[b]}
\put(23,0.5){\line(0,1){15}}
\put(19,15.5){\oval(8,1)[t]}
% \put(15,13.5){\vector(0,-1){0.5}}

\end{picture}

\caption{\label{lnd2-flowchart}
Flu"sdiagramm f"ur LindEvol-B
}
\end{figure}

Eine Auflistung der Kontrollparameter von LindEvol-B befindet sich in Tabelle \ref{lnd2-controlparams}.

% \subsection{Kontrollparameter von LindEvol-B}

\begin{table}[tb]

\noindent\begin{tabularx}{\linewidth}{|c|l|X|} \hline
Formel- & Bezeichner              & Beschreibung \\
buchstabe &                       & \\ \hline
$p_s$ & \verb|psize_init|         & Populationsgr"o"se am Anfang der Simulation \\
$l_s$ & \verb|glen_init|          & L"ange der am Simulationsstart zuf"allig generierten Genome \\
$M_r$ & \verb|m_replacement|      & Basiswert der Austauschrate \\
$M_i$ & \verb|m_insertion|        & Basiswert der Insertionsrate \\
$M_d$ & \verb|m_deletion|         & Basiswert der Deletionsrate \\
$m_f$ & \verb|m_factor|           & Modifikationsfaktor f"ur Mutationsraten \\
$d$   & \verb|p_random_death|     & Basiswahrscheinlichkeit f"ur Absterben pro Zeitschritt \\
$d_n$ & \verb|rdeath_f_numcells|  & Modifikationskoeffizient f"ur die Absterbewahrscheinlichkeit
                                    in Abh"angigkeit von der Gr"o"se der Pflanze \\
$d_e$ & \verb|rdeath_f_energy|    & Modifikationskoeffizient f"ur die Absterbewahrscheinlichkeit
                                    in Abh"angigkeit von der Gesamtenergie der Pflanze \\
$d_l$ & \verb|leanover_penalty|   & Maximalwert der Erh"ohung der Absterbewahrscheinlichkeit
                                    bei "uberh"angender Wuchsform \\
$w$   & \verb|world_width|        & Breite der Welt \\
$h$   & \verb|world_height|       & H"ohe der Welt \\
      & \verb|num_divide|         & Anzahl Codewerte f"ur \verb|divide| \\
      & \verb|num_flyingseed|     & Anzahl Codewerte f"ur \verb|flyingseed| \\
      & \verb|num_localseed|      & Anzahl Codewerte f"ur \verb|localseed| \\
      & \verb|num_mutminus|       & Anzahl Codewerte f"ur \verb|mut-| \\
      & \verb|num_mutplus|        & Anzahl Codewerte f"ur \verb|mut+| \\
      & \verb|random_seed|        & {\slshape seed} zur Initialisierung des Zufallsgenerators \\ \hline
\end{tabularx}

\caption{\label{lnd2-controlparams}
Kontrollparameter von LindEvol-B.
}
\end{table}

\section{Ergebnisse}
\label{lnd2-results}

Der Ergebnisteil zu LindEvol-B ist in drei Teile untergliedert. Im ersten Teil wird zun"achst die
Auswahl der Werte der Kontrollparameter erl"autert. Im anschlie"senden Teil wird ein Beispiellauf
detailliert besprochen. Darauf folgt eine kurze Diskussion der Auswirkungen von Variationen des
Kontrollparameters $d_l$, der die Modifikation der Absterbewahrscheinlichkeit in Abh"angigkeit von
der Ausbalanciertheit einer Pflanze bestimmt. Im letzten Teil wird die Auswertung einer Serie von
L"aufen mit ansteigenden Mutationsraten diskutiert. Dabei wird gezeigt, da"s auch bei LindEvol-B
die Evolution komplexer Pflanzenformen mit dem Auftreten hoher DVK-Werte korreliert ist, und da"s
der Bereich von Mutationsraten, in dem die Evolution komplexer Wuchsformen und hoher DVK-Werte beobachtet
wird, als Bereich einer \textsl{edge of chaos} charakterisiert werden kann.


\subsection{Auswahl der Kontrollparameter}
\label{lnd2-controlvalues}

Bei den L"aufen, die in den folgenden Teilen besprochen werden, wurden au"ser den Mutationsraten keine
anderen Kontrollparameter variiert. F"ur die Auswahl der konstanten Werte dieser Kontrollparameter
gaben folgende "Uberlegungen, von denen einige durch Testl"aufe best"atigt wurden, den Ausschlag:

\begin{itemize}

\item Abmessungen der Welt: Die Populationsgr"o"se in LindEvol-B ist ver"anderlich. Dies bedeutet,
da"s sie auch auf Null absinken kann; dann sind alle Pflanzen ausgestorben und es ist keine weitere
Evolution mehr m"oglich. Die Populationsgr"o"se ist in LindEvol-B durch die Breite der Welt
beschr"ankt. Bei geringer Breite der Welt und entsprechend kleiner Population k"onnen Fluktuationen
der Populationsgr"o"se leicht zum Aussterben der Gesamtpopulation f"uhren. Derartige Fluktuationen
k"onnen durch Pflanzen, die einen gro"sen Bereich der Welt "uberwuchern oder beschatten, ausgel"ost werden. Solche
"Uberwucherungsvorg"ange kommen bei kleineren Welten h"aufiger vor. Die Breitenausdehnung, die eine
buschf"ormige Pflanze erreicht, h"angt von ihrer H"ohe ab; daher kommen Pflanzen mit einer gr"o"seren
Breitenausdehnung umso h"aufiger vor, je gr"o"ser die H"ohe der Welt im Verh"altnis zur Breite ist.
Aus diesen Gr"unden wurden stets Welten von 500 * 50 Gitterpositionen verwendet.

\item Gr"o"se der Startpopulation: In den Testl"aufen wurde stets beobachtet, da"s ein gro"ser Teil
der am Simulationsstart generierten Pflanzen nicht vermehrungsf"ahig ist. Um die Wahrscheinlichkeit,
der Erzeugung mindestens einer vermehrungsf"ahigen Pflanze zu maximieren, wurde stets $p_{\mathit{init}}=w$,
also $p_{\mathit{init}}=500$ gesetzt.

\item Absterbebasiswahrscheinlichkeit $d$: Bei kleinen Werten der Absterbewahrscheinlichkeit sind primitive,
einzellige Formen evolution"ar ausgesprochen stabil, eine m"ogliche Evolution komplexerer Formen tritt daher erst nach
langer Anfangsphase ein. Allzu hohe Werte f"uhren dagegen zu einem Aussterben der gesamten Population in der
Anfangsphase. Testl"aufe ergaben, da"s die Populationen bei $d=0.4$ schnell aussterben. Bei $d=0.35$ k"onnen sich
hingegen primitive Formen aus der Startpopulation stabil etablieren. Je kliner der Wert von $d$ gew"ahlt wird,
desto stabiler sind die einzelligen Pflanzen und desto geringer ist folglich die Tendenz zur Evolution komplexerer
Formen. Um eine allzu starke Stabilisierung dieser
einfachen Formen zu vermeiden, wurde in den Simulationsl"aufen stets $d=0.35$ gew"ahlt.

\item Modifikatoren der Absterbewahrscheinlichkeit: In LindEvol-GA bestimmt der Energiegehalt einer Pflanze die
Vermehrungschance ihres Genoms. In Anlehnung daran soll in LindEvol-B ein hoher Energiegehalt zumindest die
"Uberlebenschancen einer Pflanze vergr"o"sern. Dies wird durch die Auswahl von $d_e=-2.5$ erreicht. Auf eine gr"o"senabh"angige
Beeinflussung der "Uberlebenschancen wurde verzichtet, also $d_n = 0.0$. F"ur die Erh"ohung der Absterbewahrscheinlichkeit
durch einseitig "uberh"angendes Wachstum wurde $d_l = 0.15$ gew"ahlt; mit diesem Wert sind einfache, diagonale
Pflanzen gegen"uber anderen Formen extrem benachteiligt, w"ahrend komplexe, buschartige Pflanzen recht stabil sein
k"onnen. Auf die Bedeutung des Wertes von $d_l$ wird in (\ref{leanover-discussion}) n"aher eingegangen.

\item Mutationsraten: Die Mutationsraten wurden, wie in LindEvol-GA, stets in einem festen Verh"altnis zueinander
gew"ahlt. Da die bitweise Austauschmutation eine erheblich geringere Ver"anderung als Insertionen und
Deletionen darstellt, sollte die Austauschrate gr"o"ser als die Insertions- und die Deletionsrate sein
Daher wurden Austausch-, Insertions- und Deletionsrate stets im Verh"altnis 5:1:1 eingesetzt.

\item Mutationsfaktor: Bei der hier gew"ahlten Gr"o"se der Welt k"onnen die Pflanzen in LindEvol-B erheblich
gr"o"ser werden als die Pflanzen in den in \ref{lndga-results} besprochenen L"aufen. Dementsprechend vergr"o"sert
sich das Maximum der Anzahl der \verb|mut-| bzw.\ \verb|mut+|-Aktionen, die eine Pflanze pro Zeitschritt ausf"uhren
kann. Daher wurde der Mutationsfaktor mit $m_f=1.5$ kleiner als in LindEvol-GA gew"ahlt.

\item Konfiguration des Genominterpreters: Hier wurden folgende Werte verwendet:

\begin{tabular}{ll}
\verb|num_divide|: 192 \\
\verb|num_flyingseed|: 16 \\
\verb|num_localseed|: 16 \\
\verb|num_mutminus|: 16 \\
\verb|num_mutplus|: 16 \\
\end{tabular}

Diese Werte wurden durch Erweiterung der bei LindEvol-GA verwendeten Konfiguration erhalten; es fanden keine
Tests mit unterschiedlichen Konfigurationen des Genominterpreters statt.

\item Stichproben bei Distanzverteilungsanalyse: Die Erstellung der Distanzverteilungen sowie die Berechnungen der
DVK erfolgten mit Stichproben aus 50 zufallsgesteuert ausgew"ahlten Pflanzen einer Population (vgl.\ \ref{ddistr-method}).

\end{itemize}


\subsection{Einzelsimulation}
\label{lnd2-individualrun}

\begin{figure}
\unitlength1cm
\begin{picture}(16,20.5)
\put(0,0){\makebox(16,20.5)[b]{\epsfxsize=16cm \epsffile{m00007n000e25l15.eps}}}
\end{picture}

\caption[Verlaufsdaten eines LindEvol-B--Laufs]
{\label{lnd2-result}
Verlaufsdiagramm des LindEvol-B--Laufs \runname{m00007n000e25l15}. Einige Boxen enthalten
zwei Kurven. In der Beschriftung dieser Boxen sind jeweils die Bezeichnungen f"ur beide
Me"sgr"o"sen, getrennt durch zwei Schr"agstriche, angegeben. Bei allen Boxen dieser Art
ist einer der beiden Werte stets kleiner oder gleich dem anderen Wert: Die Anzahl verschiedener
Genotypen kann maximal der Populationsgr"o"se entsprechen; der Energiegehalt einer Pflanze
ist durch ihre Gr"o"se nach oben beschr"ankt; und die Anzahl der fliegenden Samen kann die
Gesamtzahl der Samen nicht "ubersteigen.
}
\end{figure}

% \begin{figure}[p]
% 
% \unitlength1cm
% \begin{picture}(16,13)
% \put(0,10.5){\makebox(16,2)[b]{\epsfxsize=16cm \epsffile{lnd2-w01000.eps}}}
% \put(0,10){\makebox(16,0.5){Generation 1000}}
% \put(0,8.0){\makebox(16,2)[b]{\epsfxsize=16cm \epsffile{lnd2-w10000.eps}}}
% \put(0,7.5){\makebox(16,0.5){Generation 10000}}
% \put(0,5.5){\makebox(16,2)[b]{\epsfxsize=16cm \epsffile{lnd2-w20000.eps}}}
% \put(0,5){\makebox(16,0.5){Generation 20000}}
% \put(0,3.0){\makebox(16,2)[b]{\epsfxsize=16cm \epsffile{lnd2-w30000.eps}}}
% \put(0,2.5){\makebox(16,0.5){Generation 30000}}
% \put(0,0.5){\makebox(16,2)[b]{\epsfxsize=16cm \epsffile{lnd2-w45000.eps}}}
% \put(0,0){\makebox(16,0.5){Generation 45000}}
% \end{picture}
% 
% \caption[Weltdarstellungen aus einer LindEvol-B--Simulation]
% {\label{lnd2-worlds}
% Weltdarstellungen aus dem LindEvol-B--Lauf \runname{m00007n000e25l15}
% }
% \end{figure}

% \begin{figure}[p]
% 
% \unitlength1cm
% \begin{picture}(16,5)
% \put(0,3.0){\makebox(16,2)[b]{\epsfxsize=16cm \epsffile{m02n000e25l02w.eps}}}
% \put(0,2.5){\makebox(16,0.5){\textbf{a:} $d_l = 0.02$}}
% \put(0,0.5){\makebox(16,2)[b]{\epsfxsize=16cm \epsffile{m02n000e25l10w.eps}}}
% \put(0,0){\makebox(16,0.5){\textbf{b:} $d_l = 0.1$}}
% \end{picture}
% 
% \caption[Weltdarstellungen aus einer LindEvol-B--Simulation]
% {\label{lnd2-otherworlds}
% Weltdarstellungen aus LindEvol-B--L"aufen, jeweils nach dem Zeitschritt 49999. Bei $d_l = 0.02$
% entsteht eine Pflanze, die die gesamte Welt "uberspannt. Bei $d_l = 0.1$ weisen diagonal
% wachsende Pflanzen bereits eine ausreichende Stabilit"at auf, um das "Uberleben als Art
% zu gew"ahrleisten. Buschf"ormig verzweigte Pflanzen sind jedoch erheblich stabiler und k"onnen
% somit bis zur vollen H"ohe der Welt heranwachsen.
% }
% \end{figure}

\begin{figure}[t]

\unitlength1cm
\begin{picture}(16,13)
\put(0,9.5){\makebox(7,4)[b]{\epsfysize=4cm \epsffile{lnd2-p01000a.eps}}}
\put(0,9.0){\makebox(7,0.5){\textbf{a:} Zeitschritt 1000}}
\put(8,9.5){\makebox(7,4)[b]{\epsfysize=4cm \epsffile{lnd2-p20000a.eps}}}
\put(8,9.0){\makebox(7,0.5){\textbf{b:} Zeitschritt 20000}}
\put(0,5.0){\makebox(7,4)[b]{\epsfysize=4cm \epsffile{lnd2-p20000b.eps}}}
\put(0,4.5){\makebox(7,0.5){\textbf{c:} Zeitschritt 20000}}
\put(8,5.0){\makebox(7,4)[b]{\epsfysize=4cm \epsffile{lnd2-p30000a.eps}}}
\put(8,4.5){\makebox(7,0.5){\textbf{d:} Zeitschritt 30000}}
\put(0,0.5){\makebox(7,4)[b]{\epsfysize=4cm \epsffile{lnd2-p45000a.eps}}}
\put(0,0.0){\makebox(7,0.5){\textbf{e:} Zeitschritt 45000}}
\put(8,0.5){\makebox(7,4)[b]{\epsfysize=4cm \epsffile{lnd2-p45000c.eps}}}
\put(8,0.0){\makebox(7,0.5){\textbf{f:} Zeitschritt 45000}}
\end{picture}

\caption[Genome in LindEvol-B]
{\label{lnd2-genomes}
Genome von typischen Pflanzen aus verschiedenen Generationen in Lauf \runname{m00007n000e25l15}. Oben ist jeweils
der Ph"anotyp der Pflanze dargestellt, darunter ihr Genom in graphischer Form.
}
\end{figure}

Der Lauf \runname{m00007n000e25l15} wurde mit den Mutationsraten $m_r=0.00035$, $m_i=0.00007$ und $m_d=0.00007$
durchgef"uhrt; die restlichen Kontrollparameter waren entsprechend den Angaben in (\ref{lnd2-controlvalues})
gesetzt. Abb.\ \ref{lnd2-result} zeigt das Verlaufsdiagramm von \runname{m00007n000e25l15}. F"ur einige Me"sgr"o"sen,
die sich im Laufe des Lebens einer Pflanze ver"andern,
wurden die Maximalwerte und die Durchschnittswerte in verschiedenen Boxen dargestellt. Dies war erforderlich,
weil bei LindEvol-B keine Synchronisierung der Lebensphasen stattfindet, wie dies bei LindEvol-GA durch die
Vegetationsperioden geschieht. Somit befinden sich in einer Population Pflanzen unterschiedlichen Alters; und in die
Berechnung der Durchschnittswerte geht eine gr"o"sere Menge junger Pflanzen ein, bei denen die entsprechenden 
Werte, wie z.B.\ die Gr"o"se, typischerweise klein sind. Der Durchschnittswert einer Gr"o"se kann daher durchaus
eine Gr"o"senordnung unterhalb des Maximalwerts liegen. In diesem Fall ist er auf einer gemeinsamen Skala mit dem
Maximalwert nicht ad"aquat visualisierbar.

Bereits an den Kurven der Populationsgr"o"se und der Anzahl verschiedener Genome k"onnen verschiedene evolution"are
Spr"unge abgelesen werden. Die Populationsgr"o"se bleibt w"ahrend der ersten 35000 Zeitschritte fast konstant bei
ihrem Maximalwert von 500 (der Breite der Welt), um dann leicht abzusinken. Bei der Anzahl verschiedener Genotypen
sind zwei weitere Spr"unge zu sehen, ein stark ausgepgr"agter Anstieg beim Zeitschritt 14000 und ein weniger abrupter
weiterer Anstieg beim Zeitschritt 22500. Synchron mit dem R"uckgang der Populationsgr"o"se geht auch die Anzahl verschiedener
Genotypen wieder zur"uck.

Abb.\ \ref{lnd2-worlds} (Seite \pageref{lnd2-worlds}) zeigt Darstellungen der Welt in den verschiedenen Phasen
der Evolution. In der Anfangsphase
bildet sich zun"achst ein dichter Teppich aus einzelligen Pflanzen aus. Das Keimzellgen dieser Pflanze veranla"st
die Keimzelle zur Produktion von Samen (Abb.\ \ref{lnd2-genomes}a). Da keine Zellen erzeugt werden, werden au"ser dem
Keimzellgen keine weiteren Gene aktiviert.

Bei Generation 1000 ist eine vertikal
gewachsene Pflanze zu sehen. Diese produziert jedoch keine Samen, sie ist also steril. Da diese Pflanze eine gr"o"sere
Zahl energiereicher Zellen aufweist, ist ihre Absterbewahrscheinlichkeit sehr gering, was ihr "Uberleben seit der
Anfangsphase erkl"art. Aus der Kurve des maximalen Alters in Abb.\ \ref{lnd2-result} ist ersichtlich, da"s diese
Pflanze erst nach dem Zeitschritt 2500 abstirbt.

Beim Zeitschritt 10000 hat sich die Situation nicht wesentlich ge"andert. In Abb.\ \ref{lnd2-worlds} sind in diesem
Zeitschritt einige mehrzellige Pflanzen zu sehen. Eine Untersuchung ergab jedoch, da"s alle mehrzelligen Pflanzen
in dieser Population steril sind.

Im Zeitschritt 20000 sind die den Boden der Welt bedeckenden Pflanzen zweizellig. Abb.\ \ref{lnd2-genomes}b zeigt
das erwartungsgem"a"s einfache Entwicklungsprogramm einer solchen zweizelligen Pflanze. Der Vorteil dieser Pflanzen
gegen"uber den einzelligen Pflanzen liegt in ihrer h"oheren Lebenserwartung. W"ahrend die obere Zelle Samen produziert,
bleibt die untere Zelle, nachdem sie erneut ein Photon absorbiert hat, energiereich. Mit der Einstellung $d_e=-2.5$
f"uhrt dies dazu, da"s die Absterbewahrscheinlichkeit auf weniger als die H"alfte des Ausgangswerts sinkt. Ein solcher
R"uckgang der Absterbewahrscheinlichkeit ist in Abb.\ \ref{lnd2-result} beim Zeitschritt 14000 zu sehen, hier findet
der evolution"are Schritt zur Zweizelligkeit statt.

Die l"angere Lebenserwartung der zweistelligen Pflanzen f"uhrt dazu, da"s weniger Pflanzen pro Zeitschritt absterben
und da"s dementsprechend auch weniger Pflanzen pro Zeitschritt neu entstehen k"onnen. Der Generationenablauf verlangsamt
sich also. Auch Verdr"angungs- und Ausbreitungsprozesse laufen somit langsamer ab. Innerhalb einer Abstammungslinie
sammeln sich daher mehr genetische Unterschiede an. Dies erkl"art die beobachtete Zunahme der Anzahl verschiedener
Genotypen beim Auftreten der zweizelligen Formen. Mit diesem evolution"aren Schritt ist auch eine deutliche Ver"anderung
bei den Distanzverteilungen verbunden, gr"o"sere Distanzwerte werden infolge der st"arkeren Divergenz taxonomischer
Cluster h"aufiger.

In Abb.\ \ref{lnd2-worlds} sind bei Generation 20000 bereits etliche unbeschr"ankt vertikal wachsende Pflanzen zu sehen.
Jedoch auch hier sind s"amtliche Pflanzen dieser Art steril. Abb.\ \ref{lnd2-genomes}c zeigt das Entwicklungsprogramm
einer solchen Pflanze. Ebenso wie die bei verschiedenen LindEvol-GA--L"aufen beobachteten Pflanzen dieser Gestalt
speichern au"ser der apikalen Zelle alle Zellen Energie. In LindEvol-B f"uhrt dies zu einer hohen Lebenserwartung,
nicht jedoch zur Reproduktion. Die H"aufigkeit der Pflanzen dieser Art erkl"art sich allein durch ihre lange Lebensdauer.
Im Zeitschritt 20000 sind sie erheblich h"aufiger als im Zeitschritt 10000, weil das Entwicklungsprogramm der zweizelligen
Pflanzen bereits das Keimzellgen, das f"ur das unbeschr"ankte vertikale Wachstum erforderlich ist, enth"alt. Die 
Wahrscheinlichkeit f"ur die Entstehung eines Wachstumsprogramms f"ur eine unbeschr"ankt wachsende, vertikale Pflanze
ist somit f"ur ein Genom einer zweizelligen Pflanze h"oher als bei einem Genom einer einzelligen Pflanze.

Beim Zeitschritt 25000 ist ein starker Anstieg der durchschnittlichen Gr"o"se zu beobachten. Hier treten die ersten
unbeschr"ankt wachsenden, nicht sterilen Pflanzen auf. Abb.\ \ref{lnd2-worlds} zeigt, da"s es sich um vertikale Formen
handelt, die dicht in der Welt wachsen. Abb.\ \ref{lnd2-genomes}d zeigt das Entwicklungsprogramm dieser Pflanzen.
Nur die unterste Zelle speichert bei diesen Pflanzen Energie, die Absterbewahrscheinlichkeit dieser Pflanzen entspricht
damit derjenigen der zweizelligen Formen. Alle anderen Zellen au"ser der apikalen Zelle setzen die von ihnen absorbierte
Energie zur Samenproduktion ein. Nachdem die Pflanze eine gewisse H"ohe erreicht hat, produziert sie damit in fast jedem
Zeitschritt einen Samen (sofern sie nicht durch eine andere Pflanze "uberschattet wird). Eine zweizellige Pflanze produziert
dagegen im Mittel nur in jedem zweiten Zeitschritt einen Samen. Der Vorteil der unbeschr"ankt vertikal wachsenden Pflanzen
Pflanzen ist also nicht eine h"ohere Lebenserwartung, sondern eine effektivere Samenproduktion.

Im Zeitschritt 30000 existieren einige Pflanzen, die seitlich angelagerte Zellen aufweisen. Diese sind jedoch alle
steril; sie sind durch Mutationen des samenproduzierenden Gens (Gen 38 in Abb.\ \ref{lnd2-genomes}d) entstanden.
Die ersten nicht sterilen Formen mit seitlichen Ausw"uchsen treten kurz nach dem Zeitschritt 35000 auf. Durch diese
Entwicklung kommt es erstmalig zu Raumkonflikten und Angriffen in signifikantem Ausma"s. Dies schl"agt sich in einem
Anstieg der Anzahl der Angriffe und in einem R"uckgang der Populationsgr"o"se nieder. Durch die Angriffe sinkt auch
das durchschnittliche Alter deutlich ab.

Das Auftreten der Pflanzen mit seitlichen Verzweigungen geht mit einem Wechsel der Samenverbreitungsform einher.
W"ahrend der Phasen der Dominanz einzelliger und strikt vertikal wachsender Pflanzen findet die Reproduktion haupts"achlich
durch \verb|localseed| statt. Die Nachkommen entstehen also so nah wie m"oglich an der Mutterpflanze. Da bei strikt
vertikalem Wachstum keine r"aumliche Konkurrenz stattfindet, ist dies eher vorteilhaft f"ur die Pflanzen, weil die Strategie
der Vermehrung mittels \verb|localseed| zur Erhaltung eines konkurrenzarmen Umfeldes beitr"agt. Pflanzen, die seitliche
Verzweigungen ausbilden, greifen dabei zwangsl"aufig auch ihre Nachbarn an. F"ur solche Pflanzen ist es ung"unstig, die
Nachkommen bevorzugt in der N"ahe der Mutterpflanze zu erzeugen.

Im Zeitschritt 45000 sind die Pflanzenformen mit seitlichen Verzweigungen fest etabliert (s.\ Abb.\ \ref{lnd2-worlds}).
Die Abbildungen \ref{lnd2-genomes}e und \ref{lnd2-genomes}f zeigen die Entwicklungsprogramme von zwei verschiedenen
Formen dieses Typs. Die Entwicklungsprogramme verzweigter Formen sind erheblich komplexer als die der einfachen vertikalen
Formen. Die Samenproduktion findet haupts"achlich in den seitlichen Pflanzenbereichen statt. Mit der Entstehung der
komplexeren Entwicklungsprogramme geht eine Tendenz zur Verringerung der effektiven Mutationsraten einher,
dieser ist jedoch nicht entfernt so ausgepr"agt wie in den L"aufen von LindEvol-GA,
bei denen dieses Ph"anomen beobachtet wurde.


\subsection{Variation des Kontrollparameters $d_l$}
\label{leanover-discussion}

\begin{sloppypar}
Der Kontrollparameter $d_l$, der die Modifikation der Absterbewahrscheinlichkeit einer Pflanze in Abh"angigkeit von
ihrem "Uberhangskoeffizienten (Gl.\ \ref{leanover-eq}) steuert, erwies sich als "au\discretionary{s-}{s}{"s}ert
wichtig f"ur den qualitativen
Verlauf der Simulationen. Bei kleinen Werten f"ur diesen Kontrollparameter kommt es regelm"a"sig zur Entstehung
von Pflanzen, die die gesamte Welt "uberdecken. Abb.\ \ref{lnd2-otherworlds}a (Seite \pageref{lnd2-otherworlds}) 
zeigt ein Beispiel einer solchen
Pflanze. Hat eine Pflanze erst einmal dieses Ausma"s erreicht, hat sie einen sehr kleinen "Uberhangskoeffizienten.
Pflanzen dieser Art sind oft steril und neigen zum Energiespeichern, was ihre Absterbewahrscheinlichkeit klein h"alt.
Solche Pflanzen k"onnen als superstabil bezeichnet werden; sie sind sehr gut ausbalanciert und haben einen hohen
Energiegehalt, was zu einer entsprechend sehr kleinen Absterbewahrscheinlichkeit f"uhrt. Ferner sind solche Pflanzen
aufgrund ihrer Gr"o"se faktisch nicht erfolgreich angreifbar. Existiert eine superstabile Pflanze w"ahrend eines ausgedehnteren
Zeitraums, sterben alle anderen Pflanzen ab. Wenn die superstabile Pflanze steril ist, werden keine neuen Pflanzen
mehr gebildet. In diesem Fall sind nach dem Absterben der superstabilen Pflanze keine Individuen mehr in der Population,
die Simulation ist damit vorzeitig beendet.
\end{sloppypar}

Bei allen untersuchten Kontrollparameterkonfigurationen, bei denen massive Buschformen entstanden, kam es fr"uher
oder sp"ater zum Auftreten superstabiler Pflanzen mit nachfolgendem vorzeitigen Ende der Evolution. Bereits bei
$d_l=0.1$ sind Ans"atze zur Entstehung buschf"ormiger Pflanzen zu beobachten, wie Abb.\ \ref{lnd2-otherworlds}b
zeigt. Beide Darstellungen in Abb.\ \ref{lnd2-otherworlds} entstammen aus L"aufen, die mit den Mutationsraten $m_r=0.002$, $m_i=0.0004$
und $m_d=0.0004$ durchgef"uhrt wurden. Bei niedrigeren Mutationsraten kommt es auch bei $d_l=0.1$ zur Entstehung
superstabiler Pflanzen. Dieses Ph"anomen ist unerw"unscht, weil es in der nat"urlichen Evolution keine derart massiven, globalen
Effekte einzelner Individuen gibt. Aus diesem Grund wurde f"ur den Kontrollparameter $d_l$ f"ur die im Folgenden
besprochene Serie von L"aufen der Wert 0.15 eingesetzt, wie dies auch bei \runname{m00007n000e25l15} der Fall war.


\subsection{Systematische Untersuchung der Eigenschaften von \\
LindEvol-B in Abh"angigkeit von den Mutationsraten}

\begin{figure}

\unitlength1cm
\begin{picture}(16,20)
\put(0,14.0){\makebox(7,4)[b]{\epsfysize=4cm \epsffile{lnd2-numcells.eps}}}
\put(0,13.5){\makebox(7,0.5){\textbf{a:} Gr"o"se}}
\put(8,14.0){\makebox(7,4)[b]{\epsfysize=4cm \epsffile{lnd2-energy.eps}}}
\put(8,13.5){\makebox(7,0.5){\textbf{b:} Energiegehalt}}
\put(0,9.5){\makebox(7,4)[b]{\epsfysize=4cm \epsffile{lnd2-glength.eps}}}
\put(0,9.0){\makebox(7,0.5){\textbf{c:} Genoml"ange}}
\put(8,9.5){\makebox(7,4)[b]{\epsfysize=4cm \epsffile{lnd2-nused.eps}}}
\put(8,9.0){\makebox(7,0.5){\textbf{d:} Anzahl aktiver Gene}}
\put(0,5.0){\makebox(7,4)[b]{\epsfysize=4cm \epsffile{lnd2-age.eps}}}
\put(0,4.5){\makebox(7,0.5){\textbf{e:} Alter}}
\put(8,5.0){\makebox(7,4)[b]{\epsfysize=4cm \epsffile{lnd2-mutcounter.eps}}}
\put(8,4.5){\makebox(7,0.5){\textbf{f:} Mutationsexponent}}
\put(0,0.5){\makebox(7,4)[b]{\epsfysize=4cm \epsffile{lnd2-ddc.eps}}}
\put(0,0.0){\makebox(7,0.5){\textbf{g:} DVK}}
\put(8,0.5){\makebox(7,4)[b]{\epsfysize=4cm \epsffile{lnd2-gdv.eps}}}
\put(8,0.0){\makebox(7,0.5){\textbf{h:} genetische Diversit"at}}
\end{picture}

\caption[LindEvol-B bei verschiedenen Mutationsraten]
{\label{lnd2-mutseries}
Durchschnittswerte verschiedener Me"sgr"o"sen bei L"aufen mit verschiedenen Mutatiosnraten, jeweils ermittelt nach dem Zeitschritt 49900.
Auf der Abszisse ist in logarithmischem Ma"sstab der Wert aufgetragen, der f"ur die Insertions- und die Deletionsrate
gew"ahlt wurde, die Austauschrate war jeweils f"unfmal so hoch.
}
\end{figure}

Der Selektionsvorgang, also das Absterben von Pflanzen und ihre Reproduktion, wird in LindEvol-B nicht durch einen
expliziten Kontrollparameter, sondern durch das Wachstumsverhalten der Pflanzen, aus dem Absterbewahrscheinlichkeit
und Samenproduktion resultieren, bestimmt. F"ur die Mutation sind hingegen, wie in LindEvol-GA, die entsprechenden
Raten als explizite Kontrollparameter anzugeben. Anders als bei LindEvol-GA, wo das Evolutionsgeschehen in Abh"angigkeit
von Selektions- und Mutationsrate charakterisiert werden konnte (\ref{lndga-parameterscan}), kann bei LindEvol-B
nur der Einflu"s unterschiedlicher Mutationsraten auf die Evolution untersucht werden. Zu diesem Zweck wurde eine
Serie von L"aufen durchgef"uhrt, bei der die Mutationsraten systematisch von $2 \cdot 10^{-6}$ bis $10^{-2}$
gesteigert wurden. Die Austauschrate betrug dabei jeweils das f"unffache der Insertions- und Deletionsrate
(vgl.\ \ref{lnd2-controlvalues}).

Abb.\ \ref{lnd2-mutseries} zeigt die Ergebnisse dieses Experiments. Die Durchschnittswerte verschiedener Gr"o"sen
(Gr"o"se der Pflanzen, Energiegehalt der Pflanzen, Genoml"ange, Anzahl aktivierter Gene, Alter und Mutationsexponent)
sowie die DVK und die genetische Diversit"at wurden im Zeitschritt 49900 bestimmt.

Sowohl bei sehr geringen Mutationsraten ($m_i < 10^{-5}$) als auch bei hohen Mutationsraten ($m_i > 0.003$) entwickeln sich
nur einzellige Pflanzen. Bei hohen Mutationsraten liegt dies daran, da"s komplexere Entwicklungsprogramme durch Mutationen
leichter zerst"ort werden, wie dies bereits in LindEvol-GA beobachtet wurde (vgl.\ \ref{xlong1005section}). Bei LindEvol-B
kommt erschwerend hinzu, da"s komplexere Formen eine l"angere Generationsdauer haben, ihr Genom wird somit w"ahrend ihres
Lebens h"aufiger mutiert. Da bei hohen Mutationsraten stets einige sterile, mehrzellige Pflanzen in der Population existieren,
liegt der Durschnittswert der Gr"o"se etwas "uber eins.

Bei sehr niedrigen Mutationsraten kommt dagegen w"ahrend der hier betrachteten Anzahl von Zeitschritten keine
Evolution mehrzelliger Formen zustande. Anders als bei LindEvol-GA ist die Produktion mehrerer Zellen allein nicht
mit einem Selektionsvorteil verbunden. Nur Pflanzen, die sowohl eine m"oglichst ausbalancierte Gestalt entwickeln
und dabei auch Samen produzieren, haben gegen"uber den Einzellern Vorteile. Der evolution"are Schritt hin zu den
einfachsten Formen dieser Art, etwa den in \runname{m00007n000e25l15} beobachteten, zweizelligen Pflanzen,
wird bei den sehr geringen Mutationsraten innerhalb der ersten 50000 Zeitschritte nicht vollzogen. Damit kommt
w"ahrend des Beobachtungszeitraums keine Evolution komplexerer Formen in Gang.

\begin{sloppypar}
Im Bereich zwischen den sehr niedrigen und den hohen Mutationsraten entwickeln sich innerhalb von 50000
Zeitschritten Gemeinschaften aus mehrzelligen Pflanzen. Der zuvor besprochene Lauf \runname{m00007n000e25l15} geh"ort
zu dieser Klasse. In einigen F"allen entstehen beschr"ankt wachsende Formen, was zu nur m"a"sig erh"ohten
Durchschnittswerten der Gr"o"se und des Energiegehalts f"uhrt, w"ahrend sich in anderen F"allen, wie auch in
\runname{m00007n000e25l15}, unbeschr"ankt wachsende Pflanzen entwickeln. Bei der Evolution solcher komplexer Wuchsformen
kommt es auch zu einem Anstieg der Durchschnittswerte des Energiegehalts und des Alters.
\end{sloppypar}

Eine Korrelation zwischen der Evolution komplexer Wuchsformen und der durchschnittlichen Genoml"ange ist nicht feststellbar.
Bei den durchschnittlichen Genoml"angen f"allt lediglich ein Spitzenwert bei Mutationsraten von 0.005 auf, dieser ist
darauf zur"uckzuf"uhren, da"s das Maximum des Erwartungswerts f"ur die Verl"angerung eines Genoms nahe bei $m_i = m_d = 0.005$
liegt (vgl.\ \ref{mutationdef}). Die durchschnittliche Anzahl aktiver Gene zeigt hingegen, wie auch in LindEvol-GA,
eine deutliche Korrelation mit der Evolution komplexer Wuchsformen.

Eine deutliche Evolution des Mutationsexponenten, wie in LindEvol-GA, wurde in keinem Lauf von LindEvol-B beobachtet. Ein Grund
hierf"ur ist sicherlich die Tatsache, da"s sich die Generationen in LindEvol-B "uberlappen, so da"s stets auch ein Anteil an
jungen Pflanzen, die noch nicht das Entwicklungsstadium, in dem sie aktiv die effektiven Mutationsraten verringern, erreicht
haben, in die Durchschnittsberechnung eingeht. Der Hauptgrund f"ur die geringe Tendenz zur aktiven Verringerung der effektiven
Mutationsraten d"urfte aber sein, da"s die Pflanzen in LindEvol-B steuern k"onnen, wieviele Samen sie erzeugen. Die Rate der
Produktion von Nachkommen ist also nicht, wie in LindEvol-GA, durch eine Selektionsrate extern fest vorgegeben. Zur Erzielung
einer maximalen Ausbreitungsrate k"onnen die Pflanzen also nicht nur ihre effektiven Mutationsraten regulieren, sondern alternativ
dazu auch mehr Samen produzieren.

Dennoch ist bei den Mutationsexponenten ein Zusammenhang mit den Mutationsraten zu erkennen. Bei sehr geringen Mutationsraten,
bei denen keine Entwicklung h"oherer Formen stattfindet, bleibt auch jegliche aktive Ver"anderung der effektiven Mutationsraten
aus. Die L"aufe, bei denen die Evolution komplexer Pflanzenformen beobachtet wurde, zerfallen hinsichtlich der Evolution der
Mutationsexponenten in zwei Klassen. Im Bereich mit $m_i < 0.0002$ werden sowohl positive als auch negative Durchschnittswerte
des Mutationsexponenten beobachtet. Bei $m_i \geq 0.0002$ liegen dagegen bei s"amtlichen L"aufen, in denen komplexe Pflanzen
entstanden, die Durchschnittswerte des Mutationsexponenten im negativen Bereich. Dies ist als Hinweis auf eine im Bereich von
$m_i = 0.0002$ liegende Grenze zu werten, oberhalb derer die aktive Erniedrigung der effektiven Mutationsraten signifikante
Selektionsvorteile mit sich bringt. Wegen der intrinsischen Gestaltung der Selektionseinfl"usse, der unterschiedlichen individuellen
Lebensdauern und der daraus resultierenden unterschiedlich h"aufigen Anwendung der Mutationsoperatoren auf das Genom einer individuellen
Pflanze kann zu LindEvol-B keine mathematische Absch"atzung der Mutationsfolgen in der Art, wie dies f"ur LindEvol-GA m"oglich
war (\ref{mutconsequences}), entwickelt werden, auf deren Grundlage diese Grenze mit mathematischen Methoden charakterisiert werden
k"onnte.

Die DVK zeigt, wie auch in LindEvol-GA, eine deutliche Korrelation zu der Evolution komplexer Pflanzenformen.
Besonders ausgepr"agt ist die Korrelation zwischen der DVK und der durchschnittlichen Anzahl aktiver Gene,
die ein Ma"s f"ur die Komplexit"at der Entwicklungsprogramme ist. Die Korrelation zwischen der DVK und den
Durchschnittswerten von Gr"o"se, Energiegehalt und Alter wird dagegen von den starken Unterschieden zwischen
beschr"anktem und unbeschr"anktem Wachstum gest"ort. Wie auch bei LindEvol-GA erweist sich die genetische
Diversit"at als Ma"s, das zur Charakterisierung von L"aufen, bei denen komplexe Formen entstehen, ungeeignet ist; die genetische
Diversit"at erreicht bei den h"ochsten Mutationsraten ihre Maximalwerte, sie ist nicht geeignet, um Populationen, die im Sequenzraum
ann"ahernd gleichverteilt sind, von solchen zu unterscheiden, die eine komplex strukturierte Verteilung im Sequenzraum aufweisen.

Das Verhalten von LindEvol-B in Abh"angigkeit von den Mutationsraten entspricht in seiner Charakteristik
dem Verhalten von Zellularautomaten in Abh"angigkeit von dem $\lambda$-Parameter \cite{Langton90,Langton92}.
Mithilfe des $\lambda$-Parameters kann das Verhalten von Zellularautomaten abgesch"atzt werden.
Bei niedrigen Werten des $\lambda$-Parameters wird f"uhrt die Dynamik der Zellularautomaten zu einem Fixpunkt,
d.h.\ zu einer Konfiguration, bei der beim "Ubergang zum n"achsten Zeitschritt bei keinem Elementarautomaten mehr
eine Zustands"anderung stattfindet. Bei h"oheren Werten zeigt sich eine periodische
Dynamik. Bei hohen $\lambda$-Werten entspricht die Dynamik des Automaten einem statistischen Rauschen,
Korrelationen zwischen den Zust"anden in aufeinanderfolgenden Zeitschritten sind nicht mehr festzustellen.
Im Bereich des Phasen"ubergangs zwischen Automaten mit periodischer Dynamik und Automaten mit randomisierter
Dynamik werden Automaten mit einer komplexen Dynamik beobachtet.

Sowohl die Mutationsraten als auch der $\lambda$-Parameter bestimmen den Grad der Randomisierung im jeweiligen
System. In beiden F"allen sind komplexe Ph"anomene in einem intermedi"aren Bereich des Kontrollparameters
zu beobachten. Aufgrund dieser Gemeinsamkeit kann die Evolution komplexer Pflanzenformen in LindEvol-B als
\textsl{life at the edge of chaos} \cite{Langton92} klassifiziert werden.

In LindEvol-B ist das Verhalten des Systems bei geringer und bei hoher Randomisierung auf der
Ebene der Ph"anotypen sehr "ahnlich, in beiden F"allen dominieren einzellige Pflanzen. Die klare Korrelation
zwischen hohen DVK-Werten und der Evolution komplexer Pflanzenformen zeigt, da"s auch in LindEvol-B
hohe DVK-Werte den Bereich der \textsl{edge of chaos} anzeigen. Da zu LindEvol-B keine mathematische
Mutationsfolgenabsch"atzung m"oglich ist, kann diese \textsl{edge of chaos} nicht mit mathematischen
Methoden charakterisiert werden, wie dies bei LindEvol-GA geschehen ist (\ref{lndga-parameterscan}).


\section[Zusammenfassung]{Zusammenfassende Bewertung der Ergebnisse \\ von LindEvol-B}
\label{lnd2-summary}

Mit LindEvol-B ist die Entwicklung eines Modells der LindEvol-Familie gelungen, in dem die Selektionsrate nicht
ein extern vorzugebender Kontrollparameter ist, sondern durch intrinsische Prozesse gestaltet wird. Somit ist
die Evolution h"oherer Komplexit"at durch eine Anpassung der Lebens- und Generationsdauer m"oglich. Auch die Populationsgr"o"se
ist nicht, wie in LindEvol-GA, fest vorgegeben, sondern lediglich durch die Breite der Welt beschr"ankt. Infolgedessen
sind die Pflanzen nicht mit gleichm"a"sigen Abst"anden in der Welt angeordnet, sondern es bildet sich typischerweise
ein dichter Pflanzenteppich am Boden der Welt aus, und es kommt zu erheblich mehr Kontakten zwischen Pflanzen als in
den LindEvol-GA--L"aufen, die in Kapitel \ref{lindevol-ga} diskutiert wurden. Von diesem Unterschied abgesehen sind
die in LindEvol-B und in LindEvol-GA beobachteten Pflanzenformen recht "ahnlich. Die Evolution von Pflanzen globaler
Ausdehnung kann bei LindEvol-GA ausgeschlossen werden, indem die Breite der Welt gr"o"ser als das Zweifache der Dauer
einer Vegetationsperiode gew"ahlt wird. In LindEvol-B kann die Entstehung superstabiler Pflanzen durch die Verwendung
eines ausreichend hohen Wertes f"ur $d_l$ erschwert werden. Dies verschafft vertikal wachsenden Pflanzen einen Vorteil,
der in LindEvol-GA nicht existiert.

Einzellige, reproduktionsf"ahige Pflanzen haben in LindEvol-B eine erheblich gr"o"sere evolution"are Stabilit"at als
Einzeller in LindEvol-GA. Es wurde gezeigt, da"s in LindEvol-B sowohl bei sehr niedrigen als auch bei sehr hohen Mutationsraten
keine Evolution mehrzelliger Formen stattfindet. Nur bei intermedi"aren Mutationsraten kommt es zur Evolution komplexer,
mehrzelliger Formen. Die Evolution komplexer Pflanzenformen in LindEvol-B wurde somit als \textsl{edge of chaos} Ph"anomen
charakterisiert. In dem Intervall der Mutationsraten, bei denen mehrzellige Pflanzen evolvieren, befindet sich ein Grenzwert
der Mutationsraten oberhalb dessen eine deutliche Tendenz zur Evolution negativer Mutationsexponenten beobachtet wird.
Es wurde, wie bereits in LindEvol-GA, eine Korrelation zwischen hohen DVK-Werten und dem Bereich der
\textsl{edge of chaos} festgestellt. 

\begin{sloppypar}
Trotz der erweiterten evolution"aren M"oglichkeiten wurden in keinem Lauf von LindEvol-B Pflanzen beobachtet,
deren Gestalt sich grundlegend von den mit LindEvol-GA beobachteten Pflanzen unterscheidet. Eine naheliegende
Vermutung ist, da"s dies auf die Beschr"ankung des Pflanzenwachstums auf zwei Dimensionen zur"uckzuf"uhren ist.
Daher wurde ein Modell mit dreidimensionaler Welt entwickelt. Da lokale Strukturen bei dreidimensionaler Topologie
auf Achtzehnbitzahlen abgebildet werden (\ref{celldef}), wurde die blockorientierte Genominterpretation
erweitert. Bei dieser erweiterten Version wurde als linke Regelseite eine vollst"andig spezifizierte lokale Struktur
in drei Bytes codiert, sechs Bits blieben dabei unbenutzt. Zusammen mit dem Aktionscode umfa"ste ein Gen somit
vier Bytes. Es zeigte sich jedoch, da"s mit diesem Modell bei unterschiedlichsten Kontrollparameters"atzen keine
Evolutionsprozesse eintraten, die Anfangspopulation starb stets aus, ohne Nachkommen zu produzieren. Ebenso kamen
keine Zellteilungen zustande. Der triviale Grund f"ur diesen Befund ist, da"s bei der erweiterten blockorientierten
Genominterpretation die Wahrscheinlichkeit f"ur die zufallsgesteuerte Generierung eines Keimzellgens $2^{-18}$
betr"agt, w"ahrend sie f"ur LindEvol-B bei $2^{-8}$ liegt. Um mit derselben Wahrscheinlichkeit wie im zweidimensionalen
Modell eine Startpopulation mit vermehrungsf"ahigen Pflanzen zu erzeugen, m"u"sten in dem dreidimensionalen Modell
demnach rund tausendfach gr"o"sere Startpopulationen erzeugt werden, was die im Rahmen dieser Arbeit verf"ugbaren Rechnerkapazit"aten
massiv "uberschritten h"atte. Zu diesem technischen Hinderungsgrund gesellt sich die "Uberlegung, da"s die Wahrscheinlichkeit
f"ur die Aktivierung eines Gens bei Achtzehnbit-Zellzust"anden entsprechend obiger Rechnung erheblich kleiner ist als
bei Achtbit-Zust"anden. Das Prinzip der blockorientierten Genominterpretation hatte sich somit als ungeeignet f"ur
die dreidimensionale Topologie erwiesen. Diese Feststellung gab zur Entwicklung der promotororientierten Genominterpretation
Ansto"s.
\end{sloppypar}

% Selektionsrate intrinsisch gestaltet: veraenderliche Anzahl abgestorbener / neu gebildeter Pflanzen pro Zeitschritt.
% Keine Entstehung grundlegend neuer Formen im Vergleich zu LindEvol-GA
% Beschraenkungen verhindern Entwicklung eines 3D-Modells
% DVK erweist sich auch hier als Indikator fuer komplexe Phaenomene
% Evolution hoeherer Komplexitaet moeglich durch Anpassung der Generationsdauer (langsamere Selektion,
%     komplexere Formen)
% Nicht mathematisch charakterisierbare Grenze der Mutationsraten


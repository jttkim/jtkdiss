\chapter[LindEvol-P]{LindEvol-P: Asynchrone Generationen und promotororientierte Genominterpretation}
\label{lindevol-5}


\section[Definition]{Definition von LindEvol-P}
\label{lnd5-def}

Bei der Entwicklung von LindEvol-P wurden folgende Komponenten (\ref{modeldef}) verwendet:

\begin{itemize}
\item Topologie: Zweidimensionales Gitter (\protect\ref{topo2d}).
\item Zellparameter: Die Zellen haben einen internen Zustandsparameter von
    16 Bit Breite (\protect\ref{celldef}).
\item \begin{sloppypar}Zellaktionen: \verb|divide|, \verb|flyingseed|, \verb|localseed|,
    \verb|mut-|, \verb|mut+| sowie \verb|statebit| (\protect\ref{cellactiondef}).
    \end{sloppypar}
\item Genominterpretation: LindEvol-P arbeitet mit promotororientierter
    Genominterpretation (\protect\ref{prominterdef}).
\item Mutation: Die in LindEvol-P implementierten Mutationstypen sind
    bitweise Austauschmutationen, byteweise Insertionen und Deletionen
    sowie Genduplikationen (\protect\ref{mutationdef}).
\end{itemize}

Die Mechanismen der Vermehrung, des Absterbens und des Angriffs sind wie in
LindEvol-B (\ref{lnd2-def}). LindEvol-P unterscheidet sich nur in der
Genominterpretation von LindEvol-B, der Ablaufplan enspricht also vollst"andig
dem Flu"sdiagramm in Abb.\ \ref{lnd2-flowchart}. Tabelle \ref{lnd5-controlparams}
zeigt die Kontrollparameter von LindEvol-P.


% \subsection{Kontrollparameter von LindEvol-P}

\begin{table}[tb]

\noindent\begin{tabularx}{\linewidth}{|l|l|X|} \hline
Formel- & Bezeichner              & Beschreibung \\
buchstabe &                       & \\ \hline
$p_s$ & \verb|psize_init|         & Populationsgr"o"se am Anfang der Simulation \\
$l_s$ & \verb|glen_init|          & L"ange der am Simulationsstart zuf"allig generierten Genome \\
$M_r$ & \verb|m_replacement|      & Basiswert der Austauschrate \\
$M_i$ & \verb|m_insertion|        & Basiswert der Insertionsrate \\
$M_d$ & \verb|m_deletion|         & Basiswert der Deletionsrate \\
$M_{\mathit{dup}}$ & \verb|m_duplication| & Basiswert der Duplikationsrate \\
$m_f$ & \verb|m_factor|           & Modifikationsfaktor f"ur Mutationsraten \\
$d$   & \verb|p_random_death|     & Basiswahrscheinlichkeit f"ur Absterben pro Zeitschritt \\
$d_n$ & \verb|rdeath_f_numcells|  & Modifikationskoeffizient f"ur die Absterbewahrscheinlichkeit
                                    in Abh"angigkeit von der Gr"o"se der Pflanze \\
$d_e$ & \verb|rdeath_f_energy|    & Modifikationskoeffizient f"ur die Absterbewahrscheinlichkeit
                                    in Abh"angigkeit von der Gesamtenergie der Pflanze \\
$d_l$ & \verb|leanover_penalty|   & Maximalwert der Erh"ohung der Absterbewahrscheinlichkeit
                                    bei "uberh"angender Wuchsform \\
$w$   & \verb|world_width|        & Breite der Welt \\
$h$   & \verb|world_height|       & H"ohe der Welt \\
      & \verb|num_days|           & Anzahl Tage pro Vegetationsperiode \\
      & \verb|num_divide|         & Anzahl Codewerte f"ur \verb|divide| \\
      & \verb|num_flyingseed|     & Anzahl Codewerte f"ur \verb|flyingseed| \\
      & \verb|num_localseed|      & Anzahl Codewerte f"ur \verb|localseed| \\
      & \verb|num_mutminus|       & Anzahl Codewerte f"ur \verb|mut-| \\
      & \verb|num_mutplus|        & Anzahl Codewerte f"ur \verb|mut+| \\
      & \verb|num_statebit|       & Anzahl Codewerte f"ur \verb|statebit| \\
      & \verb|random_seed|        & {\slshape seed} zur Initialisierung des Zufallsgenerators \\ \hline
\end{tabularx}

\caption{\label{lnd5-controlparams}
Kontrollparameter von LindEvol-P.
}
\end{table}

\section{Ergebnisse}
\label{lnd5-results}

Bei der Untersuchung verschiedener L"aufe von LindEvol-P zeigte sich, da"s sich bei diesem Modell in vielen
F"allen eine erheblich komplexere Dynamik entwickelt, als dies in LindEvol-GA und LindEvol-B der Fall war.
Innerhalb eines Laufs k"onnen nacheinander unterschiedlichste Lebensgemeinschaften entstehen, was erhebliche
Schwankungen bei vielen der ermittelten Me"sgr"o"sen nach sich zieht. Zur Ermittlung von Korrelationen der verschiedenen
beobachteten Gr"o"sen untereinander und zu den Kontrollparametern sind daher erheblich umfangreichere Simulationsserien
erforderlich als bei den zuvor besprochenen Modellen. Da
die promotororientierte Genominterpretation erheblich zeitaufwendiger ist als die blockorientierte Genominterpretation,
war die Durchf"uhrung entsprechend gro"ser Serien jedoch nicht m"oglich. Aus diesem Grund beschr"ankt sich der
Ergebnisteil zu LindEvol-P auf die Darstellung und Besprechung des Laufs \runname{m02n000e25l15}, anhand dessen die
Vielfalt der evolution"aren Prozesse in LindEvol-P gut ersichtlich ist.
Dieser Lauf wurde mit den folgenden Kontrollparametern durchgef"uhrt:

\medskip
\begin{tabular}{ll}
Populationsgr"o"se am Anfang: & 50 \\
Mutationsraten: & $m_r=0.002$, $m_i=0.0004$, \\
		& $m_d=0.0004$, $m_{\mathit{dup}}=0.0$ \\
Mutationsfaktor: & 1.5 \\
Absterbewahrscheinlichkeit: & 0.35 \\
Modifikatoren d.\ Absterbewahrscheinlichk.: & $d_e=-2.5$, $d_n=0.0$, $d_l=0.15$ \\
Breite der Welt: & 500 \\
H"ohe der Welt: & 50  \\
Genoml"ange der Startpopulation: & 50 Bytes \\
Anzahl Zeitschritte:             & 50000 \\
\verb|num_divide| & 32 \\
\verb|num_mutminus| & 4 \\
\verb|num_mutplus| & 4 \\
\verb|num_flyingseed| & 4 \\
\verb|num_localseed| & 4 \\
\verb|num_statebit| & 16 \\
\verb|random_seed| & 12345 \\
\end{tabular}
\medskip

\begin{sloppypar}
Abb.\ \ref{lnd5-result} zeigt das Verlaufsdiagramm des LindEvol-P--Si\-mu\-la\-tions\-laufs \runname{m02n000e25l15},
in Abb.\ \ref{lnd5-worlds} (Seite \pageref{lnd5-worlds}) ist die Welt in verschiedenen Phasen der Evolution in \runname{m02n000e25l15}
dargestellt.
\end{sloppypar}


\begin{figure}
\unitlength1cm
\begin{picture}(16,20)
\put(0,0){\makebox(16,20)[b]{\epsfxsize=16cm \epsffile{m02n000e25l15.eps}}}
\end{picture}

\caption[Verlaufsdaten eines LindEvol-P--Laufs]
{\label{lnd5-result}
Verlaufsdiagramm des LindEvol-P--Laufs \runname{m02n000e25l15}. Einige Boxen enthalten
zwei Kurven. In der Beschriftung dieser Boxen sind jeweils die Bezeichnungen f"ur beide
Me"sgr"o"sen, getrennt durch zwei Schr"agstriche, angegeben. Bei allen Boxen dieser Art
ist einer der beiden Werte stets kleiner oder gleich dem anderen Wert: Die Anzahl verschiedener
Genotypen kann maximal der Populationsgr"o"se entsprechen; der Energiegehalt einer Pflanze
ist durch ihre Gr"o"se nach oben beschr"ankt; und die Anzahl der fliegenden Samen kann die
Gesamtzahl der Samen nicht "ubersteigen.
}
\end{figure}

% \begin{figure}
% 
% \unitlength1cm
% \begin{picture}(16,21)
% 
% \put(0,20.5){\makebox(16,2)[b]{\epsfxsize=16cm \epsffile{m02n000e25l15-0000100w.eps}}}
% \put(0,20.0){\makebox(16,0.5){\textbf{a:} Zeitschritt 100}}
% \put(0,18.0){\makebox(16,2)[b]{\epsfxsize=16cm \epsffile{m02n000e25l15-0001200w.eps}}}
% \put(0,17.5){\makebox(16,0.5){\textbf{b:} Zeitschritt 1200}}
% \put(0,15.5){\makebox(16,2)[b]{\epsfxsize=16cm \epsffile{m02n000e25l15-0003800w.eps}}}
% \put(0,15.0){\makebox(16,0.5){\textbf{c:} Zeitschritt 3800}}
% \put(0,13.0){\makebox(16,2)[b]{\epsfxsize=16cm \epsffile{m02n000e25l15-0009995w.eps}}}
% \put(0,12.5){\makebox(16,0.5){\textbf{d:} Zeitschritt 9995}}
% \put(0,10.5){\makebox(16,2)[b]{\epsfxsize=16cm \epsffile{m02n000e25l15-0010110w.eps}}}
% \put(0,10){\makebox(16,0.5){\textbf{e:} Zeitschritt 10110}}
% \put(0,8.0){\makebox(16,2)[b]{\epsfxsize=16cm \epsffile{m02n000e25l15-0010265w.eps}}}
% \put(0,7.5){\makebox(16,0.5){\textbf{f:} Zeitschritt 10265}}
% \put(0,5.5){\makebox(16,2)[b]{\epsfxsize=16cm \epsffile{m02n000e25l15-0028250w.eps}}}
% \put(0,5){\makebox(16,0.5){\textbf{g:} Zeitschritt 28250}}
% \put(0,3.0){\makebox(16,2)[b]{\epsfxsize=16cm \epsffile{m02n000e25l15-0036585w.eps}}}
% \put(0,2.5){\makebox(16,0.5){\textbf{h:} Zeitschritt 36585}}
% \put(0,0.5){\makebox(16,2)[b]{\epsfxsize=16cm \epsffile{m02n000e25l15-0039040w.eps}}}
% \put(0,0){\makebox(16,0.5){\textbf{i:} Zeitschritt 39040}}
% \end{picture}
% 
% \caption[Weltdarstellungen aus einer LindEvol-P--Simulation]
% {\label{lnd5-worlds}
% Weltdarstellungen aus dem LindEvol-P--Lauf \runname{m02n000e25l15}
% }
% \end{figure}


\subsection{Unterschiede zwischen der Evolution bei blockorientierter und bei promotororientierter Genominterpretation}
\label{lnd5-blockprodiffs}

Nach dem Start der Simulation bildet sich zun"achst ein dichter Teppich einzelliger Pflanzen aus (Abb.\ \ref{lnd5-worlds}a).
Nach etwas mehr als 1000 Zeitschritten treten die ersten mehrzelligen Pflanzen auf. Im Zeitschritt 1200 sind bereits
etliche unbeschr"ankt wachsende Pflanzen zu sehen (Abb.\ \ref{lnd5-worlds}b). Wie in LindEvol-B sind auch hier die ersten
unbeschr"ankt wachsenden Formen zun"achst steril, die Evolution reproduktionsf"ahiger, unbeschr"ankt wachsender Pflanzen
erfolgt jedoch erheblich z"ugiger als in LindEvol-B, weil dadurch, da"s bei der promotororientierten Genominterpretation
unspezifizierte Bits m"oglich sind, der zur Entwicklung unbeschr"ankt wachsender, reproduktionsf"ahiger Pflanzen erforderliche
evolution"are Schritt erheblich einfacher ist. Bei der blockorientierten Genominterpretation, die in LindEvol-B eingesetzt
wird, ist das Entwicklungsprogramm einer einzelligen, reproduktionsf"ahigen Pflanze:

\begin{verbatim}
  1: 00000000 -> flyingseed
\end{verbatim}

Das Entwicklungsprogramm einer typischen unbeschr"ankt wachsenden, reproduktionsf"ahigen Pflanze bei blockorientierter
Genominterpretation lautet:

\begin{verbatim}
  1: 00000000 -> divide 6
  2: 00000010 -> divide 6
  3: 01000010 -> flyingseed
\end{verbatim}

Um von dem Entwicklungsprogramm einer einzelligen, reproduktionsf"ahigen Pflanze zu dem einer unbeschr"ankt wachsenden,
reproduktionsf"ahigen Pflanze zu gelangen, ist also die "Anderung des Keimzellgens sowie die Bereitstellung zweier weiterer
Gene erforderlich.
Bei der promotororientierten Genominterpretation reicht ein konstitutiv exprimiertes Gen aus, um eine einzellige,
reproduktionsf"ahige Pflanze zu realisieren:

\begin{verbatim}
  1: **************** -> flyingseed
\end{verbatim}

Durch die Erg"anzung eines weiteren Gens, das alle Zellen, die nicht bereits eine Nachbarzelle unmittelbar "uber sich
haben, zur Produktion einer solchen Zelle instruiert, kann das Entwicklungsprogramm einer unbeschr"ankt wachsenden,
reproduktionsf"ahigen Pflanze entstehen:

\begin{verbatim}
  1: *********0****** -> divide 6
  2: **************** -> flyingseed
\end{verbatim}

Somit zeigt sich, da"s die variable Spezifizit"at der Gene bei promotororientierter Genominterpretation zu einer
deutlich verbesserten Evolvierbarkeit \cite{Ray92} f"uhrt. Durch den Verzicht auf die Spezifikation irrelevanter
Bits kann die Funktion zweier Gene, die bei blockorientierter Genominterpretation f"ur die Initiierung des vertikalen
Wachstums und f"ur seine Fortsetzung ben"otigt werden, bei promotororientierter Genominterpretation in einem Gen
vereinigt werden. Gene mit geringer Spezifizit"at k"onnen innerhalb eines Entwicklungsprogramms unterschiedliche
Funktionen "ubernehmen, wie das Beispiel des konstitutiv exprimierten Gens zeigt.


\subsection{Pleiotrope Mutanten eines buschf"ormigen Grundmusters}
\label{lnd5-pleiotropy}

\begin{figure}

\unitlength1cm
\begin{picture}(16,20)

\put(0,15.5){\makebox(5,4.5)[b]{\epsfxsize=5cm \epsffile{lnd5-09995_p243.eps}}}
\put(5,15.5){\makebox(11,4.5)[b]{\epsfxsize=11cm \epsffile{lnd5-09995_243.eps}}}
\put(0,15){\makebox(16,0.5){\textbf{a:} Pflanze aus Zeitschritt 9995 mit Entwicklungsprogramm}}

\put(0,10.5){\makebox(5,4.5)[b]{\epsfxsize=5cm \epsffile{lnd5-10110_p020.eps}}}
\put(5,10.5){\makebox(11,4.5)[b]{\epsfxsize=11cm \epsffile{lnd5-10110_020.eps}}}
\put(0,10){\makebox(16,0.5){\textbf{b:} Pflanze aus Zeitschritt 10110 mit Entwicklungsprogramm}}

\put(0,5.5){\makebox(5,4.5)[b]{\epsfxsize=5cm \epsffile{lnd5-10110_p105.eps}}}
\put(5,5.5){\makebox(11,4.5)[b]{\epsfxsize=11cm \epsffile{lnd5-10110_105.eps}}}
\put(0,5){\makebox(16,0.5){\textbf{c:} Pflanze aus Zeitschritt 10110 mit Entwicklungsprogramm}}

\put(0,0.5){\makebox(5,4.5)[b]{\epsfxsize=5cm \epsffile{lnd5-10265_p186.eps}}}
\put(5,0.5){\makebox(11,4.5)[b]{\epsfxsize=11cm \epsffile{lnd5-10265_186.eps}}}
\put(0,0){\makebox(16,0.5){\textbf{d:} Pflanze aus Zeitschritt 10265 mit Entwicklungsprogramm}}
\end{picture}

\caption[LindEvol-P: Pflanzen mit Entwicklungsprogrammen]{
\label{lnd5-plants}
Verschiedene busch- oder baumf"ormig wachsende Pflanzen mit den \textsl{listings} ihrer Entwicklungsprogramme.
Nach den Index eines Gens ist in runden Klammern der Index des Promotors im Genom angegeben.
Die ph"anotypisch recht unterschiedlichen Pflanzen erweisen sich als genetisch eng verwandt.
}
\end{figure}

Auf die Evolution unbeschr"ankt wachsender Pflanzen folgt bald ein weiterer evolution"arer Schritt, bei dem Pflanzen
mit einer Ausdehnung in die Breite auftreten, die sich zun"achst nur nach links erstreckt (Abb.\ \ref{lnd5-worlds}c).
Diese Entwicklung ist mit einem signifikanten Anstieg der durchschnittlichen Absterbewahrscheinlichkeit
verbunden, denn der "Uberhangkoeffizient ist bei den einseitig in die Breite wachsenden Pflanzen erheblich gr"o"ser
als 0. Viele der in die Breite wachsenden Pflanzen bilden zun"achst einen vertikalen Stamm aus und beginnen erst
sp"ater mit der Erzeugung seitlicher Ausw"uchse. Auf diese Weise bleibt der "Uberhang im Verh"altnis zur Gesamtzellmasse
geringer.

Die in die Breite wachsenden Pflanzen etablieren sich in der Population. Im Zeitraum bis etwa zum Zeitschritt 10000
entwickeln sie sich von den baumartigen Formen im Zeitschritt 3800 zu gedrungenen Zellhaufen, die nur eine geringe
H"ohe erreichen, was sich in einem R"uckgang der maximalen und der durchschnittlichen Gr"o"se niederschl"agt.

Kurz vor dem Zeitschritt 10000 tritt die erste Pflanze auf, die eine sich beidseitig ausbreitende Struktur ausbildet
(Abb.\ \ref{lnd5-worlds}d). Diese Pflanzen sind erheblich besser ausbalanciert als die einseitig nach links wachsenden
Formen. Die Gr"o"se der Pflanzen erreicht dadurch neue Maximalwerte. Aus dem Grundmotiv der beidseitig in die Breite
wachsenden Pflanze entwickelt sich innerhalb weniger hundert Zeitschritte eine erstaunliche Vielfalt von Variationen.
Abb.\ \ref{lnd5-plants} zeigt einige Vertreter dieser Formen mit ihren Entwicklungsprogrammen.

Abb.\ \ref{lnd5-plants}a zeigt eine Pflanze mit dem urspr"unglichen Entwicklungsprogramm f"ur ein stark in beide
Richtungen breitenorientiertes Wachstum. Gen 0 bewirkt das H"ohenwachstum, wenn sich unmittelbar oberhalb einer
Zelle keine Nachbarzelle befindet, wird an dieser Position eine Zelle produziert, sofern die Zelle, die Gen 0 exprimiert,
energiereich ist. Gen 2 bewirkt in entsprechender Weise, da"s eine Zelle, die keine linke Nachbarzelle hat, an dieser
Position eine Tochterzelle produziert. Dieses Gen kann nur erfolgreich exprimiert werden, wenn Gen 0 nicht aktiviert
wurde, da beide Gene energieverbrauchende Teilungsaktionen herbeif"uhren. Aufgrund von Gen 4 produzieren energiereiche
Zellen, die einen rechten Nachbarn besitzen, einen Samen. Wurden die Gene 0, 2 und 4 nicht exprimiert, tritt Gen 6
in Aktion und f"uhrt zur Produktion einer Tochterzelle rechts neben der aktuellen Zelle.

Die vergleichsweise geringe Auspr"agung des H"ohenwachstums im Vergleich zum Breitenwachstum kommt durch die Aktivit"at
des Gens 5 zustande. Die linke Seite dieses Gens spezifiziert ein gel"oschtes Energiebit. Somit bewirkt es, da"s im
Zustand einer energielosen Zelle im folgenden Zeitschritt (in dem die Zelle wieder energiereich sein kann) das Bit
6 gesetzt ist, auch wenn die Zelle keinen direkt "uber ihr gelegenen Nachbarn hat. In einer Zelle, die in einem Zeitschritt
energielos war, wird daher im folgenden Zeitschritt das Gen 0 reprimiert; sie wird, wenn sie wieder energiereich ist,
durch die Expression der Gene 2, 4 oder 6 seitliche Zellen oder einen Samen produzieren. Die Expression von Gen 0 bleibt
auf solche Zellen, die in zwei aufeinanderfolgenden Zeitschritten energiereich sind, beschr"ankt.

Abb.\ \ref{lnd5-plants}b zeigt eine Pflanze aus dem Zeitschritt 10110, die ein erheblich schlankeres, buschf"ormiges
Erscheinungsbild aufweist. In ihrem Entwicklungsprogramm ist das Grundprinzip des H"ohen- und beidseitigen Breitenwachstums
erhalten, im Vergleich zu Abb.\ \ref{lnd5-plants}a ist lediglich die Repression von Gen 0 in energielosen Zellen entfallen.
Das neu hinzugekommene Gen 10 bleibt ohne Auswirkungen auf den Ablauf der Entwicklung, weil das durch seine Aktivit"at
gesetzte Bit 8 in keinem Gen spezifiziert ist.

Bei der in Abb.\ \ref{lnd5-plants}c gezeigten Pflanze wurde das Entwicklungsprogramm w"ahrend des Wachstums durch eine
Mutation ver"andert. Der Repressor f"ur das vertikale Wachstum spezifiziert hier ein gel"oschtes Bit 11, was im Kontext
des Entwicklungsprogramms einer konstitutiven Expression gleichkommt, weil Bit 11 durch kein Gen gesetzt werden kann.
Das vertikale Wachstum wird somit vollst"andig unterdr"uckt. Da die Pflanze aber eine betr"achtliche H"ohe hat, kann
dies offensichtlich nicht vom Anfang des Lebens der Pflanze an der Fall gewesen sein. Auch die Aktivierungsh"aufigkeiten
belegen dies. Das Gen 1, welches ebenfalls konstitutiv exprimiert wird, wurde rund dreimal so h"aufig aktiviert wie
Gen 5; wenn beide von Anfang an konstitutiv exprimiert worden w"aren, h"atten sie die gleiche Aktivierungsh"aufigkeit.

Leider kann mithilfe der derzeit implementierten Simulationssoftware nicht gekl"art werden, ob eine Mutation im Gen 5
zur Abschaltung des vertikalen Wachstums gef"uhrt hat. Es ist auch denkbar, da"s ein Gen, das das Setzen von Bit 11
bewirkte, durch eine Mutation zerst"ort wurde und erst dadurch die faktisch konstitutive Expression von Gen 5 zustandekam.

Abb.\ \ref{lnd5-plants}d zeigt eine Pflanze, deren Gestalt zun"achst vermuten lassen k"onnte, da"s ihr Entwicklungsprogramm
sich von denen der bisher besprochenen Pflanzen grundlegend unterscheidet. Das \textsl{listing} zeigt jedoch ein
Entwicklungsprogramm, das dieselbe Grundstruktur wie die restlichen Entwicklungsprogramme in Abb.\ \ref{lnd5-plants} aufweist.
Neu ist lediglich ein zus"atzlicher Operator in Gen 0, der f"ur Bit 2 den gel"oschten Zustand spezifiziert. Damit wird
Gen 0 nur in Zellen exprimiert, die keinen rechtsunteren Nachbarn haben. Dies ist einerseits am rechten Rand
der Pflanze regelm"a"sig der Fall, hier findet dementsprechend haupts"achlich das H"ohenwachstum statt. Von dem nach rechts
tendierenden Hauptspro"s aus werden durch die Aktivit"at des Gens 2 nach links weisende, seitliche Strukturen ausgebildet.
Gen 6 bewirkt dagegen die Verschiebung des Hauptsprosses nach rechts.

Die seitlichen Strukturen zeigen ein geringes Ma"s an H"ohenwachstum. Auch dieses kommt durch Gen 0 zustande. Das Wachstum
der nach links ragenden Strukturen findet durch die Aktivierung von Gen 2 am Ende dieser Strukturen statt.
Da die oberen Zellschichten mehr Licht empfangen, w"achst die oberste Zellschicht einer solchen Struktur am schnellsten nach
links. Am linksoberen Ende kommt es so zur Ausbildung einer seitlichen Spitze von der Dicke einer Zellschicht. Die "au"serste
Zelle an so einer Spitze hat nur einen rechten Nachbarn, aber keinen rechtsunteren Nachbarn. Somit exprimiert sie Gen 0. Von
der dabei produzierten Zelle ausgehend w"achst dann eine neue Zellschicht der seitlichen Struktur, an deren links"au"sersten
Ende schlie"slich eine weitere Aktivierung des Gens 0 erfolgen kann.


\subsection{Weitere Pflanzenformen und Evolutionsprozesse}

Die Klasse der busch- und baumartigen Pflanzen stirbt noch vor dem Zeitschritt 11000 aus. Diese Pflanzen werden von Formen verdr"angt,
die ausschlie"slich horizontal am Boden entlangwachsen. Bei dieser Wachstumsstrategie finden viele Angriffe statt. Gleichzeitig kommt
es zu einer Verknappung der zur Produktion neuer Pflanzen verf"ugbaren Positionen und somit zu einem erheblichen R"uckgang
der Populationsgr"o"se.

Bei Zeitschritt 22000 erscheinden Pflanzen, die diagonal nach rechts in die H"ohe wachsen. Durch die Ausbildung nach links
auswachsender, seitlicher Strukturen wird eine Verringerung des "Uberhangskoeffizienten erreicht. Diese Strukturen k"onnen
sehr unterschiedlich stark ausgepr"agt sein. Verschiedene Formen dieser Art sind in Abb.\ \ref{lnd5-worlds}g zu sehen.
Nach dem Zeitschritt 36000 treten regelm"a"sig Riesenformen mit extrem umfangreichen, nach links weisenden seitlichen
Strukturen auf. Die Abbildungen \ref{lnd5-worlds}h und \ref{lnd5-worlds}i zeigen Beispiele f"ur solche Pflanzen. Die Pflanze
in Abb.\ \ref{lnd5-worlds}i "uberschattet einen sehr gro"sen Teil der Welt und l"ost dadurch eine Energiekrise aus. Die
anderen Pflanzen erhalten nur wenig Energie und k"onnen somit pro Zeitschritt nicht soviele Samen produzieren, wie Pflanzen
absterben. Damit kommt es zu einem deutlichen R"uckgang der Populationsgr"o"se. Bei wesentlich l"angerer Lebensdauer
dieser Riesenpflanze h"atte sie sich zu einer superstabilen Pflanze entwickeln k"onnen (vgl.\ \ref{leanover-discussion})

Die Tendenz zur Entstehung von Riesenpflanzen geht nach dem Zeitschritt 45000 deutlich zur"uck. Mit diesem Proze"s
ist auch eine Abnahme der Anzahl der Angriffe verbunden. Dieser Vorgang kann somit als Evolution kooperativeren
Wachstumsverhaltens eingeordnet werden.

Die durchschnittliche Spezifit"at der Gene bleibt w"ahrend des gesamten Laufs gering. Insbesondere nimmt sie im Verlauf der
Simulation nicht in signifikanter Weise zu. Der Grund daf"ur ist vermutlich, da"s Entwicklungsprogramme komplexer Pflanzen
mit wenigen Genen geringer Spezifit"at realisiert werden k"onnen, wie in (\ref{lnd5-pleiotropy}) deutlich wurde.
In LindEvol-P besteht demnach kein Selektionsdruck, der zur Evolution von Genen mit hoher Spezifit"at f"uhren k"onnte.
Die Tendenz zur Evolution von Genen geringer Spezifit"at beim Fehlen eines solchen Selektionsdrucks kann dadurch erkl"art
werden, da"s Gene geringer Spezifit"at mit geringerer Wahrscheinlichkeit durch Mutationen ver"andert werden, weil sie
weniger Operatoren haben und somit k"urzer sind.

Auffallend im Vergleich zu den Ergebnissen der Modelle mit blockorientierter Genominterpretation ist, da"s die Anzahl der
aktiven Gene bei der Evolution komplexer Pflanzenformen nur geringf"ugig ansteigt. Der Grund hierf"ur ist, da"s ein Gen
geringer Spezifit"at unterschiedliche Funktionen "ubernehmen kann, f"ur die bei der blockorientierten Genominterpretation
mehrere Gene erforderlich sind. Dies wurde in (\ref{lnd5-blockprodiffs}) anhand eines Beispiels demonstriert.

In der Kurve der Durchschnittswerte des Alters fallen bei den Zeitschritten 13000 und  38000 zwei Phasen auf, in denen das
Durchschnittsalter signifikant ansteigt. Das Minimum der Genoml"ange f"allt in diesen Phasen auf sehr kleine Werte. Diese
sehr kurzen Genome kommen zustande, wenn der Aktionsteil eines h"aufig exprimierten Gens so mutiert, da"s er den Befehl
\verb|mut+| codiert. Bei einer Pflanze signifikanter Gr"o"se wird die effektive Deletionsrate dann bis auf 1.0 erh"oht,
was bedeutet, da"s s"amtliche Bytes bei der Mutation deletiert werden. Danach besitzt die Pflanze keine Gene mehr, die
exprimiert werden k"onnten. Ihr Energiegehalt wird dadurch sehr hoch. Bei einer gut ausbalancierten Pflanze wird damit
die Absterbewahrscheinlichkeit sehr gering, so da"s die Pflanze mehrere tausend Zeitschritte lang "uberlebt, was zu den
beobachteten Peaks des durchschnittlichen Alters f"uhrt. In Abb.\ \ref{lnd5-worlds}i ist die Pflanze, die zum zweiten
Peak dieser Art f"uhrt, links von der Riesenpflanze zu sehen. Das sporadische Auftreten von Pflanzen, die ein sehr hohes
Alter erreichen, ist als Effekt der Modellierung von Mutation und Absterbewahrscheinlichkeit zu werten, der keine
Entsprechung in der Natur hat.


\section[Zusammenfassung]{Zusammenfassende Bewertung der Ergebnisse \\ von LindEvol-P}

Der Verlauf des Evolutionsgeschehens in LindEvol-P erwies sich als im Vergleich zu LindEvol-B "uberraschend komplex.
Einer der Gr"unde daf"ur ist, da"s bei promotororientierter Genominterpretation "Uberg"ange zwischen verschiedenen
Wuchsformen erheblich leichter sind als bei blockorientierter Genominterpretation. Daher kommt es zu einer schnelleren
Abfolge evolution"arer Schritte. Ein weiterer Grund ist, da"s Entwicklungsprogramme insbesondere komplexerer Pflanzen
bei promotororientierter Genominterpretation erheblich kompakter codierbar sind als bei blockorientierter Genominterpretation
(man vergleiche z.B.\ Abb.\ \ref{xlong0108genomes}f auf Seite \pageref{xlong0108genomes} mit Abb.\ \ref{lnd5-plants}b).
Formal betrachtet k"onnen in der linken Seite einer Regel bei blockorientierter Genominterpretation nur Minterme
der Zustandsbits spezifiziert werden, w"ahrend bei der promotororientierten Genominterpretation dadurch, da"s
auch unspezifizierte Bits m"oglich sind, beliebige Monome in der linken Seite einer Regel stehen k"onnen
(vgl.\ \cite{Oberschelp92,Schmidt77}). Da komplexe Evolutionsgeschehen in LindEvol-P zeigt, da"s es durch
die hier entwickelte Methode der promotororientierten Codierung von Regeln gelungen ist, diesen erweiterten Funktionsraum
in einer f"ur die evolution"are Entwicklung und Optimierung geeigneten Weise zu repr"asentieren. Das Prinzip der
promotororientierten Regelcodierung kann generell verwendet werden, um Boolesche Funktionen so zu codieren, da"s
ihre Entwicklung durch evolution"are Methoden m"oglich ist.

In LindEvol-P konnten verschiedene pleiotrope Transformationen von Ph"anotypen beobachtet werden. Bei den Modellen mit
blockorientierter Genominterpretation waren solche Mutationen nicht festzustellen. Der Grund daf"ur ist, da"s bei der
blockorientierten Genominterpretation ein Gen nur bei genau einem Zellzustand exprimiert wird. "Andert sich die linke
Seite der in einem Gen codierten Regel durch eine Mutation, hat das Muster der Zellen innerhalb der Pflanze, in denen
das Gen nach der Mutation exprimiert wird, keine "Uberlappung mit dem Expressionsmuster vor der Mutation. Bei der
promotororientierten Genominterpretation sind dagegen Erweiterungen oder Einschr"ankungen des Expressionsmusters
eines Gens durch Mutationen im Operatorbereich m"oglich. Durch solche \textsl{gain-of-function}-Mutationen k"onnen
in LindEvol-P durch geringf"ugige Ver"anderungen auf genetischer Ebene unterschiedlichste Varianten aus einer Grundform
entstehen. Pleiotrope Mutationen dieser Art werden in molekularen Systemen insbesondere bei Genen, die bei der Steuerung
morphogenetischer Prozesse mitwirken, beobachtet (vgl.\ \cite{Carroll95}, \cite{Watson}). Besonders auff"allig sind
die Gemeinsamkeiten zwischen den in \runname{m02n000e25l15} beobachteten Modifikationen der Kontrolle des vertikalen Wachstums und
den bei nat"urlichen Pflanzen beobachteten pleiotropen Effekten der Beeinflussung der apikalen Dominanz. KNOTTED-1,
ein Hom"oobox-Gen aus Mais (\textsl{Zea mays}), bewirkt bei seiner "Uberexpression in Tabak (\textsl{Nicotiana tabacum})
den Verlust der apikalen Dominanz, die transformierten Pflanzen sind erheblich kleiner als Tabakpflanzen des Wildtyps \cite{Sinha95}.
Diese Modifikationen sind hinsichtlich ihres Ausma"ses und bis zu einem gewissen Grad auch
hinsichtlich ihrer Art mit den Unterschieden zwischen den in Abb.\ \ref{lnd5-plants}b und \ref{lnd5-plants}d gezeigten
Pflanzen vergleichbar.

Bei der in Abb.\ \ref{lnd5-plants}c dargestellten Pflanze wurde eine Mutation festgestellt, durch die ein Zustandsbit,
das sonst nur bei Anwesenheit einer Nachbarzelle gesetzt ist, nun unabh"angig von der Existenz einer solchen Zelle immer
gesetzt wird. Infolge dieser Mutation befinden sich die Zellen dieser Pflanze konstitutiv in einem Zustand, in dem sie sich ohne
die Mutation nur bei bestimmten externen Bedingungen bef"anden. Diese Art von Mutation erinnert stark an den Mechanismus
mancher Onkogene wie z.B.\ \textsl{src}, \textsl{erbB} und \textsl{fms} (vgl.\ \cite{Watson}, Kapitel 26). Eine Tumorbildung
kann mit LindEvol-P jedoch nicht naturgetreu simuliert werden, weil ein Genom die Aktivit"aten aller Zellen einer Pflanze
steuert. Eine Weiterentwicklung des Modells, bei der jede Zelle ein individuelles Genom besitzt, k"onnte leicht realisiert
werden; ein solches Modell h"atte jedoch einen um mehr als das Zehnfache vergr"o"serten Speicherplatzbedarf f"ur die Genome,
auch der Rechenzeitbedarf f"ur die Manipulation dieser Genome w"are erheblich.

Mit dem im Ergebnisteil diskutierten Lauf \runname{m02n000e25l15} konnten viele Prozesse in LindEvol-P nur angedeutet
werden. Von der Durchf"uhrung und Auswertung verschiedener Serien von L"aufen sind interessante Einblicke in das
Evolutionsgeschehen von LindEvol-P zu erwarten. Von besonderem Interesse ist dabei die Charakterisierung der Evolution
der Spezifit"at in Abh"angigkeit von verschiedenen Evolutionsbedingungen.

Zur Weiterentwicklung von LindEvol-P kommt zun"achst eine Neuentwicklung der Mutationskomponente in Betracht, um die
Entstehung extrem langlebiger, genloser Pflanzen zu verhindern. Eine andere interessante Richtung der Weiterentwicklung
ist die Gestaltung von Lebensbedingungen, an die keine gute Adaptation mit wenigen Genen geringer Spezifit"at m"oglich ist.
Diese komplexen Adaptationsanforderungen sollten dabei nicht von au"sen vorgegeben
oder gar fest in das Modell eingebaut sein, sondern sich vielmehr intrinsisch in Abh"angigkeit des
pflanzlichen Wachstumsverhalten als emergentes Ph"anomen entwickeln. Ein Schritt zur Erweiterung des Raumes
f"ur eine intrinsische Gestaltung komplexer Adaptationsanforderungen ist die Entwicklung eines Modells mit
dreidimensionaler Topologie. Dies ist Gegenstand des folgenden Kapitels.


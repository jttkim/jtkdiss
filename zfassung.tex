\chapter{Zusammenfassung und Ausblick}

Mit der LindEvol-Simulationsfamilie wurde ein System etabliert, mit dem der Proze"s der
Evolution genetisch codierter Morphogeneseprogramme mit Computern simuliert werden kann.
Die diskrete Repr"asentation des Raums und das Prinzip der Bestimmung von Zellaktivit"aten
auf der Grundlage lokaler Information wurde dabei von den Zellularautomaten "ubernommen.
Das Vorbild der Lindenmayer-Systeme \cite{Prusinkiewicz94,RozenbergSalomaa86} inspirierte die
Beschreibung elementarer pflanzlicher Wachstumsprozesse durch Regeln. Anders als bei
herk"ommlichen Lindenmayer-Systemen wird jedoch auf die Repr"asentation von Pflanzen
in Zeichenketten verzichtet, die Regeln beziehen sich unmittelbar auf lokale r"aumliche
Pflanzenstrukturen.

Die Pflanzen einer Population wachsen in einer gemeinsamen Gitterwelt. Dabei konkurrieren
sie um Raum und um Energie, die sie f"ur ihr Wachstum ben"otigen. Die Wachstums- und
Reproduktionschancen einer Pflanze werden durch andere Pflanzen in der Population nachhaltig
beeinflu"st, es findet eine intrinsische Adaptation \cite{Packard89} statt.

Die aus der Genetik bekannten Mechanismen der Kontrolle der Expression \cite{Watson} von Genen wurden
in zwei Verfahren zur Interpretation von Bytesequenzen als Regelwerke zur Steuerung des Wachstums
umgesetzt. Durch das einfachere, blockorientierte Verfahren wird die Aktivierung von Genen
in Abh"angigkeit von der Position einer Zelle innerhalb des Organismus repr"asentiert.
Bei der m"achtigeren, promotororientierten Genominterpretation wurde zus"atzlich das
Prinzip der Expressionskontrolle durch Operatoren (im genetischen Sinne) realisiert.
Die Entscheidung "uber die Aktivierung eines Gens erfolgt bei der blockorientierten Genominterpretation
durch einen Minterm und bei der promotororientierten Genominterpretation durch ein Monom.

Der Vergleich der f"ur LindEvol entwickelten Interpretationsverfahren mit der Programmiersprache
\textsl{Tierran} \cite{Ray92} zeigt, da"s die Interpretationsverfahren aufgrund ihrer Gestaltung
als Simulationen genetischer Prozesse einige der Eigenschaften aufweisen, die von \textsc{Ray}
als Voraussetzung f"ur die Evolvierbarkeit einer Sprache identifiziert wurden.

Das in LindEvol verwendete Mutationsverfahren umfa"st einen Mechanismus f"ur die genetisch
gesteuerte Modifikation der effektiven Mutationsraten. Dieser Mechanismus zeichnet sich dadurch
aus, da"s seine Aktivierung, dem molekularen Vorbild entsprechend, Energie erfordert.

Mit den Komponenten des LindEvol-Systems wurde zun"achst das Modell LindEvol-GA mit blockorientierter
Genominterpretation erstellt, bei dem ein genetischer Algorithmus als Evolutionsmchanismus verwendet wurde. 
Das Modell LindEvol-B arbeitet ebenfalls mit blockorientierter Genominterpretation, anstelle des relativ
starren Evolutionsverfahren des genetischen Algorithmus tritt jedoch eine
explizitere, nicht globale Modellierung der Reproduktion und des Absterbens. Die bei LindEvol-GA
als Kontrollparameter extern vorgegebene Selektionsrate wird damit bei LindEvol-B zu einer intrinsisch modellierten
Gr"o"se. Der Einsatz der promotororientierten Genominterpretation in LindEvol-P f"uhrte zu einer
deutlichen Zunahme der Komplexit"at bei der evolution"aren Dynamik und bei den beobachteten
Pflanzenformen. Mit der promotororientierten Genominterpretation war auch die Entwicklung eines
Modells der Evolution dreidimensionaler Pflanzen m"oglich. Dies wurde mit LindEvol-P/3D realisiert.

Als Ma"s zur Charakterisierung der Komplexit"at einer evolvierenden Population wurde die
Distanzverteilungskomplexit"at (DVK) eingef"uhrt. Sie quantifiziert mithilfe der Shannon-Entropie
die Verteilung der Verwandtschaftsgrade in einer Population. Mit der DVK kann der Unterschied
zwischen Populationen mit komplexer taxonomischer Struktur, in denen viele unterschiedliche
Verwandtschaftsgrade vorkommen, unterschieden werden von randomisierten Populationen aus
nicht verwandten Individuen, und von konvergierten Populationen mit eng verwandten Individuen.

In Simulationsl"aufen mit den Modellen konnte eine Vielzahl emergenter Ph"anomene beobachtet
werden, die auch in der Natur auftreten. Zu diesen Ph"anomenen z"ahlen:

\begin{itemize}

\item Die emergente Evolution kooperativen Wachstumsverhaltens, die starke "Ahnlichkeiten mit emergenter Kooperation bei
Simulationen auf der Basis des \textsl{prisoner's dilemma} aufweist.

\item Eine emergente Evolution der aktiven Senkung der effektiven Mutationsraten. Dieses Ph"anomen
wurde dar"uberhinaus als \textsl{evolution towards the edge of chaos} charakterisiert.

\item Die evolutive Entstehung komplexer regulatorischer Netzwerke.

\item Evolution"are Adaptation durch pleiotrope Mutationen. 

\end{itemize}

Mithilfe einer einfachen mathematischen Mutationsfolgenabsch"atzung wurden zwei unterschiedliche evolution"are
Vorteile der aktiven Senkung der Mutationsraten identifiziert. Zum einen erbt bei niedrigeren Mutationsraten ein
gr"o"serer Anteil der Nachkommen eines Individuums dessen genetische Information in unver"andert funktioneller
Form. Niedrigere Mutationsraten stellen in dieser Weise keinen Selektionsvorteil, sondern einen
Reproduktionsvorteil dar. Zum anderen er"offnet die Evolution niedrigerer Mutationsraten die M"oglichkeit
der Evolution komplexerer Wachstumsprogramme, deren Evolution bei h"oheren Mutationsraten wegen der "Uberschreitung
ihrer kritischen Fehlergrenze \cite{Maynard89} nicht m"oglich war. Die komplexeren Wuchsformen k"onnen einen
Selektionsvorteil bedeuten.

Das Modell LindEvol-GA wurde als Testsystem f"ur die Standardverfahren zur Stammbaumrekonstruktion
\textsl{parsimony} und \textsl{neighbor joining} \cite{PHYLIP} eingesetzt. Es wurde demonstriert, da"s signifikante
Abweichungen von der Annahme neutraler Evolution sowie die Evolution der aktive Absenkung der
effektiven Mutationsraten zu Beeintr"achtigungen der Rekonstruktionsgenauigkeit bei den untersuchten
Verfahren f"uhren.

Es wurde gezeigt, da"s zwischen hohen DVK-Werten und vielen anderen Ma"sen und Indikatoren
der Komplexit"at, eine ausgepr"agte Korrelation besteht. Als Komplexit"atskriterien wurden
dabei die Gestalt der evolvierenden Pflanzen sowie die Anzahl ihrer aktiven Gene verwendet.
Ferner wurde bei den einfacheren Modellen eine \textsl{edge of chaos} \cite{Kauffman92,Langton92} charakterisiert. Es
zeigte sich, da"s bei dieser \textsl{edge of chaos} Maximalwerte der DVK gemessen werden.
Auch die \textsl{evolution towards the edge of chaos} durch aktive Verringerung der effektiven
Mutationsraten war stets von einem deutlichen Anstieg der DVK begleitet. Aufgrund dieser
Ergebnisse kann die DVK als ein Ma"s zur Charakterisierung der Komplexit"at evolvierender
Populationen bezeichnet werden. Die DVK kann f"ur alle evolvierenden Systeme, bei denen ein geeignetes Distanzma"s
verf"ugbar ist, berechnet werden. Da bei einer gro"sen Zahl evolvierender Systeme ein solches
Distanzma"s gefunden werden kann, ist die DVK ein Beitrag zur Entwicklung eines allgemeinen Ma"ses
der Komplexit"at evolvierender Systeme.

Das bei der Berechnung der DVK verwendete Distanzma"s sollte f"ur zwei Genome angeben,
wie oft die in dem evolvierenden System aktiven genetischen Operatoren angewendet werden m"ussen,
um das eine Genom in das andere zu transformieren. In den LindEvol-Modellen wurde ausschlie"slich
mit vegetativer Vermehrung gearbeitet. Bei vegetativer Vermehrung gen"ugen sequenzbasierte Distanzma"se wie die Hamming-Distanz
oder der Editierabstand zumindest n"aherungsweise den Anforderungen an das Distanzma"s zur
Berechnung der DVK. Bei evolvierenden Systemen, in denen auch sexuelle Fortpflanzung oder
Rekombination stattfindet, ist die Definition eines geeigneten Distanzma"ses ein ungel"ostes
Problem. Die Schwierigkeit r"uhrt daher, da"s zur Generierung eines bestimmten Genoms aus einem
anderen Genom erfoderliche Anzahl genetischer Operationen bei der Verwendung von Rekombinationsoperatoren
von der Zusammensetzung der Population abh"angt. Das Distanzma"s wird also gleichsam intrinsisch
gestaltet und ist mit sequenzanalytischen Methoden nicht ohne weiteres zu approximieren.

Eine Formulierung der DVK f"ur evolvierende Systeme mit Rekombination und sexueller Vermehrung
w"are ein weiterer Schritt zu einem allgemenen Komplexit"atsma"s f"ur evolvierende Systeme.
F"ur die Untersuchung der nat"urlichen Evolution ist diese Frage jedoch nur f"ur die Bestimmung
der intraspezifischen Komplexit"at von Bedeutung. Rekombination zwischen verschiedenen Arten
ist ein so seltenes Ereignis, da"s sie f"ur die DVK vernachl"assigt werden kann.

Zur Untersuchung der Auswirkungen von Rekombination und sexueller Vermehrung auf die mit der DVK in ihrer
in dieser Arbeit vorgestellten Formulierung bietet sich die Entwicklung neuer
LindEvol-Modelle an, die Rekombination und sexuelle Reproduktion einschlie"sen.
Diese k"onnen auch als Testsystem f"ur alternative oder erweiterte
Formulierungen der DVK dienen. Unabh"angig davon sollte die DVK anhand anderer evolvierender
Systeme, bei denen Zust"ande unterschiedlicher Komplexit"at bekannt sind, weiter untersucht werden.

Im Rahmen solcher Erweiterungen der LindEvol-Familie k"onnen auch weitere komplexe genetische
Operatoren, die auf individuelle Genome wirken, eingef"uhrt werden. Dabei ist insbesondere an
gro"se Duplikationen und an Translokationen zu denken. Diese Erweiterungen k"onnen Gelegenheiten
zur Untersuchung der Evolution von Familien paraloger Gene bieten.

Ein weiterer Anwendungsbereich von LindEvol ist die Erstellung detaillierterer Simulationen f"ur einzelne
Pflanzen. Bei einer solchen Entwicklung w"are auch die explizite Repr"asentation individueller Genome
der Zellen auch mit den heute verf"ugbaren Rechnerkapazit"aten m"oglich, so da"s die Auswirkungen lokaler,
somatischer genetischer Ver"anderungen simuliert werden k"onnten.


% -> Ray / Tierra: Robustheit gegen Mutationen, wenige Befehle, kleiner, diskreter
%     Wertebereich fuer Argumente von Befehlen.

% LindEvol-GA: Staerkere Selektion -> hoehere Robustheit gegen Mutationen (groesseres t(s,l))
%     Adaptationsmoeglichkeiten?

% LindEvol-GA -> LindEvol-B: Umwandlung Selektionsrate von extern vorgegebenem
%     Kontrollparameter zu intrinsisch gestalteter Groesse.
%     Hierdurch prinzipiell Moeglichkeit zur Variation anderer Kontrollparameter
%     (Populationsgroesse?)

% LindEvol-P: Bedeutung von gain of function.

% Offenheit: System soll Erweiterungen zulassen.
%   * Sexuelle Fortpflanzung: "pollen"-Befehl: Diffundierender Pollen hin zu "flower"
%   * Erweiterung der Promotororientierten Interpretation:
%     Prom. - Operatoren - Transkr.-Start - Struktur - Terminator.
%     Noetig wg. Depletion des 6bit Aktionscoderaums.
%   * Erweiterung um Zell-Zell-Kommunikation.
% -> LindEvol hat viele Entwicklungsmoeglichkeiten, sowohl was Modellbildung betrifft,
%     als auch im Hinblick auf Untersuchungen spezieller Prozesse (Stammbaumanalyse,
%     Konvergenzproblematik, Orhologie-Paralogie usw.

% Distanzverteilungsanalyse: Kriterien fuer Distanzmass
% DVK bei neutraler und nichtneutraler Evolution sehr aehnlich: Zeigt optimale Suche,
%     unabhaengig davon, ob features auf landscape gefunden werden koennen.
% Als universelles Mass fuer Komplexitaet evolvierender Systeme geeignet!

% Weitere Entwicklung der Modelle: Mehr morphologische Diversitaet / komplexere Oekosysteme?

% Weitere Untersuchungen zu Bandbreite des evolutionaeren Informationsuebertragungswegs?
% Problem mit LindEvol-GA: zuwenig aktive Gene, die unter Selektionsdruck sind, daher
%     koennen Effekte der Selektion nur bei sehr kurzen Genomen und entsprechend beschraenkten
%     Rekonstruktionsmoeglichkeiten untersucht werden.
% Rekonstruktionsuntersuchungen mit Modellen mit Duplikation: Bei Verwendung von Teilsequenzen
%     mit hoher Homologie koennen Probleme mit Orthologie / Paralogie untersucht werden.

% Intrinsische Adaptation: System stellt sich seine eigenen Aufgaben -- fundamentaler Unterschied
%     zu Adaptation an externe, feste Fitnessfunktionen wie in klassischen evolutionaeren Algorithmen.


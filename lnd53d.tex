\chapter[LindEvol-P/3D]{LindEvol-P/3D: Dreidimensionale Version von LindEvol-P}
\label{lnd5-3d}


\section{Definition}

LindEvol-P/3D ist eine Weiterentwicklung von LindEvol-P, bei der das Pflanzenwachstum
in einem dreidimensionalen anstatt in einem zweidimensionalen Raum stattfindet.
LindEvol-P/3D wurde unter Verwendung folgender Komponenten (vgl.\ \ref{modeldef}) erstellt:

\begin{itemize}
\item Topologie: Dreidimensionales Gitter (\protect\ref{topo3d}).
\item Zellparameter: Die Zellen haben einen internen Zustandsparameter von
    32 Bit Breite (\protect\ref{celldef}).
\item \begin{sloppypar} Zellaktionen: \verb|divide|, \verb|flyingseed|, \verb|localseed|,
    \verb|mut-|, \verb|mut+| sowie \verb|statebit| (\protect\ref{cellactiondef}).
    \end{sloppypar}
\item Genominterpretation: LindEvol-P arbeitet mit promotororientierter
    Genominterpretation (\protect\ref{prominterdef}).
\item Mutation: Die in LindEvol-P/3D implementierten Mutationstypen sind
    bitweise Austauschmutationen, byteweise Insertionen und Deletionen
    sowie Genduplikationen (\protect\ref{mutationdef}).
\end{itemize}

Unterschiede zwischen LindEvol-P und LindEvol-P/3D ergaben sich aus Anpassungen an die
dreidimensionale Topologie. Die bin"are Repr"asentation der lokalen Struktur
einer Zelle ist bei der dreidimensionalen Topologie 18 Bit breit. Um, wie bei der zweidimensionalen
Version LindEvol-P, innerhalb des Zellzustands einen Bereich zur Verf"ugung zu stellen, der
ausschlie"slich durch den internen Zustand bestimmt wird, wurde f"ur den internen Zustand
eine Breite von 32 Bit gew"ahlt. Die Berechnung zur Berechnung des
"Uberhangskoeffizienten  $L$ (s.\ Gl.\ \ref{leanover-eq}) wurde erweitert,
um beide Horizontaldimensionen zu ber"ucksichtigen. Zu diesem Zweck werden
die komponentenweisen "Uberhangskoeffizienten $L_x$ und $L_y$ wie bei
der zweidimensionalen Topologie (Gl.\ \ref{leanover-eq}) berechnet:

\begin{displaymath}
L_x = = \frac{2}{n_c(n_c - 1)} \left| \sum_{i=0}^{n_c-1} \frac{w}{2} - ((\frac{3w}{2} + x(C_i) - x(C_0)) \mbox{ mod } w) \right|
\end{displaymath}

\begin{displaymath}
L_y = = \frac{2}{n_c(n_c - 1)} \left| \sum_{i=0}^{n_c-1} \frac{w}{2} - ((\frac{3w}{2} + y(C_i) - y(C_0)) \mbox{ mod } w) \right|
\end{displaymath}

Der effektive "Uberhangskoeffizient, der in die Berechnung der Absterbewahrscheinlichkeit
eingeht, ist das arithmetische Mittel der komponentenweisen "Uberhangskoeffizienten:

\begin{equation}
\label{leanover3d-eq}
L = \frac{L_x + L_y}{2}
\end{equation}

Bei LindEvol-P/3D werden also die "Uberhangwerte in X- und in Y-Richtung
gleicherma"sen ber"ucksichtigt. Au"ser der Bitbreite des internen Zellzustands und der
Erweiterung der Berechnungsmethode f"ur den "Uberhangskoeffizienten wurde der
Modellaufbau unver"andert von LindEvol-P "ubernommen. Eine Auflistung der Kontrollparameter
von LindEvol-P/3D findet sich in Tabelle \ref{lnd5x-controlparams}.


% \subsection{Kontrollparameter von LindEvol-P/3D}

\begin{table}[tb]

\noindent\begin{tabularx}{\linewidth}{|l|l|X|} \hline
Formel- & Bezeichner              & Beschreibung \\
buchstabe &                       & \\ \hline
$p_s$ & \verb|psize_init|         & Populationsgr"o"se am Anfang der Simulation \\
$l_s$ & \verb|glen_init|          & L"ange der am Simulationsstart zuf"allig generierten Genome \\
$M_r$ & \verb|m_replacement|      & Basiswert der Austauschrate \\
$M_i$ & \verb|m_insertion|        & Basiswert der Insertionsrate \\
$M_d$ & \verb|m_deletion|         & Basiswert der Deletionsrate \\
$M_{\mathit{dup}}$ & \verb|m_duplication| & Basiswert der Duplikationsrate \\
$m_f$ & \verb|m_factor|           & Modifikationsfaktor f"ur Mutationsraten \\
$d$   & \verb|p_random_death|     & Basiswahrscheinlichkeit f"ur Absterben pro Zeitschritt \\
$d_n$ & \verb|rdeath_f_numcells|  & Modifikationskoeffizient f"ur die Absterbewahrscheinlichkeit
                                    in Abh"angigkeit von der Gr"o"se der Pflanze \\
$d_e$ & \verb|rdeath_f_energy|    & Modifikationskoeffizient f"ur die Absterbewahrscheinlichkeit
                                    in Abh"angigkeit von der Gesamtenergie der Pflanze \\
$d_l$ & \verb|leanover_penalty|   & Maximalwert der Erh"ohung der Absterbewahrscheinlichkeit
                                    bei "uberh"angender Wuchsform \\
$w$   & \verb|world_width|        & Kantenl"ange der Grundfl"ache der Welt \\
$h$   & \verb|world_height|       & H"ohe der Welt \\
      & \verb|num_days|           & Anzahl Tage pro Vegetationsperiode \\
      & \verb|num_divide|         & Anzahl Codewerte f"ur \verb|divide| \\
      & \verb|num_flyingseed|     & Anzahl Codewerte f"ur \verb|flyingseed| \\
      & \verb|num_localseed|      & Anzahl Codewerte f"ur \verb|localseed| \\
      & \verb|num_mutminus|       & Anzahl Codewerte f"ur \verb|mut-| \\
      & \verb|num_mutplus|        & Anzahl Codewerte f"ur \verb|mut+| \\
      & \verb|num_statebit|       & Anzahl Codewerte f"ur \verb|statebit| \\
      & \verb|random_seed|        & {\slshape seed} zur Initialisierung des Zufallsgenerators \\ \hline
\end{tabularx}

\caption{\label{lnd5x-controlparams}
Kontrollparameter von LindEvol-P/3D.
}
\end{table}

\section{Ergebnisse}

In L"aufen von LindEvol-P/3D ist, wie bei LindEvol-P, eine komplexe Dynamik zu beobachten. Da Simulationen mit LindEvol-P/3D
noch erheblich zeitaufwendiger als L"aufe mit LindEvol-P sind, beschr"ankt sich auch dieser Ergebnisteil auf die
Diskussion eines Laufs. Die Kontrollparameter bei diesem Lauf \runname{m02n000e25l30i} wurden soweit wie m"oglich
von dem in (\ref{lnd5-results}) besprochenen Lauf \runname{m02n000e25l15} "ubernommen:

\medskip
\begin{tabular}{ll}
Populationsgr"o"se am Anfang: & 500 \\
Mutationsraten: & $m_r=0.002$, $m_i=0.0004$, \\
		& $m_d=0.0004$, $m_{\mathit{dup}}=0.0$ \\
Mutationsfaktor: & 1.5 \\
Absterbewahrscheinlichkeit: & 0.35 \\
Modifikatoren d.\ Absterbewahrscheinlichk.: & $d_e=-2.5$, $d_n=0.0$, $d_l=0.30$ \\
Grundfl"ache der Welt: & 50 * 50 \\
H"ohe der Welt: & 20 \\
Genoml"ange der Startpopulation: & 50 Bytes \\
Anzahl Zeitschritte:             & 50000 \\
\verb|num_divide| & 20 \\
\verb|num_mutminus| & 2 \\
\verb|num_mutplus| & 2 \\
\verb|num_flyingseed| & 4 \\
\verb|num_localseed| & 4 \\
\verb|num_statebit| & 32 \\
\verb|random_seed| & 12345 \\
\end{tabular}
\medskip

Die Konfiguration des Genominterpreters mu"ste angepa"st werden, damit ausreichend viele Aktionscodes f"ur
\verb|statebit| verf"ugbar sind, um alle 32 Bits des internen Zustands adressieren zu k"onnen. Da insgesamt nur
64 verschiedene Aktionscodes zur Verf"ugung stehen, mu"sten die anderen Werte entsprechend verkleinert werden,
wobei zu beachten war, da"s auch f"ur \verb|num_divide| mindestens der Wert 18 gew"ahlt werden mu"s, damit
alle verf"ugbaren Zielpositionen f"ur die Tochterzelle bei der Teilung adressierbar sind.

Ferner wurde der Wert f"ur $d_l$ doppelt so hoch wie in den zweidimensionalen Modellen gew"ahlt. Der Grund f"ur
diese Auswahl ist, da"s bei einem Wachstum, das auch in die H"ohe geht, sich aufgrund der Gestaltung der
Nachbarschaft bei dreidimensionaler Topologie (\ref{topo3d}) eine Entfernung von der Keimzellposition nur entweder
in X- oder in Y-Richtung von der Keimzellposition stattfinden kann, nicht jedoch in beide Dimensionen gleichzeitig.
Die "Uberhangskoeffizienten sind daher bei in die H"ohe wachsenden Pflanzen in einer dreidimensionalen Welt nur
ungef"ahr halb so gro"s wie im zweidimensionalen Fall. Durch die Auswahl eines gr"o"seren $d_l$ wurde dieser Effekt
ausgeglichen.

\begin{figure}
\unitlength1cm
\begin{picture}(16,20)
\put(0,0){\makebox(16,20)[b]{\epsfxsize=16cm \epsffile{m02n000e25l30i.eps}}}
\end{picture}

\caption[Verlaufsdaten des LindEvol-P/3D--Laufs \runname{m02n000e25l30i}]
{\label{lnd53d-result}
Verlaufsdiagramm des LindEvol-P/3D--Laufs \runname{m02n000e25l30i}. In der auch bei LindEvol-B
und LindEvol-P verwendeten Weise sind in einigen Boxen zwei Kurven dargestellt (vgl. \protect\ref{lnd5-result}
}
\end{figure}

% \begin{figure}
% \unitlength1cm
% \begin{picture}(16,20)
% \put(0,15.5){\makebox(8,4.5)[b]{\epsfysize=3.7cm \epsffile{m02n000e25l30i-0008000w.eps}}}
% \put(0,15){\makebox(8,0.5){\textbf{a:} Zeitschritt 8000}}
% \put(8,15.5){\makebox(8,4.5)[b]{\epsfysize=3.7cm \epsffile{m02n000e25l30i-0016000w.eps}}}
% \put(8,15){\makebox(8,0.5){\textbf{b:} Zeitschritt 16000}}
% \put(0,10.5){\makebox(8,4.5)[b]{\epsfysize=4cm \epsffile{m02n000e25l30i-0020000w.eps}}}
% \put(8,10.5){\makebox(8,4.5)[b]{\epsfysize=4cm \epsffile{m02n000e25l30i-0020000wt.eps}}}
% \put(0,10){\makebox(16,0.5){\textbf{c:} Zeitschritt 20000, Vorderansicht und Vogelperspektive}}
% \put(0,5.5){\makebox(8,4.5)[b]{\epsfysize=4cm \epsffile{m02n000e25l30i-0024000w.eps}}}
% \put(8,5.5){\makebox(8,4.5)[b]{\epsfysize=4cm \epsffile{m02n000e25l30i-0024000wt.eps}}}
% \put(0,5){\makebox(16,0.5){\textbf{d:} Zeitschritt 24000, Vorderansicht und Vogelperspektive}}
% \put(0,0.5){\makebox(8,4.5)[b]{\epsfysize=4cm \epsffile{m02n000e25l30i-0040000w.eps}}}
% \put(8,0.5){\makebox(8,4.5)[b]{\epsfysize=4cm \epsffile{m02n000e25l30i-0040000wt.eps}}}
% \put(0,0){\makebox(16,0.5){\textbf{e:} Zeitschritt 40000, Vorderansicht und Vogelperspektive}}
% \end{picture}
% 
% \caption[Weltdarstellungen des LindEvol-P/3D--Laufs \runname{m02n000e25l30i}]
% {\label{lnd53d-worlds}
% Darstellungen der Welt in \runname{m02n000e25l30i} bei verschiedenen Zeitschritten. Welten mit komplexeren Pflanzen
% sind sowohl in der Ansicht von der Seite als auch in der Ansicht von oben (Vogelperspektive) dargestellt.
% }
% \end{figure}


Abb.\ \ref{lnd53d-result} zeigt das Verlaufsdiagramm von \runname{m02n000e25l30i}. Wie bei LindEvol-B und bei LindEvol-P
etabliert sich zun"achst eine Population einzelliger Pflanzen.
Etwa bei Zeitschritt 3500 treten zweizellige Pflanzen auf, die die Einzeller verdr"angen. Aus den zweizelligen Formen
entstehen durch Mutationen regelm"a"sig unbeschr"ankt wachsende Formen (s.\ Abb.\ \ref{lnd53d-worlds}a, Seite \pageref{lnd53d-worlds}),
die jedoch steril sind. Reproduktionsf"ahige,
unbeschr"ankt wachsende Formen, die die zweizelligen Pflanzen verdr"angen, treten jedoch erst bei Zeitschritt 11800 auf.
Diese Formen, die in Abb.\ \ref{lnd53d-worlds}b dargestellt sind, dominieren bis zum Zeitschritt 18000. Hier treten
Pflanzen auf, die aus zwei in X-Richtung benachbarten, vertikalen Zellreihen aufgebaut sind.
Die Wachstumsstrategie mit zwei Zellreihen erm"oglicht bei einem minimal erh"ohtem
"Uberhangskoeffizienten eine Verdopplung der Effektivit"at der Samenproduktion.

\begin{sloppypar}
Die evolution"are Sukzession von einzelligen Pflanzen "uber zweizellige und unbeschr"ankt wachsende, vertikale Formen
bis hin zum Auftreten von Pflanzen mit seitlicher Ausdehnung wurde auch in LindEvol-P beobachtet. Im zweidimensionalen
Modell lief dieser Proze"s jedoch erheblich schneller ab. Der Grund f"ur diese Beobachtung ist in der Erweiterung
der Menge unterschiedlicher Aktionen zu suchen. Bei LindEvol-P besteht der Satz der Aktionen aus \verb|divide| mit
acht verschiedenen Parametern, \verb|statebit| mit 16 verschiedenen Parametern, \verb|flyingseed|, \verb|localseed|,
\verb|mut-| und \verb|mut+|, also aus insgesamt 28 verschiedenen Aktionen. Bei LindEvol-P/3D gibt es dagegen 18
verschiedene Parameter f"ur \verb|divide| und 32 verschiedene Parameter f"ur \verb|statebit|, so da"s es insgesamt
54 verschiedene Aktionen gibt. Die Wahrscheinlichkeit einer Mutation, die zu einem evolution"aren Schritt f"uhrt,
ist dementsprechend geringer.
\end{sloppypar}

Im Zeitschritt 20000 (Abb.\ \ref{lnd53d-worlds}c) stehen die zweireihigen Pflanzen bereits in Konkurrenz zu Pflanzenformen,
die unbeschr"ankt in Y-Richtung wachsen. Durch die Koexistenz dieser beiden Formen kommt es zu einer Zunahme der
Angriffe. Die unbeschr"ankt in Y-Richtung wachsenden Pflanzen verschwinden jedoch wieder, was zu einem R"uckgang
der Angriffsh"aufigkeit f"uhrt.

Ein wesentlich st"arkerer Anstieg der Anzahl der Angriffe ist beim Zeitschritt 21000 zu verzeichnen. Abb.\ \ref{lnd53d-worlds}d
zeigt, da"s hier Pflanzenformen mit genetisch nicht beschr"ankterm Wachstum in X-Richtung entstanden sind.
Das Wachstum dieser Pflanzen wird durch Kollisionen mit den jeweils rechten Nachbarn gestoppt, was zu den vielen
beobachteten Angriffen f"uhrt. Die gro"sen, nach rechts unbeschr"ankt wachsenden Pflanzen "uberdecken teilweise mehrere
Gitterpositionen am Boden der Welt. Daraus resultiert eine Abnahme der Populationsgr"o"se.

Die Anzahl der Angriffe geht zwischen den Zeitschritten 26000 und 28000 wieder deutlich zur"uck. Hier etablieren sich Formen,
die nur von einem Bereich aus, den sie am Boden der Welt "uberdecken, diagonal in die H"ohe wachsen (s.\ Abb.\ \ref{lnd53d-worlds}e).
Dadurch werden Kollisionen oberhalb des Bodens der Welt vermieden. Gleichzeitig "uberdecken gro"se Pflanzen bei dieser Wuchsform
einen ausgedehnteren Bereich des Bodens, wodurch es zu einer Verringerung der Keimpositionen und somit zu einem weiteren R"uckgang
der Populationsgr"o"se kommt.


\begin{figure}[t]
{
\scriptsize
\begin{verbatim}
   1 (p=   1) [  9]: ******** ******** 1******* ******** -> flyingseed     9e 2f 54
   4 (p=  14) [ 44]: ******** ******** ******** ******** -> statebit 16    ae f0
   5 (p=  15) [ 44]: ******** ******** ******** ******** -> statebit 16    f0 f0
   8 (p=  18) [  3]: ******** ******** ******1* ******** -> divide 15      81 29 cf
  10 (p=  25) [ 29]: ******** ******** ******** ******** -> statebit 25    bc f9
  11 (p=  26) [ 44]: ******** ******** ******** ******** -> statebit 30    f9 fe
  13 (p=  28) [ 15]: ******** ******** *0****** ******** -> divide 14      a8 0e 4e
  14 (p=  31) [  9]: ******** ******** ******** *****0** -> divide 9       e0 02 49
  15 (p=  36) [  8]: ******** ******** ******** ******** -> mut-           a2 db
  16 (p=  37) [ 35]: ******** ******** 0******* ******** -> statebit 3     db 0f e3
  17 (p=  39) [ 44]: ******** ******** ******** ******** -> statebit 11    e3 6b
  18 (p=  43) [ 36]: ******** ******** ******** ******** -> statebit 26    c3 fa
  21 (p=  48) [ 36]: ***1**** ******** ******** ******** -> statebit 5     d5 3c 65
  28 (p=  59) [ 44]: ******** ******** ******** ******** -> statebit 8     dd e8
  29 (p=  60) [ 44]: ******** ******** ******** ******** -> statebit 8     e8 68
  34 (p=  73) [ 44]: ******** **0***** ******** ******** -> statebit 23    d9 15 f7
  35 (p=  75) [ 39]: ****0**1 ******** ******** ******** -> statebit 30    f7 38 1b fe
  37 (p=  81) [ 44]: ******** ******** ******** ******** -> statebit 22    a1 f6
  38 (p=  82) [ 44]: ******** ******** ******** ******** -> statebit 28    f6 7c
  44 (p=  94) [ 44]: ******** ******** ******** ******** -> statebit 30    a2 7e
  46 (p=  98) [ 44]: ******** ******** ******** ******** -> statebit 19    90 73
  52 (p= 113) [ 44]: ******** ******** ******** ******** -> statebit 24    b1 f8
  57 (p= 118) [ 44]: ******** ******** ******** ******** -> statebit 22    86 f6
  65 (p= 136) [ 44]: ******** ******** ******** ******** -> statebit 4     98 e4
  66 (p= 137) [ 44]: ******** ******** ******** ******** -> statebit 0     e4 60
  69 (p= 141) [ 44]: ******** ******** ******** ******** -> statebit 5     8d e5
  70 (p= 142) [ 44]: ******** ******** ******** ******** -> statebit 12    e5 6c
\end{verbatim}
}

\caption[\textsl{Listing} einer Pflanze aus LindEvol-P/3D]
{\label{lnd53d-listing}
\textsl{Listing} einer Pflanze aus dem Zeitschritt 40000 des LindEvol-P/3D-Laufs \runname{m02n000e25l30i}.
Nur die aktiven Gene sind aufgef"uhrt. Die Zustandsteile der
Regeln wurden im Interesse der "Ubersichtlichkeit in vier Bl"ocke zu je acht Bits eingeteilt.
}
\end{figure}

Mit diesem letzten evolution"aren Schritt ist ein deutliches Absinken der durchschnittlichen Mutationsexponenten
verbunden. Da dieses Ph"anomen nicht in LindEvol-P beobachtet wurde, wird in Abb.\ \ref{lnd53d-listing} das \textsl{listing}
des Genoms einer Pflanze aus dem Zeitschritt 40000 gezeigt. Wie in den \textsl{listing} von in LindEvol-P evolvierten
Genomen (\ref{lnd5-plants}) zeigt sich, da"s nur Gene mit geringer Spezifit"at aktiv sind. Gen 0 veranla"st alle Zellen,
die einen vertikal oberen Nachbarn (Position 15 bez"uglich der Zellen) haben, zur Produktion fliegender Samen.
Wenn eine Zelle keinen oberen Nachbarn hat,
f"uhrt Gen 8 zur Produktion eines solchen Nachbarn, wenn die Zelle einen rechten Nachbarn
(an Position 9 bez"uglich der Zelle) hat. Bei Zellen, die keinen
rechten Nachbarn haben, veranla"st Gen 13 die Produktion einer Zelle links oberhalb dieser Zellen (Position 14 bez"uglich der
Zellen), sofern sich dort nicht bereits eine Nachbarzelle befindet. Gen 14 bewirkt, da"s eine Zelle, die keinen unteren
Nachbarn (Position 2) besitzt, eine Teilung nach rechts ausf"uhrt. Sind f"ur alle vorausgehenden Gene die Aktivierungsbedingungen
nicht erf"ullt, wird die in Gen 15 codierte Aktion \verb|mut-| ausgef"uhrt.

Zusammenfassend f"uhrt Gen 8 zu einem H"ohenwachstum im linken Bereich der Pflanze; Gen 13 bewirkt ein in die H"ohe
orientiertes Wachstum nach links; und Gen 14 veranla"st ein Wachstum nach rechts auf dem Boden der Welt. Daraus ergibt
sich die aus mehreren Zellreihen bestehende, diagonale Pflanzenform. Zellen im Innern der Pflanze produzieren Samen.
Die am rechten, oberen Rand der Pflanze gelegenen Zellen f"uhren den Befehl \verb|mut-| aus.

\begin{sloppypar}
Beim Vergleich von Abb.\ \ref{lnd53d-listing} mit den \textsl{listings} der Entwicklungsprogramme in LindEvol-P
(Abb.\ \ref{lnd5-plants}) f"allt eine erheblich gr"o"sere Anzahl konstitutiv exprimierter Gene, die \verb|statebit|-Aktionen ausl"osen,
auf. In allen F"allen werden Bits gesetzt, die in keinem Zustandsteil einer f"ur die Entwicklung der Pflanze
relevanten Regel spezifiziert sind. Das vermehrte Auftreten solcher Gene in LindEvol-P/3D kann dem Umstand
zugeschrieben werden, da"s die Menge der bei der Pflanzenentwicklung relevanten Bits im Verh"altnis zur Breite
des Zellzustands bei LindEvol-P/3D kleiner als bei LindEvol-P ist. Daher ist die Wahrscheinlichkeit, da"s eine
Mutation an einem solchen Gen Auswirkungen auf die Morphogenese einer Pflanze hat, bei LindEvol-P/3D geringer;
und eine gr"o"sere Menge solcher Gene kann akkumuliert werden, ohne da"s es zu einer signifikanten Erh"ohung
des Risikos einer pleiotropen Mutation kommt.
\end{sloppypar}


\section[Zusammenfassung]{Zusammenfassende Bewertung der Ergebnisse \\ von LindEvol-P/3D}

W"ahrend die Entwicklung eines Modells mit dreidimensionaler Topologie mit blockorientierter Genominterpretation
auf der Basis von LindEvol-B scheiterte, ist sie mit LindEvol-P/3D mit promotororientierter Genominterpretation
gelungen. Dadurch, da"s bei promotororientierter Genominterpretation die Aktivit"at von Genen nicht durch
maximal spezifische Minterme, sondern durch Monome beliebiger Spezifit"at reguliert werden kann, ist die Entstehung
netzwerkartiger regulatorischer Beziehungen zwischen Genen auch bei 32 Bit breiten Zellzust"anden m"oglich.

Abgesehen von der Evolution der aktiven Verringerung der effektiven Mutationsraten wurden bei LindEvol-P/3D keine
evolution"aren Prozesse beobachtet, die bei LindEvol-P nicht festzustellen waren. Insbesondere entwickelten sich
keine Pflanzen, die sich in alle drei Dimensionen des Raums ausdehnen. Die Vermutung, da"s solche Formen auftreten
w"urden, wenn der Simulationslauf \runname{m02n000e25l30i} l"anger fortgesetzt worden w"are, liegt nahe. Ein Indiz
hierf"ur ist die im Vergleich zu LindEvol-P erheblich langsamere Sukzession evolution"arer Schritte. Es ist
auch m"oglich, da"s bei einer gr"o"seren Ausdehnung der Welt komplexere Pflanzenformen entstehen. Denkbar ist aber auch,
da"s die Beschr"ankung auf zwei Dimensionen als emergente Evolution kooperativen Wachstumsverhaltens zu werten ist.
Wegen des erheblichen Rechenzeitbedarfs f"ur Simulationen vom Umfang des Laufs \runname{m02n000e25l30i}  -- seine Durchf"uhrung
dauerte l"anger als 24 Stunden auf der Alpha-Plattform (\ref{material-hardware}) -- war die Kl"arung der Ursachen
f"ur das Ausbleiben der Evolution von Pflanzen mit dreidimensionaler r"aumlicher Ausdehnung durch die Durchf"uhrung
entsprechender L"aufe jedoch im Rahmen dieser Arbeit nicht m"oglich.


